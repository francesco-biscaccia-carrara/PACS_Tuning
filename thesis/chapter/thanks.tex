Al termine di questa tesi, voglio ringraziare di cuore tutti coloro che in questi tre anni 
hanno contribuito, nel bene o nel male, alla mia crescita personale e universitaria. 
\\
In primo luogo, voglio ringraziare il mio relatore, Leonardo Pellegrina, per la pazienza e il 
tempo dedicatomi durante la stesura della tesi. Inoltre, desidero ringraziare Cristina Menghini, 
senza la quale non avrei potuto completare una buona parte della tesi.
\\ 
Oltre ai ringraziamenti accademici, desidero ringraziare tutti coloro che in questi anni mi hanno aiutato a
crescere come persona, oltre che a sostenermi nei momenti di difficoltà: Modolo, Zanzi, Zinca, Sheldon, Colla e Kabir. 
Senza di voi questi tre anni accademici non sarebbero passati così velocemente. 
\\
Non meno importanti, voglio ringraziare tutti i miei compagni di karate: Giacomo, Elia, Nicola, Beatrice, 
Giada, Mauro e la mia allenatrice Alice. 
Anche se, sfortunatamente, non ci vediamo più con la stessa periodicità di un tempo, sarete sempre 
la mia seconda famiglia: mi avete insegnato la disciplina e la perseveranza, qualità senza le quali non
sarei dove sono ora.
\\ 
Per ultimi, ma non per importanza, voglio ringraziare la mia famiglia, in particolare mia madre Roberta
e mio padre Davide. Anche se spesso non lo dimostro, siete fondamentali nella mia vita. Oltre a sfamarmi 
e viziarmi, mi avete sempre dimostrando molta fiducia, sostenendomi in ogni mia scelta. 
Per questo e per mille altri motivi, voglio ringraziarvi dal profondo del mio cuore.
\\ 
In conclusione, ringrazio nuovamente ogni persona citata per avermi sostenuto e aver creduto in me. 
Quest'oggi voglio condividere con voi questo mio traguardo.
Un traguardo che non è un punto di arrivo, bensì la partenza di un nuovo inizio, dove desidero avervi al mio fianco!\\
\emph{Memento audere semper}
