Questa tesi presenta diversi approcci per la riduzione della polarizzazione nella navigazione nei grafi, ovvero gli algoritmi RePBubLik e ShuffLik.
RePBubLik permette di calcolare un'approssimazione accurata per il problema della riduzione del \emph{structural bias}.
In seguito alle considerazioni teoriche, son stati svolti dei confronti tra RePBubLik+, la versione più veloce e pratica dell'
algoritmo RePBubLik, e le attuali tecniche per la riduzione della polarizzaione dei grafi: ROV e 
Node2Vec. Dalle analisi, è emerso che RePBubLik+ permette di ridurre drasticamente il \emph{structural bias} 
del grafo con un numero di archi significativamente inferiore, rispetto agli altri approcci utilizzati. 
\\
In aggiunta, è stata brevemente descritta un tecnica alternativa a RePBubLik, che permette di aumentare
la navigabilità di una rete effettuando \emph{edge swapping}. 
Questo algoritmo prende il nome di ShuffLik ed esiste una sua versione più pratica, ShuffLik+.
Importante notare che ShuffLik risulta efficace in contesti dove non è possibile eseguire \emph{edge inserction} per motivi di costo o di vincoli funzionali, 
operando quindi interventi meno invasivi.
\\
Possiamo osservare diverse opportunità di sviluppo futuro:
\begin{enumerate}
    \item Nella pratica, potrebbe essere difficile stimare le probabilità di transizione all'interno del grafo. I pesi da assegnare agli archi potrebbero 
    essere appresi mediante un algoritmo di machine learning sui dataset.
    \item La determinazione di valori realistici del parametro $t$ non è semplice. È un problema rilevante individuare il valore ideale di random walk da fornire all'algoritmo, in funzione del trade-off tra guadagno e tempo di esecuzione.
    \item Nell'estensione a più ``colori'' dell'algoritmo, diventa importante determinare quale coppia di ``colori'' permette, con un numero ridotto ti archi, di ridurre notevolmente il bias strutturale.
\end{enumerate} 