%intro
\section{Experimental Setup}
The experiments are designed to compare the PACS framework with the standalone IBM ILOG CPLEX Optimization Studio (version 22.1).  
The PACS framework is implemented in C++ for performance reasons, interfacing with the solver through the C API. The source code is available under a non-commercial MIP license at \textbf{[repository link]}.  
All experiments were conducted on a cluster within the UniPD DEI Blade infrastructure, running Rocky Linux 8.10. Each node is equipped with an Intel(R) Xeon(R) E5-2623 v3 processor (4 cores, 3.00 GHz) and 16 GB of RAM. Although the infrastructure is cluster-based, the computational power of a single node is comparable to that of a general-purpose laptop, or even lower.  
For fairness in comparison, PACS executions use 4 logical threads, implemented as C++ standard threads, each running a CPLEX instance restricted to a single core. The baseline counterpart is a standalone CPLEX execution restricted to 4 cores.

\subsection{Dataset}
As anticipated in the previous sections, the benchmark for this comparison must consist of hard MIP instances. To this end, the set of hard instances from MIPLIB2017$^\text{\cite{MIPLIB}}$ has been used as the test bed.\footnote{The instance \textit{tpl-tub-ss16} was excluded, as its execution was terminated due to insufficient computational resources.}  
While CPLEX directly processes each MIP instance, the PACS framework additionally requires a random seed in order to reproduce the same randomized choices across different runs. For this purpose, PACS was evaluated on each instance using the seeds $\{38472910, 56473829, 27384910, 91827364, 8374659\}$. This setup increases statistical reliability and can be considered representative of the algorithm’s standard behavior.  
Consequently, for each instance, five independent PACS runs and one CPLEX run were executed.

\subsection{Metrics}
Both PACS and plain CPLEX terminate as soon as an incumbent solution (i.e., a feasible solution) is identified, subject to a global time limit of five minutes.
Since the PACS framework was executed five times for each instance, the reported solution quality and computation time are given as the mean across all runs.
Moreover, if the majority of PACS executions for a given instance---at least three out of five seeds---fail to produce a solution, the instance is recorded as unsolved.
Instead of directly comparing the objective values of the solutions returned by the two approaches, the MIP gap metric$^\text{\cite{MIPGAP}}$ has been adopted. The MIP gap is defined as follows: given a solution $\hat{x}$ for a MIP with an optimal solution $x$, the primal gap $\gamma(\hat{x}) \in [0,1]$ is
\begin{equation}
\gamma(\hat{x}) =
\begin{cases}
0 & \text{if } |c^T x| = |c^T \hat{x}| = 0, \\
1 & \text{if } c^T x \cdot c^T \hat{x}< 0, \\
\frac{|c^T x-c^T \hat{x}|}{\max\{|c^T \hat{x}|, |c^T x|\}} & \text{otherwise}.
\end{cases}
\end{equation}
This metric captures the relative gap between an incumbent and the best-known solution, normalized to the interval $[0,1]$.  
Since the algorithms operate under a strict time limit, the metric must be extended accordingly. Given a time limit $t_{\max}$, the primal function $p:[0,t_{\max}] \to [0,1]$ is defined as:
\begin{equation}
p(\hat{x}) =
\begin{cases}
1 & \text{if no solution is found up to time $t$}, \\
\gamma(\hat{x}(t)) & \text{if $\hat{x}(t)$ is the incumbent at time $t$}.
\end{cases}
\end{equation}
For each test, two types of plots are produced to summarize and compare performance:
\begin{enumerate}
    \item \textbf{Success Rate vs. Computation Time:}  
    The $x$-axis represents elapsed time, and the $y$-axis reports the cumulative number of instances for which an incumbent was found. The closer a curve is to the top-left corner, the more efficient the algorithm.  
    \item \textbf{Success Rate vs. MIP Gap:}  
    The $x$-axis represents the MIP gap, and the $y$-axis reports the cumulative number of instances solved with a gap less than or equal to the given value. Consistently, curves closer to the top-left corner indicate better performance 
\end{enumerate}
To provide a more compact quantitative comparison, the integral of each curve has been computed and reported in a corresponding bar chart, one for each plot.

\subsection{Tolerance Parameters}
Because both algorithms involve floating-point computations and comparisons, PACS employs a set of tolerance parameters. In these experiments, the parameters were selected to be reasonable relative to the defaults used by CPLEX:
\begin{itemize}
    \item \textbf{Zero-tolerance parameter} $\epsilon$: values smaller than $\epsilon$ are treated as zero. In all tests, $\epsilon$ was set to $10^{-5}$.
    \item \textbf{Absolute maximum constraint violation}: the maximum permissible violation of any constraint under a candidate solution $\hat{x}$. In these tests, it was set equal to $\epsilon$, i.e., $10^{-5}$.
    \item \textbf{Absolute maximum integrality violation}: the maximum deviation allowed for variables constrained to take integer values. This tolerance was likewise set to $\epsilon = 10^{-5}$.
    \item \textbf{Relative objective error}: the maximum admissible relative error between the recomputed objective value $c^T \hat{x}$ and the value stored internally by PACS. This threshold was again set to $\epsilon = 10^{-5}$.  
\end{itemize}  
In the latter three cases, if the corresponding tolerance is exceeded, the algorithm terminates with an error and the run is recorded as unsolved.  

\section{Results}
This section presents the experimental evaluation and details the progressive tuning of PACS parameters. Each test builds upon the outcomes of the previous one, providing a stepwise refinement of the algorithm.

\subsection{Fixed $\rho$ Initialization Test}\label{sec:test_fix_rho}

\begin{figure}[thpb]
    \centering
    \begin{minipage}{0.6\columnwidth}
        \centering
        \resizebox{\linewidth}{!}{%% Creator: Matplotlib, PGF backend
%%
%% To include the figure in your LaTeX document, write
%%   \input{<filename>.pgf}
%%
%% Make sure the required packages are loaded in your preamble
%%   \usepackage{pgf}
%%
%% Also ensure that all the required font packages are loaded; for instance,
%% the lmodern package is sometimes necessary when using math font.
%%   \usepackage{lmodern}
%%
%% Figures using additional raster images can only be included by \input if
%% they are in the same directory as the main LaTeX file. For loading figures
%% from other directories you can use the `import` package
%%   \usepackage{import}
%%
%% and then include the figures with
%%   \import{<path to file>}{<filename>.pgf}
%%
%% Matplotlib used the following preamble
%%   \def\mathdefault#1{#1}
%%   \everymath=\expandafter{\the\everymath\displaystyle}
%%   \IfFileExists{scrextend.sty}{
%%     \usepackage[fontsize=10.000000pt]{scrextend}
%%   }{
%%     \renewcommand{\normalsize}{\fontsize{10.000000}{12.000000}\selectfont}
%%     \normalsize
%%   }
%%   
%%   \ifdefined\pdftexversion\else  % non-pdftex case.
%%     \usepackage{fontspec}
%%     \setmainfont{DejaVuSerif.ttf}[Path=\detokenize{/home/bisca/.global/lib/python3.12/site-packages/matplotlib/mpl-data/fonts/ttf/}]
%%     \setsansfont{DejaVuSans.ttf}[Path=\detokenize{/home/bisca/.global/lib/python3.12/site-packages/matplotlib/mpl-data/fonts/ttf/}]
%%     \setmonofont{DejaVuSansMono.ttf}[Path=\detokenize{/home/bisca/.global/lib/python3.12/site-packages/matplotlib/mpl-data/fonts/ttf/}]
%%   \fi
%%   \makeatletter\@ifpackageloaded{underscore}{}{\usepackage[strings]{underscore}}\makeatother
%%
\begingroup%
\makeatletter%
\begin{pgfpicture}%
\pgfpathrectangle{\pgfpointorigin}{\pgfqpoint{12.000000in}{8.000000in}}%
\pgfusepath{use as bounding box, clip}%
\begin{pgfscope}%
\pgfsetbuttcap%
\pgfsetmiterjoin%
\definecolor{currentfill}{rgb}{1.000000,1.000000,1.000000}%
\pgfsetfillcolor{currentfill}%
\pgfsetlinewidth{0.000000pt}%
\definecolor{currentstroke}{rgb}{1.000000,1.000000,1.000000}%
\pgfsetstrokecolor{currentstroke}%
\pgfsetdash{}{0pt}%
\pgfpathmoveto{\pgfqpoint{0.000000in}{0.000000in}}%
\pgfpathlineto{\pgfqpoint{12.000000in}{0.000000in}}%
\pgfpathlineto{\pgfqpoint{12.000000in}{8.000000in}}%
\pgfpathlineto{\pgfqpoint{0.000000in}{8.000000in}}%
\pgfpathlineto{\pgfqpoint{0.000000in}{0.000000in}}%
\pgfpathclose%
\pgfusepath{fill}%
\end{pgfscope}%
\begin{pgfscope}%
\pgfsetbuttcap%
\pgfsetmiterjoin%
\definecolor{currentfill}{rgb}{1.000000,1.000000,1.000000}%
\pgfsetfillcolor{currentfill}%
\pgfsetlinewidth{0.000000pt}%
\definecolor{currentstroke}{rgb}{0.000000,0.000000,0.000000}%
\pgfsetstrokecolor{currentstroke}%
\pgfsetstrokeopacity{0.000000}%
\pgfsetdash{}{0pt}%
\pgfpathmoveto{\pgfqpoint{0.661528in}{0.586684in}}%
\pgfpathlineto{\pgfqpoint{11.811750in}{0.586684in}}%
\pgfpathlineto{\pgfqpoint{11.811750in}{7.850000in}}%
\pgfpathlineto{\pgfqpoint{0.661528in}{7.850000in}}%
\pgfpathlineto{\pgfqpoint{0.661528in}{0.586684in}}%
\pgfpathclose%
\pgfusepath{fill}%
\end{pgfscope}%
\begin{pgfscope}%
\pgfpathrectangle{\pgfqpoint{0.661528in}{0.586684in}}{\pgfqpoint{11.150222in}{7.263316in}}%
\pgfusepath{clip}%
\pgfsetbuttcap%
\pgfsetroundjoin%
\pgfsetlinewidth{0.803000pt}%
\definecolor{currentstroke}{rgb}{0.690196,0.690196,0.690196}%
\pgfsetstrokecolor{currentstroke}%
\pgfsetstrokeopacity{0.500000}%
\pgfsetdash{{2.960000pt}{1.280000pt}}{0.000000pt}%
\pgfpathmoveto{\pgfqpoint{2.408396in}{0.586684in}}%
\pgfpathlineto{\pgfqpoint{2.408396in}{7.850000in}}%
\pgfusepath{stroke}%
\end{pgfscope}%
\begin{pgfscope}%
\pgfsetbuttcap%
\pgfsetroundjoin%
\definecolor{currentfill}{rgb}{0.000000,0.000000,0.000000}%
\pgfsetfillcolor{currentfill}%
\pgfsetlinewidth{0.803000pt}%
\definecolor{currentstroke}{rgb}{0.000000,0.000000,0.000000}%
\pgfsetstrokecolor{currentstroke}%
\pgfsetdash{}{0pt}%
\pgfsys@defobject{currentmarker}{\pgfqpoint{0.000000in}{-0.048611in}}{\pgfqpoint{0.000000in}{0.000000in}}{%
\pgfpathmoveto{\pgfqpoint{0.000000in}{0.000000in}}%
\pgfpathlineto{\pgfqpoint{0.000000in}{-0.048611in}}%
\pgfusepath{stroke,fill}%
}%
\begin{pgfscope}%
\pgfsys@transformshift{2.408396in}{0.586684in}%
\pgfsys@useobject{currentmarker}{}%
\end{pgfscope}%
\end{pgfscope}%
\begin{pgfscope}%
\definecolor{textcolor}{rgb}{0.000000,0.000000,0.000000}%
\pgfsetstrokecolor{textcolor}%
\pgfsetfillcolor{textcolor}%
\pgftext[x=2.408396in,y=0.489462in,,top]{\color{textcolor}{\rmfamily\fontsize{10.000000}{12.000000}\selectfont\catcode`\^=\active\def^{\ifmmode\sp\else\^{}\fi}\catcode`\%=\active\def%{\%}50}}%
\end{pgfscope}%
\begin{pgfscope}%
\pgfpathrectangle{\pgfqpoint{0.661528in}{0.586684in}}{\pgfqpoint{11.150222in}{7.263316in}}%
\pgfusepath{clip}%
\pgfsetbuttcap%
\pgfsetroundjoin%
\pgfsetlinewidth{0.803000pt}%
\definecolor{currentstroke}{rgb}{0.690196,0.690196,0.690196}%
\pgfsetstrokecolor{currentstroke}%
\pgfsetstrokeopacity{0.500000}%
\pgfsetdash{{2.960000pt}{1.280000pt}}{0.000000pt}%
\pgfpathmoveto{\pgfqpoint{4.266766in}{0.586684in}}%
\pgfpathlineto{\pgfqpoint{4.266766in}{7.850000in}}%
\pgfusepath{stroke}%
\end{pgfscope}%
\begin{pgfscope}%
\pgfsetbuttcap%
\pgfsetroundjoin%
\definecolor{currentfill}{rgb}{0.000000,0.000000,0.000000}%
\pgfsetfillcolor{currentfill}%
\pgfsetlinewidth{0.803000pt}%
\definecolor{currentstroke}{rgb}{0.000000,0.000000,0.000000}%
\pgfsetstrokecolor{currentstroke}%
\pgfsetdash{}{0pt}%
\pgfsys@defobject{currentmarker}{\pgfqpoint{0.000000in}{-0.048611in}}{\pgfqpoint{0.000000in}{0.000000in}}{%
\pgfpathmoveto{\pgfqpoint{0.000000in}{0.000000in}}%
\pgfpathlineto{\pgfqpoint{0.000000in}{-0.048611in}}%
\pgfusepath{stroke,fill}%
}%
\begin{pgfscope}%
\pgfsys@transformshift{4.266766in}{0.586684in}%
\pgfsys@useobject{currentmarker}{}%
\end{pgfscope}%
\end{pgfscope}%
\begin{pgfscope}%
\definecolor{textcolor}{rgb}{0.000000,0.000000,0.000000}%
\pgfsetstrokecolor{textcolor}%
\pgfsetfillcolor{textcolor}%
\pgftext[x=4.266766in,y=0.489462in,,top]{\color{textcolor}{\rmfamily\fontsize{10.000000}{12.000000}\selectfont\catcode`\^=\active\def^{\ifmmode\sp\else\^{}\fi}\catcode`\%=\active\def%{\%}100}}%
\end{pgfscope}%
\begin{pgfscope}%
\pgfpathrectangle{\pgfqpoint{0.661528in}{0.586684in}}{\pgfqpoint{11.150222in}{7.263316in}}%
\pgfusepath{clip}%
\pgfsetbuttcap%
\pgfsetroundjoin%
\pgfsetlinewidth{0.803000pt}%
\definecolor{currentstroke}{rgb}{0.690196,0.690196,0.690196}%
\pgfsetstrokecolor{currentstroke}%
\pgfsetstrokeopacity{0.500000}%
\pgfsetdash{{2.960000pt}{1.280000pt}}{0.000000pt}%
\pgfpathmoveto{\pgfqpoint{6.125137in}{0.586684in}}%
\pgfpathlineto{\pgfqpoint{6.125137in}{7.850000in}}%
\pgfusepath{stroke}%
\end{pgfscope}%
\begin{pgfscope}%
\pgfsetbuttcap%
\pgfsetroundjoin%
\definecolor{currentfill}{rgb}{0.000000,0.000000,0.000000}%
\pgfsetfillcolor{currentfill}%
\pgfsetlinewidth{0.803000pt}%
\definecolor{currentstroke}{rgb}{0.000000,0.000000,0.000000}%
\pgfsetstrokecolor{currentstroke}%
\pgfsetdash{}{0pt}%
\pgfsys@defobject{currentmarker}{\pgfqpoint{0.000000in}{-0.048611in}}{\pgfqpoint{0.000000in}{0.000000in}}{%
\pgfpathmoveto{\pgfqpoint{0.000000in}{0.000000in}}%
\pgfpathlineto{\pgfqpoint{0.000000in}{-0.048611in}}%
\pgfusepath{stroke,fill}%
}%
\begin{pgfscope}%
\pgfsys@transformshift{6.125137in}{0.586684in}%
\pgfsys@useobject{currentmarker}{}%
\end{pgfscope}%
\end{pgfscope}%
\begin{pgfscope}%
\definecolor{textcolor}{rgb}{0.000000,0.000000,0.000000}%
\pgfsetstrokecolor{textcolor}%
\pgfsetfillcolor{textcolor}%
\pgftext[x=6.125137in,y=0.489462in,,top]{\color{textcolor}{\rmfamily\fontsize{10.000000}{12.000000}\selectfont\catcode`\^=\active\def^{\ifmmode\sp\else\^{}\fi}\catcode`\%=\active\def%{\%}150}}%
\end{pgfscope}%
\begin{pgfscope}%
\pgfpathrectangle{\pgfqpoint{0.661528in}{0.586684in}}{\pgfqpoint{11.150222in}{7.263316in}}%
\pgfusepath{clip}%
\pgfsetbuttcap%
\pgfsetroundjoin%
\pgfsetlinewidth{0.803000pt}%
\definecolor{currentstroke}{rgb}{0.690196,0.690196,0.690196}%
\pgfsetstrokecolor{currentstroke}%
\pgfsetstrokeopacity{0.500000}%
\pgfsetdash{{2.960000pt}{1.280000pt}}{0.000000pt}%
\pgfpathmoveto{\pgfqpoint{7.983507in}{0.586684in}}%
\pgfpathlineto{\pgfqpoint{7.983507in}{7.850000in}}%
\pgfusepath{stroke}%
\end{pgfscope}%
\begin{pgfscope}%
\pgfsetbuttcap%
\pgfsetroundjoin%
\definecolor{currentfill}{rgb}{0.000000,0.000000,0.000000}%
\pgfsetfillcolor{currentfill}%
\pgfsetlinewidth{0.803000pt}%
\definecolor{currentstroke}{rgb}{0.000000,0.000000,0.000000}%
\pgfsetstrokecolor{currentstroke}%
\pgfsetdash{}{0pt}%
\pgfsys@defobject{currentmarker}{\pgfqpoint{0.000000in}{-0.048611in}}{\pgfqpoint{0.000000in}{0.000000in}}{%
\pgfpathmoveto{\pgfqpoint{0.000000in}{0.000000in}}%
\pgfpathlineto{\pgfqpoint{0.000000in}{-0.048611in}}%
\pgfusepath{stroke,fill}%
}%
\begin{pgfscope}%
\pgfsys@transformshift{7.983507in}{0.586684in}%
\pgfsys@useobject{currentmarker}{}%
\end{pgfscope}%
\end{pgfscope}%
\begin{pgfscope}%
\definecolor{textcolor}{rgb}{0.000000,0.000000,0.000000}%
\pgfsetstrokecolor{textcolor}%
\pgfsetfillcolor{textcolor}%
\pgftext[x=7.983507in,y=0.489462in,,top]{\color{textcolor}{\rmfamily\fontsize{10.000000}{12.000000}\selectfont\catcode`\^=\active\def^{\ifmmode\sp\else\^{}\fi}\catcode`\%=\active\def%{\%}200}}%
\end{pgfscope}%
\begin{pgfscope}%
\pgfpathrectangle{\pgfqpoint{0.661528in}{0.586684in}}{\pgfqpoint{11.150222in}{7.263316in}}%
\pgfusepath{clip}%
\pgfsetbuttcap%
\pgfsetroundjoin%
\pgfsetlinewidth{0.803000pt}%
\definecolor{currentstroke}{rgb}{0.690196,0.690196,0.690196}%
\pgfsetstrokecolor{currentstroke}%
\pgfsetstrokeopacity{0.500000}%
\pgfsetdash{{2.960000pt}{1.280000pt}}{0.000000pt}%
\pgfpathmoveto{\pgfqpoint{9.841877in}{0.586684in}}%
\pgfpathlineto{\pgfqpoint{9.841877in}{7.850000in}}%
\pgfusepath{stroke}%
\end{pgfscope}%
\begin{pgfscope}%
\pgfsetbuttcap%
\pgfsetroundjoin%
\definecolor{currentfill}{rgb}{0.000000,0.000000,0.000000}%
\pgfsetfillcolor{currentfill}%
\pgfsetlinewidth{0.803000pt}%
\definecolor{currentstroke}{rgb}{0.000000,0.000000,0.000000}%
\pgfsetstrokecolor{currentstroke}%
\pgfsetdash{}{0pt}%
\pgfsys@defobject{currentmarker}{\pgfqpoint{0.000000in}{-0.048611in}}{\pgfqpoint{0.000000in}{0.000000in}}{%
\pgfpathmoveto{\pgfqpoint{0.000000in}{0.000000in}}%
\pgfpathlineto{\pgfqpoint{0.000000in}{-0.048611in}}%
\pgfusepath{stroke,fill}%
}%
\begin{pgfscope}%
\pgfsys@transformshift{9.841877in}{0.586684in}%
\pgfsys@useobject{currentmarker}{}%
\end{pgfscope}%
\end{pgfscope}%
\begin{pgfscope}%
\definecolor{textcolor}{rgb}{0.000000,0.000000,0.000000}%
\pgfsetstrokecolor{textcolor}%
\pgfsetfillcolor{textcolor}%
\pgftext[x=9.841877in,y=0.489462in,,top]{\color{textcolor}{\rmfamily\fontsize{10.000000}{12.000000}\selectfont\catcode`\^=\active\def^{\ifmmode\sp\else\^{}\fi}\catcode`\%=\active\def%{\%}250}}%
\end{pgfscope}%
\begin{pgfscope}%
\pgfpathrectangle{\pgfqpoint{0.661528in}{0.586684in}}{\pgfqpoint{11.150222in}{7.263316in}}%
\pgfusepath{clip}%
\pgfsetbuttcap%
\pgfsetroundjoin%
\pgfsetlinewidth{0.803000pt}%
\definecolor{currentstroke}{rgb}{0.690196,0.690196,0.690196}%
\pgfsetstrokecolor{currentstroke}%
\pgfsetstrokeopacity{0.500000}%
\pgfsetdash{{2.960000pt}{1.280000pt}}{0.000000pt}%
\pgfpathmoveto{\pgfqpoint{11.700248in}{0.586684in}}%
\pgfpathlineto{\pgfqpoint{11.700248in}{7.850000in}}%
\pgfusepath{stroke}%
\end{pgfscope}%
\begin{pgfscope}%
\pgfsetbuttcap%
\pgfsetroundjoin%
\definecolor{currentfill}{rgb}{0.000000,0.000000,0.000000}%
\pgfsetfillcolor{currentfill}%
\pgfsetlinewidth{0.803000pt}%
\definecolor{currentstroke}{rgb}{0.000000,0.000000,0.000000}%
\pgfsetstrokecolor{currentstroke}%
\pgfsetdash{}{0pt}%
\pgfsys@defobject{currentmarker}{\pgfqpoint{0.000000in}{-0.048611in}}{\pgfqpoint{0.000000in}{0.000000in}}{%
\pgfpathmoveto{\pgfqpoint{0.000000in}{0.000000in}}%
\pgfpathlineto{\pgfqpoint{0.000000in}{-0.048611in}}%
\pgfusepath{stroke,fill}%
}%
\begin{pgfscope}%
\pgfsys@transformshift{11.700248in}{0.586684in}%
\pgfsys@useobject{currentmarker}{}%
\end{pgfscope}%
\end{pgfscope}%
\begin{pgfscope}%
\definecolor{textcolor}{rgb}{0.000000,0.000000,0.000000}%
\pgfsetstrokecolor{textcolor}%
\pgfsetfillcolor{textcolor}%
\pgftext[x=11.700248in,y=0.489462in,,top]{\color{textcolor}{\rmfamily\fontsize{10.000000}{12.000000}\selectfont\catcode`\^=\active\def^{\ifmmode\sp\else\^{}\fi}\catcode`\%=\active\def%{\%}300}}%
\end{pgfscope}%
\begin{pgfscope}%
\definecolor{textcolor}{rgb}{0.000000,0.000000,0.000000}%
\pgfsetstrokecolor{textcolor}%
\pgfsetfillcolor{textcolor}%
\pgftext[x=6.236639in,y=0.299493in,,top]{\color{textcolor}{\rmfamily\fontsize{10.000000}{12.000000}\selectfont\catcode`\^=\active\def^{\ifmmode\sp\else\^{}\fi}\catcode`\%=\active\def%{\%}Computation Time (sec)}}%
\end{pgfscope}%
\begin{pgfscope}%
\pgfpathrectangle{\pgfqpoint{0.661528in}{0.586684in}}{\pgfqpoint{11.150222in}{7.263316in}}%
\pgfusepath{clip}%
\pgfsetbuttcap%
\pgfsetroundjoin%
\pgfsetlinewidth{0.803000pt}%
\definecolor{currentstroke}{rgb}{0.690196,0.690196,0.690196}%
\pgfsetstrokecolor{currentstroke}%
\pgfsetstrokeopacity{0.500000}%
\pgfsetdash{{2.960000pt}{1.280000pt}}{0.000000pt}%
\pgfpathmoveto{\pgfqpoint{0.661528in}{1.177480in}}%
\pgfpathlineto{\pgfqpoint{11.811750in}{1.177480in}}%
\pgfusepath{stroke}%
\end{pgfscope}%
\begin{pgfscope}%
\pgfsetbuttcap%
\pgfsetroundjoin%
\definecolor{currentfill}{rgb}{0.000000,0.000000,0.000000}%
\pgfsetfillcolor{currentfill}%
\pgfsetlinewidth{0.803000pt}%
\definecolor{currentstroke}{rgb}{0.000000,0.000000,0.000000}%
\pgfsetstrokecolor{currentstroke}%
\pgfsetdash{}{0pt}%
\pgfsys@defobject{currentmarker}{\pgfqpoint{-0.048611in}{0.000000in}}{\pgfqpoint{-0.000000in}{0.000000in}}{%
\pgfpathmoveto{\pgfqpoint{-0.000000in}{0.000000in}}%
\pgfpathlineto{\pgfqpoint{-0.048611in}{0.000000in}}%
\pgfusepath{stroke,fill}%
}%
\begin{pgfscope}%
\pgfsys@transformshift{0.661528in}{1.177480in}%
\pgfsys@useobject{currentmarker}{}%
\end{pgfscope}%
\end{pgfscope}%
\begin{pgfscope}%
\definecolor{textcolor}{rgb}{0.000000,0.000000,0.000000}%
\pgfsetstrokecolor{textcolor}%
\pgfsetfillcolor{textcolor}%
\pgftext[x=0.343426in, y=1.124719in, left, base]{\color{textcolor}{\rmfamily\fontsize{10.000000}{12.000000}\selectfont\catcode`\^=\active\def^{\ifmmode\sp\else\^{}\fi}\catcode`\%=\active\def%{\%}0.1}}%
\end{pgfscope}%
\begin{pgfscope}%
\pgfpathrectangle{\pgfqpoint{0.661528in}{0.586684in}}{\pgfqpoint{11.150222in}{7.263316in}}%
\pgfusepath{clip}%
\pgfsetbuttcap%
\pgfsetroundjoin%
\pgfsetlinewidth{0.803000pt}%
\definecolor{currentstroke}{rgb}{0.690196,0.690196,0.690196}%
\pgfsetstrokecolor{currentstroke}%
\pgfsetstrokeopacity{0.500000}%
\pgfsetdash{{2.960000pt}{1.280000pt}}{0.000000pt}%
\pgfpathmoveto{\pgfqpoint{0.661528in}{2.133180in}}%
\pgfpathlineto{\pgfqpoint{11.811750in}{2.133180in}}%
\pgfusepath{stroke}%
\end{pgfscope}%
\begin{pgfscope}%
\pgfsetbuttcap%
\pgfsetroundjoin%
\definecolor{currentfill}{rgb}{0.000000,0.000000,0.000000}%
\pgfsetfillcolor{currentfill}%
\pgfsetlinewidth{0.803000pt}%
\definecolor{currentstroke}{rgb}{0.000000,0.000000,0.000000}%
\pgfsetstrokecolor{currentstroke}%
\pgfsetdash{}{0pt}%
\pgfsys@defobject{currentmarker}{\pgfqpoint{-0.048611in}{0.000000in}}{\pgfqpoint{-0.000000in}{0.000000in}}{%
\pgfpathmoveto{\pgfqpoint{-0.000000in}{0.000000in}}%
\pgfpathlineto{\pgfqpoint{-0.048611in}{0.000000in}}%
\pgfusepath{stroke,fill}%
}%
\begin{pgfscope}%
\pgfsys@transformshift{0.661528in}{2.133180in}%
\pgfsys@useobject{currentmarker}{}%
\end{pgfscope}%
\end{pgfscope}%
\begin{pgfscope}%
\definecolor{textcolor}{rgb}{0.000000,0.000000,0.000000}%
\pgfsetstrokecolor{textcolor}%
\pgfsetfillcolor{textcolor}%
\pgftext[x=0.343426in, y=2.080418in, left, base]{\color{textcolor}{\rmfamily\fontsize{10.000000}{12.000000}\selectfont\catcode`\^=\active\def^{\ifmmode\sp\else\^{}\fi}\catcode`\%=\active\def%{\%}0.2}}%
\end{pgfscope}%
\begin{pgfscope}%
\pgfpathrectangle{\pgfqpoint{0.661528in}{0.586684in}}{\pgfqpoint{11.150222in}{7.263316in}}%
\pgfusepath{clip}%
\pgfsetbuttcap%
\pgfsetroundjoin%
\pgfsetlinewidth{0.803000pt}%
\definecolor{currentstroke}{rgb}{0.690196,0.690196,0.690196}%
\pgfsetstrokecolor{currentstroke}%
\pgfsetstrokeopacity{0.500000}%
\pgfsetdash{{2.960000pt}{1.280000pt}}{0.000000pt}%
\pgfpathmoveto{\pgfqpoint{0.661528in}{3.088879in}}%
\pgfpathlineto{\pgfqpoint{11.811750in}{3.088879in}}%
\pgfusepath{stroke}%
\end{pgfscope}%
\begin{pgfscope}%
\pgfsetbuttcap%
\pgfsetroundjoin%
\definecolor{currentfill}{rgb}{0.000000,0.000000,0.000000}%
\pgfsetfillcolor{currentfill}%
\pgfsetlinewidth{0.803000pt}%
\definecolor{currentstroke}{rgb}{0.000000,0.000000,0.000000}%
\pgfsetstrokecolor{currentstroke}%
\pgfsetdash{}{0pt}%
\pgfsys@defobject{currentmarker}{\pgfqpoint{-0.048611in}{0.000000in}}{\pgfqpoint{-0.000000in}{0.000000in}}{%
\pgfpathmoveto{\pgfqpoint{-0.000000in}{0.000000in}}%
\pgfpathlineto{\pgfqpoint{-0.048611in}{0.000000in}}%
\pgfusepath{stroke,fill}%
}%
\begin{pgfscope}%
\pgfsys@transformshift{0.661528in}{3.088879in}%
\pgfsys@useobject{currentmarker}{}%
\end{pgfscope}%
\end{pgfscope}%
\begin{pgfscope}%
\definecolor{textcolor}{rgb}{0.000000,0.000000,0.000000}%
\pgfsetstrokecolor{textcolor}%
\pgfsetfillcolor{textcolor}%
\pgftext[x=0.343426in, y=3.036117in, left, base]{\color{textcolor}{\rmfamily\fontsize{10.000000}{12.000000}\selectfont\catcode`\^=\active\def^{\ifmmode\sp\else\^{}\fi}\catcode`\%=\active\def%{\%}0.3}}%
\end{pgfscope}%
\begin{pgfscope}%
\pgfpathrectangle{\pgfqpoint{0.661528in}{0.586684in}}{\pgfqpoint{11.150222in}{7.263316in}}%
\pgfusepath{clip}%
\pgfsetbuttcap%
\pgfsetroundjoin%
\pgfsetlinewidth{0.803000pt}%
\definecolor{currentstroke}{rgb}{0.690196,0.690196,0.690196}%
\pgfsetstrokecolor{currentstroke}%
\pgfsetstrokeopacity{0.500000}%
\pgfsetdash{{2.960000pt}{1.280000pt}}{0.000000pt}%
\pgfpathmoveto{\pgfqpoint{0.661528in}{4.044578in}}%
\pgfpathlineto{\pgfqpoint{11.811750in}{4.044578in}}%
\pgfusepath{stroke}%
\end{pgfscope}%
\begin{pgfscope}%
\pgfsetbuttcap%
\pgfsetroundjoin%
\definecolor{currentfill}{rgb}{0.000000,0.000000,0.000000}%
\pgfsetfillcolor{currentfill}%
\pgfsetlinewidth{0.803000pt}%
\definecolor{currentstroke}{rgb}{0.000000,0.000000,0.000000}%
\pgfsetstrokecolor{currentstroke}%
\pgfsetdash{}{0pt}%
\pgfsys@defobject{currentmarker}{\pgfqpoint{-0.048611in}{0.000000in}}{\pgfqpoint{-0.000000in}{0.000000in}}{%
\pgfpathmoveto{\pgfqpoint{-0.000000in}{0.000000in}}%
\pgfpathlineto{\pgfqpoint{-0.048611in}{0.000000in}}%
\pgfusepath{stroke,fill}%
}%
\begin{pgfscope}%
\pgfsys@transformshift{0.661528in}{4.044578in}%
\pgfsys@useobject{currentmarker}{}%
\end{pgfscope}%
\end{pgfscope}%
\begin{pgfscope}%
\definecolor{textcolor}{rgb}{0.000000,0.000000,0.000000}%
\pgfsetstrokecolor{textcolor}%
\pgfsetfillcolor{textcolor}%
\pgftext[x=0.343426in, y=3.991817in, left, base]{\color{textcolor}{\rmfamily\fontsize{10.000000}{12.000000}\selectfont\catcode`\^=\active\def^{\ifmmode\sp\else\^{}\fi}\catcode`\%=\active\def%{\%}0.4}}%
\end{pgfscope}%
\begin{pgfscope}%
\pgfpathrectangle{\pgfqpoint{0.661528in}{0.586684in}}{\pgfqpoint{11.150222in}{7.263316in}}%
\pgfusepath{clip}%
\pgfsetbuttcap%
\pgfsetroundjoin%
\pgfsetlinewidth{0.803000pt}%
\definecolor{currentstroke}{rgb}{0.690196,0.690196,0.690196}%
\pgfsetstrokecolor{currentstroke}%
\pgfsetstrokeopacity{0.500000}%
\pgfsetdash{{2.960000pt}{1.280000pt}}{0.000000pt}%
\pgfpathmoveto{\pgfqpoint{0.661528in}{5.000278in}}%
\pgfpathlineto{\pgfqpoint{11.811750in}{5.000278in}}%
\pgfusepath{stroke}%
\end{pgfscope}%
\begin{pgfscope}%
\pgfsetbuttcap%
\pgfsetroundjoin%
\definecolor{currentfill}{rgb}{0.000000,0.000000,0.000000}%
\pgfsetfillcolor{currentfill}%
\pgfsetlinewidth{0.803000pt}%
\definecolor{currentstroke}{rgb}{0.000000,0.000000,0.000000}%
\pgfsetstrokecolor{currentstroke}%
\pgfsetdash{}{0pt}%
\pgfsys@defobject{currentmarker}{\pgfqpoint{-0.048611in}{0.000000in}}{\pgfqpoint{-0.000000in}{0.000000in}}{%
\pgfpathmoveto{\pgfqpoint{-0.000000in}{0.000000in}}%
\pgfpathlineto{\pgfqpoint{-0.048611in}{0.000000in}}%
\pgfusepath{stroke,fill}%
}%
\begin{pgfscope}%
\pgfsys@transformshift{0.661528in}{5.000278in}%
\pgfsys@useobject{currentmarker}{}%
\end{pgfscope}%
\end{pgfscope}%
\begin{pgfscope}%
\definecolor{textcolor}{rgb}{0.000000,0.000000,0.000000}%
\pgfsetstrokecolor{textcolor}%
\pgfsetfillcolor{textcolor}%
\pgftext[x=0.343426in, y=4.947516in, left, base]{\color{textcolor}{\rmfamily\fontsize{10.000000}{12.000000}\selectfont\catcode`\^=\active\def^{\ifmmode\sp\else\^{}\fi}\catcode`\%=\active\def%{\%}0.5}}%
\end{pgfscope}%
\begin{pgfscope}%
\pgfpathrectangle{\pgfqpoint{0.661528in}{0.586684in}}{\pgfqpoint{11.150222in}{7.263316in}}%
\pgfusepath{clip}%
\pgfsetbuttcap%
\pgfsetroundjoin%
\pgfsetlinewidth{0.803000pt}%
\definecolor{currentstroke}{rgb}{0.690196,0.690196,0.690196}%
\pgfsetstrokecolor{currentstroke}%
\pgfsetstrokeopacity{0.500000}%
\pgfsetdash{{2.960000pt}{1.280000pt}}{0.000000pt}%
\pgfpathmoveto{\pgfqpoint{0.661528in}{5.955977in}}%
\pgfpathlineto{\pgfqpoint{11.811750in}{5.955977in}}%
\pgfusepath{stroke}%
\end{pgfscope}%
\begin{pgfscope}%
\pgfsetbuttcap%
\pgfsetroundjoin%
\definecolor{currentfill}{rgb}{0.000000,0.000000,0.000000}%
\pgfsetfillcolor{currentfill}%
\pgfsetlinewidth{0.803000pt}%
\definecolor{currentstroke}{rgb}{0.000000,0.000000,0.000000}%
\pgfsetstrokecolor{currentstroke}%
\pgfsetdash{}{0pt}%
\pgfsys@defobject{currentmarker}{\pgfqpoint{-0.048611in}{0.000000in}}{\pgfqpoint{-0.000000in}{0.000000in}}{%
\pgfpathmoveto{\pgfqpoint{-0.000000in}{0.000000in}}%
\pgfpathlineto{\pgfqpoint{-0.048611in}{0.000000in}}%
\pgfusepath{stroke,fill}%
}%
\begin{pgfscope}%
\pgfsys@transformshift{0.661528in}{5.955977in}%
\pgfsys@useobject{currentmarker}{}%
\end{pgfscope}%
\end{pgfscope}%
\begin{pgfscope}%
\definecolor{textcolor}{rgb}{0.000000,0.000000,0.000000}%
\pgfsetstrokecolor{textcolor}%
\pgfsetfillcolor{textcolor}%
\pgftext[x=0.343426in, y=5.903216in, left, base]{\color{textcolor}{\rmfamily\fontsize{10.000000}{12.000000}\selectfont\catcode`\^=\active\def^{\ifmmode\sp\else\^{}\fi}\catcode`\%=\active\def%{\%}0.6}}%
\end{pgfscope}%
\begin{pgfscope}%
\pgfpathrectangle{\pgfqpoint{0.661528in}{0.586684in}}{\pgfqpoint{11.150222in}{7.263316in}}%
\pgfusepath{clip}%
\pgfsetbuttcap%
\pgfsetroundjoin%
\pgfsetlinewidth{0.803000pt}%
\definecolor{currentstroke}{rgb}{0.690196,0.690196,0.690196}%
\pgfsetstrokecolor{currentstroke}%
\pgfsetstrokeopacity{0.500000}%
\pgfsetdash{{2.960000pt}{1.280000pt}}{0.000000pt}%
\pgfpathmoveto{\pgfqpoint{0.661528in}{6.911677in}}%
\pgfpathlineto{\pgfqpoint{11.811750in}{6.911677in}}%
\pgfusepath{stroke}%
\end{pgfscope}%
\begin{pgfscope}%
\pgfsetbuttcap%
\pgfsetroundjoin%
\definecolor{currentfill}{rgb}{0.000000,0.000000,0.000000}%
\pgfsetfillcolor{currentfill}%
\pgfsetlinewidth{0.803000pt}%
\definecolor{currentstroke}{rgb}{0.000000,0.000000,0.000000}%
\pgfsetstrokecolor{currentstroke}%
\pgfsetdash{}{0pt}%
\pgfsys@defobject{currentmarker}{\pgfqpoint{-0.048611in}{0.000000in}}{\pgfqpoint{-0.000000in}{0.000000in}}{%
\pgfpathmoveto{\pgfqpoint{-0.000000in}{0.000000in}}%
\pgfpathlineto{\pgfqpoint{-0.048611in}{0.000000in}}%
\pgfusepath{stroke,fill}%
}%
\begin{pgfscope}%
\pgfsys@transformshift{0.661528in}{6.911677in}%
\pgfsys@useobject{currentmarker}{}%
\end{pgfscope}%
\end{pgfscope}%
\begin{pgfscope}%
\definecolor{textcolor}{rgb}{0.000000,0.000000,0.000000}%
\pgfsetstrokecolor{textcolor}%
\pgfsetfillcolor{textcolor}%
\pgftext[x=0.343426in, y=6.858915in, left, base]{\color{textcolor}{\rmfamily\fontsize{10.000000}{12.000000}\selectfont\catcode`\^=\active\def^{\ifmmode\sp\else\^{}\fi}\catcode`\%=\active\def%{\%}0.7}}%
\end{pgfscope}%
\begin{pgfscope}%
\definecolor{textcolor}{rgb}{0.000000,0.000000,0.000000}%
\pgfsetstrokecolor{textcolor}%
\pgfsetfillcolor{textcolor}%
\pgftext[x=0.287871in,y=4.218342in,,bottom,rotate=90.000000]{\color{textcolor}{\rmfamily\fontsize{10.000000}{12.000000}\selectfont\catcode`\^=\active\def^{\ifmmode\sp\else\^{}\fi}\catcode`\%=\active\def%{\%}Success Rate}}%
\end{pgfscope}%
\begin{pgfscope}%
\pgfpathrectangle{\pgfqpoint{0.661528in}{0.586684in}}{\pgfqpoint{11.150222in}{7.263316in}}%
\pgfusepath{clip}%
\pgfsetrectcap%
\pgfsetroundjoin%
\pgfsetlinewidth{2.007500pt}%
\definecolor{currentstroke}{rgb}{0.121569,0.466667,0.705882}%
\pgfsetstrokecolor{currentstroke}%
\pgfsetdash{}{0pt}%
\pgfpathmoveto{\pgfqpoint{0.661528in}{5.869096in}}%
\pgfpathlineto{\pgfqpoint{0.773030in}{6.129741in}}%
\pgfpathlineto{\pgfqpoint{0.884532in}{6.651032in}}%
\pgfpathlineto{\pgfqpoint{0.996034in}{6.737913in}}%
\pgfpathlineto{\pgfqpoint{1.107537in}{6.737913in}}%
\pgfpathlineto{\pgfqpoint{1.219039in}{6.737913in}}%
\pgfpathlineto{\pgfqpoint{1.330541in}{6.737913in}}%
\pgfpathlineto{\pgfqpoint{1.442043in}{6.824795in}}%
\pgfpathlineto{\pgfqpoint{1.553546in}{6.824795in}}%
\pgfpathlineto{\pgfqpoint{1.665048in}{6.824795in}}%
\pgfpathlineto{\pgfqpoint{1.776550in}{6.911677in}}%
\pgfpathlineto{\pgfqpoint{1.888052in}{6.911677in}}%
\pgfpathlineto{\pgfqpoint{1.999554in}{6.911677in}}%
\pgfpathlineto{\pgfqpoint{2.111057in}{6.911677in}}%
\pgfpathlineto{\pgfqpoint{2.222559in}{6.998559in}}%
\pgfpathlineto{\pgfqpoint{2.334061in}{6.998559in}}%
\pgfpathlineto{\pgfqpoint{2.445563in}{6.998559in}}%
\pgfpathlineto{\pgfqpoint{2.557066in}{6.998559in}}%
\pgfpathlineto{\pgfqpoint{2.668568in}{6.998559in}}%
\pgfpathlineto{\pgfqpoint{2.780070in}{7.085440in}}%
\pgfpathlineto{\pgfqpoint{2.891572in}{7.085440in}}%
\pgfpathlineto{\pgfqpoint{3.003074in}{7.085440in}}%
\pgfpathlineto{\pgfqpoint{3.114577in}{7.085440in}}%
\pgfpathlineto{\pgfqpoint{3.226079in}{7.085440in}}%
\pgfpathlineto{\pgfqpoint{3.337581in}{7.085440in}}%
\pgfpathlineto{\pgfqpoint{3.449083in}{7.085440in}}%
\pgfpathlineto{\pgfqpoint{3.560586in}{7.085440in}}%
\pgfpathlineto{\pgfqpoint{3.672088in}{7.085440in}}%
\pgfpathlineto{\pgfqpoint{3.783590in}{7.085440in}}%
\pgfpathlineto{\pgfqpoint{3.895092in}{7.085440in}}%
\pgfpathlineto{\pgfqpoint{4.006594in}{7.085440in}}%
\pgfpathlineto{\pgfqpoint{4.118097in}{7.085440in}}%
\pgfpathlineto{\pgfqpoint{4.229599in}{7.085440in}}%
\pgfpathlineto{\pgfqpoint{4.341101in}{7.085440in}}%
\pgfpathlineto{\pgfqpoint{4.452603in}{7.085440in}}%
\pgfpathlineto{\pgfqpoint{4.564106in}{7.085440in}}%
\pgfpathlineto{\pgfqpoint{4.675608in}{7.085440in}}%
\pgfpathlineto{\pgfqpoint{4.787110in}{7.085440in}}%
\pgfpathlineto{\pgfqpoint{4.898612in}{7.085440in}}%
\pgfpathlineto{\pgfqpoint{5.010114in}{7.085440in}}%
\pgfpathlineto{\pgfqpoint{5.121617in}{7.085440in}}%
\pgfpathlineto{\pgfqpoint{5.233119in}{7.085440in}}%
\pgfpathlineto{\pgfqpoint{5.344621in}{7.085440in}}%
\pgfpathlineto{\pgfqpoint{5.456123in}{7.085440in}}%
\pgfpathlineto{\pgfqpoint{5.567626in}{7.085440in}}%
\pgfpathlineto{\pgfqpoint{5.679128in}{7.085440in}}%
\pgfpathlineto{\pgfqpoint{5.790630in}{7.085440in}}%
\pgfpathlineto{\pgfqpoint{5.902132in}{7.085440in}}%
\pgfpathlineto{\pgfqpoint{6.013634in}{7.172322in}}%
\pgfpathlineto{\pgfqpoint{6.125137in}{7.172322in}}%
\pgfpathlineto{\pgfqpoint{6.236639in}{7.172322in}}%
\pgfpathlineto{\pgfqpoint{6.348141in}{7.259204in}}%
\pgfpathlineto{\pgfqpoint{6.459643in}{7.259204in}}%
\pgfpathlineto{\pgfqpoint{6.571146in}{7.346086in}}%
\pgfpathlineto{\pgfqpoint{6.682648in}{7.346086in}}%
\pgfpathlineto{\pgfqpoint{6.794150in}{7.346086in}}%
\pgfpathlineto{\pgfqpoint{6.905652in}{7.346086in}}%
\pgfpathlineto{\pgfqpoint{7.017154in}{7.346086in}}%
\pgfpathlineto{\pgfqpoint{7.128657in}{7.346086in}}%
\pgfpathlineto{\pgfqpoint{7.240159in}{7.346086in}}%
\pgfpathlineto{\pgfqpoint{7.351661in}{7.346086in}}%
\pgfpathlineto{\pgfqpoint{7.463163in}{7.346086in}}%
\pgfpathlineto{\pgfqpoint{7.574666in}{7.346086in}}%
\pgfpathlineto{\pgfqpoint{7.686168in}{7.346086in}}%
\pgfpathlineto{\pgfqpoint{7.797670in}{7.346086in}}%
\pgfpathlineto{\pgfqpoint{7.909172in}{7.346086in}}%
\pgfpathlineto{\pgfqpoint{8.020674in}{7.346086in}}%
\pgfpathlineto{\pgfqpoint{8.132177in}{7.346086in}}%
\pgfpathlineto{\pgfqpoint{8.243679in}{7.346086in}}%
\pgfpathlineto{\pgfqpoint{8.355181in}{7.346086in}}%
\pgfpathlineto{\pgfqpoint{8.466683in}{7.346086in}}%
\pgfpathlineto{\pgfqpoint{8.578186in}{7.346086in}}%
\pgfpathlineto{\pgfqpoint{8.689688in}{7.346086in}}%
\pgfpathlineto{\pgfqpoint{8.801190in}{7.346086in}}%
\pgfpathlineto{\pgfqpoint{8.912692in}{7.346086in}}%
\pgfpathlineto{\pgfqpoint{9.024194in}{7.346086in}}%
\pgfpathlineto{\pgfqpoint{9.135697in}{7.346086in}}%
\pgfpathlineto{\pgfqpoint{9.247199in}{7.346086in}}%
\pgfpathlineto{\pgfqpoint{9.358701in}{7.346086in}}%
\pgfpathlineto{\pgfqpoint{9.470203in}{7.346086in}}%
\pgfpathlineto{\pgfqpoint{9.581706in}{7.346086in}}%
\pgfpathlineto{\pgfqpoint{9.693208in}{7.346086in}}%
\pgfpathlineto{\pgfqpoint{9.804710in}{7.346086in}}%
\pgfpathlineto{\pgfqpoint{9.916212in}{7.346086in}}%
\pgfpathlineto{\pgfqpoint{10.027714in}{7.346086in}}%
\pgfpathlineto{\pgfqpoint{10.139217in}{7.346086in}}%
\pgfpathlineto{\pgfqpoint{10.250719in}{7.346086in}}%
\pgfpathlineto{\pgfqpoint{10.362221in}{7.346086in}}%
\pgfpathlineto{\pgfqpoint{10.473723in}{7.346086in}}%
\pgfpathlineto{\pgfqpoint{10.585226in}{7.346086in}}%
\pgfpathlineto{\pgfqpoint{10.696728in}{7.346086in}}%
\pgfpathlineto{\pgfqpoint{10.808230in}{7.346086in}}%
\pgfpathlineto{\pgfqpoint{10.919732in}{7.519849in}}%
\pgfpathlineto{\pgfqpoint{11.031234in}{7.519849in}}%
\pgfpathlineto{\pgfqpoint{11.142737in}{7.519849in}}%
\pgfpathlineto{\pgfqpoint{11.254239in}{7.519849in}}%
\pgfpathlineto{\pgfqpoint{11.365741in}{7.519849in}}%
\pgfpathlineto{\pgfqpoint{11.477243in}{7.519849in}}%
\pgfpathlineto{\pgfqpoint{11.588746in}{7.519849in}}%
\pgfpathlineto{\pgfqpoint{11.700248in}{7.519849in}}%
\pgfpathlineto{\pgfqpoint{11.811750in}{7.519849in}}%
\pgfusepath{stroke}%
\end{pgfscope}%
\begin{pgfscope}%
\pgfpathrectangle{\pgfqpoint{0.661528in}{0.586684in}}{\pgfqpoint{11.150222in}{7.263316in}}%
\pgfusepath{clip}%
\pgfsetrectcap%
\pgfsetroundjoin%
\pgfsetlinewidth{2.007500pt}%
\definecolor{currentstroke}{rgb}{1.000000,0.498039,0.054902}%
\pgfsetstrokecolor{currentstroke}%
\pgfsetdash{}{0pt}%
\pgfpathmoveto{\pgfqpoint{0.661528in}{2.046298in}}%
\pgfpathlineto{\pgfqpoint{0.773030in}{3.175761in}}%
\pgfpathlineto{\pgfqpoint{0.884532in}{3.349524in}}%
\pgfpathlineto{\pgfqpoint{0.996034in}{3.697051in}}%
\pgfpathlineto{\pgfqpoint{1.107537in}{3.957697in}}%
\pgfpathlineto{\pgfqpoint{1.219039in}{4.218342in}}%
\pgfpathlineto{\pgfqpoint{1.330541in}{4.305224in}}%
\pgfpathlineto{\pgfqpoint{1.442043in}{4.392106in}}%
\pgfpathlineto{\pgfqpoint{1.553546in}{4.392106in}}%
\pgfpathlineto{\pgfqpoint{1.665048in}{4.478987in}}%
\pgfpathlineto{\pgfqpoint{1.776550in}{4.652751in}}%
\pgfpathlineto{\pgfqpoint{1.888052in}{4.739633in}}%
\pgfpathlineto{\pgfqpoint{1.999554in}{4.826514in}}%
\pgfpathlineto{\pgfqpoint{2.111057in}{5.087160in}}%
\pgfpathlineto{\pgfqpoint{2.222559in}{5.174041in}}%
\pgfpathlineto{\pgfqpoint{2.334061in}{5.174041in}}%
\pgfpathlineto{\pgfqpoint{2.445563in}{5.260923in}}%
\pgfpathlineto{\pgfqpoint{2.557066in}{5.260923in}}%
\pgfpathlineto{\pgfqpoint{2.668568in}{5.260923in}}%
\pgfpathlineto{\pgfqpoint{2.780070in}{5.260923in}}%
\pgfpathlineto{\pgfqpoint{2.891572in}{5.260923in}}%
\pgfpathlineto{\pgfqpoint{3.003074in}{5.260923in}}%
\pgfpathlineto{\pgfqpoint{3.114577in}{5.260923in}}%
\pgfpathlineto{\pgfqpoint{3.226079in}{5.347805in}}%
\pgfpathlineto{\pgfqpoint{3.337581in}{5.347805in}}%
\pgfpathlineto{\pgfqpoint{3.449083in}{5.347805in}}%
\pgfpathlineto{\pgfqpoint{3.560586in}{5.347805in}}%
\pgfpathlineto{\pgfqpoint{3.672088in}{5.347805in}}%
\pgfpathlineto{\pgfqpoint{3.783590in}{5.347805in}}%
\pgfpathlineto{\pgfqpoint{3.895092in}{5.347805in}}%
\pgfpathlineto{\pgfqpoint{4.006594in}{5.434687in}}%
\pgfpathlineto{\pgfqpoint{4.118097in}{5.521569in}}%
\pgfpathlineto{\pgfqpoint{4.229599in}{5.521569in}}%
\pgfpathlineto{\pgfqpoint{4.341101in}{5.521569in}}%
\pgfpathlineto{\pgfqpoint{4.452603in}{5.521569in}}%
\pgfpathlineto{\pgfqpoint{4.564106in}{5.608450in}}%
\pgfpathlineto{\pgfqpoint{4.675608in}{5.695332in}}%
\pgfpathlineto{\pgfqpoint{4.787110in}{5.695332in}}%
\pgfpathlineto{\pgfqpoint{4.898612in}{5.695332in}}%
\pgfpathlineto{\pgfqpoint{5.010114in}{5.695332in}}%
\pgfpathlineto{\pgfqpoint{5.121617in}{5.695332in}}%
\pgfpathlineto{\pgfqpoint{5.233119in}{5.782214in}}%
\pgfpathlineto{\pgfqpoint{5.344621in}{5.782214in}}%
\pgfpathlineto{\pgfqpoint{5.456123in}{5.782214in}}%
\pgfpathlineto{\pgfqpoint{5.567626in}{5.782214in}}%
\pgfpathlineto{\pgfqpoint{5.679128in}{5.782214in}}%
\pgfpathlineto{\pgfqpoint{5.790630in}{5.782214in}}%
\pgfpathlineto{\pgfqpoint{5.902132in}{5.782214in}}%
\pgfpathlineto{\pgfqpoint{6.013634in}{5.782214in}}%
\pgfpathlineto{\pgfqpoint{6.125137in}{5.782214in}}%
\pgfpathlineto{\pgfqpoint{6.236639in}{5.782214in}}%
\pgfpathlineto{\pgfqpoint{6.348141in}{5.782214in}}%
\pgfpathlineto{\pgfqpoint{6.459643in}{5.869096in}}%
\pgfpathlineto{\pgfqpoint{6.571146in}{5.869096in}}%
\pgfpathlineto{\pgfqpoint{6.682648in}{5.869096in}}%
\pgfpathlineto{\pgfqpoint{6.794150in}{5.869096in}}%
\pgfpathlineto{\pgfqpoint{6.905652in}{5.869096in}}%
\pgfpathlineto{\pgfqpoint{7.017154in}{5.955977in}}%
\pgfpathlineto{\pgfqpoint{7.128657in}{5.955977in}}%
\pgfpathlineto{\pgfqpoint{7.240159in}{5.955977in}}%
\pgfpathlineto{\pgfqpoint{7.351661in}{5.955977in}}%
\pgfpathlineto{\pgfqpoint{7.463163in}{5.955977in}}%
\pgfpathlineto{\pgfqpoint{7.574666in}{5.955977in}}%
\pgfpathlineto{\pgfqpoint{7.686168in}{5.955977in}}%
\pgfpathlineto{\pgfqpoint{7.797670in}{5.955977in}}%
\pgfpathlineto{\pgfqpoint{7.909172in}{5.955977in}}%
\pgfpathlineto{\pgfqpoint{8.020674in}{5.955977in}}%
\pgfpathlineto{\pgfqpoint{8.132177in}{5.955977in}}%
\pgfpathlineto{\pgfqpoint{8.243679in}{5.955977in}}%
\pgfpathlineto{\pgfqpoint{8.355181in}{5.955977in}}%
\pgfpathlineto{\pgfqpoint{8.466683in}{5.955977in}}%
\pgfpathlineto{\pgfqpoint{8.578186in}{5.955977in}}%
\pgfpathlineto{\pgfqpoint{8.689688in}{5.955977in}}%
\pgfpathlineto{\pgfqpoint{8.801190in}{5.955977in}}%
\pgfpathlineto{\pgfqpoint{8.912692in}{5.955977in}}%
\pgfpathlineto{\pgfqpoint{9.024194in}{5.955977in}}%
\pgfpathlineto{\pgfqpoint{9.135697in}{5.955977in}}%
\pgfpathlineto{\pgfqpoint{9.247199in}{5.955977in}}%
\pgfpathlineto{\pgfqpoint{9.358701in}{5.955977in}}%
\pgfpathlineto{\pgfqpoint{9.470203in}{5.955977in}}%
\pgfpathlineto{\pgfqpoint{9.581706in}{5.955977in}}%
\pgfpathlineto{\pgfqpoint{9.693208in}{5.955977in}}%
\pgfpathlineto{\pgfqpoint{9.804710in}{5.955977in}}%
\pgfpathlineto{\pgfqpoint{9.916212in}{5.955977in}}%
\pgfpathlineto{\pgfqpoint{10.027714in}{5.955977in}}%
\pgfpathlineto{\pgfqpoint{10.139217in}{5.955977in}}%
\pgfpathlineto{\pgfqpoint{10.250719in}{5.955977in}}%
\pgfpathlineto{\pgfqpoint{10.362221in}{5.955977in}}%
\pgfpathlineto{\pgfqpoint{10.473723in}{5.955977in}}%
\pgfpathlineto{\pgfqpoint{10.585226in}{5.955977in}}%
\pgfpathlineto{\pgfqpoint{10.696728in}{5.955977in}}%
\pgfpathlineto{\pgfqpoint{10.808230in}{5.955977in}}%
\pgfpathlineto{\pgfqpoint{10.919732in}{5.955977in}}%
\pgfpathlineto{\pgfqpoint{11.031234in}{5.955977in}}%
\pgfpathlineto{\pgfqpoint{11.142737in}{5.955977in}}%
\pgfpathlineto{\pgfqpoint{11.254239in}{5.955977in}}%
\pgfpathlineto{\pgfqpoint{11.365741in}{6.042859in}}%
\pgfpathlineto{\pgfqpoint{11.477243in}{6.042859in}}%
\pgfpathlineto{\pgfqpoint{11.588746in}{6.042859in}}%
\pgfpathlineto{\pgfqpoint{11.700248in}{6.129741in}}%
\pgfpathlineto{\pgfqpoint{11.811750in}{6.129741in}}%
\pgfusepath{stroke}%
\end{pgfscope}%
\begin{pgfscope}%
\pgfpathrectangle{\pgfqpoint{0.661528in}{0.586684in}}{\pgfqpoint{11.150222in}{7.263316in}}%
\pgfusepath{clip}%
\pgfsetrectcap%
\pgfsetroundjoin%
\pgfsetlinewidth{2.007500pt}%
\definecolor{currentstroke}{rgb}{0.172549,0.627451,0.172549}%
\pgfsetstrokecolor{currentstroke}%
\pgfsetdash{}{0pt}%
\pgfpathmoveto{\pgfqpoint{0.661528in}{2.046298in}}%
\pgfpathlineto{\pgfqpoint{0.773030in}{3.262643in}}%
\pgfpathlineto{\pgfqpoint{0.884532in}{3.436406in}}%
\pgfpathlineto{\pgfqpoint{0.996034in}{3.523288in}}%
\pgfpathlineto{\pgfqpoint{1.107537in}{3.870815in}}%
\pgfpathlineto{\pgfqpoint{1.219039in}{4.218342in}}%
\pgfpathlineto{\pgfqpoint{1.330541in}{4.305224in}}%
\pgfpathlineto{\pgfqpoint{1.442043in}{4.305224in}}%
\pgfpathlineto{\pgfqpoint{1.553546in}{4.305224in}}%
\pgfpathlineto{\pgfqpoint{1.665048in}{4.305224in}}%
\pgfpathlineto{\pgfqpoint{1.776550in}{4.392106in}}%
\pgfpathlineto{\pgfqpoint{1.888052in}{4.478987in}}%
\pgfpathlineto{\pgfqpoint{1.999554in}{4.565869in}}%
\pgfpathlineto{\pgfqpoint{2.111057in}{4.652751in}}%
\pgfpathlineto{\pgfqpoint{2.222559in}{4.739633in}}%
\pgfpathlineto{\pgfqpoint{2.334061in}{5.000278in}}%
\pgfpathlineto{\pgfqpoint{2.445563in}{5.087160in}}%
\pgfpathlineto{\pgfqpoint{2.557066in}{5.174041in}}%
\pgfpathlineto{\pgfqpoint{2.668568in}{5.260923in}}%
\pgfpathlineto{\pgfqpoint{2.780070in}{5.260923in}}%
\pgfpathlineto{\pgfqpoint{2.891572in}{5.260923in}}%
\pgfpathlineto{\pgfqpoint{3.003074in}{5.260923in}}%
\pgfpathlineto{\pgfqpoint{3.114577in}{5.260923in}}%
\pgfpathlineto{\pgfqpoint{3.226079in}{5.347805in}}%
\pgfpathlineto{\pgfqpoint{3.337581in}{5.347805in}}%
\pgfpathlineto{\pgfqpoint{3.449083in}{5.347805in}}%
\pgfpathlineto{\pgfqpoint{3.560586in}{5.347805in}}%
\pgfpathlineto{\pgfqpoint{3.672088in}{5.347805in}}%
\pgfpathlineto{\pgfqpoint{3.783590in}{5.347805in}}%
\pgfpathlineto{\pgfqpoint{3.895092in}{5.347805in}}%
\pgfpathlineto{\pgfqpoint{4.006594in}{5.347805in}}%
\pgfpathlineto{\pgfqpoint{4.118097in}{5.434687in}}%
\pgfpathlineto{\pgfqpoint{4.229599in}{5.434687in}}%
\pgfpathlineto{\pgfqpoint{4.341101in}{5.434687in}}%
\pgfpathlineto{\pgfqpoint{4.452603in}{5.434687in}}%
\pgfpathlineto{\pgfqpoint{4.564106in}{5.434687in}}%
\pgfpathlineto{\pgfqpoint{4.675608in}{5.434687in}}%
\pgfpathlineto{\pgfqpoint{4.787110in}{5.434687in}}%
\pgfpathlineto{\pgfqpoint{4.898612in}{5.434687in}}%
\pgfpathlineto{\pgfqpoint{5.010114in}{5.434687in}}%
\pgfpathlineto{\pgfqpoint{5.121617in}{5.434687in}}%
\pgfpathlineto{\pgfqpoint{5.233119in}{5.434687in}}%
\pgfpathlineto{\pgfqpoint{5.344621in}{5.434687in}}%
\pgfpathlineto{\pgfqpoint{5.456123in}{5.434687in}}%
\pgfpathlineto{\pgfqpoint{5.567626in}{5.521569in}}%
\pgfpathlineto{\pgfqpoint{5.679128in}{5.521569in}}%
\pgfpathlineto{\pgfqpoint{5.790630in}{5.521569in}}%
\pgfpathlineto{\pgfqpoint{5.902132in}{5.608450in}}%
\pgfpathlineto{\pgfqpoint{6.013634in}{5.608450in}}%
\pgfpathlineto{\pgfqpoint{6.125137in}{5.608450in}}%
\pgfpathlineto{\pgfqpoint{6.236639in}{5.695332in}}%
\pgfpathlineto{\pgfqpoint{6.348141in}{5.695332in}}%
\pgfpathlineto{\pgfqpoint{6.459643in}{5.695332in}}%
\pgfpathlineto{\pgfqpoint{6.571146in}{5.782214in}}%
\pgfpathlineto{\pgfqpoint{6.682648in}{5.782214in}}%
\pgfpathlineto{\pgfqpoint{6.794150in}{5.782214in}}%
\pgfpathlineto{\pgfqpoint{6.905652in}{5.782214in}}%
\pgfpathlineto{\pgfqpoint{7.017154in}{5.782214in}}%
\pgfpathlineto{\pgfqpoint{7.128657in}{5.782214in}}%
\pgfpathlineto{\pgfqpoint{7.240159in}{5.782214in}}%
\pgfpathlineto{\pgfqpoint{7.351661in}{5.782214in}}%
\pgfpathlineto{\pgfqpoint{7.463163in}{5.869096in}}%
\pgfpathlineto{\pgfqpoint{7.574666in}{5.869096in}}%
\pgfpathlineto{\pgfqpoint{7.686168in}{5.869096in}}%
\pgfpathlineto{\pgfqpoint{7.797670in}{5.869096in}}%
\pgfpathlineto{\pgfqpoint{7.909172in}{5.955977in}}%
\pgfpathlineto{\pgfqpoint{8.020674in}{5.955977in}}%
\pgfpathlineto{\pgfqpoint{8.132177in}{5.955977in}}%
\pgfpathlineto{\pgfqpoint{8.243679in}{5.955977in}}%
\pgfpathlineto{\pgfqpoint{8.355181in}{5.955977in}}%
\pgfpathlineto{\pgfqpoint{8.466683in}{5.955977in}}%
\pgfpathlineto{\pgfqpoint{8.578186in}{5.955977in}}%
\pgfpathlineto{\pgfqpoint{8.689688in}{5.955977in}}%
\pgfpathlineto{\pgfqpoint{8.801190in}{5.955977in}}%
\pgfpathlineto{\pgfqpoint{8.912692in}{5.955977in}}%
\pgfpathlineto{\pgfqpoint{9.024194in}{5.955977in}}%
\pgfpathlineto{\pgfqpoint{9.135697in}{5.955977in}}%
\pgfpathlineto{\pgfqpoint{9.247199in}{5.955977in}}%
\pgfpathlineto{\pgfqpoint{9.358701in}{5.955977in}}%
\pgfpathlineto{\pgfqpoint{9.470203in}{5.955977in}}%
\pgfpathlineto{\pgfqpoint{9.581706in}{5.955977in}}%
\pgfpathlineto{\pgfqpoint{9.693208in}{5.955977in}}%
\pgfpathlineto{\pgfqpoint{9.804710in}{5.955977in}}%
\pgfpathlineto{\pgfqpoint{9.916212in}{5.955977in}}%
\pgfpathlineto{\pgfqpoint{10.027714in}{5.955977in}}%
\pgfpathlineto{\pgfqpoint{10.139217in}{5.955977in}}%
\pgfpathlineto{\pgfqpoint{10.250719in}{5.955977in}}%
\pgfpathlineto{\pgfqpoint{10.362221in}{6.042859in}}%
\pgfpathlineto{\pgfqpoint{10.473723in}{6.042859in}}%
\pgfpathlineto{\pgfqpoint{10.585226in}{6.042859in}}%
\pgfpathlineto{\pgfqpoint{10.696728in}{6.042859in}}%
\pgfpathlineto{\pgfqpoint{10.808230in}{6.042859in}}%
\pgfpathlineto{\pgfqpoint{10.919732in}{6.042859in}}%
\pgfpathlineto{\pgfqpoint{11.031234in}{6.042859in}}%
\pgfpathlineto{\pgfqpoint{11.142737in}{6.042859in}}%
\pgfpathlineto{\pgfqpoint{11.254239in}{6.042859in}}%
\pgfpathlineto{\pgfqpoint{11.365741in}{6.042859in}}%
\pgfpathlineto{\pgfqpoint{11.477243in}{6.042859in}}%
\pgfpathlineto{\pgfqpoint{11.588746in}{6.042859in}}%
\pgfpathlineto{\pgfqpoint{11.700248in}{6.042859in}}%
\pgfpathlineto{\pgfqpoint{11.811750in}{6.042859in}}%
\pgfusepath{stroke}%
\end{pgfscope}%
\begin{pgfscope}%
\pgfpathrectangle{\pgfqpoint{0.661528in}{0.586684in}}{\pgfqpoint{11.150222in}{7.263316in}}%
\pgfusepath{clip}%
\pgfsetrectcap%
\pgfsetroundjoin%
\pgfsetlinewidth{2.007500pt}%
\definecolor{currentstroke}{rgb}{0.839216,0.152941,0.156863}%
\pgfsetstrokecolor{currentstroke}%
\pgfsetdash{}{0pt}%
\pgfpathmoveto{\pgfqpoint{0.661528in}{2.306943in}}%
\pgfpathlineto{\pgfqpoint{0.773030in}{2.567588in}}%
\pgfpathlineto{\pgfqpoint{0.884532in}{2.828234in}}%
\pgfpathlineto{\pgfqpoint{0.996034in}{3.001997in}}%
\pgfpathlineto{\pgfqpoint{1.107537in}{3.088879in}}%
\pgfpathlineto{\pgfqpoint{1.219039in}{3.262643in}}%
\pgfpathlineto{\pgfqpoint{1.330541in}{3.262643in}}%
\pgfpathlineto{\pgfqpoint{1.442043in}{3.349524in}}%
\pgfpathlineto{\pgfqpoint{1.553546in}{3.436406in}}%
\pgfpathlineto{\pgfqpoint{1.665048in}{3.436406in}}%
\pgfpathlineto{\pgfqpoint{1.776550in}{3.436406in}}%
\pgfpathlineto{\pgfqpoint{1.888052in}{3.610170in}}%
\pgfpathlineto{\pgfqpoint{1.999554in}{3.610170in}}%
\pgfpathlineto{\pgfqpoint{2.111057in}{3.610170in}}%
\pgfpathlineto{\pgfqpoint{2.222559in}{3.783933in}}%
\pgfpathlineto{\pgfqpoint{2.334061in}{3.783933in}}%
\pgfpathlineto{\pgfqpoint{2.445563in}{3.870815in}}%
\pgfpathlineto{\pgfqpoint{2.557066in}{3.957697in}}%
\pgfpathlineto{\pgfqpoint{2.668568in}{3.957697in}}%
\pgfpathlineto{\pgfqpoint{2.780070in}{3.957697in}}%
\pgfpathlineto{\pgfqpoint{2.891572in}{4.044578in}}%
\pgfpathlineto{\pgfqpoint{3.003074in}{4.044578in}}%
\pgfpathlineto{\pgfqpoint{3.114577in}{4.044578in}}%
\pgfpathlineto{\pgfqpoint{3.226079in}{4.131460in}}%
\pgfpathlineto{\pgfqpoint{3.337581in}{4.131460in}}%
\pgfpathlineto{\pgfqpoint{3.449083in}{4.131460in}}%
\pgfpathlineto{\pgfqpoint{3.560586in}{4.131460in}}%
\pgfpathlineto{\pgfqpoint{3.672088in}{4.131460in}}%
\pgfpathlineto{\pgfqpoint{3.783590in}{4.218342in}}%
\pgfpathlineto{\pgfqpoint{3.895092in}{4.218342in}}%
\pgfpathlineto{\pgfqpoint{4.006594in}{4.218342in}}%
\pgfpathlineto{\pgfqpoint{4.118097in}{4.305224in}}%
\pgfpathlineto{\pgfqpoint{4.229599in}{4.305224in}}%
\pgfpathlineto{\pgfqpoint{4.341101in}{4.305224in}}%
\pgfpathlineto{\pgfqpoint{4.452603in}{4.305224in}}%
\pgfpathlineto{\pgfqpoint{4.564106in}{4.305224in}}%
\pgfpathlineto{\pgfqpoint{4.675608in}{4.305224in}}%
\pgfpathlineto{\pgfqpoint{4.787110in}{4.305224in}}%
\pgfpathlineto{\pgfqpoint{4.898612in}{4.305224in}}%
\pgfpathlineto{\pgfqpoint{5.010114in}{4.305224in}}%
\pgfpathlineto{\pgfqpoint{5.121617in}{4.305224in}}%
\pgfpathlineto{\pgfqpoint{5.233119in}{4.305224in}}%
\pgfpathlineto{\pgfqpoint{5.344621in}{4.305224in}}%
\pgfpathlineto{\pgfqpoint{5.456123in}{4.392106in}}%
\pgfpathlineto{\pgfqpoint{5.567626in}{4.392106in}}%
\pgfpathlineto{\pgfqpoint{5.679128in}{4.392106in}}%
\pgfpathlineto{\pgfqpoint{5.790630in}{4.392106in}}%
\pgfpathlineto{\pgfqpoint{5.902132in}{4.392106in}}%
\pgfpathlineto{\pgfqpoint{6.013634in}{4.392106in}}%
\pgfpathlineto{\pgfqpoint{6.125137in}{4.392106in}}%
\pgfpathlineto{\pgfqpoint{6.236639in}{4.392106in}}%
\pgfpathlineto{\pgfqpoint{6.348141in}{4.392106in}}%
\pgfpathlineto{\pgfqpoint{6.459643in}{4.478987in}}%
\pgfpathlineto{\pgfqpoint{6.571146in}{4.478987in}}%
\pgfpathlineto{\pgfqpoint{6.682648in}{4.565869in}}%
\pgfpathlineto{\pgfqpoint{6.794150in}{4.565869in}}%
\pgfpathlineto{\pgfqpoint{6.905652in}{4.565869in}}%
\pgfpathlineto{\pgfqpoint{7.017154in}{4.565869in}}%
\pgfpathlineto{\pgfqpoint{7.128657in}{4.565869in}}%
\pgfpathlineto{\pgfqpoint{7.240159in}{4.565869in}}%
\pgfpathlineto{\pgfqpoint{7.351661in}{4.565869in}}%
\pgfpathlineto{\pgfqpoint{7.463163in}{4.565869in}}%
\pgfpathlineto{\pgfqpoint{7.574666in}{4.565869in}}%
\pgfpathlineto{\pgfqpoint{7.686168in}{4.565869in}}%
\pgfpathlineto{\pgfqpoint{7.797670in}{4.565869in}}%
\pgfpathlineto{\pgfqpoint{7.909172in}{4.565869in}}%
\pgfpathlineto{\pgfqpoint{8.020674in}{4.565869in}}%
\pgfpathlineto{\pgfqpoint{8.132177in}{4.565869in}}%
\pgfpathlineto{\pgfqpoint{8.243679in}{4.565869in}}%
\pgfpathlineto{\pgfqpoint{8.355181in}{4.565869in}}%
\pgfpathlineto{\pgfqpoint{8.466683in}{4.565869in}}%
\pgfpathlineto{\pgfqpoint{8.578186in}{4.565869in}}%
\pgfpathlineto{\pgfqpoint{8.689688in}{4.565869in}}%
\pgfpathlineto{\pgfqpoint{8.801190in}{4.565869in}}%
\pgfpathlineto{\pgfqpoint{8.912692in}{4.565869in}}%
\pgfpathlineto{\pgfqpoint{9.024194in}{4.565869in}}%
\pgfpathlineto{\pgfqpoint{9.135697in}{4.565869in}}%
\pgfpathlineto{\pgfqpoint{9.247199in}{4.565869in}}%
\pgfpathlineto{\pgfqpoint{9.358701in}{4.565869in}}%
\pgfpathlineto{\pgfqpoint{9.470203in}{4.565869in}}%
\pgfpathlineto{\pgfqpoint{9.581706in}{4.565869in}}%
\pgfpathlineto{\pgfqpoint{9.693208in}{4.652751in}}%
\pgfpathlineto{\pgfqpoint{9.804710in}{4.652751in}}%
\pgfpathlineto{\pgfqpoint{9.916212in}{4.652751in}}%
\pgfpathlineto{\pgfqpoint{10.027714in}{4.652751in}}%
\pgfpathlineto{\pgfqpoint{10.139217in}{4.652751in}}%
\pgfpathlineto{\pgfqpoint{10.250719in}{4.652751in}}%
\pgfpathlineto{\pgfqpoint{10.362221in}{4.652751in}}%
\pgfpathlineto{\pgfqpoint{10.473723in}{4.652751in}}%
\pgfpathlineto{\pgfqpoint{10.585226in}{4.652751in}}%
\pgfpathlineto{\pgfqpoint{10.696728in}{4.652751in}}%
\pgfpathlineto{\pgfqpoint{10.808230in}{4.652751in}}%
\pgfpathlineto{\pgfqpoint{10.919732in}{4.652751in}}%
\pgfpathlineto{\pgfqpoint{11.031234in}{4.652751in}}%
\pgfpathlineto{\pgfqpoint{11.142737in}{4.652751in}}%
\pgfpathlineto{\pgfqpoint{11.254239in}{4.652751in}}%
\pgfpathlineto{\pgfqpoint{11.365741in}{4.652751in}}%
\pgfpathlineto{\pgfqpoint{11.477243in}{4.652751in}}%
\pgfpathlineto{\pgfqpoint{11.588746in}{4.652751in}}%
\pgfpathlineto{\pgfqpoint{11.700248in}{4.652751in}}%
\pgfpathlineto{\pgfqpoint{11.811750in}{4.652751in}}%
\pgfusepath{stroke}%
\end{pgfscope}%
\begin{pgfscope}%
\pgfpathrectangle{\pgfqpoint{0.661528in}{0.586684in}}{\pgfqpoint{11.150222in}{7.263316in}}%
\pgfusepath{clip}%
\pgfsetrectcap%
\pgfsetroundjoin%
\pgfsetlinewidth{2.007500pt}%
\definecolor{currentstroke}{rgb}{0.580392,0.403922,0.741176}%
\pgfsetstrokecolor{currentstroke}%
\pgfsetdash{}{0pt}%
\pgfpathmoveto{\pgfqpoint{0.661528in}{1.525007in}}%
\pgfpathlineto{\pgfqpoint{0.773030in}{1.959416in}}%
\pgfpathlineto{\pgfqpoint{0.884532in}{1.959416in}}%
\pgfpathlineto{\pgfqpoint{0.996034in}{2.046298in}}%
\pgfpathlineto{\pgfqpoint{1.107537in}{2.133180in}}%
\pgfpathlineto{\pgfqpoint{1.219039in}{2.133180in}}%
\pgfpathlineto{\pgfqpoint{1.330541in}{2.133180in}}%
\pgfpathlineto{\pgfqpoint{1.442043in}{2.133180in}}%
\pgfpathlineto{\pgfqpoint{1.553546in}{2.133180in}}%
\pgfpathlineto{\pgfqpoint{1.665048in}{2.133180in}}%
\pgfpathlineto{\pgfqpoint{1.776550in}{2.220061in}}%
\pgfpathlineto{\pgfqpoint{1.888052in}{2.220061in}}%
\pgfpathlineto{\pgfqpoint{1.999554in}{2.220061in}}%
\pgfpathlineto{\pgfqpoint{2.111057in}{2.220061in}}%
\pgfpathlineto{\pgfqpoint{2.222559in}{2.220061in}}%
\pgfpathlineto{\pgfqpoint{2.334061in}{2.220061in}}%
\pgfpathlineto{\pgfqpoint{2.445563in}{2.306943in}}%
\pgfpathlineto{\pgfqpoint{2.557066in}{2.306943in}}%
\pgfpathlineto{\pgfqpoint{2.668568in}{2.306943in}}%
\pgfpathlineto{\pgfqpoint{2.780070in}{2.393825in}}%
\pgfpathlineto{\pgfqpoint{2.891572in}{2.393825in}}%
\pgfpathlineto{\pgfqpoint{3.003074in}{2.393825in}}%
\pgfpathlineto{\pgfqpoint{3.114577in}{2.393825in}}%
\pgfpathlineto{\pgfqpoint{3.226079in}{2.393825in}}%
\pgfpathlineto{\pgfqpoint{3.337581in}{2.393825in}}%
\pgfpathlineto{\pgfqpoint{3.449083in}{2.567588in}}%
\pgfpathlineto{\pgfqpoint{3.560586in}{2.567588in}}%
\pgfpathlineto{\pgfqpoint{3.672088in}{2.567588in}}%
\pgfpathlineto{\pgfqpoint{3.783590in}{2.654470in}}%
\pgfpathlineto{\pgfqpoint{3.895092in}{2.654470in}}%
\pgfpathlineto{\pgfqpoint{4.006594in}{2.654470in}}%
\pgfpathlineto{\pgfqpoint{4.118097in}{2.654470in}}%
\pgfpathlineto{\pgfqpoint{4.229599in}{2.654470in}}%
\pgfpathlineto{\pgfqpoint{4.341101in}{2.654470in}}%
\pgfpathlineto{\pgfqpoint{4.452603in}{2.654470in}}%
\pgfpathlineto{\pgfqpoint{4.564106in}{2.654470in}}%
\pgfpathlineto{\pgfqpoint{4.675608in}{2.654470in}}%
\pgfpathlineto{\pgfqpoint{4.787110in}{2.654470in}}%
\pgfpathlineto{\pgfqpoint{4.898612in}{2.654470in}}%
\pgfpathlineto{\pgfqpoint{5.010114in}{2.741352in}}%
\pgfpathlineto{\pgfqpoint{5.121617in}{2.741352in}}%
\pgfpathlineto{\pgfqpoint{5.233119in}{2.828234in}}%
\pgfpathlineto{\pgfqpoint{5.344621in}{2.828234in}}%
\pgfpathlineto{\pgfqpoint{5.456123in}{2.828234in}}%
\pgfpathlineto{\pgfqpoint{5.567626in}{2.828234in}}%
\pgfpathlineto{\pgfqpoint{5.679128in}{2.828234in}}%
\pgfpathlineto{\pgfqpoint{5.790630in}{2.828234in}}%
\pgfpathlineto{\pgfqpoint{5.902132in}{2.828234in}}%
\pgfpathlineto{\pgfqpoint{6.013634in}{2.828234in}}%
\pgfpathlineto{\pgfqpoint{6.125137in}{3.001997in}}%
\pgfpathlineto{\pgfqpoint{6.236639in}{3.001997in}}%
\pgfpathlineto{\pgfqpoint{6.348141in}{3.001997in}}%
\pgfpathlineto{\pgfqpoint{6.459643in}{3.001997in}}%
\pgfpathlineto{\pgfqpoint{6.571146in}{3.001997in}}%
\pgfpathlineto{\pgfqpoint{6.682648in}{3.001997in}}%
\pgfpathlineto{\pgfqpoint{6.794150in}{3.001997in}}%
\pgfpathlineto{\pgfqpoint{6.905652in}{3.001997in}}%
\pgfpathlineto{\pgfqpoint{7.017154in}{3.001997in}}%
\pgfpathlineto{\pgfqpoint{7.128657in}{3.001997in}}%
\pgfpathlineto{\pgfqpoint{7.240159in}{3.001997in}}%
\pgfpathlineto{\pgfqpoint{7.351661in}{3.001997in}}%
\pgfpathlineto{\pgfqpoint{7.463163in}{3.001997in}}%
\pgfpathlineto{\pgfqpoint{7.574666in}{3.001997in}}%
\pgfpathlineto{\pgfqpoint{7.686168in}{3.001997in}}%
\pgfpathlineto{\pgfqpoint{7.797670in}{3.088879in}}%
\pgfpathlineto{\pgfqpoint{7.909172in}{3.088879in}}%
\pgfpathlineto{\pgfqpoint{8.020674in}{3.088879in}}%
\pgfpathlineto{\pgfqpoint{8.132177in}{3.088879in}}%
\pgfpathlineto{\pgfqpoint{8.243679in}{3.088879in}}%
\pgfpathlineto{\pgfqpoint{8.355181in}{3.088879in}}%
\pgfpathlineto{\pgfqpoint{8.466683in}{3.088879in}}%
\pgfpathlineto{\pgfqpoint{8.578186in}{3.088879in}}%
\pgfpathlineto{\pgfqpoint{8.689688in}{3.088879in}}%
\pgfpathlineto{\pgfqpoint{8.801190in}{3.088879in}}%
\pgfpathlineto{\pgfqpoint{8.912692in}{3.088879in}}%
\pgfpathlineto{\pgfqpoint{9.024194in}{3.088879in}}%
\pgfpathlineto{\pgfqpoint{9.135697in}{3.088879in}}%
\pgfpathlineto{\pgfqpoint{9.247199in}{3.088879in}}%
\pgfpathlineto{\pgfqpoint{9.358701in}{3.088879in}}%
\pgfpathlineto{\pgfqpoint{9.470203in}{3.088879in}}%
\pgfpathlineto{\pgfqpoint{9.581706in}{3.088879in}}%
\pgfpathlineto{\pgfqpoint{9.693208in}{3.088879in}}%
\pgfpathlineto{\pgfqpoint{9.804710in}{3.088879in}}%
\pgfpathlineto{\pgfqpoint{9.916212in}{3.088879in}}%
\pgfpathlineto{\pgfqpoint{10.027714in}{3.088879in}}%
\pgfpathlineto{\pgfqpoint{10.139217in}{3.088879in}}%
\pgfpathlineto{\pgfqpoint{10.250719in}{3.088879in}}%
\pgfpathlineto{\pgfqpoint{10.362221in}{3.088879in}}%
\pgfpathlineto{\pgfqpoint{10.473723in}{3.088879in}}%
\pgfpathlineto{\pgfqpoint{10.585226in}{3.088879in}}%
\pgfpathlineto{\pgfqpoint{10.696728in}{3.088879in}}%
\pgfpathlineto{\pgfqpoint{10.808230in}{3.088879in}}%
\pgfpathlineto{\pgfqpoint{10.919732in}{3.088879in}}%
\pgfpathlineto{\pgfqpoint{11.031234in}{3.088879in}}%
\pgfpathlineto{\pgfqpoint{11.142737in}{3.088879in}}%
\pgfpathlineto{\pgfqpoint{11.254239in}{3.088879in}}%
\pgfpathlineto{\pgfqpoint{11.365741in}{3.088879in}}%
\pgfpathlineto{\pgfqpoint{11.477243in}{3.088879in}}%
\pgfpathlineto{\pgfqpoint{11.588746in}{3.088879in}}%
\pgfpathlineto{\pgfqpoint{11.700248in}{3.088879in}}%
\pgfpathlineto{\pgfqpoint{11.811750in}{3.088879in}}%
\pgfusepath{stroke}%
\end{pgfscope}%
\begin{pgfscope}%
\pgfpathrectangle{\pgfqpoint{0.661528in}{0.586684in}}{\pgfqpoint{11.150222in}{7.263316in}}%
\pgfusepath{clip}%
\pgfsetrectcap%
\pgfsetroundjoin%
\pgfsetlinewidth{2.007500pt}%
\definecolor{currentstroke}{rgb}{0.549020,0.337255,0.294118}%
\pgfsetstrokecolor{currentstroke}%
\pgfsetdash{}{0pt}%
\pgfpathmoveto{\pgfqpoint{0.661528in}{0.916835in}}%
\pgfpathlineto{\pgfqpoint{0.773030in}{1.264362in}}%
\pgfpathlineto{\pgfqpoint{0.884532in}{1.264362in}}%
\pgfpathlineto{\pgfqpoint{0.996034in}{1.438125in}}%
\pgfpathlineto{\pgfqpoint{1.107537in}{1.438125in}}%
\pgfpathlineto{\pgfqpoint{1.219039in}{1.438125in}}%
\pgfpathlineto{\pgfqpoint{1.330541in}{1.438125in}}%
\pgfpathlineto{\pgfqpoint{1.442043in}{1.438125in}}%
\pgfpathlineto{\pgfqpoint{1.553546in}{1.438125in}}%
\pgfpathlineto{\pgfqpoint{1.665048in}{1.438125in}}%
\pgfpathlineto{\pgfqpoint{1.776550in}{1.438125in}}%
\pgfpathlineto{\pgfqpoint{1.888052in}{1.438125in}}%
\pgfpathlineto{\pgfqpoint{1.999554in}{1.438125in}}%
\pgfpathlineto{\pgfqpoint{2.111057in}{1.438125in}}%
\pgfpathlineto{\pgfqpoint{2.222559in}{1.438125in}}%
\pgfpathlineto{\pgfqpoint{2.334061in}{1.438125in}}%
\pgfpathlineto{\pgfqpoint{2.445563in}{1.438125in}}%
\pgfpathlineto{\pgfqpoint{2.557066in}{1.438125in}}%
\pgfpathlineto{\pgfqpoint{2.668568in}{1.438125in}}%
\pgfpathlineto{\pgfqpoint{2.780070in}{1.525007in}}%
\pgfpathlineto{\pgfqpoint{2.891572in}{1.525007in}}%
\pgfpathlineto{\pgfqpoint{3.003074in}{1.525007in}}%
\pgfpathlineto{\pgfqpoint{3.114577in}{1.611889in}}%
\pgfpathlineto{\pgfqpoint{3.226079in}{1.611889in}}%
\pgfpathlineto{\pgfqpoint{3.337581in}{1.611889in}}%
\pgfpathlineto{\pgfqpoint{3.449083in}{1.611889in}}%
\pgfpathlineto{\pgfqpoint{3.560586in}{1.611889in}}%
\pgfpathlineto{\pgfqpoint{3.672088in}{1.611889in}}%
\pgfpathlineto{\pgfqpoint{3.783590in}{1.611889in}}%
\pgfpathlineto{\pgfqpoint{3.895092in}{1.611889in}}%
\pgfpathlineto{\pgfqpoint{4.006594in}{1.611889in}}%
\pgfpathlineto{\pgfqpoint{4.118097in}{1.611889in}}%
\pgfpathlineto{\pgfqpoint{4.229599in}{1.611889in}}%
\pgfpathlineto{\pgfqpoint{4.341101in}{1.611889in}}%
\pgfpathlineto{\pgfqpoint{4.452603in}{1.611889in}}%
\pgfpathlineto{\pgfqpoint{4.564106in}{1.611889in}}%
\pgfpathlineto{\pgfqpoint{4.675608in}{1.611889in}}%
\pgfpathlineto{\pgfqpoint{4.787110in}{1.611889in}}%
\pgfpathlineto{\pgfqpoint{4.898612in}{1.611889in}}%
\pgfpathlineto{\pgfqpoint{5.010114in}{1.611889in}}%
\pgfpathlineto{\pgfqpoint{5.121617in}{1.611889in}}%
\pgfpathlineto{\pgfqpoint{5.233119in}{1.611889in}}%
\pgfpathlineto{\pgfqpoint{5.344621in}{1.698771in}}%
\pgfpathlineto{\pgfqpoint{5.456123in}{1.698771in}}%
\pgfpathlineto{\pgfqpoint{5.567626in}{1.698771in}}%
\pgfpathlineto{\pgfqpoint{5.679128in}{1.698771in}}%
\pgfpathlineto{\pgfqpoint{5.790630in}{1.698771in}}%
\pgfpathlineto{\pgfqpoint{5.902132in}{1.698771in}}%
\pgfpathlineto{\pgfqpoint{6.013634in}{1.698771in}}%
\pgfpathlineto{\pgfqpoint{6.125137in}{1.698771in}}%
\pgfpathlineto{\pgfqpoint{6.236639in}{1.698771in}}%
\pgfpathlineto{\pgfqpoint{6.348141in}{1.698771in}}%
\pgfpathlineto{\pgfqpoint{6.459643in}{1.698771in}}%
\pgfpathlineto{\pgfqpoint{6.571146in}{1.698771in}}%
\pgfpathlineto{\pgfqpoint{6.682648in}{1.698771in}}%
\pgfpathlineto{\pgfqpoint{6.794150in}{1.698771in}}%
\pgfpathlineto{\pgfqpoint{6.905652in}{1.698771in}}%
\pgfpathlineto{\pgfqpoint{7.017154in}{1.698771in}}%
\pgfpathlineto{\pgfqpoint{7.128657in}{1.698771in}}%
\pgfpathlineto{\pgfqpoint{7.240159in}{1.698771in}}%
\pgfpathlineto{\pgfqpoint{7.351661in}{1.698771in}}%
\pgfpathlineto{\pgfqpoint{7.463163in}{1.785652in}}%
\pgfpathlineto{\pgfqpoint{7.574666in}{1.785652in}}%
\pgfpathlineto{\pgfqpoint{7.686168in}{1.785652in}}%
\pgfpathlineto{\pgfqpoint{7.797670in}{1.785652in}}%
\pgfpathlineto{\pgfqpoint{7.909172in}{1.785652in}}%
\pgfpathlineto{\pgfqpoint{8.020674in}{1.785652in}}%
\pgfpathlineto{\pgfqpoint{8.132177in}{1.785652in}}%
\pgfpathlineto{\pgfqpoint{8.243679in}{1.785652in}}%
\pgfpathlineto{\pgfqpoint{8.355181in}{1.785652in}}%
\pgfpathlineto{\pgfqpoint{8.466683in}{1.872534in}}%
\pgfpathlineto{\pgfqpoint{8.578186in}{1.872534in}}%
\pgfpathlineto{\pgfqpoint{8.689688in}{1.872534in}}%
\pgfpathlineto{\pgfqpoint{8.801190in}{1.872534in}}%
\pgfpathlineto{\pgfqpoint{8.912692in}{1.872534in}}%
\pgfpathlineto{\pgfqpoint{9.024194in}{1.872534in}}%
\pgfpathlineto{\pgfqpoint{9.135697in}{1.872534in}}%
\pgfpathlineto{\pgfqpoint{9.247199in}{1.872534in}}%
\pgfpathlineto{\pgfqpoint{9.358701in}{1.872534in}}%
\pgfpathlineto{\pgfqpoint{9.470203in}{1.872534in}}%
\pgfpathlineto{\pgfqpoint{9.581706in}{1.872534in}}%
\pgfpathlineto{\pgfqpoint{9.693208in}{1.872534in}}%
\pgfpathlineto{\pgfqpoint{9.804710in}{1.872534in}}%
\pgfpathlineto{\pgfqpoint{9.916212in}{1.872534in}}%
\pgfpathlineto{\pgfqpoint{10.027714in}{1.872534in}}%
\pgfpathlineto{\pgfqpoint{10.139217in}{1.872534in}}%
\pgfpathlineto{\pgfqpoint{10.250719in}{1.872534in}}%
\pgfpathlineto{\pgfqpoint{10.362221in}{1.872534in}}%
\pgfpathlineto{\pgfqpoint{10.473723in}{1.872534in}}%
\pgfpathlineto{\pgfqpoint{10.585226in}{1.872534in}}%
\pgfpathlineto{\pgfqpoint{10.696728in}{1.872534in}}%
\pgfpathlineto{\pgfqpoint{10.808230in}{1.872534in}}%
\pgfpathlineto{\pgfqpoint{10.919732in}{1.872534in}}%
\pgfpathlineto{\pgfqpoint{11.031234in}{1.872534in}}%
\pgfpathlineto{\pgfqpoint{11.142737in}{1.872534in}}%
\pgfpathlineto{\pgfqpoint{11.254239in}{1.872534in}}%
\pgfpathlineto{\pgfqpoint{11.365741in}{1.872534in}}%
\pgfpathlineto{\pgfqpoint{11.477243in}{1.872534in}}%
\pgfpathlineto{\pgfqpoint{11.588746in}{1.872534in}}%
\pgfpathlineto{\pgfqpoint{11.700248in}{1.872534in}}%
\pgfpathlineto{\pgfqpoint{11.811750in}{1.872534in}}%
\pgfusepath{stroke}%
\end{pgfscope}%
\begin{pgfscope}%
\pgfsetrectcap%
\pgfsetmiterjoin%
\pgfsetlinewidth{0.803000pt}%
\definecolor{currentstroke}{rgb}{0.000000,0.000000,0.000000}%
\pgfsetstrokecolor{currentstroke}%
\pgfsetdash{}{0pt}%
\pgfpathmoveto{\pgfqpoint{0.661528in}{0.586684in}}%
\pgfpathlineto{\pgfqpoint{0.661528in}{7.850000in}}%
\pgfusepath{stroke}%
\end{pgfscope}%
\begin{pgfscope}%
\pgfsetrectcap%
\pgfsetmiterjoin%
\pgfsetlinewidth{0.803000pt}%
\definecolor{currentstroke}{rgb}{0.000000,0.000000,0.000000}%
\pgfsetstrokecolor{currentstroke}%
\pgfsetdash{}{0pt}%
\pgfpathmoveto{\pgfqpoint{11.811750in}{0.586684in}}%
\pgfpathlineto{\pgfqpoint{11.811750in}{7.850000in}}%
\pgfusepath{stroke}%
\end{pgfscope}%
\begin{pgfscope}%
\pgfsetrectcap%
\pgfsetmiterjoin%
\pgfsetlinewidth{0.803000pt}%
\definecolor{currentstroke}{rgb}{0.000000,0.000000,0.000000}%
\pgfsetstrokecolor{currentstroke}%
\pgfsetdash{}{0pt}%
\pgfpathmoveto{\pgfqpoint{0.661528in}{0.586684in}}%
\pgfpathlineto{\pgfqpoint{11.811750in}{0.586684in}}%
\pgfusepath{stroke}%
\end{pgfscope}%
\begin{pgfscope}%
\pgfsetrectcap%
\pgfsetmiterjoin%
\pgfsetlinewidth{0.803000pt}%
\definecolor{currentstroke}{rgb}{0.000000,0.000000,0.000000}%
\pgfsetstrokecolor{currentstroke}%
\pgfsetdash{}{0pt}%
\pgfpathmoveto{\pgfqpoint{0.661528in}{7.850000in}}%
\pgfpathlineto{\pgfqpoint{11.811750in}{7.850000in}}%
\pgfusepath{stroke}%
\end{pgfscope}%
\begin{pgfscope}%
\pgfsetbuttcap%
\pgfsetmiterjoin%
\definecolor{currentfill}{rgb}{1.000000,1.000000,1.000000}%
\pgfsetfillcolor{currentfill}%
\pgfsetfillopacity{0.800000}%
\pgfsetlinewidth{1.003750pt}%
\definecolor{currentstroke}{rgb}{0.800000,0.800000,0.800000}%
\pgfsetstrokecolor{currentstroke}%
\pgfsetstrokeopacity{0.800000}%
\pgfsetdash{}{0pt}%
\pgfpathmoveto{\pgfqpoint{0.758750in}{6.496418in}}%
\pgfpathlineto{\pgfqpoint{2.230687in}{6.496418in}}%
\pgfpathquadraticcurveto{\pgfqpoint{2.258465in}{6.496418in}}{\pgfqpoint{2.258465in}{6.524196in}}%
\pgfpathlineto{\pgfqpoint{2.258465in}{7.752778in}}%
\pgfpathquadraticcurveto{\pgfqpoint{2.258465in}{7.780556in}}{\pgfqpoint{2.230687in}{7.780556in}}%
\pgfpathlineto{\pgfqpoint{0.758750in}{7.780556in}}%
\pgfpathquadraticcurveto{\pgfqpoint{0.730972in}{7.780556in}}{\pgfqpoint{0.730972in}{7.752778in}}%
\pgfpathlineto{\pgfqpoint{0.730972in}{6.524196in}}%
\pgfpathquadraticcurveto{\pgfqpoint{0.730972in}{6.496418in}}{\pgfqpoint{0.758750in}{6.496418in}}%
\pgfpathlineto{\pgfqpoint{0.758750in}{6.496418in}}%
\pgfpathclose%
\pgfusepath{stroke,fill}%
\end{pgfscope}%
\begin{pgfscope}%
\pgfsetrectcap%
\pgfsetroundjoin%
\pgfsetlinewidth{2.007500pt}%
\definecolor{currentstroke}{rgb}{0.121569,0.466667,0.705882}%
\pgfsetstrokecolor{currentstroke}%
\pgfsetdash{}{0pt}%
\pgfpathmoveto{\pgfqpoint{0.786528in}{7.668088in}}%
\pgfpathlineto{\pgfqpoint{0.925417in}{7.668088in}}%
\pgfpathlineto{\pgfqpoint{1.064306in}{7.668088in}}%
\pgfusepath{stroke}%
\end{pgfscope}%
\begin{pgfscope}%
\definecolor{textcolor}{rgb}{0.000000,0.000000,0.000000}%
\pgfsetstrokecolor{textcolor}%
\pgfsetfillcolor{textcolor}%
\pgftext[x=1.175417in,y=7.619477in,left,base]{\color{textcolor}{\rmfamily\fontsize{10.000000}{12.000000}\selectfont\catcode`\^=\active\def^{\ifmmode\sp\else\^{}\fi}\catcode`\%=\active\def%{\%}CPLEX}}%
\end{pgfscope}%
\begin{pgfscope}%
\pgfsetrectcap%
\pgfsetroundjoin%
\pgfsetlinewidth{2.007500pt}%
\definecolor{currentstroke}{rgb}{1.000000,0.498039,0.054902}%
\pgfsetstrokecolor{currentstroke}%
\pgfsetdash{}{0pt}%
\pgfpathmoveto{\pgfqpoint{0.786528in}{7.464231in}}%
\pgfpathlineto{\pgfqpoint{0.925417in}{7.464231in}}%
\pgfpathlineto{\pgfqpoint{1.064306in}{7.464231in}}%
\pgfusepath{stroke}%
\end{pgfscope}%
\begin{pgfscope}%
\definecolor{textcolor}{rgb}{0.000000,0.000000,0.000000}%
\pgfsetstrokecolor{textcolor}%
\pgfsetfillcolor{textcolor}%
\pgftext[x=1.175417in,y=7.415620in,left,base]{\color{textcolor}{\rmfamily\fontsize{10.000000}{12.000000}\selectfont\catcode`\^=\active\def^{\ifmmode\sp\else\^{}\fi}\catcode`\%=\active\def%{\%}PACS, $\rho =.1$}}%
\end{pgfscope}%
\begin{pgfscope}%
\pgfsetrectcap%
\pgfsetroundjoin%
\pgfsetlinewidth{2.007500pt}%
\definecolor{currentstroke}{rgb}{0.172549,0.627451,0.172549}%
\pgfsetstrokecolor{currentstroke}%
\pgfsetdash{}{0pt}%
\pgfpathmoveto{\pgfqpoint{0.786528in}{7.256508in}}%
\pgfpathlineto{\pgfqpoint{0.925417in}{7.256508in}}%
\pgfpathlineto{\pgfqpoint{1.064306in}{7.256508in}}%
\pgfusepath{stroke}%
\end{pgfscope}%
\begin{pgfscope}%
\definecolor{textcolor}{rgb}{0.000000,0.000000,0.000000}%
\pgfsetstrokecolor{textcolor}%
\pgfsetfillcolor{textcolor}%
\pgftext[x=1.175417in,y=7.207897in,left,base]{\color{textcolor}{\rmfamily\fontsize{10.000000}{12.000000}\selectfont\catcode`\^=\active\def^{\ifmmode\sp\else\^{}\fi}\catcode`\%=\active\def%{\%}PACS, $\rho =.25$}}%
\end{pgfscope}%
\begin{pgfscope}%
\pgfsetrectcap%
\pgfsetroundjoin%
\pgfsetlinewidth{2.007500pt}%
\definecolor{currentstroke}{rgb}{0.839216,0.152941,0.156863}%
\pgfsetstrokecolor{currentstroke}%
\pgfsetdash{}{0pt}%
\pgfpathmoveto{\pgfqpoint{0.786528in}{7.048785in}}%
\pgfpathlineto{\pgfqpoint{0.925417in}{7.048785in}}%
\pgfpathlineto{\pgfqpoint{1.064306in}{7.048785in}}%
\pgfusepath{stroke}%
\end{pgfscope}%
\begin{pgfscope}%
\definecolor{textcolor}{rgb}{0.000000,0.000000,0.000000}%
\pgfsetstrokecolor{textcolor}%
\pgfsetfillcolor{textcolor}%
\pgftext[x=1.175417in,y=7.000174in,left,base]{\color{textcolor}{\rmfamily\fontsize{10.000000}{12.000000}\selectfont\catcode`\^=\active\def^{\ifmmode\sp\else\^{}\fi}\catcode`\%=\active\def%{\%}PACS, $\rho =.5$}}%
\end{pgfscope}%
\begin{pgfscope}%
\pgfsetrectcap%
\pgfsetroundjoin%
\pgfsetlinewidth{2.007500pt}%
\definecolor{currentstroke}{rgb}{0.580392,0.403922,0.741176}%
\pgfsetstrokecolor{currentstroke}%
\pgfsetdash{}{0pt}%
\pgfpathmoveto{\pgfqpoint{0.786528in}{6.841062in}}%
\pgfpathlineto{\pgfqpoint{0.925417in}{6.841062in}}%
\pgfpathlineto{\pgfqpoint{1.064306in}{6.841062in}}%
\pgfusepath{stroke}%
\end{pgfscope}%
\begin{pgfscope}%
\definecolor{textcolor}{rgb}{0.000000,0.000000,0.000000}%
\pgfsetstrokecolor{textcolor}%
\pgfsetfillcolor{textcolor}%
\pgftext[x=1.175417in,y=6.792451in,left,base]{\color{textcolor}{\rmfamily\fontsize{10.000000}{12.000000}\selectfont\catcode`\^=\active\def^{\ifmmode\sp\else\^{}\fi}\catcode`\%=\active\def%{\%}PACS, $\rho =.75$}}%
\end{pgfscope}%
\begin{pgfscope}%
\pgfsetrectcap%
\pgfsetroundjoin%
\pgfsetlinewidth{2.007500pt}%
\definecolor{currentstroke}{rgb}{0.549020,0.337255,0.294118}%
\pgfsetstrokecolor{currentstroke}%
\pgfsetdash{}{0pt}%
\pgfpathmoveto{\pgfqpoint{0.786528in}{6.633340in}}%
\pgfpathlineto{\pgfqpoint{0.925417in}{6.633340in}}%
\pgfpathlineto{\pgfqpoint{1.064306in}{6.633340in}}%
\pgfusepath{stroke}%
\end{pgfscope}%
\begin{pgfscope}%
\definecolor{textcolor}{rgb}{0.000000,0.000000,0.000000}%
\pgfsetstrokecolor{textcolor}%
\pgfsetfillcolor{textcolor}%
\pgftext[x=1.175417in,y=6.584729in,left,base]{\color{textcolor}{\rmfamily\fontsize{10.000000}{12.000000}\selectfont\catcode`\^=\active\def^{\ifmmode\sp\else\^{}\fi}\catcode`\%=\active\def%{\%}PACS, $\rho =.9$}}%
\end{pgfscope}%
\end{pgfpicture}%
\makeatother%
\endgroup%
}
    \end{minipage}%
    \hfill
    \begin{minipage}{0.4\columnwidth}
        \centering
        \resizebox{\linewidth}{!}{%% Creator: Matplotlib, PGF backend
%%
%% To include the figure in your LaTeX document, write
%%   \input{<filename>.pgf}
%%
%% Make sure the required packages are loaded in your preamble
%%   \usepackage{pgf}
%%
%% Also ensure that all the required font packages are loaded; for instance,
%% the lmodern package is sometimes necessary when using math font.
%%   \usepackage{lmodern}
%%
%% Figures using additional raster images can only be included by \input if
%% they are in the same directory as the main LaTeX file. For loading figures
%% from other directories you can use the `import` package
%%   \usepackage{import}
%%
%% and then include the figures with
%%   \import{<path to file>}{<filename>.pgf}
%%
%% Matplotlib used the following preamble
%%   \def\mathdefault#1{#1}
%%   \everymath=\expandafter{\the\everymath\displaystyle}
%%   \IfFileExists{scrextend.sty}{
%%     \usepackage[fontsize=10.000000pt]{scrextend}
%%   }{
%%     \renewcommand{\normalsize}{\fontsize{10.000000}{12.000000}\selectfont}
%%     \normalsize
%%   }
%%   
%%   \ifdefined\pdftexversion\else  % non-pdftex case.
%%     \usepackage{fontspec}
%%     \setmainfont{DejaVuSerif.ttf}[Path=\detokenize{/home/bisca/.global/lib/python3.12/site-packages/matplotlib/mpl-data/fonts/ttf/}]
%%     \setsansfont{DejaVuSans.ttf}[Path=\detokenize{/home/bisca/.global/lib/python3.12/site-packages/matplotlib/mpl-data/fonts/ttf/}]
%%     \setmonofont{DejaVuSansMono.ttf}[Path=\detokenize{/home/bisca/.global/lib/python3.12/site-packages/matplotlib/mpl-data/fonts/ttf/}]
%%   \fi
%%   \makeatletter\@ifpackageloaded{underscore}{}{\usepackage[strings]{underscore}}\makeatother
%%
\begingroup%
\makeatletter%
\begin{pgfpicture}%
\pgfpathrectangle{\pgfpointorigin}{\pgfqpoint{8.000000in}{5.000000in}}%
\pgfusepath{use as bounding box, clip}%
\begin{pgfscope}%
\pgfsetbuttcap%
\pgfsetmiterjoin%
\definecolor{currentfill}{rgb}{1.000000,1.000000,1.000000}%
\pgfsetfillcolor{currentfill}%
\pgfsetlinewidth{0.000000pt}%
\definecolor{currentstroke}{rgb}{1.000000,1.000000,1.000000}%
\pgfsetstrokecolor{currentstroke}%
\pgfsetdash{}{0pt}%
\pgfpathmoveto{\pgfqpoint{0.000000in}{0.000000in}}%
\pgfpathlineto{\pgfqpoint{8.000000in}{0.000000in}}%
\pgfpathlineto{\pgfqpoint{8.000000in}{5.000000in}}%
\pgfpathlineto{\pgfqpoint{0.000000in}{5.000000in}}%
\pgfpathlineto{\pgfqpoint{0.000000in}{0.000000in}}%
\pgfpathclose%
\pgfusepath{fill}%
\end{pgfscope}%
\begin{pgfscope}%
\pgfsetbuttcap%
\pgfsetmiterjoin%
\definecolor{currentfill}{rgb}{1.000000,1.000000,1.000000}%
\pgfsetfillcolor{currentfill}%
\pgfsetlinewidth{0.000000pt}%
\definecolor{currentstroke}{rgb}{0.000000,0.000000,0.000000}%
\pgfsetstrokecolor{currentstroke}%
\pgfsetstrokeopacity{0.000000}%
\pgfsetdash{}{0pt}%
\pgfpathmoveto{\pgfqpoint{0.706528in}{0.395972in}}%
\pgfpathlineto{\pgfqpoint{7.850000in}{0.395972in}}%
\pgfpathlineto{\pgfqpoint{7.850000in}{4.850000in}}%
\pgfpathlineto{\pgfqpoint{0.706528in}{4.850000in}}%
\pgfpathlineto{\pgfqpoint{0.706528in}{0.395972in}}%
\pgfpathclose%
\pgfusepath{fill}%
\end{pgfscope}%
\begin{pgfscope}%
\pgfpathrectangle{\pgfqpoint{0.706528in}{0.395972in}}{\pgfqpoint{7.143472in}{4.454028in}}%
\pgfusepath{clip}%
\pgfsetbuttcap%
\pgfsetmiterjoin%
\definecolor{currentfill}{rgb}{0.121569,0.466667,0.705882}%
\pgfsetfillcolor{currentfill}%
\pgfsetlinewidth{0.000000pt}%
\definecolor{currentstroke}{rgb}{0.000000,0.000000,0.000000}%
\pgfsetstrokecolor{currentstroke}%
\pgfsetstrokeopacity{0.000000}%
\pgfsetdash{}{0pt}%
\pgfpathmoveto{\pgfqpoint{1.031231in}{0.395972in}}%
\pgfpathlineto{\pgfqpoint{1.926964in}{0.395972in}}%
\pgfpathlineto{\pgfqpoint{1.926964in}{4.637903in}}%
\pgfpathlineto{\pgfqpoint{1.031231in}{4.637903in}}%
\pgfpathlineto{\pgfqpoint{1.031231in}{0.395972in}}%
\pgfpathclose%
\pgfusepath{fill}%
\end{pgfscope}%
\begin{pgfscope}%
\pgfpathrectangle{\pgfqpoint{0.706528in}{0.395972in}}{\pgfqpoint{7.143472in}{4.454028in}}%
\pgfusepath{clip}%
\pgfsetbuttcap%
\pgfsetmiterjoin%
\definecolor{currentfill}{rgb}{1.000000,0.498039,0.054902}%
\pgfsetfillcolor{currentfill}%
\pgfsetlinewidth{0.000000pt}%
\definecolor{currentstroke}{rgb}{0.000000,0.000000,0.000000}%
\pgfsetstrokecolor{currentstroke}%
\pgfsetstrokeopacity{0.000000}%
\pgfsetdash{}{0pt}%
\pgfpathmoveto{\pgfqpoint{2.150898in}{0.395972in}}%
\pgfpathlineto{\pgfqpoint{3.046631in}{0.395972in}}%
\pgfpathlineto{\pgfqpoint{3.046631in}{3.645973in}}%
\pgfpathlineto{\pgfqpoint{2.150898in}{3.645973in}}%
\pgfpathlineto{\pgfqpoint{2.150898in}{0.395972in}}%
\pgfpathclose%
\pgfusepath{fill}%
\end{pgfscope}%
\begin{pgfscope}%
\pgfpathrectangle{\pgfqpoint{0.706528in}{0.395972in}}{\pgfqpoint{7.143472in}{4.454028in}}%
\pgfusepath{clip}%
\pgfsetbuttcap%
\pgfsetmiterjoin%
\definecolor{currentfill}{rgb}{0.172549,0.627451,0.172549}%
\pgfsetfillcolor{currentfill}%
\pgfsetlinewidth{0.000000pt}%
\definecolor{currentstroke}{rgb}{0.000000,0.000000,0.000000}%
\pgfsetstrokecolor{currentstroke}%
\pgfsetstrokeopacity{0.000000}%
\pgfsetdash{}{0pt}%
\pgfpathmoveto{\pgfqpoint{3.270564in}{0.395972in}}%
\pgfpathlineto{\pgfqpoint{4.166297in}{0.395972in}}%
\pgfpathlineto{\pgfqpoint{4.166297in}{3.598020in}}%
\pgfpathlineto{\pgfqpoint{3.270564in}{3.598020in}}%
\pgfpathlineto{\pgfqpoint{3.270564in}{0.395972in}}%
\pgfpathclose%
\pgfusepath{fill}%
\end{pgfscope}%
\begin{pgfscope}%
\pgfpathrectangle{\pgfqpoint{0.706528in}{0.395972in}}{\pgfqpoint{7.143472in}{4.454028in}}%
\pgfusepath{clip}%
\pgfsetbuttcap%
\pgfsetmiterjoin%
\definecolor{currentfill}{rgb}{0.839216,0.152941,0.156863}%
\pgfsetfillcolor{currentfill}%
\pgfsetlinewidth{0.000000pt}%
\definecolor{currentstroke}{rgb}{0.000000,0.000000,0.000000}%
\pgfsetstrokecolor{currentstroke}%
\pgfsetstrokeopacity{0.000000}%
\pgfsetdash{}{0pt}%
\pgfpathmoveto{\pgfqpoint{4.390231in}{0.395972in}}%
\pgfpathlineto{\pgfqpoint{5.285964in}{0.395972in}}%
\pgfpathlineto{\pgfqpoint{5.285964in}{2.862285in}}%
\pgfpathlineto{\pgfqpoint{4.390231in}{2.862285in}}%
\pgfpathlineto{\pgfqpoint{4.390231in}{0.395972in}}%
\pgfpathclose%
\pgfusepath{fill}%
\end{pgfscope}%
\begin{pgfscope}%
\pgfpathrectangle{\pgfqpoint{0.706528in}{0.395972in}}{\pgfqpoint{7.143472in}{4.454028in}}%
\pgfusepath{clip}%
\pgfsetbuttcap%
\pgfsetmiterjoin%
\definecolor{currentfill}{rgb}{0.580392,0.403922,0.741176}%
\pgfsetfillcolor{currentfill}%
\pgfsetlinewidth{0.000000pt}%
\definecolor{currentstroke}{rgb}{0.000000,0.000000,0.000000}%
\pgfsetstrokecolor{currentstroke}%
\pgfsetstrokeopacity{0.000000}%
\pgfsetdash{}{0pt}%
\pgfpathmoveto{\pgfqpoint{5.509897in}{0.395972in}}%
\pgfpathlineto{\pgfqpoint{6.405630in}{0.395972in}}%
\pgfpathlineto{\pgfqpoint{6.405630in}{1.949572in}}%
\pgfpathlineto{\pgfqpoint{5.509897in}{1.949572in}}%
\pgfpathlineto{\pgfqpoint{5.509897in}{0.395972in}}%
\pgfpathclose%
\pgfusepath{fill}%
\end{pgfscope}%
\begin{pgfscope}%
\pgfpathrectangle{\pgfqpoint{0.706528in}{0.395972in}}{\pgfqpoint{7.143472in}{4.454028in}}%
\pgfusepath{clip}%
\pgfsetbuttcap%
\pgfsetmiterjoin%
\definecolor{currentfill}{rgb}{0.549020,0.337255,0.294118}%
\pgfsetfillcolor{currentfill}%
\pgfsetlinewidth{0.000000pt}%
\definecolor{currentstroke}{rgb}{0.000000,0.000000,0.000000}%
\pgfsetstrokecolor{currentstroke}%
\pgfsetstrokeopacity{0.000000}%
\pgfsetdash{}{0pt}%
\pgfpathmoveto{\pgfqpoint{6.629564in}{0.395972in}}%
\pgfpathlineto{\pgfqpoint{7.525297in}{0.395972in}}%
\pgfpathlineto{\pgfqpoint{7.525297in}{1.286961in}}%
\pgfpathlineto{\pgfqpoint{6.629564in}{1.286961in}}%
\pgfpathlineto{\pgfqpoint{6.629564in}{0.395972in}}%
\pgfpathclose%
\pgfusepath{fill}%
\end{pgfscope}%
\begin{pgfscope}%
\pgfsetbuttcap%
\pgfsetroundjoin%
\definecolor{currentfill}{rgb}{0.000000,0.000000,0.000000}%
\pgfsetfillcolor{currentfill}%
\pgfsetlinewidth{0.803000pt}%
\definecolor{currentstroke}{rgb}{0.000000,0.000000,0.000000}%
\pgfsetstrokecolor{currentstroke}%
\pgfsetdash{}{0pt}%
\pgfsys@defobject{currentmarker}{\pgfqpoint{0.000000in}{-0.048611in}}{\pgfqpoint{0.000000in}{0.000000in}}{%
\pgfpathmoveto{\pgfqpoint{0.000000in}{0.000000in}}%
\pgfpathlineto{\pgfqpoint{0.000000in}{-0.048611in}}%
\pgfusepath{stroke,fill}%
}%
\begin{pgfscope}%
\pgfsys@transformshift{1.479098in}{0.395972in}%
\pgfsys@useobject{currentmarker}{}%
\end{pgfscope}%
\end{pgfscope}%
\begin{pgfscope}%
\definecolor{textcolor}{rgb}{0.000000,0.000000,0.000000}%
\pgfsetstrokecolor{textcolor}%
\pgfsetfillcolor{textcolor}%
\pgftext[x=1.479098in,y=0.298750in,,top]{\color{textcolor}{\rmfamily\fontsize{10.000000}{12.000000}\selectfont\catcode`\^=\active\def^{\ifmmode\sp\else\^{}\fi}\catcode`\%=\active\def%{\%}CPLEX}}%
\end{pgfscope}%
\begin{pgfscope}%
\pgfsetbuttcap%
\pgfsetroundjoin%
\definecolor{currentfill}{rgb}{0.000000,0.000000,0.000000}%
\pgfsetfillcolor{currentfill}%
\pgfsetlinewidth{0.803000pt}%
\definecolor{currentstroke}{rgb}{0.000000,0.000000,0.000000}%
\pgfsetstrokecolor{currentstroke}%
\pgfsetdash{}{0pt}%
\pgfsys@defobject{currentmarker}{\pgfqpoint{0.000000in}{-0.048611in}}{\pgfqpoint{0.000000in}{0.000000in}}{%
\pgfpathmoveto{\pgfqpoint{0.000000in}{0.000000in}}%
\pgfpathlineto{\pgfqpoint{0.000000in}{-0.048611in}}%
\pgfusepath{stroke,fill}%
}%
\begin{pgfscope}%
\pgfsys@transformshift{2.598764in}{0.395972in}%
\pgfsys@useobject{currentmarker}{}%
\end{pgfscope}%
\end{pgfscope}%
\begin{pgfscope}%
\definecolor{textcolor}{rgb}{0.000000,0.000000,0.000000}%
\pgfsetstrokecolor{textcolor}%
\pgfsetfillcolor{textcolor}%
\pgftext[x=2.598764in,y=0.298750in,,top]{\color{textcolor}{\rmfamily\fontsize{10.000000}{12.000000}\selectfont\catcode`\^=\active\def^{\ifmmode\sp\else\^{}\fi}\catcode`\%=\active\def%{\%}PACS, $\rho=.1$}}%
\end{pgfscope}%
\begin{pgfscope}%
\pgfsetbuttcap%
\pgfsetroundjoin%
\definecolor{currentfill}{rgb}{0.000000,0.000000,0.000000}%
\pgfsetfillcolor{currentfill}%
\pgfsetlinewidth{0.803000pt}%
\definecolor{currentstroke}{rgb}{0.000000,0.000000,0.000000}%
\pgfsetstrokecolor{currentstroke}%
\pgfsetdash{}{0pt}%
\pgfsys@defobject{currentmarker}{\pgfqpoint{0.000000in}{-0.048611in}}{\pgfqpoint{0.000000in}{0.000000in}}{%
\pgfpathmoveto{\pgfqpoint{0.000000in}{0.000000in}}%
\pgfpathlineto{\pgfqpoint{0.000000in}{-0.048611in}}%
\pgfusepath{stroke,fill}%
}%
\begin{pgfscope}%
\pgfsys@transformshift{3.718431in}{0.395972in}%
\pgfsys@useobject{currentmarker}{}%
\end{pgfscope}%
\end{pgfscope}%
\begin{pgfscope}%
\definecolor{textcolor}{rgb}{0.000000,0.000000,0.000000}%
\pgfsetstrokecolor{textcolor}%
\pgfsetfillcolor{textcolor}%
\pgftext[x=3.718431in,y=0.298750in,,top]{\color{textcolor}{\rmfamily\fontsize{10.000000}{12.000000}\selectfont\catcode`\^=\active\def^{\ifmmode\sp\else\^{}\fi}\catcode`\%=\active\def%{\%}PACS, $\rho=.25$}}%
\end{pgfscope}%
\begin{pgfscope}%
\pgfsetbuttcap%
\pgfsetroundjoin%
\definecolor{currentfill}{rgb}{0.000000,0.000000,0.000000}%
\pgfsetfillcolor{currentfill}%
\pgfsetlinewidth{0.803000pt}%
\definecolor{currentstroke}{rgb}{0.000000,0.000000,0.000000}%
\pgfsetstrokecolor{currentstroke}%
\pgfsetdash{}{0pt}%
\pgfsys@defobject{currentmarker}{\pgfqpoint{0.000000in}{-0.048611in}}{\pgfqpoint{0.000000in}{0.000000in}}{%
\pgfpathmoveto{\pgfqpoint{0.000000in}{0.000000in}}%
\pgfpathlineto{\pgfqpoint{0.000000in}{-0.048611in}}%
\pgfusepath{stroke,fill}%
}%
\begin{pgfscope}%
\pgfsys@transformshift{4.838097in}{0.395972in}%
\pgfsys@useobject{currentmarker}{}%
\end{pgfscope}%
\end{pgfscope}%
\begin{pgfscope}%
\definecolor{textcolor}{rgb}{0.000000,0.000000,0.000000}%
\pgfsetstrokecolor{textcolor}%
\pgfsetfillcolor{textcolor}%
\pgftext[x=4.838097in,y=0.298750in,,top]{\color{textcolor}{\rmfamily\fontsize{10.000000}{12.000000}\selectfont\catcode`\^=\active\def^{\ifmmode\sp\else\^{}\fi}\catcode`\%=\active\def%{\%}PACS, $\rho=.5$}}%
\end{pgfscope}%
\begin{pgfscope}%
\pgfsetbuttcap%
\pgfsetroundjoin%
\definecolor{currentfill}{rgb}{0.000000,0.000000,0.000000}%
\pgfsetfillcolor{currentfill}%
\pgfsetlinewidth{0.803000pt}%
\definecolor{currentstroke}{rgb}{0.000000,0.000000,0.000000}%
\pgfsetstrokecolor{currentstroke}%
\pgfsetdash{}{0pt}%
\pgfsys@defobject{currentmarker}{\pgfqpoint{0.000000in}{-0.048611in}}{\pgfqpoint{0.000000in}{0.000000in}}{%
\pgfpathmoveto{\pgfqpoint{0.000000in}{0.000000in}}%
\pgfpathlineto{\pgfqpoint{0.000000in}{-0.048611in}}%
\pgfusepath{stroke,fill}%
}%
\begin{pgfscope}%
\pgfsys@transformshift{5.957764in}{0.395972in}%
\pgfsys@useobject{currentmarker}{}%
\end{pgfscope}%
\end{pgfscope}%
\begin{pgfscope}%
\definecolor{textcolor}{rgb}{0.000000,0.000000,0.000000}%
\pgfsetstrokecolor{textcolor}%
\pgfsetfillcolor{textcolor}%
\pgftext[x=5.957764in,y=0.298750in,,top]{\color{textcolor}{\rmfamily\fontsize{10.000000}{12.000000}\selectfont\catcode`\^=\active\def^{\ifmmode\sp\else\^{}\fi}\catcode`\%=\active\def%{\%}PACS, $\rho=.75$}}%
\end{pgfscope}%
\begin{pgfscope}%
\pgfsetbuttcap%
\pgfsetroundjoin%
\definecolor{currentfill}{rgb}{0.000000,0.000000,0.000000}%
\pgfsetfillcolor{currentfill}%
\pgfsetlinewidth{0.803000pt}%
\definecolor{currentstroke}{rgb}{0.000000,0.000000,0.000000}%
\pgfsetstrokecolor{currentstroke}%
\pgfsetdash{}{0pt}%
\pgfsys@defobject{currentmarker}{\pgfqpoint{0.000000in}{-0.048611in}}{\pgfqpoint{0.000000in}{0.000000in}}{%
\pgfpathmoveto{\pgfqpoint{0.000000in}{0.000000in}}%
\pgfpathlineto{\pgfqpoint{0.000000in}{-0.048611in}}%
\pgfusepath{stroke,fill}%
}%
\begin{pgfscope}%
\pgfsys@transformshift{7.077430in}{0.395972in}%
\pgfsys@useobject{currentmarker}{}%
\end{pgfscope}%
\end{pgfscope}%
\begin{pgfscope}%
\definecolor{textcolor}{rgb}{0.000000,0.000000,0.000000}%
\pgfsetstrokecolor{textcolor}%
\pgfsetfillcolor{textcolor}%
\pgftext[x=7.077430in,y=0.298750in,,top]{\color{textcolor}{\rmfamily\fontsize{10.000000}{12.000000}\selectfont\catcode`\^=\active\def^{\ifmmode\sp\else\^{}\fi}\catcode`\%=\active\def%{\%}PACS, $\rho=.9$}}%
\end{pgfscope}%
\begin{pgfscope}%
\pgfpathrectangle{\pgfqpoint{0.706528in}{0.395972in}}{\pgfqpoint{7.143472in}{4.454028in}}%
\pgfusepath{clip}%
\pgfsetbuttcap%
\pgfsetroundjoin%
\pgfsetlinewidth{0.803000pt}%
\definecolor{currentstroke}{rgb}{0.690196,0.690196,0.690196}%
\pgfsetstrokecolor{currentstroke}%
\pgfsetstrokeopacity{0.500000}%
\pgfsetdash{{2.960000pt}{1.280000pt}}{0.000000pt}%
\pgfpathmoveto{\pgfqpoint{0.706528in}{0.395972in}}%
\pgfpathlineto{\pgfqpoint{7.850000in}{0.395972in}}%
\pgfusepath{stroke}%
\end{pgfscope}%
\begin{pgfscope}%
\pgfsetbuttcap%
\pgfsetroundjoin%
\definecolor{currentfill}{rgb}{0.000000,0.000000,0.000000}%
\pgfsetfillcolor{currentfill}%
\pgfsetlinewidth{0.803000pt}%
\definecolor{currentstroke}{rgb}{0.000000,0.000000,0.000000}%
\pgfsetstrokecolor{currentstroke}%
\pgfsetdash{}{0pt}%
\pgfsys@defobject{currentmarker}{\pgfqpoint{-0.048611in}{0.000000in}}{\pgfqpoint{-0.000000in}{0.000000in}}{%
\pgfpathmoveto{\pgfqpoint{-0.000000in}{0.000000in}}%
\pgfpathlineto{\pgfqpoint{-0.048611in}{0.000000in}}%
\pgfusepath{stroke,fill}%
}%
\begin{pgfscope}%
\pgfsys@transformshift{0.706528in}{0.395972in}%
\pgfsys@useobject{currentmarker}{}%
\end{pgfscope}%
\end{pgfscope}%
\begin{pgfscope}%
\definecolor{textcolor}{rgb}{0.000000,0.000000,0.000000}%
\pgfsetstrokecolor{textcolor}%
\pgfsetfillcolor{textcolor}%
\pgftext[x=0.520940in, y=0.343211in, left, base]{\color{textcolor}{\rmfamily\fontsize{10.000000}{12.000000}\selectfont\catcode`\^=\active\def^{\ifmmode\sp\else\^{}\fi}\catcode`\%=\active\def%{\%}0}}%
\end{pgfscope}%
\begin{pgfscope}%
\pgfpathrectangle{\pgfqpoint{0.706528in}{0.395972in}}{\pgfqpoint{7.143472in}{4.454028in}}%
\pgfusepath{clip}%
\pgfsetbuttcap%
\pgfsetroundjoin%
\pgfsetlinewidth{0.803000pt}%
\definecolor{currentstroke}{rgb}{0.690196,0.690196,0.690196}%
\pgfsetstrokecolor{currentstroke}%
\pgfsetstrokeopacity{0.500000}%
\pgfsetdash{{2.960000pt}{1.280000pt}}{0.000000pt}%
\pgfpathmoveto{\pgfqpoint{0.706528in}{1.367414in}}%
\pgfpathlineto{\pgfqpoint{7.850000in}{1.367414in}}%
\pgfusepath{stroke}%
\end{pgfscope}%
\begin{pgfscope}%
\pgfsetbuttcap%
\pgfsetroundjoin%
\definecolor{currentfill}{rgb}{0.000000,0.000000,0.000000}%
\pgfsetfillcolor{currentfill}%
\pgfsetlinewidth{0.803000pt}%
\definecolor{currentstroke}{rgb}{0.000000,0.000000,0.000000}%
\pgfsetstrokecolor{currentstroke}%
\pgfsetdash{}{0pt}%
\pgfsys@defobject{currentmarker}{\pgfqpoint{-0.048611in}{0.000000in}}{\pgfqpoint{-0.000000in}{0.000000in}}{%
\pgfpathmoveto{\pgfqpoint{-0.000000in}{0.000000in}}%
\pgfpathlineto{\pgfqpoint{-0.048611in}{0.000000in}}%
\pgfusepath{stroke,fill}%
}%
\begin{pgfscope}%
\pgfsys@transformshift{0.706528in}{1.367414in}%
\pgfsys@useobject{currentmarker}{}%
\end{pgfscope}%
\end{pgfscope}%
\begin{pgfscope}%
\definecolor{textcolor}{rgb}{0.000000,0.000000,0.000000}%
\pgfsetstrokecolor{textcolor}%
\pgfsetfillcolor{textcolor}%
\pgftext[x=0.432575in, y=1.314652in, left, base]{\color{textcolor}{\rmfamily\fontsize{10.000000}{12.000000}\selectfont\catcode`\^=\active\def^{\ifmmode\sp\else\^{}\fi}\catcode`\%=\active\def%{\%}50}}%
\end{pgfscope}%
\begin{pgfscope}%
\pgfpathrectangle{\pgfqpoint{0.706528in}{0.395972in}}{\pgfqpoint{7.143472in}{4.454028in}}%
\pgfusepath{clip}%
\pgfsetbuttcap%
\pgfsetroundjoin%
\pgfsetlinewidth{0.803000pt}%
\definecolor{currentstroke}{rgb}{0.690196,0.690196,0.690196}%
\pgfsetstrokecolor{currentstroke}%
\pgfsetstrokeopacity{0.500000}%
\pgfsetdash{{2.960000pt}{1.280000pt}}{0.000000pt}%
\pgfpathmoveto{\pgfqpoint{0.706528in}{2.338855in}}%
\pgfpathlineto{\pgfqpoint{7.850000in}{2.338855in}}%
\pgfusepath{stroke}%
\end{pgfscope}%
\begin{pgfscope}%
\pgfsetbuttcap%
\pgfsetroundjoin%
\definecolor{currentfill}{rgb}{0.000000,0.000000,0.000000}%
\pgfsetfillcolor{currentfill}%
\pgfsetlinewidth{0.803000pt}%
\definecolor{currentstroke}{rgb}{0.000000,0.000000,0.000000}%
\pgfsetstrokecolor{currentstroke}%
\pgfsetdash{}{0pt}%
\pgfsys@defobject{currentmarker}{\pgfqpoint{-0.048611in}{0.000000in}}{\pgfqpoint{-0.000000in}{0.000000in}}{%
\pgfpathmoveto{\pgfqpoint{-0.000000in}{0.000000in}}%
\pgfpathlineto{\pgfqpoint{-0.048611in}{0.000000in}}%
\pgfusepath{stroke,fill}%
}%
\begin{pgfscope}%
\pgfsys@transformshift{0.706528in}{2.338855in}%
\pgfsys@useobject{currentmarker}{}%
\end{pgfscope}%
\end{pgfscope}%
\begin{pgfscope}%
\definecolor{textcolor}{rgb}{0.000000,0.000000,0.000000}%
\pgfsetstrokecolor{textcolor}%
\pgfsetfillcolor{textcolor}%
\pgftext[x=0.344210in, y=2.286093in, left, base]{\color{textcolor}{\rmfamily\fontsize{10.000000}{12.000000}\selectfont\catcode`\^=\active\def^{\ifmmode\sp\else\^{}\fi}\catcode`\%=\active\def%{\%}100}}%
\end{pgfscope}%
\begin{pgfscope}%
\pgfpathrectangle{\pgfqpoint{0.706528in}{0.395972in}}{\pgfqpoint{7.143472in}{4.454028in}}%
\pgfusepath{clip}%
\pgfsetbuttcap%
\pgfsetroundjoin%
\pgfsetlinewidth{0.803000pt}%
\definecolor{currentstroke}{rgb}{0.690196,0.690196,0.690196}%
\pgfsetstrokecolor{currentstroke}%
\pgfsetstrokeopacity{0.500000}%
\pgfsetdash{{2.960000pt}{1.280000pt}}{0.000000pt}%
\pgfpathmoveto{\pgfqpoint{0.706528in}{3.310296in}}%
\pgfpathlineto{\pgfqpoint{7.850000in}{3.310296in}}%
\pgfusepath{stroke}%
\end{pgfscope}%
\begin{pgfscope}%
\pgfsetbuttcap%
\pgfsetroundjoin%
\definecolor{currentfill}{rgb}{0.000000,0.000000,0.000000}%
\pgfsetfillcolor{currentfill}%
\pgfsetlinewidth{0.803000pt}%
\definecolor{currentstroke}{rgb}{0.000000,0.000000,0.000000}%
\pgfsetstrokecolor{currentstroke}%
\pgfsetdash{}{0pt}%
\pgfsys@defobject{currentmarker}{\pgfqpoint{-0.048611in}{0.000000in}}{\pgfqpoint{-0.000000in}{0.000000in}}{%
\pgfpathmoveto{\pgfqpoint{-0.000000in}{0.000000in}}%
\pgfpathlineto{\pgfqpoint{-0.048611in}{0.000000in}}%
\pgfusepath{stroke,fill}%
}%
\begin{pgfscope}%
\pgfsys@transformshift{0.706528in}{3.310296in}%
\pgfsys@useobject{currentmarker}{}%
\end{pgfscope}%
\end{pgfscope}%
\begin{pgfscope}%
\definecolor{textcolor}{rgb}{0.000000,0.000000,0.000000}%
\pgfsetstrokecolor{textcolor}%
\pgfsetfillcolor{textcolor}%
\pgftext[x=0.344210in, y=3.257535in, left, base]{\color{textcolor}{\rmfamily\fontsize{10.000000}{12.000000}\selectfont\catcode`\^=\active\def^{\ifmmode\sp\else\^{}\fi}\catcode`\%=\active\def%{\%}150}}%
\end{pgfscope}%
\begin{pgfscope}%
\pgfpathrectangle{\pgfqpoint{0.706528in}{0.395972in}}{\pgfqpoint{7.143472in}{4.454028in}}%
\pgfusepath{clip}%
\pgfsetbuttcap%
\pgfsetroundjoin%
\pgfsetlinewidth{0.803000pt}%
\definecolor{currentstroke}{rgb}{0.690196,0.690196,0.690196}%
\pgfsetstrokecolor{currentstroke}%
\pgfsetstrokeopacity{0.500000}%
\pgfsetdash{{2.960000pt}{1.280000pt}}{0.000000pt}%
\pgfpathmoveto{\pgfqpoint{0.706528in}{4.281738in}}%
\pgfpathlineto{\pgfqpoint{7.850000in}{4.281738in}}%
\pgfusepath{stroke}%
\end{pgfscope}%
\begin{pgfscope}%
\pgfsetbuttcap%
\pgfsetroundjoin%
\definecolor{currentfill}{rgb}{0.000000,0.000000,0.000000}%
\pgfsetfillcolor{currentfill}%
\pgfsetlinewidth{0.803000pt}%
\definecolor{currentstroke}{rgb}{0.000000,0.000000,0.000000}%
\pgfsetstrokecolor{currentstroke}%
\pgfsetdash{}{0pt}%
\pgfsys@defobject{currentmarker}{\pgfqpoint{-0.048611in}{0.000000in}}{\pgfqpoint{-0.000000in}{0.000000in}}{%
\pgfpathmoveto{\pgfqpoint{-0.000000in}{0.000000in}}%
\pgfpathlineto{\pgfqpoint{-0.048611in}{0.000000in}}%
\pgfusepath{stroke,fill}%
}%
\begin{pgfscope}%
\pgfsys@transformshift{0.706528in}{4.281738in}%
\pgfsys@useobject{currentmarker}{}%
\end{pgfscope}%
\end{pgfscope}%
\begin{pgfscope}%
\definecolor{textcolor}{rgb}{0.000000,0.000000,0.000000}%
\pgfsetstrokecolor{textcolor}%
\pgfsetfillcolor{textcolor}%
\pgftext[x=0.344210in, y=4.228976in, left, base]{\color{textcolor}{\rmfamily\fontsize{10.000000}{12.000000}\selectfont\catcode`\^=\active\def^{\ifmmode\sp\else\^{}\fi}\catcode`\%=\active\def%{\%}200}}%
\end{pgfscope}%
\begin{pgfscope}%
\definecolor{textcolor}{rgb}{0.000000,0.000000,0.000000}%
\pgfsetstrokecolor{textcolor}%
\pgfsetfillcolor{textcolor}%
\pgftext[x=0.288654in,y=2.622986in,,bottom,rotate=90.000000]{\color{textcolor}{\rmfamily\fontsize{10.000000}{12.000000}\selectfont\catcode`\^=\active\def^{\ifmmode\sp\else\^{}\fi}\catcode`\%=\active\def%{\%}Integral Value}}%
\end{pgfscope}%
\begin{pgfscope}%
\pgfsetrectcap%
\pgfsetmiterjoin%
\pgfsetlinewidth{0.803000pt}%
\definecolor{currentstroke}{rgb}{0.000000,0.000000,0.000000}%
\pgfsetstrokecolor{currentstroke}%
\pgfsetdash{}{0pt}%
\pgfpathmoveto{\pgfqpoint{0.706528in}{0.395972in}}%
\pgfpathlineto{\pgfqpoint{0.706528in}{4.850000in}}%
\pgfusepath{stroke}%
\end{pgfscope}%
\begin{pgfscope}%
\pgfsetrectcap%
\pgfsetmiterjoin%
\pgfsetlinewidth{0.803000pt}%
\definecolor{currentstroke}{rgb}{0.000000,0.000000,0.000000}%
\pgfsetstrokecolor{currentstroke}%
\pgfsetdash{}{0pt}%
\pgfpathmoveto{\pgfqpoint{7.850000in}{0.395972in}}%
\pgfpathlineto{\pgfqpoint{7.850000in}{4.850000in}}%
\pgfusepath{stroke}%
\end{pgfscope}%
\begin{pgfscope}%
\pgfsetrectcap%
\pgfsetmiterjoin%
\pgfsetlinewidth{0.803000pt}%
\definecolor{currentstroke}{rgb}{0.000000,0.000000,0.000000}%
\pgfsetstrokecolor{currentstroke}%
\pgfsetdash{}{0pt}%
\pgfpathmoveto{\pgfqpoint{0.706528in}{0.395972in}}%
\pgfpathlineto{\pgfqpoint{7.850000in}{0.395972in}}%
\pgfusepath{stroke}%
\end{pgfscope}%
\begin{pgfscope}%
\pgfsetrectcap%
\pgfsetmiterjoin%
\pgfsetlinewidth{0.803000pt}%
\definecolor{currentstroke}{rgb}{0.000000,0.000000,0.000000}%
\pgfsetstrokecolor{currentstroke}%
\pgfsetdash{}{0pt}%
\pgfpathmoveto{\pgfqpoint{0.706528in}{4.850000in}}%
\pgfpathlineto{\pgfqpoint{7.850000in}{4.850000in}}%
\pgfusepath{stroke}%
\end{pgfscope}%
\end{pgfpicture}%
\makeatother%
\endgroup%
}
    \end{minipage}
    \caption{Success Rate vs. Computation Time for Fixed $\rho$ Initialization Test}
    \label{fig:PACS_STD_SuccRate}
\end{figure}


\begin{figure}[thpb]
    \centering
    \begin{minipage}{0.6\columnwidth}
        \centering
        \resizebox{\linewidth}{!}{%% Creator: Matplotlib, PGF backend
%%
%% To include the figure in your LaTeX document, write
%%   \input{<filename>.pgf}
%%
%% Make sure the required packages are loaded in your preamble
%%   \usepackage{pgf}
%%
%% Also ensure that all the required font packages are loaded; for instance,
%% the lmodern package is sometimes necessary when using math font.
%%   \usepackage{lmodern}
%%
%% Figures using additional raster images can only be included by \input if
%% they are in the same directory as the main LaTeX file. For loading figures
%% from other directories you can use the `import` package
%%   \usepackage{import}
%%
%% and then include the figures with
%%   \import{<path to file>}{<filename>.pgf}
%%
%% Matplotlib used the following preamble
%%   \def\mathdefault#1{#1}
%%   \everymath=\expandafter{\the\everymath\displaystyle}
%%   \IfFileExists{scrextend.sty}{
%%     \usepackage[fontsize=10.000000pt]{scrextend}
%%   }{
%%     \renewcommand{\normalsize}{\fontsize{10.000000}{12.000000}\selectfont}
%%     \normalsize
%%   }
%%   
%%   \ifdefined\pdftexversion\else  % non-pdftex case.
%%     \usepackage{fontspec}
%%     \setmainfont{DejaVuSerif.ttf}[Path=\detokenize{/home/bisca/.global/lib/python3.12/site-packages/matplotlib/mpl-data/fonts/ttf/}]
%%     \setsansfont{DejaVuSans.ttf}[Path=\detokenize{/home/bisca/.global/lib/python3.12/site-packages/matplotlib/mpl-data/fonts/ttf/}]
%%     \setmonofont{DejaVuSansMono.ttf}[Path=\detokenize{/home/bisca/.global/lib/python3.12/site-packages/matplotlib/mpl-data/fonts/ttf/}]
%%   \fi
%%   \makeatletter\@ifpackageloaded{underscore}{}{\usepackage[strings]{underscore}}\makeatother
%%
\begingroup%
\makeatletter%
\begin{pgfpicture}%
\pgfpathrectangle{\pgfpointorigin}{\pgfqpoint{12.000000in}{8.000000in}}%
\pgfusepath{use as bounding box, clip}%
\begin{pgfscope}%
\pgfsetbuttcap%
\pgfsetmiterjoin%
\definecolor{currentfill}{rgb}{1.000000,1.000000,1.000000}%
\pgfsetfillcolor{currentfill}%
\pgfsetlinewidth{0.000000pt}%
\definecolor{currentstroke}{rgb}{1.000000,1.000000,1.000000}%
\pgfsetstrokecolor{currentstroke}%
\pgfsetdash{}{0pt}%
\pgfpathmoveto{\pgfqpoint{0.000000in}{0.000000in}}%
\pgfpathlineto{\pgfqpoint{12.000000in}{0.000000in}}%
\pgfpathlineto{\pgfqpoint{12.000000in}{8.000000in}}%
\pgfpathlineto{\pgfqpoint{0.000000in}{8.000000in}}%
\pgfpathlineto{\pgfqpoint{0.000000in}{0.000000in}}%
\pgfpathclose%
\pgfusepath{fill}%
\end{pgfscope}%
\begin{pgfscope}%
\pgfsetbuttcap%
\pgfsetmiterjoin%
\definecolor{currentfill}{rgb}{1.000000,1.000000,1.000000}%
\pgfsetfillcolor{currentfill}%
\pgfsetlinewidth{0.000000pt}%
\definecolor{currentstroke}{rgb}{0.000000,0.000000,0.000000}%
\pgfsetstrokecolor{currentstroke}%
\pgfsetstrokeopacity{0.000000}%
\pgfsetdash{}{0pt}%
\pgfpathmoveto{\pgfqpoint{0.661528in}{0.586684in}}%
\pgfpathlineto{\pgfqpoint{11.718125in}{0.586684in}}%
\pgfpathlineto{\pgfqpoint{11.718125in}{7.850000in}}%
\pgfpathlineto{\pgfqpoint{0.661528in}{7.850000in}}%
\pgfpathlineto{\pgfqpoint{0.661528in}{0.586684in}}%
\pgfpathclose%
\pgfusepath{fill}%
\end{pgfscope}%
\begin{pgfscope}%
\pgfpathrectangle{\pgfqpoint{0.661528in}{0.586684in}}{\pgfqpoint{11.056597in}{7.263316in}}%
\pgfusepath{clip}%
\pgfsetbuttcap%
\pgfsetroundjoin%
\pgfsetlinewidth{0.803000pt}%
\definecolor{currentstroke}{rgb}{0.690196,0.690196,0.690196}%
\pgfsetstrokecolor{currentstroke}%
\pgfsetstrokeopacity{0.500000}%
\pgfsetdash{{2.960000pt}{1.280000pt}}{0.000000pt}%
\pgfpathmoveto{\pgfqpoint{2.783501in}{0.586684in}}%
\pgfpathlineto{\pgfqpoint{2.783501in}{7.850000in}}%
\pgfusepath{stroke}%
\end{pgfscope}%
\begin{pgfscope}%
\pgfsetbuttcap%
\pgfsetroundjoin%
\definecolor{currentfill}{rgb}{0.000000,0.000000,0.000000}%
\pgfsetfillcolor{currentfill}%
\pgfsetlinewidth{0.803000pt}%
\definecolor{currentstroke}{rgb}{0.000000,0.000000,0.000000}%
\pgfsetstrokecolor{currentstroke}%
\pgfsetdash{}{0pt}%
\pgfsys@defobject{currentmarker}{\pgfqpoint{0.000000in}{-0.048611in}}{\pgfqpoint{0.000000in}{0.000000in}}{%
\pgfpathmoveto{\pgfqpoint{0.000000in}{0.000000in}}%
\pgfpathlineto{\pgfqpoint{0.000000in}{-0.048611in}}%
\pgfusepath{stroke,fill}%
}%
\begin{pgfscope}%
\pgfsys@transformshift{2.783501in}{0.586684in}%
\pgfsys@useobject{currentmarker}{}%
\end{pgfscope}%
\end{pgfscope}%
\begin{pgfscope}%
\definecolor{textcolor}{rgb}{0.000000,0.000000,0.000000}%
\pgfsetstrokecolor{textcolor}%
\pgfsetfillcolor{textcolor}%
\pgftext[x=2.783501in,y=0.489462in,,top]{\color{textcolor}{\rmfamily\fontsize{10.000000}{12.000000}\selectfont\catcode`\^=\active\def^{\ifmmode\sp\else\^{}\fi}\catcode`\%=\active\def%{\%}20}}%
\end{pgfscope}%
\begin{pgfscope}%
\pgfpathrectangle{\pgfqpoint{0.661528in}{0.586684in}}{\pgfqpoint{11.056597in}{7.263316in}}%
\pgfusepath{clip}%
\pgfsetbuttcap%
\pgfsetroundjoin%
\pgfsetlinewidth{0.803000pt}%
\definecolor{currentstroke}{rgb}{0.690196,0.690196,0.690196}%
\pgfsetstrokecolor{currentstroke}%
\pgfsetstrokeopacity{0.500000}%
\pgfsetdash{{2.960000pt}{1.280000pt}}{0.000000pt}%
\pgfpathmoveto{\pgfqpoint{5.017157in}{0.586684in}}%
\pgfpathlineto{\pgfqpoint{5.017157in}{7.850000in}}%
\pgfusepath{stroke}%
\end{pgfscope}%
\begin{pgfscope}%
\pgfsetbuttcap%
\pgfsetroundjoin%
\definecolor{currentfill}{rgb}{0.000000,0.000000,0.000000}%
\pgfsetfillcolor{currentfill}%
\pgfsetlinewidth{0.803000pt}%
\definecolor{currentstroke}{rgb}{0.000000,0.000000,0.000000}%
\pgfsetstrokecolor{currentstroke}%
\pgfsetdash{}{0pt}%
\pgfsys@defobject{currentmarker}{\pgfqpoint{0.000000in}{-0.048611in}}{\pgfqpoint{0.000000in}{0.000000in}}{%
\pgfpathmoveto{\pgfqpoint{0.000000in}{0.000000in}}%
\pgfpathlineto{\pgfqpoint{0.000000in}{-0.048611in}}%
\pgfusepath{stroke,fill}%
}%
\begin{pgfscope}%
\pgfsys@transformshift{5.017157in}{0.586684in}%
\pgfsys@useobject{currentmarker}{}%
\end{pgfscope}%
\end{pgfscope}%
\begin{pgfscope}%
\definecolor{textcolor}{rgb}{0.000000,0.000000,0.000000}%
\pgfsetstrokecolor{textcolor}%
\pgfsetfillcolor{textcolor}%
\pgftext[x=5.017157in,y=0.489462in,,top]{\color{textcolor}{\rmfamily\fontsize{10.000000}{12.000000}\selectfont\catcode`\^=\active\def^{\ifmmode\sp\else\^{}\fi}\catcode`\%=\active\def%{\%}40}}%
\end{pgfscope}%
\begin{pgfscope}%
\pgfpathrectangle{\pgfqpoint{0.661528in}{0.586684in}}{\pgfqpoint{11.056597in}{7.263316in}}%
\pgfusepath{clip}%
\pgfsetbuttcap%
\pgfsetroundjoin%
\pgfsetlinewidth{0.803000pt}%
\definecolor{currentstroke}{rgb}{0.690196,0.690196,0.690196}%
\pgfsetstrokecolor{currentstroke}%
\pgfsetstrokeopacity{0.500000}%
\pgfsetdash{{2.960000pt}{1.280000pt}}{0.000000pt}%
\pgfpathmoveto{\pgfqpoint{7.250813in}{0.586684in}}%
\pgfpathlineto{\pgfqpoint{7.250813in}{7.850000in}}%
\pgfusepath{stroke}%
\end{pgfscope}%
\begin{pgfscope}%
\pgfsetbuttcap%
\pgfsetroundjoin%
\definecolor{currentfill}{rgb}{0.000000,0.000000,0.000000}%
\pgfsetfillcolor{currentfill}%
\pgfsetlinewidth{0.803000pt}%
\definecolor{currentstroke}{rgb}{0.000000,0.000000,0.000000}%
\pgfsetstrokecolor{currentstroke}%
\pgfsetdash{}{0pt}%
\pgfsys@defobject{currentmarker}{\pgfqpoint{0.000000in}{-0.048611in}}{\pgfqpoint{0.000000in}{0.000000in}}{%
\pgfpathmoveto{\pgfqpoint{0.000000in}{0.000000in}}%
\pgfpathlineto{\pgfqpoint{0.000000in}{-0.048611in}}%
\pgfusepath{stroke,fill}%
}%
\begin{pgfscope}%
\pgfsys@transformshift{7.250813in}{0.586684in}%
\pgfsys@useobject{currentmarker}{}%
\end{pgfscope}%
\end{pgfscope}%
\begin{pgfscope}%
\definecolor{textcolor}{rgb}{0.000000,0.000000,0.000000}%
\pgfsetstrokecolor{textcolor}%
\pgfsetfillcolor{textcolor}%
\pgftext[x=7.250813in,y=0.489462in,,top]{\color{textcolor}{\rmfamily\fontsize{10.000000}{12.000000}\selectfont\catcode`\^=\active\def^{\ifmmode\sp\else\^{}\fi}\catcode`\%=\active\def%{\%}60}}%
\end{pgfscope}%
\begin{pgfscope}%
\pgfpathrectangle{\pgfqpoint{0.661528in}{0.586684in}}{\pgfqpoint{11.056597in}{7.263316in}}%
\pgfusepath{clip}%
\pgfsetbuttcap%
\pgfsetroundjoin%
\pgfsetlinewidth{0.803000pt}%
\definecolor{currentstroke}{rgb}{0.690196,0.690196,0.690196}%
\pgfsetstrokecolor{currentstroke}%
\pgfsetstrokeopacity{0.500000}%
\pgfsetdash{{2.960000pt}{1.280000pt}}{0.000000pt}%
\pgfpathmoveto{\pgfqpoint{9.484469in}{0.586684in}}%
\pgfpathlineto{\pgfqpoint{9.484469in}{7.850000in}}%
\pgfusepath{stroke}%
\end{pgfscope}%
\begin{pgfscope}%
\pgfsetbuttcap%
\pgfsetroundjoin%
\definecolor{currentfill}{rgb}{0.000000,0.000000,0.000000}%
\pgfsetfillcolor{currentfill}%
\pgfsetlinewidth{0.803000pt}%
\definecolor{currentstroke}{rgb}{0.000000,0.000000,0.000000}%
\pgfsetstrokecolor{currentstroke}%
\pgfsetdash{}{0pt}%
\pgfsys@defobject{currentmarker}{\pgfqpoint{0.000000in}{-0.048611in}}{\pgfqpoint{0.000000in}{0.000000in}}{%
\pgfpathmoveto{\pgfqpoint{0.000000in}{0.000000in}}%
\pgfpathlineto{\pgfqpoint{0.000000in}{-0.048611in}}%
\pgfusepath{stroke,fill}%
}%
\begin{pgfscope}%
\pgfsys@transformshift{9.484469in}{0.586684in}%
\pgfsys@useobject{currentmarker}{}%
\end{pgfscope}%
\end{pgfscope}%
\begin{pgfscope}%
\definecolor{textcolor}{rgb}{0.000000,0.000000,0.000000}%
\pgfsetstrokecolor{textcolor}%
\pgfsetfillcolor{textcolor}%
\pgftext[x=9.484469in,y=0.489462in,,top]{\color{textcolor}{\rmfamily\fontsize{10.000000}{12.000000}\selectfont\catcode`\^=\active\def^{\ifmmode\sp\else\^{}\fi}\catcode`\%=\active\def%{\%}80}}%
\end{pgfscope}%
\begin{pgfscope}%
\pgfpathrectangle{\pgfqpoint{0.661528in}{0.586684in}}{\pgfqpoint{11.056597in}{7.263316in}}%
\pgfusepath{clip}%
\pgfsetbuttcap%
\pgfsetroundjoin%
\pgfsetlinewidth{0.803000pt}%
\definecolor{currentstroke}{rgb}{0.690196,0.690196,0.690196}%
\pgfsetstrokecolor{currentstroke}%
\pgfsetstrokeopacity{0.500000}%
\pgfsetdash{{2.960000pt}{1.280000pt}}{0.000000pt}%
\pgfpathmoveto{\pgfqpoint{11.718125in}{0.586684in}}%
\pgfpathlineto{\pgfqpoint{11.718125in}{7.850000in}}%
\pgfusepath{stroke}%
\end{pgfscope}%
\begin{pgfscope}%
\pgfsetbuttcap%
\pgfsetroundjoin%
\definecolor{currentfill}{rgb}{0.000000,0.000000,0.000000}%
\pgfsetfillcolor{currentfill}%
\pgfsetlinewidth{0.803000pt}%
\definecolor{currentstroke}{rgb}{0.000000,0.000000,0.000000}%
\pgfsetstrokecolor{currentstroke}%
\pgfsetdash{}{0pt}%
\pgfsys@defobject{currentmarker}{\pgfqpoint{0.000000in}{-0.048611in}}{\pgfqpoint{0.000000in}{0.000000in}}{%
\pgfpathmoveto{\pgfqpoint{0.000000in}{0.000000in}}%
\pgfpathlineto{\pgfqpoint{0.000000in}{-0.048611in}}%
\pgfusepath{stroke,fill}%
}%
\begin{pgfscope}%
\pgfsys@transformshift{11.718125in}{0.586684in}%
\pgfsys@useobject{currentmarker}{}%
\end{pgfscope}%
\end{pgfscope}%
\begin{pgfscope}%
\definecolor{textcolor}{rgb}{0.000000,0.000000,0.000000}%
\pgfsetstrokecolor{textcolor}%
\pgfsetfillcolor{textcolor}%
\pgftext[x=11.718125in,y=0.489462in,,top]{\color{textcolor}{\rmfamily\fontsize{10.000000}{12.000000}\selectfont\catcode`\^=\active\def^{\ifmmode\sp\else\^{}\fi}\catcode`\%=\active\def%{\%}100}}%
\end{pgfscope}%
\begin{pgfscope}%
\definecolor{textcolor}{rgb}{0.000000,0.000000,0.000000}%
\pgfsetstrokecolor{textcolor}%
\pgfsetfillcolor{textcolor}%
\pgftext[x=6.189826in,y=0.299493in,,top]{\color{textcolor}{\rmfamily\fontsize{10.000000}{12.000000}\selectfont\catcode`\^=\active\def^{\ifmmode\sp\else\^{}\fi}\catcode`\%=\active\def%{\%}MIP Gap (%)}}%
\end{pgfscope}%
\begin{pgfscope}%
\pgfpathrectangle{\pgfqpoint{0.661528in}{0.586684in}}{\pgfqpoint{11.056597in}{7.263316in}}%
\pgfusepath{clip}%
\pgfsetbuttcap%
\pgfsetroundjoin%
\pgfsetlinewidth{0.803000pt}%
\definecolor{currentstroke}{rgb}{0.690196,0.690196,0.690196}%
\pgfsetstrokecolor{currentstroke}%
\pgfsetstrokeopacity{0.500000}%
\pgfsetdash{{2.960000pt}{1.280000pt}}{0.000000pt}%
\pgfpathmoveto{\pgfqpoint{0.661528in}{0.916835in}}%
\pgfpathlineto{\pgfqpoint{11.718125in}{0.916835in}}%
\pgfusepath{stroke}%
\end{pgfscope}%
\begin{pgfscope}%
\pgfsetbuttcap%
\pgfsetroundjoin%
\definecolor{currentfill}{rgb}{0.000000,0.000000,0.000000}%
\pgfsetfillcolor{currentfill}%
\pgfsetlinewidth{0.803000pt}%
\definecolor{currentstroke}{rgb}{0.000000,0.000000,0.000000}%
\pgfsetstrokecolor{currentstroke}%
\pgfsetdash{}{0pt}%
\pgfsys@defobject{currentmarker}{\pgfqpoint{-0.048611in}{0.000000in}}{\pgfqpoint{-0.000000in}{0.000000in}}{%
\pgfpathmoveto{\pgfqpoint{-0.000000in}{0.000000in}}%
\pgfpathlineto{\pgfqpoint{-0.048611in}{0.000000in}}%
\pgfusepath{stroke,fill}%
}%
\begin{pgfscope}%
\pgfsys@transformshift{0.661528in}{0.916835in}%
\pgfsys@useobject{currentmarker}{}%
\end{pgfscope}%
\end{pgfscope}%
\begin{pgfscope}%
\definecolor{textcolor}{rgb}{0.000000,0.000000,0.000000}%
\pgfsetstrokecolor{textcolor}%
\pgfsetfillcolor{textcolor}%
\pgftext[x=0.343426in, y=0.864073in, left, base]{\color{textcolor}{\rmfamily\fontsize{10.000000}{12.000000}\selectfont\catcode`\^=\active\def^{\ifmmode\sp\else\^{}\fi}\catcode`\%=\active\def%{\%}0.0}}%
\end{pgfscope}%
\begin{pgfscope}%
\pgfpathrectangle{\pgfqpoint{0.661528in}{0.586684in}}{\pgfqpoint{11.056597in}{7.263316in}}%
\pgfusepath{clip}%
\pgfsetbuttcap%
\pgfsetroundjoin%
\pgfsetlinewidth{0.803000pt}%
\definecolor{currentstroke}{rgb}{0.690196,0.690196,0.690196}%
\pgfsetstrokecolor{currentstroke}%
\pgfsetstrokeopacity{0.500000}%
\pgfsetdash{{2.960000pt}{1.280000pt}}{0.000000pt}%
\pgfpathmoveto{\pgfqpoint{0.661528in}{2.237438in}}%
\pgfpathlineto{\pgfqpoint{11.718125in}{2.237438in}}%
\pgfusepath{stroke}%
\end{pgfscope}%
\begin{pgfscope}%
\pgfsetbuttcap%
\pgfsetroundjoin%
\definecolor{currentfill}{rgb}{0.000000,0.000000,0.000000}%
\pgfsetfillcolor{currentfill}%
\pgfsetlinewidth{0.803000pt}%
\definecolor{currentstroke}{rgb}{0.000000,0.000000,0.000000}%
\pgfsetstrokecolor{currentstroke}%
\pgfsetdash{}{0pt}%
\pgfsys@defobject{currentmarker}{\pgfqpoint{-0.048611in}{0.000000in}}{\pgfqpoint{-0.000000in}{0.000000in}}{%
\pgfpathmoveto{\pgfqpoint{-0.000000in}{0.000000in}}%
\pgfpathlineto{\pgfqpoint{-0.048611in}{0.000000in}}%
\pgfusepath{stroke,fill}%
}%
\begin{pgfscope}%
\pgfsys@transformshift{0.661528in}{2.237438in}%
\pgfsys@useobject{currentmarker}{}%
\end{pgfscope}%
\end{pgfscope}%
\begin{pgfscope}%
\definecolor{textcolor}{rgb}{0.000000,0.000000,0.000000}%
\pgfsetstrokecolor{textcolor}%
\pgfsetfillcolor{textcolor}%
\pgftext[x=0.343426in, y=2.184676in, left, base]{\color{textcolor}{\rmfamily\fontsize{10.000000}{12.000000}\selectfont\catcode`\^=\active\def^{\ifmmode\sp\else\^{}\fi}\catcode`\%=\active\def%{\%}0.2}}%
\end{pgfscope}%
\begin{pgfscope}%
\pgfpathrectangle{\pgfqpoint{0.661528in}{0.586684in}}{\pgfqpoint{11.056597in}{7.263316in}}%
\pgfusepath{clip}%
\pgfsetbuttcap%
\pgfsetroundjoin%
\pgfsetlinewidth{0.803000pt}%
\definecolor{currentstroke}{rgb}{0.690196,0.690196,0.690196}%
\pgfsetstrokecolor{currentstroke}%
\pgfsetstrokeopacity{0.500000}%
\pgfsetdash{{2.960000pt}{1.280000pt}}{0.000000pt}%
\pgfpathmoveto{\pgfqpoint{0.661528in}{3.558041in}}%
\pgfpathlineto{\pgfqpoint{11.718125in}{3.558041in}}%
\pgfusepath{stroke}%
\end{pgfscope}%
\begin{pgfscope}%
\pgfsetbuttcap%
\pgfsetroundjoin%
\definecolor{currentfill}{rgb}{0.000000,0.000000,0.000000}%
\pgfsetfillcolor{currentfill}%
\pgfsetlinewidth{0.803000pt}%
\definecolor{currentstroke}{rgb}{0.000000,0.000000,0.000000}%
\pgfsetstrokecolor{currentstroke}%
\pgfsetdash{}{0pt}%
\pgfsys@defobject{currentmarker}{\pgfqpoint{-0.048611in}{0.000000in}}{\pgfqpoint{-0.000000in}{0.000000in}}{%
\pgfpathmoveto{\pgfqpoint{-0.000000in}{0.000000in}}%
\pgfpathlineto{\pgfqpoint{-0.048611in}{0.000000in}}%
\pgfusepath{stroke,fill}%
}%
\begin{pgfscope}%
\pgfsys@transformshift{0.661528in}{3.558041in}%
\pgfsys@useobject{currentmarker}{}%
\end{pgfscope}%
\end{pgfscope}%
\begin{pgfscope}%
\definecolor{textcolor}{rgb}{0.000000,0.000000,0.000000}%
\pgfsetstrokecolor{textcolor}%
\pgfsetfillcolor{textcolor}%
\pgftext[x=0.343426in, y=3.505279in, left, base]{\color{textcolor}{\rmfamily\fontsize{10.000000}{12.000000}\selectfont\catcode`\^=\active\def^{\ifmmode\sp\else\^{}\fi}\catcode`\%=\active\def%{\%}0.4}}%
\end{pgfscope}%
\begin{pgfscope}%
\pgfpathrectangle{\pgfqpoint{0.661528in}{0.586684in}}{\pgfqpoint{11.056597in}{7.263316in}}%
\pgfusepath{clip}%
\pgfsetbuttcap%
\pgfsetroundjoin%
\pgfsetlinewidth{0.803000pt}%
\definecolor{currentstroke}{rgb}{0.690196,0.690196,0.690196}%
\pgfsetstrokecolor{currentstroke}%
\pgfsetstrokeopacity{0.500000}%
\pgfsetdash{{2.960000pt}{1.280000pt}}{0.000000pt}%
\pgfpathmoveto{\pgfqpoint{0.661528in}{4.878643in}}%
\pgfpathlineto{\pgfqpoint{11.718125in}{4.878643in}}%
\pgfusepath{stroke}%
\end{pgfscope}%
\begin{pgfscope}%
\pgfsetbuttcap%
\pgfsetroundjoin%
\definecolor{currentfill}{rgb}{0.000000,0.000000,0.000000}%
\pgfsetfillcolor{currentfill}%
\pgfsetlinewidth{0.803000pt}%
\definecolor{currentstroke}{rgb}{0.000000,0.000000,0.000000}%
\pgfsetstrokecolor{currentstroke}%
\pgfsetdash{}{0pt}%
\pgfsys@defobject{currentmarker}{\pgfqpoint{-0.048611in}{0.000000in}}{\pgfqpoint{-0.000000in}{0.000000in}}{%
\pgfpathmoveto{\pgfqpoint{-0.000000in}{0.000000in}}%
\pgfpathlineto{\pgfqpoint{-0.048611in}{0.000000in}}%
\pgfusepath{stroke,fill}%
}%
\begin{pgfscope}%
\pgfsys@transformshift{0.661528in}{4.878643in}%
\pgfsys@useobject{currentmarker}{}%
\end{pgfscope}%
\end{pgfscope}%
\begin{pgfscope}%
\definecolor{textcolor}{rgb}{0.000000,0.000000,0.000000}%
\pgfsetstrokecolor{textcolor}%
\pgfsetfillcolor{textcolor}%
\pgftext[x=0.343426in, y=4.825882in, left, base]{\color{textcolor}{\rmfamily\fontsize{10.000000}{12.000000}\selectfont\catcode`\^=\active\def^{\ifmmode\sp\else\^{}\fi}\catcode`\%=\active\def%{\%}0.6}}%
\end{pgfscope}%
\begin{pgfscope}%
\pgfpathrectangle{\pgfqpoint{0.661528in}{0.586684in}}{\pgfqpoint{11.056597in}{7.263316in}}%
\pgfusepath{clip}%
\pgfsetbuttcap%
\pgfsetroundjoin%
\pgfsetlinewidth{0.803000pt}%
\definecolor{currentstroke}{rgb}{0.690196,0.690196,0.690196}%
\pgfsetstrokecolor{currentstroke}%
\pgfsetstrokeopacity{0.500000}%
\pgfsetdash{{2.960000pt}{1.280000pt}}{0.000000pt}%
\pgfpathmoveto{\pgfqpoint{0.661528in}{6.199246in}}%
\pgfpathlineto{\pgfqpoint{11.718125in}{6.199246in}}%
\pgfusepath{stroke}%
\end{pgfscope}%
\begin{pgfscope}%
\pgfsetbuttcap%
\pgfsetroundjoin%
\definecolor{currentfill}{rgb}{0.000000,0.000000,0.000000}%
\pgfsetfillcolor{currentfill}%
\pgfsetlinewidth{0.803000pt}%
\definecolor{currentstroke}{rgb}{0.000000,0.000000,0.000000}%
\pgfsetstrokecolor{currentstroke}%
\pgfsetdash{}{0pt}%
\pgfsys@defobject{currentmarker}{\pgfqpoint{-0.048611in}{0.000000in}}{\pgfqpoint{-0.000000in}{0.000000in}}{%
\pgfpathmoveto{\pgfqpoint{-0.000000in}{0.000000in}}%
\pgfpathlineto{\pgfqpoint{-0.048611in}{0.000000in}}%
\pgfusepath{stroke,fill}%
}%
\begin{pgfscope}%
\pgfsys@transformshift{0.661528in}{6.199246in}%
\pgfsys@useobject{currentmarker}{}%
\end{pgfscope}%
\end{pgfscope}%
\begin{pgfscope}%
\definecolor{textcolor}{rgb}{0.000000,0.000000,0.000000}%
\pgfsetstrokecolor{textcolor}%
\pgfsetfillcolor{textcolor}%
\pgftext[x=0.343426in, y=6.146485in, left, base]{\color{textcolor}{\rmfamily\fontsize{10.000000}{12.000000}\selectfont\catcode`\^=\active\def^{\ifmmode\sp\else\^{}\fi}\catcode`\%=\active\def%{\%}0.8}}%
\end{pgfscope}%
\begin{pgfscope}%
\pgfpathrectangle{\pgfqpoint{0.661528in}{0.586684in}}{\pgfqpoint{11.056597in}{7.263316in}}%
\pgfusepath{clip}%
\pgfsetbuttcap%
\pgfsetroundjoin%
\pgfsetlinewidth{0.803000pt}%
\definecolor{currentstroke}{rgb}{0.690196,0.690196,0.690196}%
\pgfsetstrokecolor{currentstroke}%
\pgfsetstrokeopacity{0.500000}%
\pgfsetdash{{2.960000pt}{1.280000pt}}{0.000000pt}%
\pgfpathmoveto{\pgfqpoint{0.661528in}{7.519849in}}%
\pgfpathlineto{\pgfqpoint{11.718125in}{7.519849in}}%
\pgfusepath{stroke}%
\end{pgfscope}%
\begin{pgfscope}%
\pgfsetbuttcap%
\pgfsetroundjoin%
\definecolor{currentfill}{rgb}{0.000000,0.000000,0.000000}%
\pgfsetfillcolor{currentfill}%
\pgfsetlinewidth{0.803000pt}%
\definecolor{currentstroke}{rgb}{0.000000,0.000000,0.000000}%
\pgfsetstrokecolor{currentstroke}%
\pgfsetdash{}{0pt}%
\pgfsys@defobject{currentmarker}{\pgfqpoint{-0.048611in}{0.000000in}}{\pgfqpoint{-0.000000in}{0.000000in}}{%
\pgfpathmoveto{\pgfqpoint{-0.000000in}{0.000000in}}%
\pgfpathlineto{\pgfqpoint{-0.048611in}{0.000000in}}%
\pgfusepath{stroke,fill}%
}%
\begin{pgfscope}%
\pgfsys@transformshift{0.661528in}{7.519849in}%
\pgfsys@useobject{currentmarker}{}%
\end{pgfscope}%
\end{pgfscope}%
\begin{pgfscope}%
\definecolor{textcolor}{rgb}{0.000000,0.000000,0.000000}%
\pgfsetstrokecolor{textcolor}%
\pgfsetfillcolor{textcolor}%
\pgftext[x=0.343426in, y=7.467088in, left, base]{\color{textcolor}{\rmfamily\fontsize{10.000000}{12.000000}\selectfont\catcode`\^=\active\def^{\ifmmode\sp\else\^{}\fi}\catcode`\%=\active\def%{\%}1.0}}%
\end{pgfscope}%
\begin{pgfscope}%
\definecolor{textcolor}{rgb}{0.000000,0.000000,0.000000}%
\pgfsetstrokecolor{textcolor}%
\pgfsetfillcolor{textcolor}%
\pgftext[x=0.287871in,y=4.218342in,,bottom,rotate=90.000000]{\color{textcolor}{\rmfamily\fontsize{10.000000}{12.000000}\selectfont\catcode`\^=\active\def^{\ifmmode\sp\else\^{}\fi}\catcode`\%=\active\def%{\%}Success Rate}}%
\end{pgfscope}%
\begin{pgfscope}%
\pgfpathrectangle{\pgfqpoint{0.661528in}{0.586684in}}{\pgfqpoint{11.056597in}{7.263316in}}%
\pgfusepath{clip}%
\pgfsetrectcap%
\pgfsetroundjoin%
\pgfsetlinewidth{2.007500pt}%
\definecolor{currentstroke}{rgb}{0.121569,0.466667,0.705882}%
\pgfsetstrokecolor{currentstroke}%
\pgfsetdash{}{0pt}%
\pgfpathmoveto{\pgfqpoint{0.651528in}{0.916835in}}%
\pgfpathlineto{\pgfqpoint{0.661528in}{0.916835in}}%
\pgfpathlineto{\pgfqpoint{0.773211in}{0.916835in}}%
\pgfpathlineto{\pgfqpoint{0.884893in}{0.916835in}}%
\pgfpathlineto{\pgfqpoint{0.996576in}{0.976862in}}%
\pgfpathlineto{\pgfqpoint{1.108259in}{1.036890in}}%
\pgfpathlineto{\pgfqpoint{1.219942in}{1.096917in}}%
\pgfpathlineto{\pgfqpoint{1.331625in}{1.156944in}}%
\pgfpathlineto{\pgfqpoint{1.443307in}{1.156944in}}%
\pgfpathlineto{\pgfqpoint{1.554990in}{1.156944in}}%
\pgfpathlineto{\pgfqpoint{1.666673in}{1.156944in}}%
\pgfpathlineto{\pgfqpoint{1.778356in}{1.216972in}}%
\pgfpathlineto{\pgfqpoint{1.890039in}{1.216972in}}%
\pgfpathlineto{\pgfqpoint{2.001721in}{1.216972in}}%
\pgfpathlineto{\pgfqpoint{2.113404in}{1.216972in}}%
\pgfpathlineto{\pgfqpoint{2.225087in}{1.276999in}}%
\pgfpathlineto{\pgfqpoint{2.336770in}{1.337027in}}%
\pgfpathlineto{\pgfqpoint{2.448453in}{1.337027in}}%
\pgfpathlineto{\pgfqpoint{2.560135in}{1.457081in}}%
\pgfpathlineto{\pgfqpoint{2.671818in}{1.457081in}}%
\pgfpathlineto{\pgfqpoint{2.783501in}{1.517109in}}%
\pgfpathlineto{\pgfqpoint{2.895184in}{1.517109in}}%
\pgfpathlineto{\pgfqpoint{3.006867in}{1.517109in}}%
\pgfpathlineto{\pgfqpoint{3.118549in}{1.517109in}}%
\pgfpathlineto{\pgfqpoint{3.230232in}{1.517109in}}%
\pgfpathlineto{\pgfqpoint{3.341915in}{1.517109in}}%
\pgfpathlineto{\pgfqpoint{3.453598in}{1.517109in}}%
\pgfpathlineto{\pgfqpoint{3.565281in}{1.517109in}}%
\pgfpathlineto{\pgfqpoint{3.676963in}{1.577136in}}%
\pgfpathlineto{\pgfqpoint{3.788646in}{1.577136in}}%
\pgfpathlineto{\pgfqpoint{3.900329in}{1.577136in}}%
\pgfpathlineto{\pgfqpoint{4.012012in}{1.577136in}}%
\pgfpathlineto{\pgfqpoint{4.123695in}{1.697191in}}%
\pgfpathlineto{\pgfqpoint{4.235377in}{1.757218in}}%
\pgfpathlineto{\pgfqpoint{4.347060in}{1.817246in}}%
\pgfpathlineto{\pgfqpoint{4.458743in}{1.877273in}}%
\pgfpathlineto{\pgfqpoint{4.570426in}{1.877273in}}%
\pgfpathlineto{\pgfqpoint{4.682109in}{1.937301in}}%
\pgfpathlineto{\pgfqpoint{4.793791in}{1.937301in}}%
\pgfpathlineto{\pgfqpoint{4.905474in}{1.937301in}}%
\pgfpathlineto{\pgfqpoint{5.017157in}{1.997328in}}%
\pgfpathlineto{\pgfqpoint{5.128840in}{1.997328in}}%
\pgfpathlineto{\pgfqpoint{5.240523in}{1.997328in}}%
\pgfpathlineto{\pgfqpoint{5.352205in}{1.997328in}}%
\pgfpathlineto{\pgfqpoint{5.463888in}{1.997328in}}%
\pgfpathlineto{\pgfqpoint{5.575571in}{2.057355in}}%
\pgfpathlineto{\pgfqpoint{5.687254in}{2.117383in}}%
\pgfpathlineto{\pgfqpoint{5.798937in}{2.117383in}}%
\pgfpathlineto{\pgfqpoint{5.910619in}{2.117383in}}%
\pgfpathlineto{\pgfqpoint{6.022302in}{2.117383in}}%
\pgfpathlineto{\pgfqpoint{6.133985in}{2.117383in}}%
\pgfpathlineto{\pgfqpoint{6.245668in}{2.117383in}}%
\pgfpathlineto{\pgfqpoint{6.357351in}{2.177410in}}%
\pgfpathlineto{\pgfqpoint{6.469033in}{2.177410in}}%
\pgfpathlineto{\pgfqpoint{6.580716in}{2.177410in}}%
\pgfpathlineto{\pgfqpoint{6.692399in}{2.177410in}}%
\pgfpathlineto{\pgfqpoint{6.804082in}{2.177410in}}%
\pgfpathlineto{\pgfqpoint{6.915765in}{2.177410in}}%
\pgfpathlineto{\pgfqpoint{7.027447in}{2.237438in}}%
\pgfpathlineto{\pgfqpoint{7.139130in}{2.237438in}}%
\pgfpathlineto{\pgfqpoint{7.250813in}{2.237438in}}%
\pgfpathlineto{\pgfqpoint{7.362496in}{2.297465in}}%
\pgfpathlineto{\pgfqpoint{7.474179in}{2.297465in}}%
\pgfpathlineto{\pgfqpoint{7.585861in}{2.357492in}}%
\pgfpathlineto{\pgfqpoint{7.697544in}{2.357492in}}%
\pgfpathlineto{\pgfqpoint{7.809227in}{2.417520in}}%
\pgfpathlineto{\pgfqpoint{7.920910in}{2.477547in}}%
\pgfpathlineto{\pgfqpoint{8.032593in}{2.477547in}}%
\pgfpathlineto{\pgfqpoint{8.144275in}{2.537575in}}%
\pgfpathlineto{\pgfqpoint{8.255958in}{2.537575in}}%
\pgfpathlineto{\pgfqpoint{8.367641in}{2.597602in}}%
\pgfpathlineto{\pgfqpoint{8.479324in}{2.597602in}}%
\pgfpathlineto{\pgfqpoint{8.591007in}{2.657629in}}%
\pgfpathlineto{\pgfqpoint{8.702689in}{2.717657in}}%
\pgfpathlineto{\pgfqpoint{8.814372in}{2.777684in}}%
\pgfpathlineto{\pgfqpoint{8.926055in}{2.837712in}}%
\pgfpathlineto{\pgfqpoint{9.037738in}{2.837712in}}%
\pgfpathlineto{\pgfqpoint{9.149421in}{2.837712in}}%
\pgfpathlineto{\pgfqpoint{9.261103in}{2.897739in}}%
\pgfpathlineto{\pgfqpoint{9.372786in}{2.897739in}}%
\pgfpathlineto{\pgfqpoint{9.484469in}{2.897739in}}%
\pgfpathlineto{\pgfqpoint{9.596152in}{2.897739in}}%
\pgfpathlineto{\pgfqpoint{9.707835in}{2.897739in}}%
\pgfpathlineto{\pgfqpoint{9.819517in}{2.897739in}}%
\pgfpathlineto{\pgfqpoint{9.931200in}{2.897739in}}%
\pgfpathlineto{\pgfqpoint{10.042883in}{2.957767in}}%
\pgfpathlineto{\pgfqpoint{10.154566in}{3.017794in}}%
\pgfpathlineto{\pgfqpoint{10.266249in}{3.077821in}}%
\pgfpathlineto{\pgfqpoint{10.377931in}{3.257904in}}%
\pgfpathlineto{\pgfqpoint{10.489614in}{3.257904in}}%
\pgfpathlineto{\pgfqpoint{10.601297in}{3.257904in}}%
\pgfpathlineto{\pgfqpoint{10.712980in}{3.257904in}}%
\pgfpathlineto{\pgfqpoint{10.824663in}{3.377958in}}%
\pgfpathlineto{\pgfqpoint{10.936345in}{3.437986in}}%
\pgfpathlineto{\pgfqpoint{11.048028in}{3.498013in}}%
\pgfpathlineto{\pgfqpoint{11.159711in}{3.618068in}}%
\pgfpathlineto{\pgfqpoint{11.271394in}{3.738123in}}%
\pgfpathlineto{\pgfqpoint{11.383077in}{3.798150in}}%
\pgfpathlineto{\pgfqpoint{11.494759in}{3.858178in}}%
\pgfpathlineto{\pgfqpoint{11.606442in}{3.978232in}}%
\pgfpathlineto{\pgfqpoint{11.718125in}{7.519849in}}%
\pgfusepath{stroke}%
\end{pgfscope}%
\begin{pgfscope}%
\pgfpathrectangle{\pgfqpoint{0.661528in}{0.586684in}}{\pgfqpoint{11.056597in}{7.263316in}}%
\pgfusepath{clip}%
\pgfsetrectcap%
\pgfsetroundjoin%
\pgfsetlinewidth{2.007500pt}%
\definecolor{currentstroke}{rgb}{1.000000,0.498039,0.054902}%
\pgfsetstrokecolor{currentstroke}%
\pgfsetdash{}{0pt}%
\pgfpathmoveto{\pgfqpoint{0.651528in}{1.135445in}}%
\pgfpathlineto{\pgfqpoint{0.661528in}{1.156944in}}%
\pgfpathlineto{\pgfqpoint{0.773211in}{1.276999in}}%
\pgfpathlineto{\pgfqpoint{0.884893in}{1.457081in}}%
\pgfpathlineto{\pgfqpoint{0.996576in}{1.577136in}}%
\pgfpathlineto{\pgfqpoint{1.108259in}{1.577136in}}%
\pgfpathlineto{\pgfqpoint{1.219942in}{1.637164in}}%
\pgfpathlineto{\pgfqpoint{1.331625in}{1.637164in}}%
\pgfpathlineto{\pgfqpoint{1.443307in}{1.757218in}}%
\pgfpathlineto{\pgfqpoint{1.554990in}{1.757218in}}%
\pgfpathlineto{\pgfqpoint{1.666673in}{1.757218in}}%
\pgfpathlineto{\pgfqpoint{1.778356in}{1.817246in}}%
\pgfpathlineto{\pgfqpoint{1.890039in}{1.877273in}}%
\pgfpathlineto{\pgfqpoint{2.001721in}{1.997328in}}%
\pgfpathlineto{\pgfqpoint{2.113404in}{2.117383in}}%
\pgfpathlineto{\pgfqpoint{2.225087in}{2.177410in}}%
\pgfpathlineto{\pgfqpoint{2.336770in}{2.237438in}}%
\pgfpathlineto{\pgfqpoint{2.448453in}{2.237438in}}%
\pgfpathlineto{\pgfqpoint{2.560135in}{2.237438in}}%
\pgfpathlineto{\pgfqpoint{2.671818in}{2.237438in}}%
\pgfpathlineto{\pgfqpoint{2.783501in}{2.237438in}}%
\pgfpathlineto{\pgfqpoint{2.895184in}{2.297465in}}%
\pgfpathlineto{\pgfqpoint{3.006867in}{2.357492in}}%
\pgfpathlineto{\pgfqpoint{3.118549in}{2.537575in}}%
\pgfpathlineto{\pgfqpoint{3.230232in}{2.537575in}}%
\pgfpathlineto{\pgfqpoint{3.341915in}{2.537575in}}%
\pgfpathlineto{\pgfqpoint{3.453598in}{2.537575in}}%
\pgfpathlineto{\pgfqpoint{3.565281in}{2.537575in}}%
\pgfpathlineto{\pgfqpoint{3.676963in}{2.657629in}}%
\pgfpathlineto{\pgfqpoint{3.788646in}{2.777684in}}%
\pgfpathlineto{\pgfqpoint{3.900329in}{2.897739in}}%
\pgfpathlineto{\pgfqpoint{4.012012in}{2.957767in}}%
\pgfpathlineto{\pgfqpoint{4.123695in}{2.957767in}}%
\pgfpathlineto{\pgfqpoint{4.235377in}{2.957767in}}%
\pgfpathlineto{\pgfqpoint{4.347060in}{2.957767in}}%
\pgfpathlineto{\pgfqpoint{4.458743in}{2.957767in}}%
\pgfpathlineto{\pgfqpoint{4.570426in}{3.017794in}}%
\pgfpathlineto{\pgfqpoint{4.682109in}{3.077821in}}%
\pgfpathlineto{\pgfqpoint{4.793791in}{3.077821in}}%
\pgfpathlineto{\pgfqpoint{4.905474in}{3.077821in}}%
\pgfpathlineto{\pgfqpoint{5.017157in}{3.137849in}}%
\pgfpathlineto{\pgfqpoint{5.128840in}{3.137849in}}%
\pgfpathlineto{\pgfqpoint{5.240523in}{3.137849in}}%
\pgfpathlineto{\pgfqpoint{5.352205in}{3.137849in}}%
\pgfpathlineto{\pgfqpoint{5.463888in}{3.137849in}}%
\pgfpathlineto{\pgfqpoint{5.575571in}{3.137849in}}%
\pgfpathlineto{\pgfqpoint{5.687254in}{3.197876in}}%
\pgfpathlineto{\pgfqpoint{5.798937in}{3.197876in}}%
\pgfpathlineto{\pgfqpoint{5.910619in}{3.197876in}}%
\pgfpathlineto{\pgfqpoint{6.022302in}{3.257904in}}%
\pgfpathlineto{\pgfqpoint{6.133985in}{3.257904in}}%
\pgfpathlineto{\pgfqpoint{6.245668in}{3.317931in}}%
\pgfpathlineto{\pgfqpoint{6.357351in}{3.317931in}}%
\pgfpathlineto{\pgfqpoint{6.469033in}{3.377958in}}%
\pgfpathlineto{\pgfqpoint{6.580716in}{3.377958in}}%
\pgfpathlineto{\pgfqpoint{6.692399in}{3.377958in}}%
\pgfpathlineto{\pgfqpoint{6.804082in}{3.377958in}}%
\pgfpathlineto{\pgfqpoint{6.915765in}{3.377958in}}%
\pgfpathlineto{\pgfqpoint{7.027447in}{3.377958in}}%
\pgfpathlineto{\pgfqpoint{7.139130in}{3.437986in}}%
\pgfpathlineto{\pgfqpoint{7.250813in}{3.437986in}}%
\pgfpathlineto{\pgfqpoint{7.362496in}{3.437986in}}%
\pgfpathlineto{\pgfqpoint{7.474179in}{3.437986in}}%
\pgfpathlineto{\pgfqpoint{7.585861in}{3.498013in}}%
\pgfpathlineto{\pgfqpoint{7.697544in}{3.498013in}}%
\pgfpathlineto{\pgfqpoint{7.809227in}{3.558041in}}%
\pgfpathlineto{\pgfqpoint{7.920910in}{3.618068in}}%
\pgfpathlineto{\pgfqpoint{8.032593in}{3.618068in}}%
\pgfpathlineto{\pgfqpoint{8.144275in}{3.678095in}}%
\pgfpathlineto{\pgfqpoint{8.255958in}{3.678095in}}%
\pgfpathlineto{\pgfqpoint{8.367641in}{3.678095in}}%
\pgfpathlineto{\pgfqpoint{8.479324in}{3.678095in}}%
\pgfpathlineto{\pgfqpoint{8.591007in}{3.678095in}}%
\pgfpathlineto{\pgfqpoint{8.702689in}{3.738123in}}%
\pgfpathlineto{\pgfqpoint{8.814372in}{3.798150in}}%
\pgfpathlineto{\pgfqpoint{8.926055in}{3.798150in}}%
\pgfpathlineto{\pgfqpoint{9.037738in}{3.798150in}}%
\pgfpathlineto{\pgfqpoint{9.149421in}{3.858178in}}%
\pgfpathlineto{\pgfqpoint{9.261103in}{3.858178in}}%
\pgfpathlineto{\pgfqpoint{9.372786in}{3.858178in}}%
\pgfpathlineto{\pgfqpoint{9.484469in}{3.858178in}}%
\pgfpathlineto{\pgfqpoint{9.596152in}{3.858178in}}%
\pgfpathlineto{\pgfqpoint{9.707835in}{3.858178in}}%
\pgfpathlineto{\pgfqpoint{9.819517in}{3.858178in}}%
\pgfpathlineto{\pgfqpoint{9.931200in}{3.858178in}}%
\pgfpathlineto{\pgfqpoint{10.042883in}{3.858178in}}%
\pgfpathlineto{\pgfqpoint{10.154566in}{3.858178in}}%
\pgfpathlineto{\pgfqpoint{10.266249in}{3.858178in}}%
\pgfpathlineto{\pgfqpoint{10.377931in}{3.858178in}}%
\pgfpathlineto{\pgfqpoint{10.489614in}{3.858178in}}%
\pgfpathlineto{\pgfqpoint{10.601297in}{3.918205in}}%
\pgfpathlineto{\pgfqpoint{10.712980in}{3.918205in}}%
\pgfpathlineto{\pgfqpoint{10.824663in}{3.918205in}}%
\pgfpathlineto{\pgfqpoint{10.936345in}{3.918205in}}%
\pgfpathlineto{\pgfqpoint{11.048028in}{3.918205in}}%
\pgfpathlineto{\pgfqpoint{11.159711in}{3.978232in}}%
\pgfpathlineto{\pgfqpoint{11.271394in}{4.038260in}}%
\pgfpathlineto{\pgfqpoint{11.383077in}{4.158315in}}%
\pgfpathlineto{\pgfqpoint{11.494759in}{4.218342in}}%
\pgfpathlineto{\pgfqpoint{11.606442in}{4.398424in}}%
\pgfpathlineto{\pgfqpoint{11.718125in}{7.519849in}}%
\pgfusepath{stroke}%
\end{pgfscope}%
\begin{pgfscope}%
\pgfpathrectangle{\pgfqpoint{0.661528in}{0.586684in}}{\pgfqpoint{11.056597in}{7.263316in}}%
\pgfusepath{clip}%
\pgfsetrectcap%
\pgfsetroundjoin%
\pgfsetlinewidth{2.007500pt}%
\definecolor{currentstroke}{rgb}{0.172549,0.627451,0.172549}%
\pgfsetstrokecolor{currentstroke}%
\pgfsetdash{}{0pt}%
\pgfpathmoveto{\pgfqpoint{0.651528in}{1.026140in}}%
\pgfpathlineto{\pgfqpoint{0.661528in}{1.036890in}}%
\pgfpathlineto{\pgfqpoint{0.773211in}{1.156944in}}%
\pgfpathlineto{\pgfqpoint{0.884893in}{1.156944in}}%
\pgfpathlineto{\pgfqpoint{0.996576in}{1.216972in}}%
\pgfpathlineto{\pgfqpoint{1.108259in}{1.276999in}}%
\pgfpathlineto{\pgfqpoint{1.219942in}{1.337027in}}%
\pgfpathlineto{\pgfqpoint{1.331625in}{1.337027in}}%
\pgfpathlineto{\pgfqpoint{1.443307in}{1.397054in}}%
\pgfpathlineto{\pgfqpoint{1.554990in}{1.457081in}}%
\pgfpathlineto{\pgfqpoint{1.666673in}{1.457081in}}%
\pgfpathlineto{\pgfqpoint{1.778356in}{1.577136in}}%
\pgfpathlineto{\pgfqpoint{1.890039in}{1.577136in}}%
\pgfpathlineto{\pgfqpoint{2.001721in}{1.697191in}}%
\pgfpathlineto{\pgfqpoint{2.113404in}{1.697191in}}%
\pgfpathlineto{\pgfqpoint{2.225087in}{1.757218in}}%
\pgfpathlineto{\pgfqpoint{2.336770in}{1.877273in}}%
\pgfpathlineto{\pgfqpoint{2.448453in}{1.877273in}}%
\pgfpathlineto{\pgfqpoint{2.560135in}{1.877273in}}%
\pgfpathlineto{\pgfqpoint{2.671818in}{2.057355in}}%
\pgfpathlineto{\pgfqpoint{2.783501in}{2.237438in}}%
\pgfpathlineto{\pgfqpoint{2.895184in}{2.297465in}}%
\pgfpathlineto{\pgfqpoint{3.006867in}{2.297465in}}%
\pgfpathlineto{\pgfqpoint{3.118549in}{2.297465in}}%
\pgfpathlineto{\pgfqpoint{3.230232in}{2.297465in}}%
\pgfpathlineto{\pgfqpoint{3.341915in}{2.297465in}}%
\pgfpathlineto{\pgfqpoint{3.453598in}{2.297465in}}%
\pgfpathlineto{\pgfqpoint{3.565281in}{2.297465in}}%
\pgfpathlineto{\pgfqpoint{3.676963in}{2.297465in}}%
\pgfpathlineto{\pgfqpoint{3.788646in}{2.297465in}}%
\pgfpathlineto{\pgfqpoint{3.900329in}{2.297465in}}%
\pgfpathlineto{\pgfqpoint{4.012012in}{2.357492in}}%
\pgfpathlineto{\pgfqpoint{4.123695in}{2.357492in}}%
\pgfpathlineto{\pgfqpoint{4.235377in}{2.357492in}}%
\pgfpathlineto{\pgfqpoint{4.347060in}{2.477547in}}%
\pgfpathlineto{\pgfqpoint{4.458743in}{2.537575in}}%
\pgfpathlineto{\pgfqpoint{4.570426in}{2.537575in}}%
\pgfpathlineto{\pgfqpoint{4.682109in}{2.597602in}}%
\pgfpathlineto{\pgfqpoint{4.793791in}{2.597602in}}%
\pgfpathlineto{\pgfqpoint{4.905474in}{2.597602in}}%
\pgfpathlineto{\pgfqpoint{5.017157in}{2.717657in}}%
\pgfpathlineto{\pgfqpoint{5.128840in}{2.717657in}}%
\pgfpathlineto{\pgfqpoint{5.240523in}{2.717657in}}%
\pgfpathlineto{\pgfqpoint{5.352205in}{2.717657in}}%
\pgfpathlineto{\pgfqpoint{5.463888in}{2.717657in}}%
\pgfpathlineto{\pgfqpoint{5.575571in}{2.777684in}}%
\pgfpathlineto{\pgfqpoint{5.687254in}{2.837712in}}%
\pgfpathlineto{\pgfqpoint{5.798937in}{2.837712in}}%
\pgfpathlineto{\pgfqpoint{5.910619in}{2.837712in}}%
\pgfpathlineto{\pgfqpoint{6.022302in}{2.837712in}}%
\pgfpathlineto{\pgfqpoint{6.133985in}{2.837712in}}%
\pgfpathlineto{\pgfqpoint{6.245668in}{2.837712in}}%
\pgfpathlineto{\pgfqpoint{6.357351in}{2.897739in}}%
\pgfpathlineto{\pgfqpoint{6.469033in}{2.957767in}}%
\pgfpathlineto{\pgfqpoint{6.580716in}{2.957767in}}%
\pgfpathlineto{\pgfqpoint{6.692399in}{2.957767in}}%
\pgfpathlineto{\pgfqpoint{6.804082in}{3.017794in}}%
\pgfpathlineto{\pgfqpoint{6.915765in}{3.017794in}}%
\pgfpathlineto{\pgfqpoint{7.027447in}{3.077821in}}%
\pgfpathlineto{\pgfqpoint{7.139130in}{3.137849in}}%
\pgfpathlineto{\pgfqpoint{7.250813in}{3.137849in}}%
\pgfpathlineto{\pgfqpoint{7.362496in}{3.197876in}}%
\pgfpathlineto{\pgfqpoint{7.474179in}{3.197876in}}%
\pgfpathlineto{\pgfqpoint{7.585861in}{3.197876in}}%
\pgfpathlineto{\pgfqpoint{7.697544in}{3.197876in}}%
\pgfpathlineto{\pgfqpoint{7.809227in}{3.197876in}}%
\pgfpathlineto{\pgfqpoint{7.920910in}{3.257904in}}%
\pgfpathlineto{\pgfqpoint{8.032593in}{3.317931in}}%
\pgfpathlineto{\pgfqpoint{8.144275in}{3.317931in}}%
\pgfpathlineto{\pgfqpoint{8.255958in}{3.377958in}}%
\pgfpathlineto{\pgfqpoint{8.367641in}{3.377958in}}%
\pgfpathlineto{\pgfqpoint{8.479324in}{3.377958in}}%
\pgfpathlineto{\pgfqpoint{8.591007in}{3.437986in}}%
\pgfpathlineto{\pgfqpoint{8.702689in}{3.437986in}}%
\pgfpathlineto{\pgfqpoint{8.814372in}{3.498013in}}%
\pgfpathlineto{\pgfqpoint{8.926055in}{3.498013in}}%
\pgfpathlineto{\pgfqpoint{9.037738in}{3.498013in}}%
\pgfpathlineto{\pgfqpoint{9.149421in}{3.498013in}}%
\pgfpathlineto{\pgfqpoint{9.261103in}{3.558041in}}%
\pgfpathlineto{\pgfqpoint{9.372786in}{3.678095in}}%
\pgfpathlineto{\pgfqpoint{9.484469in}{3.678095in}}%
\pgfpathlineto{\pgfqpoint{9.596152in}{3.678095in}}%
\pgfpathlineto{\pgfqpoint{9.707835in}{3.678095in}}%
\pgfpathlineto{\pgfqpoint{9.819517in}{3.678095in}}%
\pgfpathlineto{\pgfqpoint{9.931200in}{3.678095in}}%
\pgfpathlineto{\pgfqpoint{10.042883in}{3.798150in}}%
\pgfpathlineto{\pgfqpoint{10.154566in}{3.798150in}}%
\pgfpathlineto{\pgfqpoint{10.266249in}{3.798150in}}%
\pgfpathlineto{\pgfqpoint{10.377931in}{3.798150in}}%
\pgfpathlineto{\pgfqpoint{10.489614in}{3.858178in}}%
\pgfpathlineto{\pgfqpoint{10.601297in}{3.858178in}}%
\pgfpathlineto{\pgfqpoint{10.712980in}{3.858178in}}%
\pgfpathlineto{\pgfqpoint{10.824663in}{3.858178in}}%
\pgfpathlineto{\pgfqpoint{10.936345in}{3.858178in}}%
\pgfpathlineto{\pgfqpoint{11.048028in}{3.858178in}}%
\pgfpathlineto{\pgfqpoint{11.159711in}{3.918205in}}%
\pgfpathlineto{\pgfqpoint{11.271394in}{3.918205in}}%
\pgfpathlineto{\pgfqpoint{11.383077in}{3.918205in}}%
\pgfpathlineto{\pgfqpoint{11.494759in}{3.978232in}}%
\pgfpathlineto{\pgfqpoint{11.606442in}{4.218342in}}%
\pgfpathlineto{\pgfqpoint{11.718125in}{7.519849in}}%
\pgfusepath{stroke}%
\end{pgfscope}%
\begin{pgfscope}%
\pgfpathrectangle{\pgfqpoint{0.661528in}{0.586684in}}{\pgfqpoint{11.056597in}{7.263316in}}%
\pgfusepath{clip}%
\pgfsetrectcap%
\pgfsetroundjoin%
\pgfsetlinewidth{2.007500pt}%
\definecolor{currentstroke}{rgb}{0.839216,0.152941,0.156863}%
\pgfsetstrokecolor{currentstroke}%
\pgfsetdash{}{0pt}%
\pgfpathmoveto{\pgfqpoint{0.651528in}{0.916835in}}%
\pgfpathlineto{\pgfqpoint{0.661528in}{0.916835in}}%
\pgfpathlineto{\pgfqpoint{0.773211in}{0.916835in}}%
\pgfpathlineto{\pgfqpoint{0.884893in}{0.916835in}}%
\pgfpathlineto{\pgfqpoint{0.996576in}{0.976862in}}%
\pgfpathlineto{\pgfqpoint{1.108259in}{0.976862in}}%
\pgfpathlineto{\pgfqpoint{1.219942in}{0.976862in}}%
\pgfpathlineto{\pgfqpoint{1.331625in}{0.976862in}}%
\pgfpathlineto{\pgfqpoint{1.443307in}{0.976862in}}%
\pgfpathlineto{\pgfqpoint{1.554990in}{1.036890in}}%
\pgfpathlineto{\pgfqpoint{1.666673in}{1.036890in}}%
\pgfpathlineto{\pgfqpoint{1.778356in}{1.036890in}}%
\pgfpathlineto{\pgfqpoint{1.890039in}{1.096917in}}%
\pgfpathlineto{\pgfqpoint{2.001721in}{1.156944in}}%
\pgfpathlineto{\pgfqpoint{2.113404in}{1.276999in}}%
\pgfpathlineto{\pgfqpoint{2.225087in}{1.276999in}}%
\pgfpathlineto{\pgfqpoint{2.336770in}{1.276999in}}%
\pgfpathlineto{\pgfqpoint{2.448453in}{1.276999in}}%
\pgfpathlineto{\pgfqpoint{2.560135in}{1.337027in}}%
\pgfpathlineto{\pgfqpoint{2.671818in}{1.337027in}}%
\pgfpathlineto{\pgfqpoint{2.783501in}{1.337027in}}%
\pgfpathlineto{\pgfqpoint{2.895184in}{1.337027in}}%
\pgfpathlineto{\pgfqpoint{3.006867in}{1.337027in}}%
\pgfpathlineto{\pgfqpoint{3.118549in}{1.337027in}}%
\pgfpathlineto{\pgfqpoint{3.230232in}{1.517109in}}%
\pgfpathlineto{\pgfqpoint{3.341915in}{1.577136in}}%
\pgfpathlineto{\pgfqpoint{3.453598in}{1.577136in}}%
\pgfpathlineto{\pgfqpoint{3.565281in}{1.577136in}}%
\pgfpathlineto{\pgfqpoint{3.676963in}{1.637164in}}%
\pgfpathlineto{\pgfqpoint{3.788646in}{1.637164in}}%
\pgfpathlineto{\pgfqpoint{3.900329in}{1.637164in}}%
\pgfpathlineto{\pgfqpoint{4.012012in}{1.637164in}}%
\pgfpathlineto{\pgfqpoint{4.123695in}{1.697191in}}%
\pgfpathlineto{\pgfqpoint{4.235377in}{1.757218in}}%
\pgfpathlineto{\pgfqpoint{4.347060in}{1.877273in}}%
\pgfpathlineto{\pgfqpoint{4.458743in}{1.877273in}}%
\pgfpathlineto{\pgfqpoint{4.570426in}{1.937301in}}%
\pgfpathlineto{\pgfqpoint{4.682109in}{1.937301in}}%
\pgfpathlineto{\pgfqpoint{4.793791in}{1.937301in}}%
\pgfpathlineto{\pgfqpoint{4.905474in}{1.937301in}}%
\pgfpathlineto{\pgfqpoint{5.017157in}{1.937301in}}%
\pgfpathlineto{\pgfqpoint{5.128840in}{1.937301in}}%
\pgfpathlineto{\pgfqpoint{5.240523in}{1.937301in}}%
\pgfpathlineto{\pgfqpoint{5.352205in}{1.937301in}}%
\pgfpathlineto{\pgfqpoint{5.463888in}{1.997328in}}%
\pgfpathlineto{\pgfqpoint{5.575571in}{1.997328in}}%
\pgfpathlineto{\pgfqpoint{5.687254in}{1.997328in}}%
\pgfpathlineto{\pgfqpoint{5.798937in}{2.057355in}}%
\pgfpathlineto{\pgfqpoint{5.910619in}{2.057355in}}%
\pgfpathlineto{\pgfqpoint{6.022302in}{2.057355in}}%
\pgfpathlineto{\pgfqpoint{6.133985in}{2.057355in}}%
\pgfpathlineto{\pgfqpoint{6.245668in}{2.117383in}}%
\pgfpathlineto{\pgfqpoint{6.357351in}{2.117383in}}%
\pgfpathlineto{\pgfqpoint{6.469033in}{2.177410in}}%
\pgfpathlineto{\pgfqpoint{6.580716in}{2.177410in}}%
\pgfpathlineto{\pgfqpoint{6.692399in}{2.237438in}}%
\pgfpathlineto{\pgfqpoint{6.804082in}{2.357492in}}%
\pgfpathlineto{\pgfqpoint{6.915765in}{2.357492in}}%
\pgfpathlineto{\pgfqpoint{7.027447in}{2.357492in}}%
\pgfpathlineto{\pgfqpoint{7.139130in}{2.357492in}}%
\pgfpathlineto{\pgfqpoint{7.250813in}{2.417520in}}%
\pgfpathlineto{\pgfqpoint{7.362496in}{2.417520in}}%
\pgfpathlineto{\pgfqpoint{7.474179in}{2.417520in}}%
\pgfpathlineto{\pgfqpoint{7.585861in}{2.417520in}}%
\pgfpathlineto{\pgfqpoint{7.697544in}{2.417520in}}%
\pgfpathlineto{\pgfqpoint{7.809227in}{2.477547in}}%
\pgfpathlineto{\pgfqpoint{7.920910in}{2.477547in}}%
\pgfpathlineto{\pgfqpoint{8.032593in}{2.477547in}}%
\pgfpathlineto{\pgfqpoint{8.144275in}{2.477547in}}%
\pgfpathlineto{\pgfqpoint{8.255958in}{2.477547in}}%
\pgfpathlineto{\pgfqpoint{8.367641in}{2.537575in}}%
\pgfpathlineto{\pgfqpoint{8.479324in}{2.537575in}}%
\pgfpathlineto{\pgfqpoint{8.591007in}{2.597602in}}%
\pgfpathlineto{\pgfqpoint{8.702689in}{2.597602in}}%
\pgfpathlineto{\pgfqpoint{8.814372in}{2.597602in}}%
\pgfpathlineto{\pgfqpoint{8.926055in}{2.597602in}}%
\pgfpathlineto{\pgfqpoint{9.037738in}{2.657629in}}%
\pgfpathlineto{\pgfqpoint{9.149421in}{2.657629in}}%
\pgfpathlineto{\pgfqpoint{9.261103in}{2.717657in}}%
\pgfpathlineto{\pgfqpoint{9.372786in}{2.717657in}}%
\pgfpathlineto{\pgfqpoint{9.484469in}{2.777684in}}%
\pgfpathlineto{\pgfqpoint{9.596152in}{2.777684in}}%
\pgfpathlineto{\pgfqpoint{9.707835in}{2.777684in}}%
\pgfpathlineto{\pgfqpoint{9.819517in}{2.837712in}}%
\pgfpathlineto{\pgfqpoint{9.931200in}{2.837712in}}%
\pgfpathlineto{\pgfqpoint{10.042883in}{2.897739in}}%
\pgfpathlineto{\pgfqpoint{10.154566in}{2.897739in}}%
\pgfpathlineto{\pgfqpoint{10.266249in}{2.897739in}}%
\pgfpathlineto{\pgfqpoint{10.377931in}{2.897739in}}%
\pgfpathlineto{\pgfqpoint{10.489614in}{2.957767in}}%
\pgfpathlineto{\pgfqpoint{10.601297in}{2.957767in}}%
\pgfpathlineto{\pgfqpoint{10.712980in}{2.957767in}}%
\pgfpathlineto{\pgfqpoint{10.824663in}{2.957767in}}%
\pgfpathlineto{\pgfqpoint{10.936345in}{3.077821in}}%
\pgfpathlineto{\pgfqpoint{11.048028in}{3.257904in}}%
\pgfpathlineto{\pgfqpoint{11.159711in}{3.257904in}}%
\pgfpathlineto{\pgfqpoint{11.271394in}{3.317931in}}%
\pgfpathlineto{\pgfqpoint{11.383077in}{3.317931in}}%
\pgfpathlineto{\pgfqpoint{11.494759in}{3.317931in}}%
\pgfpathlineto{\pgfqpoint{11.606442in}{3.317931in}}%
\pgfpathlineto{\pgfqpoint{11.718125in}{7.519849in}}%
\pgfusepath{stroke}%
\end{pgfscope}%
\begin{pgfscope}%
\pgfpathrectangle{\pgfqpoint{0.661528in}{0.586684in}}{\pgfqpoint{11.056597in}{7.263316in}}%
\pgfusepath{clip}%
\pgfsetrectcap%
\pgfsetroundjoin%
\pgfsetlinewidth{2.007500pt}%
\definecolor{currentstroke}{rgb}{0.580392,0.403922,0.741176}%
\pgfsetstrokecolor{currentstroke}%
\pgfsetdash{}{0pt}%
\pgfpathmoveto{\pgfqpoint{0.651528in}{0.916835in}}%
\pgfpathlineto{\pgfqpoint{0.661528in}{0.916835in}}%
\pgfpathlineto{\pgfqpoint{0.773211in}{0.916835in}}%
\pgfpathlineto{\pgfqpoint{0.884893in}{0.916835in}}%
\pgfpathlineto{\pgfqpoint{0.996576in}{0.916835in}}%
\pgfpathlineto{\pgfqpoint{1.108259in}{0.916835in}}%
\pgfpathlineto{\pgfqpoint{1.219942in}{0.916835in}}%
\pgfpathlineto{\pgfqpoint{1.331625in}{0.916835in}}%
\pgfpathlineto{\pgfqpoint{1.443307in}{0.916835in}}%
\pgfpathlineto{\pgfqpoint{1.554990in}{0.916835in}}%
\pgfpathlineto{\pgfqpoint{1.666673in}{0.916835in}}%
\pgfpathlineto{\pgfqpoint{1.778356in}{0.976862in}}%
\pgfpathlineto{\pgfqpoint{1.890039in}{0.976862in}}%
\pgfpathlineto{\pgfqpoint{2.001721in}{0.976862in}}%
\pgfpathlineto{\pgfqpoint{2.113404in}{0.976862in}}%
\pgfpathlineto{\pgfqpoint{2.225087in}{0.976862in}}%
\pgfpathlineto{\pgfqpoint{2.336770in}{0.976862in}}%
\pgfpathlineto{\pgfqpoint{2.448453in}{0.976862in}}%
\pgfpathlineto{\pgfqpoint{2.560135in}{0.976862in}}%
\pgfpathlineto{\pgfqpoint{2.671818in}{1.036890in}}%
\pgfpathlineto{\pgfqpoint{2.783501in}{1.036890in}}%
\pgfpathlineto{\pgfqpoint{2.895184in}{1.036890in}}%
\pgfpathlineto{\pgfqpoint{3.006867in}{1.036890in}}%
\pgfpathlineto{\pgfqpoint{3.118549in}{1.036890in}}%
\pgfpathlineto{\pgfqpoint{3.230232in}{1.036890in}}%
\pgfpathlineto{\pgfqpoint{3.341915in}{1.036890in}}%
\pgfpathlineto{\pgfqpoint{3.453598in}{1.036890in}}%
\pgfpathlineto{\pgfqpoint{3.565281in}{1.036890in}}%
\pgfpathlineto{\pgfqpoint{3.676963in}{1.096917in}}%
\pgfpathlineto{\pgfqpoint{3.788646in}{1.156944in}}%
\pgfpathlineto{\pgfqpoint{3.900329in}{1.156944in}}%
\pgfpathlineto{\pgfqpoint{4.012012in}{1.156944in}}%
\pgfpathlineto{\pgfqpoint{4.123695in}{1.156944in}}%
\pgfpathlineto{\pgfqpoint{4.235377in}{1.156944in}}%
\pgfpathlineto{\pgfqpoint{4.347060in}{1.216972in}}%
\pgfpathlineto{\pgfqpoint{4.458743in}{1.276999in}}%
\pgfpathlineto{\pgfqpoint{4.570426in}{1.276999in}}%
\pgfpathlineto{\pgfqpoint{4.682109in}{1.276999in}}%
\pgfpathlineto{\pgfqpoint{4.793791in}{1.276999in}}%
\pgfpathlineto{\pgfqpoint{4.905474in}{1.337027in}}%
\pgfpathlineto{\pgfqpoint{5.017157in}{1.397054in}}%
\pgfpathlineto{\pgfqpoint{5.128840in}{1.397054in}}%
\pgfpathlineto{\pgfqpoint{5.240523in}{1.397054in}}%
\pgfpathlineto{\pgfqpoint{5.352205in}{1.457081in}}%
\pgfpathlineto{\pgfqpoint{5.463888in}{1.457081in}}%
\pgfpathlineto{\pgfqpoint{5.575571in}{1.457081in}}%
\pgfpathlineto{\pgfqpoint{5.687254in}{1.457081in}}%
\pgfpathlineto{\pgfqpoint{5.798937in}{1.457081in}}%
\pgfpathlineto{\pgfqpoint{5.910619in}{1.457081in}}%
\pgfpathlineto{\pgfqpoint{6.022302in}{1.457081in}}%
\pgfpathlineto{\pgfqpoint{6.133985in}{1.457081in}}%
\pgfpathlineto{\pgfqpoint{6.245668in}{1.457081in}}%
\pgfpathlineto{\pgfqpoint{6.357351in}{1.457081in}}%
\pgfpathlineto{\pgfqpoint{6.469033in}{1.457081in}}%
\pgfpathlineto{\pgfqpoint{6.580716in}{1.457081in}}%
\pgfpathlineto{\pgfqpoint{6.692399in}{1.457081in}}%
\pgfpathlineto{\pgfqpoint{6.804082in}{1.457081in}}%
\pgfpathlineto{\pgfqpoint{6.915765in}{1.517109in}}%
\pgfpathlineto{\pgfqpoint{7.027447in}{1.517109in}}%
\pgfpathlineto{\pgfqpoint{7.139130in}{1.517109in}}%
\pgfpathlineto{\pgfqpoint{7.250813in}{1.517109in}}%
\pgfpathlineto{\pgfqpoint{7.362496in}{1.517109in}}%
\pgfpathlineto{\pgfqpoint{7.474179in}{1.517109in}}%
\pgfpathlineto{\pgfqpoint{7.585861in}{1.517109in}}%
\pgfpathlineto{\pgfqpoint{7.697544in}{1.577136in}}%
\pgfpathlineto{\pgfqpoint{7.809227in}{1.697191in}}%
\pgfpathlineto{\pgfqpoint{7.920910in}{1.697191in}}%
\pgfpathlineto{\pgfqpoint{8.032593in}{1.697191in}}%
\pgfpathlineto{\pgfqpoint{8.144275in}{1.697191in}}%
\pgfpathlineto{\pgfqpoint{8.255958in}{1.697191in}}%
\pgfpathlineto{\pgfqpoint{8.367641in}{1.697191in}}%
\pgfpathlineto{\pgfqpoint{8.479324in}{1.757218in}}%
\pgfpathlineto{\pgfqpoint{8.591007in}{1.817246in}}%
\pgfpathlineto{\pgfqpoint{8.702689in}{1.817246in}}%
\pgfpathlineto{\pgfqpoint{8.814372in}{1.817246in}}%
\pgfpathlineto{\pgfqpoint{8.926055in}{1.817246in}}%
\pgfpathlineto{\pgfqpoint{9.037738in}{1.877273in}}%
\pgfpathlineto{\pgfqpoint{9.149421in}{1.877273in}}%
\pgfpathlineto{\pgfqpoint{9.261103in}{1.877273in}}%
\pgfpathlineto{\pgfqpoint{9.372786in}{1.937301in}}%
\pgfpathlineto{\pgfqpoint{9.484469in}{1.997328in}}%
\pgfpathlineto{\pgfqpoint{9.596152in}{1.997328in}}%
\pgfpathlineto{\pgfqpoint{9.707835in}{1.997328in}}%
\pgfpathlineto{\pgfqpoint{9.819517in}{1.997328in}}%
\pgfpathlineto{\pgfqpoint{9.931200in}{2.057355in}}%
\pgfpathlineto{\pgfqpoint{10.042883in}{2.117383in}}%
\pgfpathlineto{\pgfqpoint{10.154566in}{2.117383in}}%
\pgfpathlineto{\pgfqpoint{10.266249in}{2.117383in}}%
\pgfpathlineto{\pgfqpoint{10.377931in}{2.117383in}}%
\pgfpathlineto{\pgfqpoint{10.489614in}{2.177410in}}%
\pgfpathlineto{\pgfqpoint{10.601297in}{2.177410in}}%
\pgfpathlineto{\pgfqpoint{10.712980in}{2.177410in}}%
\pgfpathlineto{\pgfqpoint{10.824663in}{2.177410in}}%
\pgfpathlineto{\pgfqpoint{10.936345in}{2.177410in}}%
\pgfpathlineto{\pgfqpoint{11.048028in}{2.177410in}}%
\pgfpathlineto{\pgfqpoint{11.159711in}{2.237438in}}%
\pgfpathlineto{\pgfqpoint{11.271394in}{2.237438in}}%
\pgfpathlineto{\pgfqpoint{11.383077in}{2.297465in}}%
\pgfpathlineto{\pgfqpoint{11.494759in}{2.417520in}}%
\pgfpathlineto{\pgfqpoint{11.606442in}{2.417520in}}%
\pgfpathlineto{\pgfqpoint{11.718125in}{7.519849in}}%
\pgfusepath{stroke}%
\end{pgfscope}%
\begin{pgfscope}%
\pgfpathrectangle{\pgfqpoint{0.661528in}{0.586684in}}{\pgfqpoint{11.056597in}{7.263316in}}%
\pgfusepath{clip}%
\pgfsetrectcap%
\pgfsetroundjoin%
\pgfsetlinewidth{2.007500pt}%
\definecolor{currentstroke}{rgb}{0.549020,0.337255,0.294118}%
\pgfsetstrokecolor{currentstroke}%
\pgfsetdash{}{0pt}%
\pgfpathmoveto{\pgfqpoint{0.651528in}{0.916835in}}%
\pgfpathlineto{\pgfqpoint{0.661528in}{0.916835in}}%
\pgfpathlineto{\pgfqpoint{0.773211in}{0.916835in}}%
\pgfpathlineto{\pgfqpoint{0.884893in}{0.916835in}}%
\pgfpathlineto{\pgfqpoint{0.996576in}{0.916835in}}%
\pgfpathlineto{\pgfqpoint{1.108259in}{0.916835in}}%
\pgfpathlineto{\pgfqpoint{1.219942in}{0.916835in}}%
\pgfpathlineto{\pgfqpoint{1.331625in}{0.916835in}}%
\pgfpathlineto{\pgfqpoint{1.443307in}{0.916835in}}%
\pgfpathlineto{\pgfqpoint{1.554990in}{0.916835in}}%
\pgfpathlineto{\pgfqpoint{1.666673in}{0.916835in}}%
\pgfpathlineto{\pgfqpoint{1.778356in}{0.916835in}}%
\pgfpathlineto{\pgfqpoint{1.890039in}{0.916835in}}%
\pgfpathlineto{\pgfqpoint{2.001721in}{0.976862in}}%
\pgfpathlineto{\pgfqpoint{2.113404in}{0.976862in}}%
\pgfpathlineto{\pgfqpoint{2.225087in}{1.036890in}}%
\pgfpathlineto{\pgfqpoint{2.336770in}{1.036890in}}%
\pgfpathlineto{\pgfqpoint{2.448453in}{1.036890in}}%
\pgfpathlineto{\pgfqpoint{2.560135in}{1.036890in}}%
\pgfpathlineto{\pgfqpoint{2.671818in}{1.036890in}}%
\pgfpathlineto{\pgfqpoint{2.783501in}{1.036890in}}%
\pgfpathlineto{\pgfqpoint{2.895184in}{1.036890in}}%
\pgfpathlineto{\pgfqpoint{3.006867in}{1.036890in}}%
\pgfpathlineto{\pgfqpoint{3.118549in}{1.036890in}}%
\pgfpathlineto{\pgfqpoint{3.230232in}{1.036890in}}%
\pgfpathlineto{\pgfqpoint{3.341915in}{1.036890in}}%
\pgfpathlineto{\pgfqpoint{3.453598in}{1.036890in}}%
\pgfpathlineto{\pgfqpoint{3.565281in}{1.036890in}}%
\pgfpathlineto{\pgfqpoint{3.676963in}{1.036890in}}%
\pgfpathlineto{\pgfqpoint{3.788646in}{1.036890in}}%
\pgfpathlineto{\pgfqpoint{3.900329in}{1.036890in}}%
\pgfpathlineto{\pgfqpoint{4.012012in}{1.036890in}}%
\pgfpathlineto{\pgfqpoint{4.123695in}{1.096917in}}%
\pgfpathlineto{\pgfqpoint{4.235377in}{1.096917in}}%
\pgfpathlineto{\pgfqpoint{4.347060in}{1.096917in}}%
\pgfpathlineto{\pgfqpoint{4.458743in}{1.096917in}}%
\pgfpathlineto{\pgfqpoint{4.570426in}{1.096917in}}%
\pgfpathlineto{\pgfqpoint{4.682109in}{1.096917in}}%
\pgfpathlineto{\pgfqpoint{4.793791in}{1.096917in}}%
\pgfpathlineto{\pgfqpoint{4.905474in}{1.096917in}}%
\pgfpathlineto{\pgfqpoint{5.017157in}{1.096917in}}%
\pgfpathlineto{\pgfqpoint{5.128840in}{1.156944in}}%
\pgfpathlineto{\pgfqpoint{5.240523in}{1.156944in}}%
\pgfpathlineto{\pgfqpoint{5.352205in}{1.156944in}}%
\pgfpathlineto{\pgfqpoint{5.463888in}{1.156944in}}%
\pgfpathlineto{\pgfqpoint{5.575571in}{1.156944in}}%
\pgfpathlineto{\pgfqpoint{5.687254in}{1.156944in}}%
\pgfpathlineto{\pgfqpoint{5.798937in}{1.156944in}}%
\pgfpathlineto{\pgfqpoint{5.910619in}{1.156944in}}%
\pgfpathlineto{\pgfqpoint{6.022302in}{1.156944in}}%
\pgfpathlineto{\pgfqpoint{6.133985in}{1.216972in}}%
\pgfpathlineto{\pgfqpoint{6.245668in}{1.216972in}}%
\pgfpathlineto{\pgfqpoint{6.357351in}{1.216972in}}%
\pgfpathlineto{\pgfqpoint{6.469033in}{1.216972in}}%
\pgfpathlineto{\pgfqpoint{6.580716in}{1.216972in}}%
\pgfpathlineto{\pgfqpoint{6.692399in}{1.216972in}}%
\pgfpathlineto{\pgfqpoint{6.804082in}{1.216972in}}%
\pgfpathlineto{\pgfqpoint{6.915765in}{1.216972in}}%
\pgfpathlineto{\pgfqpoint{7.027447in}{1.216972in}}%
\pgfpathlineto{\pgfqpoint{7.139130in}{1.216972in}}%
\pgfpathlineto{\pgfqpoint{7.250813in}{1.216972in}}%
\pgfpathlineto{\pgfqpoint{7.362496in}{1.216972in}}%
\pgfpathlineto{\pgfqpoint{7.474179in}{1.276999in}}%
\pgfpathlineto{\pgfqpoint{7.585861in}{1.276999in}}%
\pgfpathlineto{\pgfqpoint{7.697544in}{1.276999in}}%
\pgfpathlineto{\pgfqpoint{7.809227in}{1.276999in}}%
\pgfpathlineto{\pgfqpoint{7.920910in}{1.276999in}}%
\pgfpathlineto{\pgfqpoint{8.032593in}{1.276999in}}%
\pgfpathlineto{\pgfqpoint{8.144275in}{1.276999in}}%
\pgfpathlineto{\pgfqpoint{8.255958in}{1.337027in}}%
\pgfpathlineto{\pgfqpoint{8.367641in}{1.457081in}}%
\pgfpathlineto{\pgfqpoint{8.479324in}{1.457081in}}%
\pgfpathlineto{\pgfqpoint{8.591007in}{1.517109in}}%
\pgfpathlineto{\pgfqpoint{8.702689in}{1.517109in}}%
\pgfpathlineto{\pgfqpoint{8.814372in}{1.517109in}}%
\pgfpathlineto{\pgfqpoint{8.926055in}{1.517109in}}%
\pgfpathlineto{\pgfqpoint{9.037738in}{1.517109in}}%
\pgfpathlineto{\pgfqpoint{9.149421in}{1.517109in}}%
\pgfpathlineto{\pgfqpoint{9.261103in}{1.517109in}}%
\pgfpathlineto{\pgfqpoint{9.372786in}{1.577136in}}%
\pgfpathlineto{\pgfqpoint{9.484469in}{1.577136in}}%
\pgfpathlineto{\pgfqpoint{9.596152in}{1.577136in}}%
\pgfpathlineto{\pgfqpoint{9.707835in}{1.577136in}}%
\pgfpathlineto{\pgfqpoint{9.819517in}{1.577136in}}%
\pgfpathlineto{\pgfqpoint{9.931200in}{1.577136in}}%
\pgfpathlineto{\pgfqpoint{10.042883in}{1.637164in}}%
\pgfpathlineto{\pgfqpoint{10.154566in}{1.637164in}}%
\pgfpathlineto{\pgfqpoint{10.266249in}{1.697191in}}%
\pgfpathlineto{\pgfqpoint{10.377931in}{1.697191in}}%
\pgfpathlineto{\pgfqpoint{10.489614in}{1.697191in}}%
\pgfpathlineto{\pgfqpoint{10.601297in}{1.697191in}}%
\pgfpathlineto{\pgfqpoint{10.712980in}{1.697191in}}%
\pgfpathlineto{\pgfqpoint{10.824663in}{1.697191in}}%
\pgfpathlineto{\pgfqpoint{10.936345in}{1.697191in}}%
\pgfpathlineto{\pgfqpoint{11.048028in}{1.697191in}}%
\pgfpathlineto{\pgfqpoint{11.159711in}{1.697191in}}%
\pgfpathlineto{\pgfqpoint{11.271394in}{1.697191in}}%
\pgfpathlineto{\pgfqpoint{11.383077in}{1.697191in}}%
\pgfpathlineto{\pgfqpoint{11.494759in}{1.697191in}}%
\pgfpathlineto{\pgfqpoint{11.606442in}{1.697191in}}%
\pgfpathlineto{\pgfqpoint{11.718125in}{7.519849in}}%
\pgfusepath{stroke}%
\end{pgfscope}%
\begin{pgfscope}%
\pgfsetrectcap%
\pgfsetmiterjoin%
\pgfsetlinewidth{0.803000pt}%
\definecolor{currentstroke}{rgb}{0.000000,0.000000,0.000000}%
\pgfsetstrokecolor{currentstroke}%
\pgfsetdash{}{0pt}%
\pgfpathmoveto{\pgfqpoint{0.661528in}{0.586684in}}%
\pgfpathlineto{\pgfqpoint{0.661528in}{7.850000in}}%
\pgfusepath{stroke}%
\end{pgfscope}%
\begin{pgfscope}%
\pgfsetrectcap%
\pgfsetmiterjoin%
\pgfsetlinewidth{0.803000pt}%
\definecolor{currentstroke}{rgb}{0.000000,0.000000,0.000000}%
\pgfsetstrokecolor{currentstroke}%
\pgfsetdash{}{0pt}%
\pgfpathmoveto{\pgfqpoint{11.718125in}{0.586684in}}%
\pgfpathlineto{\pgfqpoint{11.718125in}{7.850000in}}%
\pgfusepath{stroke}%
\end{pgfscope}%
\begin{pgfscope}%
\pgfsetrectcap%
\pgfsetmiterjoin%
\pgfsetlinewidth{0.803000pt}%
\definecolor{currentstroke}{rgb}{0.000000,0.000000,0.000000}%
\pgfsetstrokecolor{currentstroke}%
\pgfsetdash{}{0pt}%
\pgfpathmoveto{\pgfqpoint{0.661528in}{0.586684in}}%
\pgfpathlineto{\pgfqpoint{11.718125in}{0.586684in}}%
\pgfusepath{stroke}%
\end{pgfscope}%
\begin{pgfscope}%
\pgfsetrectcap%
\pgfsetmiterjoin%
\pgfsetlinewidth{0.803000pt}%
\definecolor{currentstroke}{rgb}{0.000000,0.000000,0.000000}%
\pgfsetstrokecolor{currentstroke}%
\pgfsetdash{}{0pt}%
\pgfpathmoveto{\pgfqpoint{0.661528in}{7.850000in}}%
\pgfpathlineto{\pgfqpoint{11.718125in}{7.850000in}}%
\pgfusepath{stroke}%
\end{pgfscope}%
\begin{pgfscope}%
\pgfsetbuttcap%
\pgfsetmiterjoin%
\definecolor{currentfill}{rgb}{1.000000,1.000000,1.000000}%
\pgfsetfillcolor{currentfill}%
\pgfsetfillopacity{0.800000}%
\pgfsetlinewidth{1.003750pt}%
\definecolor{currentstroke}{rgb}{0.800000,0.800000,0.800000}%
\pgfsetstrokecolor{currentstroke}%
\pgfsetstrokeopacity{0.800000}%
\pgfsetdash{}{0pt}%
\pgfpathmoveto{\pgfqpoint{0.758750in}{6.496418in}}%
\pgfpathlineto{\pgfqpoint{2.230687in}{6.496418in}}%
\pgfpathquadraticcurveto{\pgfqpoint{2.258465in}{6.496418in}}{\pgfqpoint{2.258465in}{6.524196in}}%
\pgfpathlineto{\pgfqpoint{2.258465in}{7.752778in}}%
\pgfpathquadraticcurveto{\pgfqpoint{2.258465in}{7.780556in}}{\pgfqpoint{2.230687in}{7.780556in}}%
\pgfpathlineto{\pgfqpoint{0.758750in}{7.780556in}}%
\pgfpathquadraticcurveto{\pgfqpoint{0.730972in}{7.780556in}}{\pgfqpoint{0.730972in}{7.752778in}}%
\pgfpathlineto{\pgfqpoint{0.730972in}{6.524196in}}%
\pgfpathquadraticcurveto{\pgfqpoint{0.730972in}{6.496418in}}{\pgfqpoint{0.758750in}{6.496418in}}%
\pgfpathlineto{\pgfqpoint{0.758750in}{6.496418in}}%
\pgfpathclose%
\pgfusepath{stroke,fill}%
\end{pgfscope}%
\begin{pgfscope}%
\pgfsetrectcap%
\pgfsetroundjoin%
\pgfsetlinewidth{2.007500pt}%
\definecolor{currentstroke}{rgb}{0.121569,0.466667,0.705882}%
\pgfsetstrokecolor{currentstroke}%
\pgfsetdash{}{0pt}%
\pgfpathmoveto{\pgfqpoint{0.786528in}{7.668088in}}%
\pgfpathlineto{\pgfqpoint{0.925417in}{7.668088in}}%
\pgfpathlineto{\pgfqpoint{1.064306in}{7.668088in}}%
\pgfusepath{stroke}%
\end{pgfscope}%
\begin{pgfscope}%
\definecolor{textcolor}{rgb}{0.000000,0.000000,0.000000}%
\pgfsetstrokecolor{textcolor}%
\pgfsetfillcolor{textcolor}%
\pgftext[x=1.175417in,y=7.619477in,left,base]{\color{textcolor}{\rmfamily\fontsize{10.000000}{12.000000}\selectfont\catcode`\^=\active\def^{\ifmmode\sp\else\^{}\fi}\catcode`\%=\active\def%{\%}CPLEX}}%
\end{pgfscope}%
\begin{pgfscope}%
\pgfsetrectcap%
\pgfsetroundjoin%
\pgfsetlinewidth{2.007500pt}%
\definecolor{currentstroke}{rgb}{1.000000,0.498039,0.054902}%
\pgfsetstrokecolor{currentstroke}%
\pgfsetdash{}{0pt}%
\pgfpathmoveto{\pgfqpoint{0.786528in}{7.464231in}}%
\pgfpathlineto{\pgfqpoint{0.925417in}{7.464231in}}%
\pgfpathlineto{\pgfqpoint{1.064306in}{7.464231in}}%
\pgfusepath{stroke}%
\end{pgfscope}%
\begin{pgfscope}%
\definecolor{textcolor}{rgb}{0.000000,0.000000,0.000000}%
\pgfsetstrokecolor{textcolor}%
\pgfsetfillcolor{textcolor}%
\pgftext[x=1.175417in,y=7.415620in,left,base]{\color{textcolor}{\rmfamily\fontsize{10.000000}{12.000000}\selectfont\catcode`\^=\active\def^{\ifmmode\sp\else\^{}\fi}\catcode`\%=\active\def%{\%}PACS, $\rho =.1$}}%
\end{pgfscope}%
\begin{pgfscope}%
\pgfsetrectcap%
\pgfsetroundjoin%
\pgfsetlinewidth{2.007500pt}%
\definecolor{currentstroke}{rgb}{0.172549,0.627451,0.172549}%
\pgfsetstrokecolor{currentstroke}%
\pgfsetdash{}{0pt}%
\pgfpathmoveto{\pgfqpoint{0.786528in}{7.256508in}}%
\pgfpathlineto{\pgfqpoint{0.925417in}{7.256508in}}%
\pgfpathlineto{\pgfqpoint{1.064306in}{7.256508in}}%
\pgfusepath{stroke}%
\end{pgfscope}%
\begin{pgfscope}%
\definecolor{textcolor}{rgb}{0.000000,0.000000,0.000000}%
\pgfsetstrokecolor{textcolor}%
\pgfsetfillcolor{textcolor}%
\pgftext[x=1.175417in,y=7.207897in,left,base]{\color{textcolor}{\rmfamily\fontsize{10.000000}{12.000000}\selectfont\catcode`\^=\active\def^{\ifmmode\sp\else\^{}\fi}\catcode`\%=\active\def%{\%}PACS, $\rho =.25$}}%
\end{pgfscope}%
\begin{pgfscope}%
\pgfsetrectcap%
\pgfsetroundjoin%
\pgfsetlinewidth{2.007500pt}%
\definecolor{currentstroke}{rgb}{0.839216,0.152941,0.156863}%
\pgfsetstrokecolor{currentstroke}%
\pgfsetdash{}{0pt}%
\pgfpathmoveto{\pgfqpoint{0.786528in}{7.048785in}}%
\pgfpathlineto{\pgfqpoint{0.925417in}{7.048785in}}%
\pgfpathlineto{\pgfqpoint{1.064306in}{7.048785in}}%
\pgfusepath{stroke}%
\end{pgfscope}%
\begin{pgfscope}%
\definecolor{textcolor}{rgb}{0.000000,0.000000,0.000000}%
\pgfsetstrokecolor{textcolor}%
\pgfsetfillcolor{textcolor}%
\pgftext[x=1.175417in,y=7.000174in,left,base]{\color{textcolor}{\rmfamily\fontsize{10.000000}{12.000000}\selectfont\catcode`\^=\active\def^{\ifmmode\sp\else\^{}\fi}\catcode`\%=\active\def%{\%}PACS, $\rho =.5$}}%
\end{pgfscope}%
\begin{pgfscope}%
\pgfsetrectcap%
\pgfsetroundjoin%
\pgfsetlinewidth{2.007500pt}%
\definecolor{currentstroke}{rgb}{0.580392,0.403922,0.741176}%
\pgfsetstrokecolor{currentstroke}%
\pgfsetdash{}{0pt}%
\pgfpathmoveto{\pgfqpoint{0.786528in}{6.841062in}}%
\pgfpathlineto{\pgfqpoint{0.925417in}{6.841062in}}%
\pgfpathlineto{\pgfqpoint{1.064306in}{6.841062in}}%
\pgfusepath{stroke}%
\end{pgfscope}%
\begin{pgfscope}%
\definecolor{textcolor}{rgb}{0.000000,0.000000,0.000000}%
\pgfsetstrokecolor{textcolor}%
\pgfsetfillcolor{textcolor}%
\pgftext[x=1.175417in,y=6.792451in,left,base]{\color{textcolor}{\rmfamily\fontsize{10.000000}{12.000000}\selectfont\catcode`\^=\active\def^{\ifmmode\sp\else\^{}\fi}\catcode`\%=\active\def%{\%}PACS, $\rho =.75$}}%
\end{pgfscope}%
\begin{pgfscope}%
\pgfsetrectcap%
\pgfsetroundjoin%
\pgfsetlinewidth{2.007500pt}%
\definecolor{currentstroke}{rgb}{0.549020,0.337255,0.294118}%
\pgfsetstrokecolor{currentstroke}%
\pgfsetdash{}{0pt}%
\pgfpathmoveto{\pgfqpoint{0.786528in}{6.633340in}}%
\pgfpathlineto{\pgfqpoint{0.925417in}{6.633340in}}%
\pgfpathlineto{\pgfqpoint{1.064306in}{6.633340in}}%
\pgfusepath{stroke}%
\end{pgfscope}%
\begin{pgfscope}%
\definecolor{textcolor}{rgb}{0.000000,0.000000,0.000000}%
\pgfsetstrokecolor{textcolor}%
\pgfsetfillcolor{textcolor}%
\pgftext[x=1.175417in,y=6.584729in,left,base]{\color{textcolor}{\rmfamily\fontsize{10.000000}{12.000000}\selectfont\catcode`\^=\active\def^{\ifmmode\sp\else\^{}\fi}\catcode`\%=\active\def%{\%}PACS, $\rho =.9$}}%
\end{pgfscope}%
\end{pgfpicture}%
\makeatother%
\endgroup%
}
    \end{minipage}%
    \hfill
    \begin{minipage}{0.4\columnwidth}
        \centering
        \resizebox{\linewidth}{!}{%% Creator: Matplotlib, PGF backend
%%
%% To include the figure in your LaTeX document, write
%%   \input{<filename>.pgf}
%%
%% Make sure the required packages are loaded in your preamble
%%   \usepackage{pgf}
%%
%% Also ensure that all the required font packages are loaded; for instance,
%% the lmodern package is sometimes necessary when using math font.
%%   \usepackage{lmodern}
%%
%% Figures using additional raster images can only be included by \input if
%% they are in the same directory as the main LaTeX file. For loading figures
%% from other directories you can use the `import` package
%%   \usepackage{import}
%%
%% and then include the figures with
%%   \import{<path to file>}{<filename>.pgf}
%%
%% Matplotlib used the following preamble
%%   \def\mathdefault#1{#1}
%%   \everymath=\expandafter{\the\everymath\displaystyle}
%%   \IfFileExists{scrextend.sty}{
%%     \usepackage[fontsize=10.000000pt]{scrextend}
%%   }{
%%     \renewcommand{\normalsize}{\fontsize{10.000000}{12.000000}\selectfont}
%%     \normalsize
%%   }
%%   
%%   \ifdefined\pdftexversion\else  % non-pdftex case.
%%     \usepackage{fontspec}
%%     \setmainfont{DejaVuSerif.ttf}[Path=\detokenize{/home/bisca/.global/lib/python3.12/site-packages/matplotlib/mpl-data/fonts/ttf/}]
%%     \setsansfont{DejaVuSans.ttf}[Path=\detokenize{/home/bisca/.global/lib/python3.12/site-packages/matplotlib/mpl-data/fonts/ttf/}]
%%     \setmonofont{DejaVuSansMono.ttf}[Path=\detokenize{/home/bisca/.global/lib/python3.12/site-packages/matplotlib/mpl-data/fonts/ttf/}]
%%   \fi
%%   \makeatletter\@ifpackageloaded{underscore}{}{\usepackage[strings]{underscore}}\makeatother
%%
\begingroup%
\makeatletter%
\begin{pgfpicture}%
\pgfpathrectangle{\pgfpointorigin}{\pgfqpoint{8.000000in}{5.000000in}}%
\pgfusepath{use as bounding box, clip}%
\begin{pgfscope}%
\pgfsetbuttcap%
\pgfsetmiterjoin%
\definecolor{currentfill}{rgb}{1.000000,1.000000,1.000000}%
\pgfsetfillcolor{currentfill}%
\pgfsetlinewidth{0.000000pt}%
\definecolor{currentstroke}{rgb}{1.000000,1.000000,1.000000}%
\pgfsetstrokecolor{currentstroke}%
\pgfsetdash{}{0pt}%
\pgfpathmoveto{\pgfqpoint{0.000000in}{0.000000in}}%
\pgfpathlineto{\pgfqpoint{8.000000in}{0.000000in}}%
\pgfpathlineto{\pgfqpoint{8.000000in}{5.000000in}}%
\pgfpathlineto{\pgfqpoint{0.000000in}{5.000000in}}%
\pgfpathlineto{\pgfqpoint{0.000000in}{0.000000in}}%
\pgfpathclose%
\pgfusepath{fill}%
\end{pgfscope}%
\begin{pgfscope}%
\pgfsetbuttcap%
\pgfsetmiterjoin%
\definecolor{currentfill}{rgb}{1.000000,1.000000,1.000000}%
\pgfsetfillcolor{currentfill}%
\pgfsetlinewidth{0.000000pt}%
\definecolor{currentstroke}{rgb}{0.000000,0.000000,0.000000}%
\pgfsetstrokecolor{currentstroke}%
\pgfsetstrokeopacity{0.000000}%
\pgfsetdash{}{0pt}%
\pgfpathmoveto{\pgfqpoint{0.619028in}{0.395972in}}%
\pgfpathlineto{\pgfqpoint{7.850000in}{0.395972in}}%
\pgfpathlineto{\pgfqpoint{7.850000in}{4.850000in}}%
\pgfpathlineto{\pgfqpoint{0.619028in}{4.850000in}}%
\pgfpathlineto{\pgfqpoint{0.619028in}{0.395972in}}%
\pgfpathclose%
\pgfusepath{fill}%
\end{pgfscope}%
\begin{pgfscope}%
\pgfpathrectangle{\pgfqpoint{0.619028in}{0.395972in}}{\pgfqpoint{7.230972in}{4.454028in}}%
\pgfusepath{clip}%
\pgfsetbuttcap%
\pgfsetmiterjoin%
\definecolor{currentfill}{rgb}{0.121569,0.466667,0.705882}%
\pgfsetfillcolor{currentfill}%
\pgfsetlinewidth{0.000000pt}%
\definecolor{currentstroke}{rgb}{0.000000,0.000000,0.000000}%
\pgfsetstrokecolor{currentstroke}%
\pgfsetstrokeopacity{0.000000}%
\pgfsetdash{}{0pt}%
\pgfpathmoveto{\pgfqpoint{0.947708in}{0.395972in}}%
\pgfpathlineto{\pgfqpoint{1.854413in}{0.395972in}}%
\pgfpathlineto{\pgfqpoint{1.854413in}{2.867338in}}%
\pgfpathlineto{\pgfqpoint{0.947708in}{2.867338in}}%
\pgfpathlineto{\pgfqpoint{0.947708in}{0.395972in}}%
\pgfpathclose%
\pgfusepath{fill}%
\end{pgfscope}%
\begin{pgfscope}%
\pgfpathrectangle{\pgfqpoint{0.619028in}{0.395972in}}{\pgfqpoint{7.230972in}{4.454028in}}%
\pgfusepath{clip}%
\pgfsetbuttcap%
\pgfsetmiterjoin%
\definecolor{currentfill}{rgb}{1.000000,0.498039,0.054902}%
\pgfsetfillcolor{currentfill}%
\pgfsetlinewidth{0.000000pt}%
\definecolor{currentstroke}{rgb}{0.000000,0.000000,0.000000}%
\pgfsetstrokecolor{currentstroke}%
\pgfsetstrokeopacity{0.000000}%
\pgfsetdash{}{0pt}%
\pgfpathmoveto{\pgfqpoint{2.081090in}{0.395972in}}%
\pgfpathlineto{\pgfqpoint{2.987795in}{0.395972in}}%
\pgfpathlineto{\pgfqpoint{2.987795in}{4.637903in}}%
\pgfpathlineto{\pgfqpoint{2.081090in}{4.637903in}}%
\pgfpathlineto{\pgfqpoint{2.081090in}{0.395972in}}%
\pgfpathclose%
\pgfusepath{fill}%
\end{pgfscope}%
\begin{pgfscope}%
\pgfpathrectangle{\pgfqpoint{0.619028in}{0.395972in}}{\pgfqpoint{7.230972in}{4.454028in}}%
\pgfusepath{clip}%
\pgfsetbuttcap%
\pgfsetmiterjoin%
\definecolor{currentfill}{rgb}{0.172549,0.627451,0.172549}%
\pgfsetfillcolor{currentfill}%
\pgfsetlinewidth{0.000000pt}%
\definecolor{currentstroke}{rgb}{0.000000,0.000000,0.000000}%
\pgfsetstrokecolor{currentstroke}%
\pgfsetstrokeopacity{0.000000}%
\pgfsetdash{}{0pt}%
\pgfpathmoveto{\pgfqpoint{3.214471in}{0.395972in}}%
\pgfpathlineto{\pgfqpoint{4.121176in}{0.395972in}}%
\pgfpathlineto{\pgfqpoint{4.121176in}{4.075175in}}%
\pgfpathlineto{\pgfqpoint{3.214471in}{4.075175in}}%
\pgfpathlineto{\pgfqpoint{3.214471in}{0.395972in}}%
\pgfpathclose%
\pgfusepath{fill}%
\end{pgfscope}%
\begin{pgfscope}%
\pgfpathrectangle{\pgfqpoint{0.619028in}{0.395972in}}{\pgfqpoint{7.230972in}{4.454028in}}%
\pgfusepath{clip}%
\pgfsetbuttcap%
\pgfsetmiterjoin%
\definecolor{currentfill}{rgb}{0.839216,0.152941,0.156863}%
\pgfsetfillcolor{currentfill}%
\pgfsetlinewidth{0.000000pt}%
\definecolor{currentstroke}{rgb}{0.000000,0.000000,0.000000}%
\pgfsetstrokecolor{currentstroke}%
\pgfsetstrokeopacity{0.000000}%
\pgfsetdash{}{0pt}%
\pgfpathmoveto{\pgfqpoint{4.347852in}{0.395972in}}%
\pgfpathlineto{\pgfqpoint{5.254557in}{0.395972in}}%
\pgfpathlineto{\pgfqpoint{5.254557in}{2.708382in}}%
\pgfpathlineto{\pgfqpoint{4.347852in}{2.708382in}}%
\pgfpathlineto{\pgfqpoint{4.347852in}{0.395972in}}%
\pgfpathclose%
\pgfusepath{fill}%
\end{pgfscope}%
\begin{pgfscope}%
\pgfpathrectangle{\pgfqpoint{0.619028in}{0.395972in}}{\pgfqpoint{7.230972in}{4.454028in}}%
\pgfusepath{clip}%
\pgfsetbuttcap%
\pgfsetmiterjoin%
\definecolor{currentfill}{rgb}{0.580392,0.403922,0.741176}%
\pgfsetfillcolor{currentfill}%
\pgfsetlinewidth{0.000000pt}%
\definecolor{currentstroke}{rgb}{0.000000,0.000000,0.000000}%
\pgfsetstrokecolor{currentstroke}%
\pgfsetstrokeopacity{0.000000}%
\pgfsetdash{}{0pt}%
\pgfpathmoveto{\pgfqpoint{5.481233in}{0.395972in}}%
\pgfpathlineto{\pgfqpoint{6.387938in}{0.395972in}}%
\pgfpathlineto{\pgfqpoint{6.387938in}{1.558558in}}%
\pgfpathlineto{\pgfqpoint{5.481233in}{1.558558in}}%
\pgfpathlineto{\pgfqpoint{5.481233in}{0.395972in}}%
\pgfpathclose%
\pgfusepath{fill}%
\end{pgfscope}%
\begin{pgfscope}%
\pgfpathrectangle{\pgfqpoint{0.619028in}{0.395972in}}{\pgfqpoint{7.230972in}{4.454028in}}%
\pgfusepath{clip}%
\pgfsetbuttcap%
\pgfsetmiterjoin%
\definecolor{currentfill}{rgb}{0.549020,0.337255,0.294118}%
\pgfsetfillcolor{currentfill}%
\pgfsetlinewidth{0.000000pt}%
\definecolor{currentstroke}{rgb}{0.000000,0.000000,0.000000}%
\pgfsetstrokecolor{currentstroke}%
\pgfsetstrokeopacity{0.000000}%
\pgfsetdash{}{0pt}%
\pgfpathmoveto{\pgfqpoint{6.614614in}{0.395972in}}%
\pgfpathlineto{\pgfqpoint{7.521319in}{0.395972in}}%
\pgfpathlineto{\pgfqpoint{7.521319in}{1.103734in}}%
\pgfpathlineto{\pgfqpoint{6.614614in}{1.103734in}}%
\pgfpathlineto{\pgfqpoint{6.614614in}{0.395972in}}%
\pgfpathclose%
\pgfusepath{fill}%
\end{pgfscope}%
\begin{pgfscope}%
\pgfsetbuttcap%
\pgfsetroundjoin%
\definecolor{currentfill}{rgb}{0.000000,0.000000,0.000000}%
\pgfsetfillcolor{currentfill}%
\pgfsetlinewidth{0.803000pt}%
\definecolor{currentstroke}{rgb}{0.000000,0.000000,0.000000}%
\pgfsetstrokecolor{currentstroke}%
\pgfsetdash{}{0pt}%
\pgfsys@defobject{currentmarker}{\pgfqpoint{0.000000in}{-0.048611in}}{\pgfqpoint{0.000000in}{0.000000in}}{%
\pgfpathmoveto{\pgfqpoint{0.000000in}{0.000000in}}%
\pgfpathlineto{\pgfqpoint{0.000000in}{-0.048611in}}%
\pgfusepath{stroke,fill}%
}%
\begin{pgfscope}%
\pgfsys@transformshift{1.401061in}{0.395972in}%
\pgfsys@useobject{currentmarker}{}%
\end{pgfscope}%
\end{pgfscope}%
\begin{pgfscope}%
\definecolor{textcolor}{rgb}{0.000000,0.000000,0.000000}%
\pgfsetstrokecolor{textcolor}%
\pgfsetfillcolor{textcolor}%
\pgftext[x=1.401061in,y=0.298750in,,top]{\color{textcolor}{\rmfamily\fontsize{10.000000}{12.000000}\selectfont\catcode`\^=\active\def^{\ifmmode\sp\else\^{}\fi}\catcode`\%=\active\def%{\%}CPLEX}}%
\end{pgfscope}%
\begin{pgfscope}%
\pgfsetbuttcap%
\pgfsetroundjoin%
\definecolor{currentfill}{rgb}{0.000000,0.000000,0.000000}%
\pgfsetfillcolor{currentfill}%
\pgfsetlinewidth{0.803000pt}%
\definecolor{currentstroke}{rgb}{0.000000,0.000000,0.000000}%
\pgfsetstrokecolor{currentstroke}%
\pgfsetdash{}{0pt}%
\pgfsys@defobject{currentmarker}{\pgfqpoint{0.000000in}{-0.048611in}}{\pgfqpoint{0.000000in}{0.000000in}}{%
\pgfpathmoveto{\pgfqpoint{0.000000in}{0.000000in}}%
\pgfpathlineto{\pgfqpoint{0.000000in}{-0.048611in}}%
\pgfusepath{stroke,fill}%
}%
\begin{pgfscope}%
\pgfsys@transformshift{2.534442in}{0.395972in}%
\pgfsys@useobject{currentmarker}{}%
\end{pgfscope}%
\end{pgfscope}%
\begin{pgfscope}%
\definecolor{textcolor}{rgb}{0.000000,0.000000,0.000000}%
\pgfsetstrokecolor{textcolor}%
\pgfsetfillcolor{textcolor}%
\pgftext[x=2.534442in,y=0.298750in,,top]{\color{textcolor}{\rmfamily\fontsize{10.000000}{12.000000}\selectfont\catcode`\^=\active\def^{\ifmmode\sp\else\^{}\fi}\catcode`\%=\active\def%{\%}PACS, $\rho=0.1$}}%
\end{pgfscope}%
\begin{pgfscope}%
\pgfsetbuttcap%
\pgfsetroundjoin%
\definecolor{currentfill}{rgb}{0.000000,0.000000,0.000000}%
\pgfsetfillcolor{currentfill}%
\pgfsetlinewidth{0.803000pt}%
\definecolor{currentstroke}{rgb}{0.000000,0.000000,0.000000}%
\pgfsetstrokecolor{currentstroke}%
\pgfsetdash{}{0pt}%
\pgfsys@defobject{currentmarker}{\pgfqpoint{0.000000in}{-0.048611in}}{\pgfqpoint{0.000000in}{0.000000in}}{%
\pgfpathmoveto{\pgfqpoint{0.000000in}{0.000000in}}%
\pgfpathlineto{\pgfqpoint{0.000000in}{-0.048611in}}%
\pgfusepath{stroke,fill}%
}%
\begin{pgfscope}%
\pgfsys@transformshift{3.667823in}{0.395972in}%
\pgfsys@useobject{currentmarker}{}%
\end{pgfscope}%
\end{pgfscope}%
\begin{pgfscope}%
\definecolor{textcolor}{rgb}{0.000000,0.000000,0.000000}%
\pgfsetstrokecolor{textcolor}%
\pgfsetfillcolor{textcolor}%
\pgftext[x=3.667823in,y=0.298750in,,top]{\color{textcolor}{\rmfamily\fontsize{10.000000}{12.000000}\selectfont\catcode`\^=\active\def^{\ifmmode\sp\else\^{}\fi}\catcode`\%=\active\def%{\%}PACS, $\rho=0.25$}}%
\end{pgfscope}%
\begin{pgfscope}%
\pgfsetbuttcap%
\pgfsetroundjoin%
\definecolor{currentfill}{rgb}{0.000000,0.000000,0.000000}%
\pgfsetfillcolor{currentfill}%
\pgfsetlinewidth{0.803000pt}%
\definecolor{currentstroke}{rgb}{0.000000,0.000000,0.000000}%
\pgfsetstrokecolor{currentstroke}%
\pgfsetdash{}{0pt}%
\pgfsys@defobject{currentmarker}{\pgfqpoint{0.000000in}{-0.048611in}}{\pgfqpoint{0.000000in}{0.000000in}}{%
\pgfpathmoveto{\pgfqpoint{0.000000in}{0.000000in}}%
\pgfpathlineto{\pgfqpoint{0.000000in}{-0.048611in}}%
\pgfusepath{stroke,fill}%
}%
\begin{pgfscope}%
\pgfsys@transformshift{4.801205in}{0.395972in}%
\pgfsys@useobject{currentmarker}{}%
\end{pgfscope}%
\end{pgfscope}%
\begin{pgfscope}%
\definecolor{textcolor}{rgb}{0.000000,0.000000,0.000000}%
\pgfsetstrokecolor{textcolor}%
\pgfsetfillcolor{textcolor}%
\pgftext[x=4.801205in,y=0.298750in,,top]{\color{textcolor}{\rmfamily\fontsize{10.000000}{12.000000}\selectfont\catcode`\^=\active\def^{\ifmmode\sp\else\^{}\fi}\catcode`\%=\active\def%{\%}PACS, $\rho=0.5$}}%
\end{pgfscope}%
\begin{pgfscope}%
\pgfsetbuttcap%
\pgfsetroundjoin%
\definecolor{currentfill}{rgb}{0.000000,0.000000,0.000000}%
\pgfsetfillcolor{currentfill}%
\pgfsetlinewidth{0.803000pt}%
\definecolor{currentstroke}{rgb}{0.000000,0.000000,0.000000}%
\pgfsetstrokecolor{currentstroke}%
\pgfsetdash{}{0pt}%
\pgfsys@defobject{currentmarker}{\pgfqpoint{0.000000in}{-0.048611in}}{\pgfqpoint{0.000000in}{0.000000in}}{%
\pgfpathmoveto{\pgfqpoint{0.000000in}{0.000000in}}%
\pgfpathlineto{\pgfqpoint{0.000000in}{-0.048611in}}%
\pgfusepath{stroke,fill}%
}%
\begin{pgfscope}%
\pgfsys@transformshift{5.934586in}{0.395972in}%
\pgfsys@useobject{currentmarker}{}%
\end{pgfscope}%
\end{pgfscope}%
\begin{pgfscope}%
\definecolor{textcolor}{rgb}{0.000000,0.000000,0.000000}%
\pgfsetstrokecolor{textcolor}%
\pgfsetfillcolor{textcolor}%
\pgftext[x=5.934586in,y=0.298750in,,top]{\color{textcolor}{\rmfamily\fontsize{10.000000}{12.000000}\selectfont\catcode`\^=\active\def^{\ifmmode\sp\else\^{}\fi}\catcode`\%=\active\def%{\%}PACS, $\rho=0.75$}}%
\end{pgfscope}%
\begin{pgfscope}%
\pgfsetbuttcap%
\pgfsetroundjoin%
\definecolor{currentfill}{rgb}{0.000000,0.000000,0.000000}%
\pgfsetfillcolor{currentfill}%
\pgfsetlinewidth{0.803000pt}%
\definecolor{currentstroke}{rgb}{0.000000,0.000000,0.000000}%
\pgfsetstrokecolor{currentstroke}%
\pgfsetdash{}{0pt}%
\pgfsys@defobject{currentmarker}{\pgfqpoint{0.000000in}{-0.048611in}}{\pgfqpoint{0.000000in}{0.000000in}}{%
\pgfpathmoveto{\pgfqpoint{0.000000in}{0.000000in}}%
\pgfpathlineto{\pgfqpoint{0.000000in}{-0.048611in}}%
\pgfusepath{stroke,fill}%
}%
\begin{pgfscope}%
\pgfsys@transformshift{7.067967in}{0.395972in}%
\pgfsys@useobject{currentmarker}{}%
\end{pgfscope}%
\end{pgfscope}%
\begin{pgfscope}%
\definecolor{textcolor}{rgb}{0.000000,0.000000,0.000000}%
\pgfsetstrokecolor{textcolor}%
\pgfsetfillcolor{textcolor}%
\pgftext[x=7.067967in,y=0.298750in,,top]{\color{textcolor}{\rmfamily\fontsize{10.000000}{12.000000}\selectfont\catcode`\^=\active\def^{\ifmmode\sp\else\^{}\fi}\catcode`\%=\active\def%{\%}PACS, $\rho=0.9$}}%
\end{pgfscope}%
\begin{pgfscope}%
\pgfpathrectangle{\pgfqpoint{0.619028in}{0.395972in}}{\pgfqpoint{7.230972in}{4.454028in}}%
\pgfusepath{clip}%
\pgfsetbuttcap%
\pgfsetroundjoin%
\pgfsetlinewidth{0.803000pt}%
\definecolor{currentstroke}{rgb}{0.690196,0.690196,0.690196}%
\pgfsetstrokecolor{currentstroke}%
\pgfsetstrokeopacity{0.500000}%
\pgfsetdash{{2.960000pt}{1.280000pt}}{0.000000pt}%
\pgfpathmoveto{\pgfqpoint{0.619028in}{0.395972in}}%
\pgfpathlineto{\pgfqpoint{7.850000in}{0.395972in}}%
\pgfusepath{stroke}%
\end{pgfscope}%
\begin{pgfscope}%
\pgfsetbuttcap%
\pgfsetroundjoin%
\definecolor{currentfill}{rgb}{0.000000,0.000000,0.000000}%
\pgfsetfillcolor{currentfill}%
\pgfsetlinewidth{0.803000pt}%
\definecolor{currentstroke}{rgb}{0.000000,0.000000,0.000000}%
\pgfsetstrokecolor{currentstroke}%
\pgfsetdash{}{0pt}%
\pgfsys@defobject{currentmarker}{\pgfqpoint{-0.048611in}{0.000000in}}{\pgfqpoint{-0.000000in}{0.000000in}}{%
\pgfpathmoveto{\pgfqpoint{-0.000000in}{0.000000in}}%
\pgfpathlineto{\pgfqpoint{-0.048611in}{0.000000in}}%
\pgfusepath{stroke,fill}%
}%
\begin{pgfscope}%
\pgfsys@transformshift{0.619028in}{0.395972in}%
\pgfsys@useobject{currentmarker}{}%
\end{pgfscope}%
\end{pgfscope}%
\begin{pgfscope}%
\definecolor{textcolor}{rgb}{0.000000,0.000000,0.000000}%
\pgfsetstrokecolor{textcolor}%
\pgfsetfillcolor{textcolor}%
\pgftext[x=0.433440in, y=0.343211in, left, base]{\color{textcolor}{\rmfamily\fontsize{10.000000}{12.000000}\selectfont\catcode`\^=\active\def^{\ifmmode\sp\else\^{}\fi}\catcode`\%=\active\def%{\%}0}}%
\end{pgfscope}%
\begin{pgfscope}%
\pgfpathrectangle{\pgfqpoint{0.619028in}{0.395972in}}{\pgfqpoint{7.230972in}{4.454028in}}%
\pgfusepath{clip}%
\pgfsetbuttcap%
\pgfsetroundjoin%
\pgfsetlinewidth{0.803000pt}%
\definecolor{currentstroke}{rgb}{0.690196,0.690196,0.690196}%
\pgfsetstrokecolor{currentstroke}%
\pgfsetstrokeopacity{0.500000}%
\pgfsetdash{{2.960000pt}{1.280000pt}}{0.000000pt}%
\pgfpathmoveto{\pgfqpoint{0.619028in}{1.034118in}}%
\pgfpathlineto{\pgfqpoint{7.850000in}{1.034118in}}%
\pgfusepath{stroke}%
\end{pgfscope}%
\begin{pgfscope}%
\pgfsetbuttcap%
\pgfsetroundjoin%
\definecolor{currentfill}{rgb}{0.000000,0.000000,0.000000}%
\pgfsetfillcolor{currentfill}%
\pgfsetlinewidth{0.803000pt}%
\definecolor{currentstroke}{rgb}{0.000000,0.000000,0.000000}%
\pgfsetstrokecolor{currentstroke}%
\pgfsetdash{}{0pt}%
\pgfsys@defobject{currentmarker}{\pgfqpoint{-0.048611in}{0.000000in}}{\pgfqpoint{-0.000000in}{0.000000in}}{%
\pgfpathmoveto{\pgfqpoint{-0.000000in}{0.000000in}}%
\pgfpathlineto{\pgfqpoint{-0.048611in}{0.000000in}}%
\pgfusepath{stroke,fill}%
}%
\begin{pgfscope}%
\pgfsys@transformshift{0.619028in}{1.034118in}%
\pgfsys@useobject{currentmarker}{}%
\end{pgfscope}%
\end{pgfscope}%
\begin{pgfscope}%
\definecolor{textcolor}{rgb}{0.000000,0.000000,0.000000}%
\pgfsetstrokecolor{textcolor}%
\pgfsetfillcolor{textcolor}%
\pgftext[x=0.433440in, y=0.981357in, left, base]{\color{textcolor}{\rmfamily\fontsize{10.000000}{12.000000}\selectfont\catcode`\^=\active\def^{\ifmmode\sp\else\^{}\fi}\catcode`\%=\active\def%{\%}5}}%
\end{pgfscope}%
\begin{pgfscope}%
\pgfpathrectangle{\pgfqpoint{0.619028in}{0.395972in}}{\pgfqpoint{7.230972in}{4.454028in}}%
\pgfusepath{clip}%
\pgfsetbuttcap%
\pgfsetroundjoin%
\pgfsetlinewidth{0.803000pt}%
\definecolor{currentstroke}{rgb}{0.690196,0.690196,0.690196}%
\pgfsetstrokecolor{currentstroke}%
\pgfsetstrokeopacity{0.500000}%
\pgfsetdash{{2.960000pt}{1.280000pt}}{0.000000pt}%
\pgfpathmoveto{\pgfqpoint{0.619028in}{1.672264in}}%
\pgfpathlineto{\pgfqpoint{7.850000in}{1.672264in}}%
\pgfusepath{stroke}%
\end{pgfscope}%
\begin{pgfscope}%
\pgfsetbuttcap%
\pgfsetroundjoin%
\definecolor{currentfill}{rgb}{0.000000,0.000000,0.000000}%
\pgfsetfillcolor{currentfill}%
\pgfsetlinewidth{0.803000pt}%
\definecolor{currentstroke}{rgb}{0.000000,0.000000,0.000000}%
\pgfsetstrokecolor{currentstroke}%
\pgfsetdash{}{0pt}%
\pgfsys@defobject{currentmarker}{\pgfqpoint{-0.048611in}{0.000000in}}{\pgfqpoint{-0.000000in}{0.000000in}}{%
\pgfpathmoveto{\pgfqpoint{-0.000000in}{0.000000in}}%
\pgfpathlineto{\pgfqpoint{-0.048611in}{0.000000in}}%
\pgfusepath{stroke,fill}%
}%
\begin{pgfscope}%
\pgfsys@transformshift{0.619028in}{1.672264in}%
\pgfsys@useobject{currentmarker}{}%
\end{pgfscope}%
\end{pgfscope}%
\begin{pgfscope}%
\definecolor{textcolor}{rgb}{0.000000,0.000000,0.000000}%
\pgfsetstrokecolor{textcolor}%
\pgfsetfillcolor{textcolor}%
\pgftext[x=0.345075in, y=1.619503in, left, base]{\color{textcolor}{\rmfamily\fontsize{10.000000}{12.000000}\selectfont\catcode`\^=\active\def^{\ifmmode\sp\else\^{}\fi}\catcode`\%=\active\def%{\%}10}}%
\end{pgfscope}%
\begin{pgfscope}%
\pgfpathrectangle{\pgfqpoint{0.619028in}{0.395972in}}{\pgfqpoint{7.230972in}{4.454028in}}%
\pgfusepath{clip}%
\pgfsetbuttcap%
\pgfsetroundjoin%
\pgfsetlinewidth{0.803000pt}%
\definecolor{currentstroke}{rgb}{0.690196,0.690196,0.690196}%
\pgfsetstrokecolor{currentstroke}%
\pgfsetstrokeopacity{0.500000}%
\pgfsetdash{{2.960000pt}{1.280000pt}}{0.000000pt}%
\pgfpathmoveto{\pgfqpoint{0.619028in}{2.310411in}}%
\pgfpathlineto{\pgfqpoint{7.850000in}{2.310411in}}%
\pgfusepath{stroke}%
\end{pgfscope}%
\begin{pgfscope}%
\pgfsetbuttcap%
\pgfsetroundjoin%
\definecolor{currentfill}{rgb}{0.000000,0.000000,0.000000}%
\pgfsetfillcolor{currentfill}%
\pgfsetlinewidth{0.803000pt}%
\definecolor{currentstroke}{rgb}{0.000000,0.000000,0.000000}%
\pgfsetstrokecolor{currentstroke}%
\pgfsetdash{}{0pt}%
\pgfsys@defobject{currentmarker}{\pgfqpoint{-0.048611in}{0.000000in}}{\pgfqpoint{-0.000000in}{0.000000in}}{%
\pgfpathmoveto{\pgfqpoint{-0.000000in}{0.000000in}}%
\pgfpathlineto{\pgfqpoint{-0.048611in}{0.000000in}}%
\pgfusepath{stroke,fill}%
}%
\begin{pgfscope}%
\pgfsys@transformshift{0.619028in}{2.310411in}%
\pgfsys@useobject{currentmarker}{}%
\end{pgfscope}%
\end{pgfscope}%
\begin{pgfscope}%
\definecolor{textcolor}{rgb}{0.000000,0.000000,0.000000}%
\pgfsetstrokecolor{textcolor}%
\pgfsetfillcolor{textcolor}%
\pgftext[x=0.345075in, y=2.257649in, left, base]{\color{textcolor}{\rmfamily\fontsize{10.000000}{12.000000}\selectfont\catcode`\^=\active\def^{\ifmmode\sp\else\^{}\fi}\catcode`\%=\active\def%{\%}15}}%
\end{pgfscope}%
\begin{pgfscope}%
\pgfpathrectangle{\pgfqpoint{0.619028in}{0.395972in}}{\pgfqpoint{7.230972in}{4.454028in}}%
\pgfusepath{clip}%
\pgfsetbuttcap%
\pgfsetroundjoin%
\pgfsetlinewidth{0.803000pt}%
\definecolor{currentstroke}{rgb}{0.690196,0.690196,0.690196}%
\pgfsetstrokecolor{currentstroke}%
\pgfsetstrokeopacity{0.500000}%
\pgfsetdash{{2.960000pt}{1.280000pt}}{0.000000pt}%
\pgfpathmoveto{\pgfqpoint{0.619028in}{2.948557in}}%
\pgfpathlineto{\pgfqpoint{7.850000in}{2.948557in}}%
\pgfusepath{stroke}%
\end{pgfscope}%
\begin{pgfscope}%
\pgfsetbuttcap%
\pgfsetroundjoin%
\definecolor{currentfill}{rgb}{0.000000,0.000000,0.000000}%
\pgfsetfillcolor{currentfill}%
\pgfsetlinewidth{0.803000pt}%
\definecolor{currentstroke}{rgb}{0.000000,0.000000,0.000000}%
\pgfsetstrokecolor{currentstroke}%
\pgfsetdash{}{0pt}%
\pgfsys@defobject{currentmarker}{\pgfqpoint{-0.048611in}{0.000000in}}{\pgfqpoint{-0.000000in}{0.000000in}}{%
\pgfpathmoveto{\pgfqpoint{-0.000000in}{0.000000in}}%
\pgfpathlineto{\pgfqpoint{-0.048611in}{0.000000in}}%
\pgfusepath{stroke,fill}%
}%
\begin{pgfscope}%
\pgfsys@transformshift{0.619028in}{2.948557in}%
\pgfsys@useobject{currentmarker}{}%
\end{pgfscope}%
\end{pgfscope}%
\begin{pgfscope}%
\definecolor{textcolor}{rgb}{0.000000,0.000000,0.000000}%
\pgfsetstrokecolor{textcolor}%
\pgfsetfillcolor{textcolor}%
\pgftext[x=0.345075in, y=2.895795in, left, base]{\color{textcolor}{\rmfamily\fontsize{10.000000}{12.000000}\selectfont\catcode`\^=\active\def^{\ifmmode\sp\else\^{}\fi}\catcode`\%=\active\def%{\%}20}}%
\end{pgfscope}%
\begin{pgfscope}%
\pgfpathrectangle{\pgfqpoint{0.619028in}{0.395972in}}{\pgfqpoint{7.230972in}{4.454028in}}%
\pgfusepath{clip}%
\pgfsetbuttcap%
\pgfsetroundjoin%
\pgfsetlinewidth{0.803000pt}%
\definecolor{currentstroke}{rgb}{0.690196,0.690196,0.690196}%
\pgfsetstrokecolor{currentstroke}%
\pgfsetstrokeopacity{0.500000}%
\pgfsetdash{{2.960000pt}{1.280000pt}}{0.000000pt}%
\pgfpathmoveto{\pgfqpoint{0.619028in}{3.586703in}}%
\pgfpathlineto{\pgfqpoint{7.850000in}{3.586703in}}%
\pgfusepath{stroke}%
\end{pgfscope}%
\begin{pgfscope}%
\pgfsetbuttcap%
\pgfsetroundjoin%
\definecolor{currentfill}{rgb}{0.000000,0.000000,0.000000}%
\pgfsetfillcolor{currentfill}%
\pgfsetlinewidth{0.803000pt}%
\definecolor{currentstroke}{rgb}{0.000000,0.000000,0.000000}%
\pgfsetstrokecolor{currentstroke}%
\pgfsetdash{}{0pt}%
\pgfsys@defobject{currentmarker}{\pgfqpoint{-0.048611in}{0.000000in}}{\pgfqpoint{-0.000000in}{0.000000in}}{%
\pgfpathmoveto{\pgfqpoint{-0.000000in}{0.000000in}}%
\pgfpathlineto{\pgfqpoint{-0.048611in}{0.000000in}}%
\pgfusepath{stroke,fill}%
}%
\begin{pgfscope}%
\pgfsys@transformshift{0.619028in}{3.586703in}%
\pgfsys@useobject{currentmarker}{}%
\end{pgfscope}%
\end{pgfscope}%
\begin{pgfscope}%
\definecolor{textcolor}{rgb}{0.000000,0.000000,0.000000}%
\pgfsetstrokecolor{textcolor}%
\pgfsetfillcolor{textcolor}%
\pgftext[x=0.345075in, y=3.533941in, left, base]{\color{textcolor}{\rmfamily\fontsize{10.000000}{12.000000}\selectfont\catcode`\^=\active\def^{\ifmmode\sp\else\^{}\fi}\catcode`\%=\active\def%{\%}25}}%
\end{pgfscope}%
\begin{pgfscope}%
\pgfpathrectangle{\pgfqpoint{0.619028in}{0.395972in}}{\pgfqpoint{7.230972in}{4.454028in}}%
\pgfusepath{clip}%
\pgfsetbuttcap%
\pgfsetroundjoin%
\pgfsetlinewidth{0.803000pt}%
\definecolor{currentstroke}{rgb}{0.690196,0.690196,0.690196}%
\pgfsetstrokecolor{currentstroke}%
\pgfsetstrokeopacity{0.500000}%
\pgfsetdash{{2.960000pt}{1.280000pt}}{0.000000pt}%
\pgfpathmoveto{\pgfqpoint{0.619028in}{4.224849in}}%
\pgfpathlineto{\pgfqpoint{7.850000in}{4.224849in}}%
\pgfusepath{stroke}%
\end{pgfscope}%
\begin{pgfscope}%
\pgfsetbuttcap%
\pgfsetroundjoin%
\definecolor{currentfill}{rgb}{0.000000,0.000000,0.000000}%
\pgfsetfillcolor{currentfill}%
\pgfsetlinewidth{0.803000pt}%
\definecolor{currentstroke}{rgb}{0.000000,0.000000,0.000000}%
\pgfsetstrokecolor{currentstroke}%
\pgfsetdash{}{0pt}%
\pgfsys@defobject{currentmarker}{\pgfqpoint{-0.048611in}{0.000000in}}{\pgfqpoint{-0.000000in}{0.000000in}}{%
\pgfpathmoveto{\pgfqpoint{-0.000000in}{0.000000in}}%
\pgfpathlineto{\pgfqpoint{-0.048611in}{0.000000in}}%
\pgfusepath{stroke,fill}%
}%
\begin{pgfscope}%
\pgfsys@transformshift{0.619028in}{4.224849in}%
\pgfsys@useobject{currentmarker}{}%
\end{pgfscope}%
\end{pgfscope}%
\begin{pgfscope}%
\definecolor{textcolor}{rgb}{0.000000,0.000000,0.000000}%
\pgfsetstrokecolor{textcolor}%
\pgfsetfillcolor{textcolor}%
\pgftext[x=0.345075in, y=4.172087in, left, base]{\color{textcolor}{\rmfamily\fontsize{10.000000}{12.000000}\selectfont\catcode`\^=\active\def^{\ifmmode\sp\else\^{}\fi}\catcode`\%=\active\def%{\%}30}}%
\end{pgfscope}%
\begin{pgfscope}%
\definecolor{textcolor}{rgb}{0.000000,0.000000,0.000000}%
\pgfsetstrokecolor{textcolor}%
\pgfsetfillcolor{textcolor}%
\pgftext[x=0.289519in,y=2.622986in,,bottom,rotate=90.000000]{\color{textcolor}{\rmfamily\fontsize{10.000000}{12.000000}\selectfont\catcode`\^=\active\def^{\ifmmode\sp\else\^{}\fi}\catcode`\%=\active\def%{\%}Integral Value}}%
\end{pgfscope}%
\begin{pgfscope}%
\pgfsetrectcap%
\pgfsetmiterjoin%
\pgfsetlinewidth{0.803000pt}%
\definecolor{currentstroke}{rgb}{0.000000,0.000000,0.000000}%
\pgfsetstrokecolor{currentstroke}%
\pgfsetdash{}{0pt}%
\pgfpathmoveto{\pgfqpoint{0.619028in}{0.395972in}}%
\pgfpathlineto{\pgfqpoint{0.619028in}{4.850000in}}%
\pgfusepath{stroke}%
\end{pgfscope}%
\begin{pgfscope}%
\pgfsetrectcap%
\pgfsetmiterjoin%
\pgfsetlinewidth{0.803000pt}%
\definecolor{currentstroke}{rgb}{0.000000,0.000000,0.000000}%
\pgfsetstrokecolor{currentstroke}%
\pgfsetdash{}{0pt}%
\pgfpathmoveto{\pgfqpoint{7.850000in}{0.395972in}}%
\pgfpathlineto{\pgfqpoint{7.850000in}{4.850000in}}%
\pgfusepath{stroke}%
\end{pgfscope}%
\begin{pgfscope}%
\pgfsetrectcap%
\pgfsetmiterjoin%
\pgfsetlinewidth{0.803000pt}%
\definecolor{currentstroke}{rgb}{0.000000,0.000000,0.000000}%
\pgfsetstrokecolor{currentstroke}%
\pgfsetdash{}{0pt}%
\pgfpathmoveto{\pgfqpoint{0.619028in}{0.395972in}}%
\pgfpathlineto{\pgfqpoint{7.850000in}{0.395972in}}%
\pgfusepath{stroke}%
\end{pgfscope}%
\begin{pgfscope}%
\pgfsetrectcap%
\pgfsetmiterjoin%
\pgfsetlinewidth{0.803000pt}%
\definecolor{currentstroke}{rgb}{0.000000,0.000000,0.000000}%
\pgfsetstrokecolor{currentstroke}%
\pgfsetdash{}{0pt}%
\pgfpathmoveto{\pgfqpoint{0.619028in}{4.850000in}}%
\pgfpathlineto{\pgfqpoint{7.850000in}{4.850000in}}%
\pgfusepath{stroke}%
\end{pgfscope}%
\end{pgfpicture}%
\makeatother%
\endgroup%
}
    \end{minipage}
    \caption{Success Rate vs. MIP Gap Plot for Fixed $\rho$ Initialization Test}
    \label{fig:PACS_STD_MGAP_Int}
\end{figure}

\subsection{Dynamic Adjustment of Parameter $\rho$}\label{sec:test_dyn_rho}
\subsection{Fixed vs. Dynamically Adjusted $\rho$ Test}\label{sec:test_fixvsdyn_rho}
\subsection{Initial Solution Construction Test}\label{sec:test_init_sol}
\subsection{Slack Upper Bound Enforcement Test}\label{sec:test_slack_UB}
\subsection{Budget Constraint Removal Test}\label{sec:test_bdg_constr}
\subsection{WalkMIP Strategy Test}\label{sec:test_walkMIP}



% ----- TEMPLATE -----
% \begin{figure}[thpb]
%     \centering
%     \begin{minipage}{0.6\columnwidth}
%         \centering
%         \resizebox{\linewidth}{!}{%% Creator: Matplotlib, PGF backend
%%
%% To include the figure in your LaTeX document, write
%%   \input{<filename>.pgf}
%%
%% Make sure the required packages are loaded in your preamble
%%   \usepackage{pgf}
%%
%% Also ensure that all the required font packages are loaded; for instance,
%% the lmodern package is sometimes necessary when using math font.
%%   \usepackage{lmodern}
%%
%% Figures using additional raster images can only be included by \input if
%% they are in the same directory as the main LaTeX file. For loading figures
%% from other directories you can use the `import` package
%%   \usepackage{import}
%%
%% and then include the figures with
%%   \import{<path to file>}{<filename>.pgf}
%%
%% Matplotlib used the following preamble
%%   \def\mathdefault#1{#1}
%%   \everymath=\expandafter{\the\everymath\displaystyle}
%%   \IfFileExists{scrextend.sty}{
%%     \usepackage[fontsize=10.000000pt]{scrextend}
%%   }{
%%     \renewcommand{\normalsize}{\fontsize{10.000000}{12.000000}\selectfont}
%%     \normalsize
%%   }
%%   
%%   \ifdefined\pdftexversion\else  % non-pdftex case.
%%     \usepackage{fontspec}
%%     \setmainfont{DejaVuSerif.ttf}[Path=\detokenize{/home/bisca/.global/lib/python3.12/site-packages/matplotlib/mpl-data/fonts/ttf/}]
%%     \setsansfont{DejaVuSans.ttf}[Path=\detokenize{/home/bisca/.global/lib/python3.12/site-packages/matplotlib/mpl-data/fonts/ttf/}]
%%     \setmonofont{DejaVuSansMono.ttf}[Path=\detokenize{/home/bisca/.global/lib/python3.12/site-packages/matplotlib/mpl-data/fonts/ttf/}]
%%   \fi
%%   \makeatletter\@ifpackageloaded{underscore}{}{\usepackage[strings]{underscore}}\makeatother
%%
\begingroup%
\makeatletter%
\begin{pgfpicture}%
\pgfpathrectangle{\pgfpointorigin}{\pgfqpoint{12.000000in}{8.000000in}}%
\pgfusepath{use as bounding box, clip}%
\begin{pgfscope}%
\pgfsetbuttcap%
\pgfsetmiterjoin%
\definecolor{currentfill}{rgb}{1.000000,1.000000,1.000000}%
\pgfsetfillcolor{currentfill}%
\pgfsetlinewidth{0.000000pt}%
\definecolor{currentstroke}{rgb}{1.000000,1.000000,1.000000}%
\pgfsetstrokecolor{currentstroke}%
\pgfsetdash{}{0pt}%
\pgfpathmoveto{\pgfqpoint{0.000000in}{0.000000in}}%
\pgfpathlineto{\pgfqpoint{12.000000in}{0.000000in}}%
\pgfpathlineto{\pgfqpoint{12.000000in}{8.000000in}}%
\pgfpathlineto{\pgfqpoint{0.000000in}{8.000000in}}%
\pgfpathlineto{\pgfqpoint{0.000000in}{0.000000in}}%
\pgfpathclose%
\pgfusepath{fill}%
\end{pgfscope}%
\begin{pgfscope}%
\pgfsetbuttcap%
\pgfsetmiterjoin%
\definecolor{currentfill}{rgb}{1.000000,1.000000,1.000000}%
\pgfsetfillcolor{currentfill}%
\pgfsetlinewidth{0.000000pt}%
\definecolor{currentstroke}{rgb}{0.000000,0.000000,0.000000}%
\pgfsetstrokecolor{currentstroke}%
\pgfsetstrokeopacity{0.000000}%
\pgfsetdash{}{0pt}%
\pgfpathmoveto{\pgfqpoint{0.661528in}{0.586684in}}%
\pgfpathlineto{\pgfqpoint{11.811750in}{0.586684in}}%
\pgfpathlineto{\pgfqpoint{11.811750in}{7.850000in}}%
\pgfpathlineto{\pgfqpoint{0.661528in}{7.850000in}}%
\pgfpathlineto{\pgfqpoint{0.661528in}{0.586684in}}%
\pgfpathclose%
\pgfusepath{fill}%
\end{pgfscope}%
\begin{pgfscope}%
\pgfpathrectangle{\pgfqpoint{0.661528in}{0.586684in}}{\pgfqpoint{11.150222in}{7.263316in}}%
\pgfusepath{clip}%
\pgfsetbuttcap%
\pgfsetroundjoin%
\pgfsetlinewidth{0.803000pt}%
\definecolor{currentstroke}{rgb}{0.690196,0.690196,0.690196}%
\pgfsetstrokecolor{currentstroke}%
\pgfsetstrokeopacity{0.500000}%
\pgfsetdash{{2.960000pt}{1.280000pt}}{0.000000pt}%
\pgfpathmoveto{\pgfqpoint{2.408396in}{0.586684in}}%
\pgfpathlineto{\pgfqpoint{2.408396in}{7.850000in}}%
\pgfusepath{stroke}%
\end{pgfscope}%
\begin{pgfscope}%
\pgfsetbuttcap%
\pgfsetroundjoin%
\definecolor{currentfill}{rgb}{0.000000,0.000000,0.000000}%
\pgfsetfillcolor{currentfill}%
\pgfsetlinewidth{0.803000pt}%
\definecolor{currentstroke}{rgb}{0.000000,0.000000,0.000000}%
\pgfsetstrokecolor{currentstroke}%
\pgfsetdash{}{0pt}%
\pgfsys@defobject{currentmarker}{\pgfqpoint{0.000000in}{-0.048611in}}{\pgfqpoint{0.000000in}{0.000000in}}{%
\pgfpathmoveto{\pgfqpoint{0.000000in}{0.000000in}}%
\pgfpathlineto{\pgfqpoint{0.000000in}{-0.048611in}}%
\pgfusepath{stroke,fill}%
}%
\begin{pgfscope}%
\pgfsys@transformshift{2.408396in}{0.586684in}%
\pgfsys@useobject{currentmarker}{}%
\end{pgfscope}%
\end{pgfscope}%
\begin{pgfscope}%
\definecolor{textcolor}{rgb}{0.000000,0.000000,0.000000}%
\pgfsetstrokecolor{textcolor}%
\pgfsetfillcolor{textcolor}%
\pgftext[x=2.408396in,y=0.489462in,,top]{\color{textcolor}{\rmfamily\fontsize{10.000000}{12.000000}\selectfont\catcode`\^=\active\def^{\ifmmode\sp\else\^{}\fi}\catcode`\%=\active\def%{\%}50}}%
\end{pgfscope}%
\begin{pgfscope}%
\pgfpathrectangle{\pgfqpoint{0.661528in}{0.586684in}}{\pgfqpoint{11.150222in}{7.263316in}}%
\pgfusepath{clip}%
\pgfsetbuttcap%
\pgfsetroundjoin%
\pgfsetlinewidth{0.803000pt}%
\definecolor{currentstroke}{rgb}{0.690196,0.690196,0.690196}%
\pgfsetstrokecolor{currentstroke}%
\pgfsetstrokeopacity{0.500000}%
\pgfsetdash{{2.960000pt}{1.280000pt}}{0.000000pt}%
\pgfpathmoveto{\pgfqpoint{4.266766in}{0.586684in}}%
\pgfpathlineto{\pgfqpoint{4.266766in}{7.850000in}}%
\pgfusepath{stroke}%
\end{pgfscope}%
\begin{pgfscope}%
\pgfsetbuttcap%
\pgfsetroundjoin%
\definecolor{currentfill}{rgb}{0.000000,0.000000,0.000000}%
\pgfsetfillcolor{currentfill}%
\pgfsetlinewidth{0.803000pt}%
\definecolor{currentstroke}{rgb}{0.000000,0.000000,0.000000}%
\pgfsetstrokecolor{currentstroke}%
\pgfsetdash{}{0pt}%
\pgfsys@defobject{currentmarker}{\pgfqpoint{0.000000in}{-0.048611in}}{\pgfqpoint{0.000000in}{0.000000in}}{%
\pgfpathmoveto{\pgfqpoint{0.000000in}{0.000000in}}%
\pgfpathlineto{\pgfqpoint{0.000000in}{-0.048611in}}%
\pgfusepath{stroke,fill}%
}%
\begin{pgfscope}%
\pgfsys@transformshift{4.266766in}{0.586684in}%
\pgfsys@useobject{currentmarker}{}%
\end{pgfscope}%
\end{pgfscope}%
\begin{pgfscope}%
\definecolor{textcolor}{rgb}{0.000000,0.000000,0.000000}%
\pgfsetstrokecolor{textcolor}%
\pgfsetfillcolor{textcolor}%
\pgftext[x=4.266766in,y=0.489462in,,top]{\color{textcolor}{\rmfamily\fontsize{10.000000}{12.000000}\selectfont\catcode`\^=\active\def^{\ifmmode\sp\else\^{}\fi}\catcode`\%=\active\def%{\%}100}}%
\end{pgfscope}%
\begin{pgfscope}%
\pgfpathrectangle{\pgfqpoint{0.661528in}{0.586684in}}{\pgfqpoint{11.150222in}{7.263316in}}%
\pgfusepath{clip}%
\pgfsetbuttcap%
\pgfsetroundjoin%
\pgfsetlinewidth{0.803000pt}%
\definecolor{currentstroke}{rgb}{0.690196,0.690196,0.690196}%
\pgfsetstrokecolor{currentstroke}%
\pgfsetstrokeopacity{0.500000}%
\pgfsetdash{{2.960000pt}{1.280000pt}}{0.000000pt}%
\pgfpathmoveto{\pgfqpoint{6.125137in}{0.586684in}}%
\pgfpathlineto{\pgfqpoint{6.125137in}{7.850000in}}%
\pgfusepath{stroke}%
\end{pgfscope}%
\begin{pgfscope}%
\pgfsetbuttcap%
\pgfsetroundjoin%
\definecolor{currentfill}{rgb}{0.000000,0.000000,0.000000}%
\pgfsetfillcolor{currentfill}%
\pgfsetlinewidth{0.803000pt}%
\definecolor{currentstroke}{rgb}{0.000000,0.000000,0.000000}%
\pgfsetstrokecolor{currentstroke}%
\pgfsetdash{}{0pt}%
\pgfsys@defobject{currentmarker}{\pgfqpoint{0.000000in}{-0.048611in}}{\pgfqpoint{0.000000in}{0.000000in}}{%
\pgfpathmoveto{\pgfqpoint{0.000000in}{0.000000in}}%
\pgfpathlineto{\pgfqpoint{0.000000in}{-0.048611in}}%
\pgfusepath{stroke,fill}%
}%
\begin{pgfscope}%
\pgfsys@transformshift{6.125137in}{0.586684in}%
\pgfsys@useobject{currentmarker}{}%
\end{pgfscope}%
\end{pgfscope}%
\begin{pgfscope}%
\definecolor{textcolor}{rgb}{0.000000,0.000000,0.000000}%
\pgfsetstrokecolor{textcolor}%
\pgfsetfillcolor{textcolor}%
\pgftext[x=6.125137in,y=0.489462in,,top]{\color{textcolor}{\rmfamily\fontsize{10.000000}{12.000000}\selectfont\catcode`\^=\active\def^{\ifmmode\sp\else\^{}\fi}\catcode`\%=\active\def%{\%}150}}%
\end{pgfscope}%
\begin{pgfscope}%
\pgfpathrectangle{\pgfqpoint{0.661528in}{0.586684in}}{\pgfqpoint{11.150222in}{7.263316in}}%
\pgfusepath{clip}%
\pgfsetbuttcap%
\pgfsetroundjoin%
\pgfsetlinewidth{0.803000pt}%
\definecolor{currentstroke}{rgb}{0.690196,0.690196,0.690196}%
\pgfsetstrokecolor{currentstroke}%
\pgfsetstrokeopacity{0.500000}%
\pgfsetdash{{2.960000pt}{1.280000pt}}{0.000000pt}%
\pgfpathmoveto{\pgfqpoint{7.983507in}{0.586684in}}%
\pgfpathlineto{\pgfqpoint{7.983507in}{7.850000in}}%
\pgfusepath{stroke}%
\end{pgfscope}%
\begin{pgfscope}%
\pgfsetbuttcap%
\pgfsetroundjoin%
\definecolor{currentfill}{rgb}{0.000000,0.000000,0.000000}%
\pgfsetfillcolor{currentfill}%
\pgfsetlinewidth{0.803000pt}%
\definecolor{currentstroke}{rgb}{0.000000,0.000000,0.000000}%
\pgfsetstrokecolor{currentstroke}%
\pgfsetdash{}{0pt}%
\pgfsys@defobject{currentmarker}{\pgfqpoint{0.000000in}{-0.048611in}}{\pgfqpoint{0.000000in}{0.000000in}}{%
\pgfpathmoveto{\pgfqpoint{0.000000in}{0.000000in}}%
\pgfpathlineto{\pgfqpoint{0.000000in}{-0.048611in}}%
\pgfusepath{stroke,fill}%
}%
\begin{pgfscope}%
\pgfsys@transformshift{7.983507in}{0.586684in}%
\pgfsys@useobject{currentmarker}{}%
\end{pgfscope}%
\end{pgfscope}%
\begin{pgfscope}%
\definecolor{textcolor}{rgb}{0.000000,0.000000,0.000000}%
\pgfsetstrokecolor{textcolor}%
\pgfsetfillcolor{textcolor}%
\pgftext[x=7.983507in,y=0.489462in,,top]{\color{textcolor}{\rmfamily\fontsize{10.000000}{12.000000}\selectfont\catcode`\^=\active\def^{\ifmmode\sp\else\^{}\fi}\catcode`\%=\active\def%{\%}200}}%
\end{pgfscope}%
\begin{pgfscope}%
\pgfpathrectangle{\pgfqpoint{0.661528in}{0.586684in}}{\pgfqpoint{11.150222in}{7.263316in}}%
\pgfusepath{clip}%
\pgfsetbuttcap%
\pgfsetroundjoin%
\pgfsetlinewidth{0.803000pt}%
\definecolor{currentstroke}{rgb}{0.690196,0.690196,0.690196}%
\pgfsetstrokecolor{currentstroke}%
\pgfsetstrokeopacity{0.500000}%
\pgfsetdash{{2.960000pt}{1.280000pt}}{0.000000pt}%
\pgfpathmoveto{\pgfqpoint{9.841877in}{0.586684in}}%
\pgfpathlineto{\pgfqpoint{9.841877in}{7.850000in}}%
\pgfusepath{stroke}%
\end{pgfscope}%
\begin{pgfscope}%
\pgfsetbuttcap%
\pgfsetroundjoin%
\definecolor{currentfill}{rgb}{0.000000,0.000000,0.000000}%
\pgfsetfillcolor{currentfill}%
\pgfsetlinewidth{0.803000pt}%
\definecolor{currentstroke}{rgb}{0.000000,0.000000,0.000000}%
\pgfsetstrokecolor{currentstroke}%
\pgfsetdash{}{0pt}%
\pgfsys@defobject{currentmarker}{\pgfqpoint{0.000000in}{-0.048611in}}{\pgfqpoint{0.000000in}{0.000000in}}{%
\pgfpathmoveto{\pgfqpoint{0.000000in}{0.000000in}}%
\pgfpathlineto{\pgfqpoint{0.000000in}{-0.048611in}}%
\pgfusepath{stroke,fill}%
}%
\begin{pgfscope}%
\pgfsys@transformshift{9.841877in}{0.586684in}%
\pgfsys@useobject{currentmarker}{}%
\end{pgfscope}%
\end{pgfscope}%
\begin{pgfscope}%
\definecolor{textcolor}{rgb}{0.000000,0.000000,0.000000}%
\pgfsetstrokecolor{textcolor}%
\pgfsetfillcolor{textcolor}%
\pgftext[x=9.841877in,y=0.489462in,,top]{\color{textcolor}{\rmfamily\fontsize{10.000000}{12.000000}\selectfont\catcode`\^=\active\def^{\ifmmode\sp\else\^{}\fi}\catcode`\%=\active\def%{\%}250}}%
\end{pgfscope}%
\begin{pgfscope}%
\pgfpathrectangle{\pgfqpoint{0.661528in}{0.586684in}}{\pgfqpoint{11.150222in}{7.263316in}}%
\pgfusepath{clip}%
\pgfsetbuttcap%
\pgfsetroundjoin%
\pgfsetlinewidth{0.803000pt}%
\definecolor{currentstroke}{rgb}{0.690196,0.690196,0.690196}%
\pgfsetstrokecolor{currentstroke}%
\pgfsetstrokeopacity{0.500000}%
\pgfsetdash{{2.960000pt}{1.280000pt}}{0.000000pt}%
\pgfpathmoveto{\pgfqpoint{11.700248in}{0.586684in}}%
\pgfpathlineto{\pgfqpoint{11.700248in}{7.850000in}}%
\pgfusepath{stroke}%
\end{pgfscope}%
\begin{pgfscope}%
\pgfsetbuttcap%
\pgfsetroundjoin%
\definecolor{currentfill}{rgb}{0.000000,0.000000,0.000000}%
\pgfsetfillcolor{currentfill}%
\pgfsetlinewidth{0.803000pt}%
\definecolor{currentstroke}{rgb}{0.000000,0.000000,0.000000}%
\pgfsetstrokecolor{currentstroke}%
\pgfsetdash{}{0pt}%
\pgfsys@defobject{currentmarker}{\pgfqpoint{0.000000in}{-0.048611in}}{\pgfqpoint{0.000000in}{0.000000in}}{%
\pgfpathmoveto{\pgfqpoint{0.000000in}{0.000000in}}%
\pgfpathlineto{\pgfqpoint{0.000000in}{-0.048611in}}%
\pgfusepath{stroke,fill}%
}%
\begin{pgfscope}%
\pgfsys@transformshift{11.700248in}{0.586684in}%
\pgfsys@useobject{currentmarker}{}%
\end{pgfscope}%
\end{pgfscope}%
\begin{pgfscope}%
\definecolor{textcolor}{rgb}{0.000000,0.000000,0.000000}%
\pgfsetstrokecolor{textcolor}%
\pgfsetfillcolor{textcolor}%
\pgftext[x=11.700248in,y=0.489462in,,top]{\color{textcolor}{\rmfamily\fontsize{10.000000}{12.000000}\selectfont\catcode`\^=\active\def^{\ifmmode\sp\else\^{}\fi}\catcode`\%=\active\def%{\%}300}}%
\end{pgfscope}%
\begin{pgfscope}%
\definecolor{textcolor}{rgb}{0.000000,0.000000,0.000000}%
\pgfsetstrokecolor{textcolor}%
\pgfsetfillcolor{textcolor}%
\pgftext[x=6.236639in,y=0.299493in,,top]{\color{textcolor}{\rmfamily\fontsize{10.000000}{12.000000}\selectfont\catcode`\^=\active\def^{\ifmmode\sp\else\^{}\fi}\catcode`\%=\active\def%{\%}Computation Time (sec)}}%
\end{pgfscope}%
\begin{pgfscope}%
\pgfpathrectangle{\pgfqpoint{0.661528in}{0.586684in}}{\pgfqpoint{11.150222in}{7.263316in}}%
\pgfusepath{clip}%
\pgfsetbuttcap%
\pgfsetroundjoin%
\pgfsetlinewidth{0.803000pt}%
\definecolor{currentstroke}{rgb}{0.690196,0.690196,0.690196}%
\pgfsetstrokecolor{currentstroke}%
\pgfsetstrokeopacity{0.500000}%
\pgfsetdash{{2.960000pt}{1.280000pt}}{0.000000pt}%
\pgfpathmoveto{\pgfqpoint{0.661528in}{1.177480in}}%
\pgfpathlineto{\pgfqpoint{11.811750in}{1.177480in}}%
\pgfusepath{stroke}%
\end{pgfscope}%
\begin{pgfscope}%
\pgfsetbuttcap%
\pgfsetroundjoin%
\definecolor{currentfill}{rgb}{0.000000,0.000000,0.000000}%
\pgfsetfillcolor{currentfill}%
\pgfsetlinewidth{0.803000pt}%
\definecolor{currentstroke}{rgb}{0.000000,0.000000,0.000000}%
\pgfsetstrokecolor{currentstroke}%
\pgfsetdash{}{0pt}%
\pgfsys@defobject{currentmarker}{\pgfqpoint{-0.048611in}{0.000000in}}{\pgfqpoint{-0.000000in}{0.000000in}}{%
\pgfpathmoveto{\pgfqpoint{-0.000000in}{0.000000in}}%
\pgfpathlineto{\pgfqpoint{-0.048611in}{0.000000in}}%
\pgfusepath{stroke,fill}%
}%
\begin{pgfscope}%
\pgfsys@transformshift{0.661528in}{1.177480in}%
\pgfsys@useobject{currentmarker}{}%
\end{pgfscope}%
\end{pgfscope}%
\begin{pgfscope}%
\definecolor{textcolor}{rgb}{0.000000,0.000000,0.000000}%
\pgfsetstrokecolor{textcolor}%
\pgfsetfillcolor{textcolor}%
\pgftext[x=0.343426in, y=1.124719in, left, base]{\color{textcolor}{\rmfamily\fontsize{10.000000}{12.000000}\selectfont\catcode`\^=\active\def^{\ifmmode\sp\else\^{}\fi}\catcode`\%=\active\def%{\%}0.1}}%
\end{pgfscope}%
\begin{pgfscope}%
\pgfpathrectangle{\pgfqpoint{0.661528in}{0.586684in}}{\pgfqpoint{11.150222in}{7.263316in}}%
\pgfusepath{clip}%
\pgfsetbuttcap%
\pgfsetroundjoin%
\pgfsetlinewidth{0.803000pt}%
\definecolor{currentstroke}{rgb}{0.690196,0.690196,0.690196}%
\pgfsetstrokecolor{currentstroke}%
\pgfsetstrokeopacity{0.500000}%
\pgfsetdash{{2.960000pt}{1.280000pt}}{0.000000pt}%
\pgfpathmoveto{\pgfqpoint{0.661528in}{2.133180in}}%
\pgfpathlineto{\pgfqpoint{11.811750in}{2.133180in}}%
\pgfusepath{stroke}%
\end{pgfscope}%
\begin{pgfscope}%
\pgfsetbuttcap%
\pgfsetroundjoin%
\definecolor{currentfill}{rgb}{0.000000,0.000000,0.000000}%
\pgfsetfillcolor{currentfill}%
\pgfsetlinewidth{0.803000pt}%
\definecolor{currentstroke}{rgb}{0.000000,0.000000,0.000000}%
\pgfsetstrokecolor{currentstroke}%
\pgfsetdash{}{0pt}%
\pgfsys@defobject{currentmarker}{\pgfqpoint{-0.048611in}{0.000000in}}{\pgfqpoint{-0.000000in}{0.000000in}}{%
\pgfpathmoveto{\pgfqpoint{-0.000000in}{0.000000in}}%
\pgfpathlineto{\pgfqpoint{-0.048611in}{0.000000in}}%
\pgfusepath{stroke,fill}%
}%
\begin{pgfscope}%
\pgfsys@transformshift{0.661528in}{2.133180in}%
\pgfsys@useobject{currentmarker}{}%
\end{pgfscope}%
\end{pgfscope}%
\begin{pgfscope}%
\definecolor{textcolor}{rgb}{0.000000,0.000000,0.000000}%
\pgfsetstrokecolor{textcolor}%
\pgfsetfillcolor{textcolor}%
\pgftext[x=0.343426in, y=2.080418in, left, base]{\color{textcolor}{\rmfamily\fontsize{10.000000}{12.000000}\selectfont\catcode`\^=\active\def^{\ifmmode\sp\else\^{}\fi}\catcode`\%=\active\def%{\%}0.2}}%
\end{pgfscope}%
\begin{pgfscope}%
\pgfpathrectangle{\pgfqpoint{0.661528in}{0.586684in}}{\pgfqpoint{11.150222in}{7.263316in}}%
\pgfusepath{clip}%
\pgfsetbuttcap%
\pgfsetroundjoin%
\pgfsetlinewidth{0.803000pt}%
\definecolor{currentstroke}{rgb}{0.690196,0.690196,0.690196}%
\pgfsetstrokecolor{currentstroke}%
\pgfsetstrokeopacity{0.500000}%
\pgfsetdash{{2.960000pt}{1.280000pt}}{0.000000pt}%
\pgfpathmoveto{\pgfqpoint{0.661528in}{3.088879in}}%
\pgfpathlineto{\pgfqpoint{11.811750in}{3.088879in}}%
\pgfusepath{stroke}%
\end{pgfscope}%
\begin{pgfscope}%
\pgfsetbuttcap%
\pgfsetroundjoin%
\definecolor{currentfill}{rgb}{0.000000,0.000000,0.000000}%
\pgfsetfillcolor{currentfill}%
\pgfsetlinewidth{0.803000pt}%
\definecolor{currentstroke}{rgb}{0.000000,0.000000,0.000000}%
\pgfsetstrokecolor{currentstroke}%
\pgfsetdash{}{0pt}%
\pgfsys@defobject{currentmarker}{\pgfqpoint{-0.048611in}{0.000000in}}{\pgfqpoint{-0.000000in}{0.000000in}}{%
\pgfpathmoveto{\pgfqpoint{-0.000000in}{0.000000in}}%
\pgfpathlineto{\pgfqpoint{-0.048611in}{0.000000in}}%
\pgfusepath{stroke,fill}%
}%
\begin{pgfscope}%
\pgfsys@transformshift{0.661528in}{3.088879in}%
\pgfsys@useobject{currentmarker}{}%
\end{pgfscope}%
\end{pgfscope}%
\begin{pgfscope}%
\definecolor{textcolor}{rgb}{0.000000,0.000000,0.000000}%
\pgfsetstrokecolor{textcolor}%
\pgfsetfillcolor{textcolor}%
\pgftext[x=0.343426in, y=3.036117in, left, base]{\color{textcolor}{\rmfamily\fontsize{10.000000}{12.000000}\selectfont\catcode`\^=\active\def^{\ifmmode\sp\else\^{}\fi}\catcode`\%=\active\def%{\%}0.3}}%
\end{pgfscope}%
\begin{pgfscope}%
\pgfpathrectangle{\pgfqpoint{0.661528in}{0.586684in}}{\pgfqpoint{11.150222in}{7.263316in}}%
\pgfusepath{clip}%
\pgfsetbuttcap%
\pgfsetroundjoin%
\pgfsetlinewidth{0.803000pt}%
\definecolor{currentstroke}{rgb}{0.690196,0.690196,0.690196}%
\pgfsetstrokecolor{currentstroke}%
\pgfsetstrokeopacity{0.500000}%
\pgfsetdash{{2.960000pt}{1.280000pt}}{0.000000pt}%
\pgfpathmoveto{\pgfqpoint{0.661528in}{4.044578in}}%
\pgfpathlineto{\pgfqpoint{11.811750in}{4.044578in}}%
\pgfusepath{stroke}%
\end{pgfscope}%
\begin{pgfscope}%
\pgfsetbuttcap%
\pgfsetroundjoin%
\definecolor{currentfill}{rgb}{0.000000,0.000000,0.000000}%
\pgfsetfillcolor{currentfill}%
\pgfsetlinewidth{0.803000pt}%
\definecolor{currentstroke}{rgb}{0.000000,0.000000,0.000000}%
\pgfsetstrokecolor{currentstroke}%
\pgfsetdash{}{0pt}%
\pgfsys@defobject{currentmarker}{\pgfqpoint{-0.048611in}{0.000000in}}{\pgfqpoint{-0.000000in}{0.000000in}}{%
\pgfpathmoveto{\pgfqpoint{-0.000000in}{0.000000in}}%
\pgfpathlineto{\pgfqpoint{-0.048611in}{0.000000in}}%
\pgfusepath{stroke,fill}%
}%
\begin{pgfscope}%
\pgfsys@transformshift{0.661528in}{4.044578in}%
\pgfsys@useobject{currentmarker}{}%
\end{pgfscope}%
\end{pgfscope}%
\begin{pgfscope}%
\definecolor{textcolor}{rgb}{0.000000,0.000000,0.000000}%
\pgfsetstrokecolor{textcolor}%
\pgfsetfillcolor{textcolor}%
\pgftext[x=0.343426in, y=3.991817in, left, base]{\color{textcolor}{\rmfamily\fontsize{10.000000}{12.000000}\selectfont\catcode`\^=\active\def^{\ifmmode\sp\else\^{}\fi}\catcode`\%=\active\def%{\%}0.4}}%
\end{pgfscope}%
\begin{pgfscope}%
\pgfpathrectangle{\pgfqpoint{0.661528in}{0.586684in}}{\pgfqpoint{11.150222in}{7.263316in}}%
\pgfusepath{clip}%
\pgfsetbuttcap%
\pgfsetroundjoin%
\pgfsetlinewidth{0.803000pt}%
\definecolor{currentstroke}{rgb}{0.690196,0.690196,0.690196}%
\pgfsetstrokecolor{currentstroke}%
\pgfsetstrokeopacity{0.500000}%
\pgfsetdash{{2.960000pt}{1.280000pt}}{0.000000pt}%
\pgfpathmoveto{\pgfqpoint{0.661528in}{5.000278in}}%
\pgfpathlineto{\pgfqpoint{11.811750in}{5.000278in}}%
\pgfusepath{stroke}%
\end{pgfscope}%
\begin{pgfscope}%
\pgfsetbuttcap%
\pgfsetroundjoin%
\definecolor{currentfill}{rgb}{0.000000,0.000000,0.000000}%
\pgfsetfillcolor{currentfill}%
\pgfsetlinewidth{0.803000pt}%
\definecolor{currentstroke}{rgb}{0.000000,0.000000,0.000000}%
\pgfsetstrokecolor{currentstroke}%
\pgfsetdash{}{0pt}%
\pgfsys@defobject{currentmarker}{\pgfqpoint{-0.048611in}{0.000000in}}{\pgfqpoint{-0.000000in}{0.000000in}}{%
\pgfpathmoveto{\pgfqpoint{-0.000000in}{0.000000in}}%
\pgfpathlineto{\pgfqpoint{-0.048611in}{0.000000in}}%
\pgfusepath{stroke,fill}%
}%
\begin{pgfscope}%
\pgfsys@transformshift{0.661528in}{5.000278in}%
\pgfsys@useobject{currentmarker}{}%
\end{pgfscope}%
\end{pgfscope}%
\begin{pgfscope}%
\definecolor{textcolor}{rgb}{0.000000,0.000000,0.000000}%
\pgfsetstrokecolor{textcolor}%
\pgfsetfillcolor{textcolor}%
\pgftext[x=0.343426in, y=4.947516in, left, base]{\color{textcolor}{\rmfamily\fontsize{10.000000}{12.000000}\selectfont\catcode`\^=\active\def^{\ifmmode\sp\else\^{}\fi}\catcode`\%=\active\def%{\%}0.5}}%
\end{pgfscope}%
\begin{pgfscope}%
\pgfpathrectangle{\pgfqpoint{0.661528in}{0.586684in}}{\pgfqpoint{11.150222in}{7.263316in}}%
\pgfusepath{clip}%
\pgfsetbuttcap%
\pgfsetroundjoin%
\pgfsetlinewidth{0.803000pt}%
\definecolor{currentstroke}{rgb}{0.690196,0.690196,0.690196}%
\pgfsetstrokecolor{currentstroke}%
\pgfsetstrokeopacity{0.500000}%
\pgfsetdash{{2.960000pt}{1.280000pt}}{0.000000pt}%
\pgfpathmoveto{\pgfqpoint{0.661528in}{5.955977in}}%
\pgfpathlineto{\pgfqpoint{11.811750in}{5.955977in}}%
\pgfusepath{stroke}%
\end{pgfscope}%
\begin{pgfscope}%
\pgfsetbuttcap%
\pgfsetroundjoin%
\definecolor{currentfill}{rgb}{0.000000,0.000000,0.000000}%
\pgfsetfillcolor{currentfill}%
\pgfsetlinewidth{0.803000pt}%
\definecolor{currentstroke}{rgb}{0.000000,0.000000,0.000000}%
\pgfsetstrokecolor{currentstroke}%
\pgfsetdash{}{0pt}%
\pgfsys@defobject{currentmarker}{\pgfqpoint{-0.048611in}{0.000000in}}{\pgfqpoint{-0.000000in}{0.000000in}}{%
\pgfpathmoveto{\pgfqpoint{-0.000000in}{0.000000in}}%
\pgfpathlineto{\pgfqpoint{-0.048611in}{0.000000in}}%
\pgfusepath{stroke,fill}%
}%
\begin{pgfscope}%
\pgfsys@transformshift{0.661528in}{5.955977in}%
\pgfsys@useobject{currentmarker}{}%
\end{pgfscope}%
\end{pgfscope}%
\begin{pgfscope}%
\definecolor{textcolor}{rgb}{0.000000,0.000000,0.000000}%
\pgfsetstrokecolor{textcolor}%
\pgfsetfillcolor{textcolor}%
\pgftext[x=0.343426in, y=5.903216in, left, base]{\color{textcolor}{\rmfamily\fontsize{10.000000}{12.000000}\selectfont\catcode`\^=\active\def^{\ifmmode\sp\else\^{}\fi}\catcode`\%=\active\def%{\%}0.6}}%
\end{pgfscope}%
\begin{pgfscope}%
\pgfpathrectangle{\pgfqpoint{0.661528in}{0.586684in}}{\pgfqpoint{11.150222in}{7.263316in}}%
\pgfusepath{clip}%
\pgfsetbuttcap%
\pgfsetroundjoin%
\pgfsetlinewidth{0.803000pt}%
\definecolor{currentstroke}{rgb}{0.690196,0.690196,0.690196}%
\pgfsetstrokecolor{currentstroke}%
\pgfsetstrokeopacity{0.500000}%
\pgfsetdash{{2.960000pt}{1.280000pt}}{0.000000pt}%
\pgfpathmoveto{\pgfqpoint{0.661528in}{6.911677in}}%
\pgfpathlineto{\pgfqpoint{11.811750in}{6.911677in}}%
\pgfusepath{stroke}%
\end{pgfscope}%
\begin{pgfscope}%
\pgfsetbuttcap%
\pgfsetroundjoin%
\definecolor{currentfill}{rgb}{0.000000,0.000000,0.000000}%
\pgfsetfillcolor{currentfill}%
\pgfsetlinewidth{0.803000pt}%
\definecolor{currentstroke}{rgb}{0.000000,0.000000,0.000000}%
\pgfsetstrokecolor{currentstroke}%
\pgfsetdash{}{0pt}%
\pgfsys@defobject{currentmarker}{\pgfqpoint{-0.048611in}{0.000000in}}{\pgfqpoint{-0.000000in}{0.000000in}}{%
\pgfpathmoveto{\pgfqpoint{-0.000000in}{0.000000in}}%
\pgfpathlineto{\pgfqpoint{-0.048611in}{0.000000in}}%
\pgfusepath{stroke,fill}%
}%
\begin{pgfscope}%
\pgfsys@transformshift{0.661528in}{6.911677in}%
\pgfsys@useobject{currentmarker}{}%
\end{pgfscope}%
\end{pgfscope}%
\begin{pgfscope}%
\definecolor{textcolor}{rgb}{0.000000,0.000000,0.000000}%
\pgfsetstrokecolor{textcolor}%
\pgfsetfillcolor{textcolor}%
\pgftext[x=0.343426in, y=6.858915in, left, base]{\color{textcolor}{\rmfamily\fontsize{10.000000}{12.000000}\selectfont\catcode`\^=\active\def^{\ifmmode\sp\else\^{}\fi}\catcode`\%=\active\def%{\%}0.7}}%
\end{pgfscope}%
\begin{pgfscope}%
\definecolor{textcolor}{rgb}{0.000000,0.000000,0.000000}%
\pgfsetstrokecolor{textcolor}%
\pgfsetfillcolor{textcolor}%
\pgftext[x=0.287871in,y=4.218342in,,bottom,rotate=90.000000]{\color{textcolor}{\rmfamily\fontsize{10.000000}{12.000000}\selectfont\catcode`\^=\active\def^{\ifmmode\sp\else\^{}\fi}\catcode`\%=\active\def%{\%}Success Rate}}%
\end{pgfscope}%
\begin{pgfscope}%
\pgfpathrectangle{\pgfqpoint{0.661528in}{0.586684in}}{\pgfqpoint{11.150222in}{7.263316in}}%
\pgfusepath{clip}%
\pgfsetrectcap%
\pgfsetroundjoin%
\pgfsetlinewidth{2.007500pt}%
\definecolor{currentstroke}{rgb}{0.121569,0.466667,0.705882}%
\pgfsetstrokecolor{currentstroke}%
\pgfsetdash{}{0pt}%
\pgfpathmoveto{\pgfqpoint{0.661528in}{5.869096in}}%
\pgfpathlineto{\pgfqpoint{0.773030in}{6.129741in}}%
\pgfpathlineto{\pgfqpoint{0.884532in}{6.651032in}}%
\pgfpathlineto{\pgfqpoint{0.996034in}{6.737913in}}%
\pgfpathlineto{\pgfqpoint{1.107537in}{6.737913in}}%
\pgfpathlineto{\pgfqpoint{1.219039in}{6.737913in}}%
\pgfpathlineto{\pgfqpoint{1.330541in}{6.737913in}}%
\pgfpathlineto{\pgfqpoint{1.442043in}{6.824795in}}%
\pgfpathlineto{\pgfqpoint{1.553546in}{6.824795in}}%
\pgfpathlineto{\pgfqpoint{1.665048in}{6.824795in}}%
\pgfpathlineto{\pgfqpoint{1.776550in}{6.911677in}}%
\pgfpathlineto{\pgfqpoint{1.888052in}{6.911677in}}%
\pgfpathlineto{\pgfqpoint{1.999554in}{6.911677in}}%
\pgfpathlineto{\pgfqpoint{2.111057in}{6.911677in}}%
\pgfpathlineto{\pgfqpoint{2.222559in}{6.998559in}}%
\pgfpathlineto{\pgfqpoint{2.334061in}{6.998559in}}%
\pgfpathlineto{\pgfqpoint{2.445563in}{6.998559in}}%
\pgfpathlineto{\pgfqpoint{2.557066in}{6.998559in}}%
\pgfpathlineto{\pgfqpoint{2.668568in}{6.998559in}}%
\pgfpathlineto{\pgfqpoint{2.780070in}{7.085440in}}%
\pgfpathlineto{\pgfqpoint{2.891572in}{7.085440in}}%
\pgfpathlineto{\pgfqpoint{3.003074in}{7.085440in}}%
\pgfpathlineto{\pgfqpoint{3.114577in}{7.085440in}}%
\pgfpathlineto{\pgfqpoint{3.226079in}{7.085440in}}%
\pgfpathlineto{\pgfqpoint{3.337581in}{7.085440in}}%
\pgfpathlineto{\pgfqpoint{3.449083in}{7.085440in}}%
\pgfpathlineto{\pgfqpoint{3.560586in}{7.085440in}}%
\pgfpathlineto{\pgfqpoint{3.672088in}{7.085440in}}%
\pgfpathlineto{\pgfqpoint{3.783590in}{7.085440in}}%
\pgfpathlineto{\pgfqpoint{3.895092in}{7.085440in}}%
\pgfpathlineto{\pgfqpoint{4.006594in}{7.085440in}}%
\pgfpathlineto{\pgfqpoint{4.118097in}{7.085440in}}%
\pgfpathlineto{\pgfqpoint{4.229599in}{7.085440in}}%
\pgfpathlineto{\pgfqpoint{4.341101in}{7.085440in}}%
\pgfpathlineto{\pgfqpoint{4.452603in}{7.085440in}}%
\pgfpathlineto{\pgfqpoint{4.564106in}{7.085440in}}%
\pgfpathlineto{\pgfqpoint{4.675608in}{7.085440in}}%
\pgfpathlineto{\pgfqpoint{4.787110in}{7.085440in}}%
\pgfpathlineto{\pgfqpoint{4.898612in}{7.085440in}}%
\pgfpathlineto{\pgfqpoint{5.010114in}{7.085440in}}%
\pgfpathlineto{\pgfqpoint{5.121617in}{7.085440in}}%
\pgfpathlineto{\pgfqpoint{5.233119in}{7.085440in}}%
\pgfpathlineto{\pgfqpoint{5.344621in}{7.085440in}}%
\pgfpathlineto{\pgfqpoint{5.456123in}{7.085440in}}%
\pgfpathlineto{\pgfqpoint{5.567626in}{7.085440in}}%
\pgfpathlineto{\pgfqpoint{5.679128in}{7.085440in}}%
\pgfpathlineto{\pgfqpoint{5.790630in}{7.085440in}}%
\pgfpathlineto{\pgfqpoint{5.902132in}{7.085440in}}%
\pgfpathlineto{\pgfqpoint{6.013634in}{7.172322in}}%
\pgfpathlineto{\pgfqpoint{6.125137in}{7.172322in}}%
\pgfpathlineto{\pgfqpoint{6.236639in}{7.172322in}}%
\pgfpathlineto{\pgfqpoint{6.348141in}{7.259204in}}%
\pgfpathlineto{\pgfqpoint{6.459643in}{7.259204in}}%
\pgfpathlineto{\pgfqpoint{6.571146in}{7.346086in}}%
\pgfpathlineto{\pgfqpoint{6.682648in}{7.346086in}}%
\pgfpathlineto{\pgfqpoint{6.794150in}{7.346086in}}%
\pgfpathlineto{\pgfqpoint{6.905652in}{7.346086in}}%
\pgfpathlineto{\pgfqpoint{7.017154in}{7.346086in}}%
\pgfpathlineto{\pgfqpoint{7.128657in}{7.346086in}}%
\pgfpathlineto{\pgfqpoint{7.240159in}{7.346086in}}%
\pgfpathlineto{\pgfqpoint{7.351661in}{7.346086in}}%
\pgfpathlineto{\pgfqpoint{7.463163in}{7.346086in}}%
\pgfpathlineto{\pgfqpoint{7.574666in}{7.346086in}}%
\pgfpathlineto{\pgfqpoint{7.686168in}{7.346086in}}%
\pgfpathlineto{\pgfqpoint{7.797670in}{7.346086in}}%
\pgfpathlineto{\pgfqpoint{7.909172in}{7.346086in}}%
\pgfpathlineto{\pgfqpoint{8.020674in}{7.346086in}}%
\pgfpathlineto{\pgfqpoint{8.132177in}{7.346086in}}%
\pgfpathlineto{\pgfqpoint{8.243679in}{7.346086in}}%
\pgfpathlineto{\pgfqpoint{8.355181in}{7.346086in}}%
\pgfpathlineto{\pgfqpoint{8.466683in}{7.346086in}}%
\pgfpathlineto{\pgfqpoint{8.578186in}{7.346086in}}%
\pgfpathlineto{\pgfqpoint{8.689688in}{7.346086in}}%
\pgfpathlineto{\pgfqpoint{8.801190in}{7.346086in}}%
\pgfpathlineto{\pgfqpoint{8.912692in}{7.346086in}}%
\pgfpathlineto{\pgfqpoint{9.024194in}{7.346086in}}%
\pgfpathlineto{\pgfqpoint{9.135697in}{7.346086in}}%
\pgfpathlineto{\pgfqpoint{9.247199in}{7.346086in}}%
\pgfpathlineto{\pgfqpoint{9.358701in}{7.346086in}}%
\pgfpathlineto{\pgfqpoint{9.470203in}{7.346086in}}%
\pgfpathlineto{\pgfqpoint{9.581706in}{7.346086in}}%
\pgfpathlineto{\pgfqpoint{9.693208in}{7.346086in}}%
\pgfpathlineto{\pgfqpoint{9.804710in}{7.346086in}}%
\pgfpathlineto{\pgfqpoint{9.916212in}{7.346086in}}%
\pgfpathlineto{\pgfqpoint{10.027714in}{7.346086in}}%
\pgfpathlineto{\pgfqpoint{10.139217in}{7.346086in}}%
\pgfpathlineto{\pgfqpoint{10.250719in}{7.346086in}}%
\pgfpathlineto{\pgfqpoint{10.362221in}{7.346086in}}%
\pgfpathlineto{\pgfqpoint{10.473723in}{7.346086in}}%
\pgfpathlineto{\pgfqpoint{10.585226in}{7.346086in}}%
\pgfpathlineto{\pgfqpoint{10.696728in}{7.346086in}}%
\pgfpathlineto{\pgfqpoint{10.808230in}{7.346086in}}%
\pgfpathlineto{\pgfqpoint{10.919732in}{7.519849in}}%
\pgfpathlineto{\pgfqpoint{11.031234in}{7.519849in}}%
\pgfpathlineto{\pgfqpoint{11.142737in}{7.519849in}}%
\pgfpathlineto{\pgfqpoint{11.254239in}{7.519849in}}%
\pgfpathlineto{\pgfqpoint{11.365741in}{7.519849in}}%
\pgfpathlineto{\pgfqpoint{11.477243in}{7.519849in}}%
\pgfpathlineto{\pgfqpoint{11.588746in}{7.519849in}}%
\pgfpathlineto{\pgfqpoint{11.700248in}{7.519849in}}%
\pgfpathlineto{\pgfqpoint{11.811750in}{7.519849in}}%
\pgfusepath{stroke}%
\end{pgfscope}%
\begin{pgfscope}%
\pgfpathrectangle{\pgfqpoint{0.661528in}{0.586684in}}{\pgfqpoint{11.150222in}{7.263316in}}%
\pgfusepath{clip}%
\pgfsetrectcap%
\pgfsetroundjoin%
\pgfsetlinewidth{2.007500pt}%
\definecolor{currentstroke}{rgb}{1.000000,0.498039,0.054902}%
\pgfsetstrokecolor{currentstroke}%
\pgfsetdash{}{0pt}%
\pgfpathmoveto{\pgfqpoint{0.661528in}{2.046298in}}%
\pgfpathlineto{\pgfqpoint{0.773030in}{3.175761in}}%
\pgfpathlineto{\pgfqpoint{0.884532in}{3.349524in}}%
\pgfpathlineto{\pgfqpoint{0.996034in}{3.697051in}}%
\pgfpathlineto{\pgfqpoint{1.107537in}{3.957697in}}%
\pgfpathlineto{\pgfqpoint{1.219039in}{4.218342in}}%
\pgfpathlineto{\pgfqpoint{1.330541in}{4.305224in}}%
\pgfpathlineto{\pgfqpoint{1.442043in}{4.392106in}}%
\pgfpathlineto{\pgfqpoint{1.553546in}{4.392106in}}%
\pgfpathlineto{\pgfqpoint{1.665048in}{4.478987in}}%
\pgfpathlineto{\pgfqpoint{1.776550in}{4.652751in}}%
\pgfpathlineto{\pgfqpoint{1.888052in}{4.739633in}}%
\pgfpathlineto{\pgfqpoint{1.999554in}{4.826514in}}%
\pgfpathlineto{\pgfqpoint{2.111057in}{5.087160in}}%
\pgfpathlineto{\pgfqpoint{2.222559in}{5.174041in}}%
\pgfpathlineto{\pgfqpoint{2.334061in}{5.174041in}}%
\pgfpathlineto{\pgfqpoint{2.445563in}{5.260923in}}%
\pgfpathlineto{\pgfqpoint{2.557066in}{5.260923in}}%
\pgfpathlineto{\pgfqpoint{2.668568in}{5.260923in}}%
\pgfpathlineto{\pgfqpoint{2.780070in}{5.260923in}}%
\pgfpathlineto{\pgfqpoint{2.891572in}{5.260923in}}%
\pgfpathlineto{\pgfqpoint{3.003074in}{5.260923in}}%
\pgfpathlineto{\pgfqpoint{3.114577in}{5.260923in}}%
\pgfpathlineto{\pgfqpoint{3.226079in}{5.347805in}}%
\pgfpathlineto{\pgfqpoint{3.337581in}{5.347805in}}%
\pgfpathlineto{\pgfqpoint{3.449083in}{5.347805in}}%
\pgfpathlineto{\pgfqpoint{3.560586in}{5.347805in}}%
\pgfpathlineto{\pgfqpoint{3.672088in}{5.347805in}}%
\pgfpathlineto{\pgfqpoint{3.783590in}{5.347805in}}%
\pgfpathlineto{\pgfqpoint{3.895092in}{5.347805in}}%
\pgfpathlineto{\pgfqpoint{4.006594in}{5.434687in}}%
\pgfpathlineto{\pgfqpoint{4.118097in}{5.521569in}}%
\pgfpathlineto{\pgfqpoint{4.229599in}{5.521569in}}%
\pgfpathlineto{\pgfqpoint{4.341101in}{5.521569in}}%
\pgfpathlineto{\pgfqpoint{4.452603in}{5.521569in}}%
\pgfpathlineto{\pgfqpoint{4.564106in}{5.608450in}}%
\pgfpathlineto{\pgfqpoint{4.675608in}{5.695332in}}%
\pgfpathlineto{\pgfqpoint{4.787110in}{5.695332in}}%
\pgfpathlineto{\pgfqpoint{4.898612in}{5.695332in}}%
\pgfpathlineto{\pgfqpoint{5.010114in}{5.695332in}}%
\pgfpathlineto{\pgfqpoint{5.121617in}{5.695332in}}%
\pgfpathlineto{\pgfqpoint{5.233119in}{5.782214in}}%
\pgfpathlineto{\pgfqpoint{5.344621in}{5.782214in}}%
\pgfpathlineto{\pgfqpoint{5.456123in}{5.782214in}}%
\pgfpathlineto{\pgfqpoint{5.567626in}{5.782214in}}%
\pgfpathlineto{\pgfqpoint{5.679128in}{5.782214in}}%
\pgfpathlineto{\pgfqpoint{5.790630in}{5.782214in}}%
\pgfpathlineto{\pgfqpoint{5.902132in}{5.782214in}}%
\pgfpathlineto{\pgfqpoint{6.013634in}{5.782214in}}%
\pgfpathlineto{\pgfqpoint{6.125137in}{5.782214in}}%
\pgfpathlineto{\pgfqpoint{6.236639in}{5.782214in}}%
\pgfpathlineto{\pgfqpoint{6.348141in}{5.782214in}}%
\pgfpathlineto{\pgfqpoint{6.459643in}{5.869096in}}%
\pgfpathlineto{\pgfqpoint{6.571146in}{5.869096in}}%
\pgfpathlineto{\pgfqpoint{6.682648in}{5.869096in}}%
\pgfpathlineto{\pgfqpoint{6.794150in}{5.869096in}}%
\pgfpathlineto{\pgfqpoint{6.905652in}{5.869096in}}%
\pgfpathlineto{\pgfqpoint{7.017154in}{5.955977in}}%
\pgfpathlineto{\pgfqpoint{7.128657in}{5.955977in}}%
\pgfpathlineto{\pgfqpoint{7.240159in}{5.955977in}}%
\pgfpathlineto{\pgfqpoint{7.351661in}{5.955977in}}%
\pgfpathlineto{\pgfqpoint{7.463163in}{5.955977in}}%
\pgfpathlineto{\pgfqpoint{7.574666in}{5.955977in}}%
\pgfpathlineto{\pgfqpoint{7.686168in}{5.955977in}}%
\pgfpathlineto{\pgfqpoint{7.797670in}{5.955977in}}%
\pgfpathlineto{\pgfqpoint{7.909172in}{5.955977in}}%
\pgfpathlineto{\pgfqpoint{8.020674in}{5.955977in}}%
\pgfpathlineto{\pgfqpoint{8.132177in}{5.955977in}}%
\pgfpathlineto{\pgfqpoint{8.243679in}{5.955977in}}%
\pgfpathlineto{\pgfqpoint{8.355181in}{5.955977in}}%
\pgfpathlineto{\pgfqpoint{8.466683in}{5.955977in}}%
\pgfpathlineto{\pgfqpoint{8.578186in}{5.955977in}}%
\pgfpathlineto{\pgfqpoint{8.689688in}{5.955977in}}%
\pgfpathlineto{\pgfqpoint{8.801190in}{5.955977in}}%
\pgfpathlineto{\pgfqpoint{8.912692in}{5.955977in}}%
\pgfpathlineto{\pgfqpoint{9.024194in}{5.955977in}}%
\pgfpathlineto{\pgfqpoint{9.135697in}{5.955977in}}%
\pgfpathlineto{\pgfqpoint{9.247199in}{5.955977in}}%
\pgfpathlineto{\pgfqpoint{9.358701in}{5.955977in}}%
\pgfpathlineto{\pgfqpoint{9.470203in}{5.955977in}}%
\pgfpathlineto{\pgfqpoint{9.581706in}{5.955977in}}%
\pgfpathlineto{\pgfqpoint{9.693208in}{5.955977in}}%
\pgfpathlineto{\pgfqpoint{9.804710in}{5.955977in}}%
\pgfpathlineto{\pgfqpoint{9.916212in}{5.955977in}}%
\pgfpathlineto{\pgfqpoint{10.027714in}{5.955977in}}%
\pgfpathlineto{\pgfqpoint{10.139217in}{5.955977in}}%
\pgfpathlineto{\pgfqpoint{10.250719in}{5.955977in}}%
\pgfpathlineto{\pgfqpoint{10.362221in}{5.955977in}}%
\pgfpathlineto{\pgfqpoint{10.473723in}{5.955977in}}%
\pgfpathlineto{\pgfqpoint{10.585226in}{5.955977in}}%
\pgfpathlineto{\pgfqpoint{10.696728in}{5.955977in}}%
\pgfpathlineto{\pgfqpoint{10.808230in}{5.955977in}}%
\pgfpathlineto{\pgfqpoint{10.919732in}{5.955977in}}%
\pgfpathlineto{\pgfqpoint{11.031234in}{5.955977in}}%
\pgfpathlineto{\pgfqpoint{11.142737in}{5.955977in}}%
\pgfpathlineto{\pgfqpoint{11.254239in}{5.955977in}}%
\pgfpathlineto{\pgfqpoint{11.365741in}{6.042859in}}%
\pgfpathlineto{\pgfqpoint{11.477243in}{6.042859in}}%
\pgfpathlineto{\pgfqpoint{11.588746in}{6.042859in}}%
\pgfpathlineto{\pgfqpoint{11.700248in}{6.129741in}}%
\pgfpathlineto{\pgfqpoint{11.811750in}{6.129741in}}%
\pgfusepath{stroke}%
\end{pgfscope}%
\begin{pgfscope}%
\pgfpathrectangle{\pgfqpoint{0.661528in}{0.586684in}}{\pgfqpoint{11.150222in}{7.263316in}}%
\pgfusepath{clip}%
\pgfsetrectcap%
\pgfsetroundjoin%
\pgfsetlinewidth{2.007500pt}%
\definecolor{currentstroke}{rgb}{0.172549,0.627451,0.172549}%
\pgfsetstrokecolor{currentstroke}%
\pgfsetdash{}{0pt}%
\pgfpathmoveto{\pgfqpoint{0.661528in}{2.046298in}}%
\pgfpathlineto{\pgfqpoint{0.773030in}{3.262643in}}%
\pgfpathlineto{\pgfqpoint{0.884532in}{3.436406in}}%
\pgfpathlineto{\pgfqpoint{0.996034in}{3.523288in}}%
\pgfpathlineto{\pgfqpoint{1.107537in}{3.870815in}}%
\pgfpathlineto{\pgfqpoint{1.219039in}{4.218342in}}%
\pgfpathlineto{\pgfqpoint{1.330541in}{4.305224in}}%
\pgfpathlineto{\pgfqpoint{1.442043in}{4.305224in}}%
\pgfpathlineto{\pgfqpoint{1.553546in}{4.305224in}}%
\pgfpathlineto{\pgfqpoint{1.665048in}{4.305224in}}%
\pgfpathlineto{\pgfqpoint{1.776550in}{4.392106in}}%
\pgfpathlineto{\pgfqpoint{1.888052in}{4.478987in}}%
\pgfpathlineto{\pgfqpoint{1.999554in}{4.565869in}}%
\pgfpathlineto{\pgfqpoint{2.111057in}{4.652751in}}%
\pgfpathlineto{\pgfqpoint{2.222559in}{4.739633in}}%
\pgfpathlineto{\pgfqpoint{2.334061in}{5.000278in}}%
\pgfpathlineto{\pgfqpoint{2.445563in}{5.087160in}}%
\pgfpathlineto{\pgfqpoint{2.557066in}{5.174041in}}%
\pgfpathlineto{\pgfqpoint{2.668568in}{5.260923in}}%
\pgfpathlineto{\pgfqpoint{2.780070in}{5.260923in}}%
\pgfpathlineto{\pgfqpoint{2.891572in}{5.260923in}}%
\pgfpathlineto{\pgfqpoint{3.003074in}{5.260923in}}%
\pgfpathlineto{\pgfqpoint{3.114577in}{5.260923in}}%
\pgfpathlineto{\pgfqpoint{3.226079in}{5.347805in}}%
\pgfpathlineto{\pgfqpoint{3.337581in}{5.347805in}}%
\pgfpathlineto{\pgfqpoint{3.449083in}{5.347805in}}%
\pgfpathlineto{\pgfqpoint{3.560586in}{5.347805in}}%
\pgfpathlineto{\pgfqpoint{3.672088in}{5.347805in}}%
\pgfpathlineto{\pgfqpoint{3.783590in}{5.347805in}}%
\pgfpathlineto{\pgfqpoint{3.895092in}{5.347805in}}%
\pgfpathlineto{\pgfqpoint{4.006594in}{5.347805in}}%
\pgfpathlineto{\pgfqpoint{4.118097in}{5.434687in}}%
\pgfpathlineto{\pgfqpoint{4.229599in}{5.434687in}}%
\pgfpathlineto{\pgfqpoint{4.341101in}{5.434687in}}%
\pgfpathlineto{\pgfqpoint{4.452603in}{5.434687in}}%
\pgfpathlineto{\pgfqpoint{4.564106in}{5.434687in}}%
\pgfpathlineto{\pgfqpoint{4.675608in}{5.434687in}}%
\pgfpathlineto{\pgfqpoint{4.787110in}{5.434687in}}%
\pgfpathlineto{\pgfqpoint{4.898612in}{5.434687in}}%
\pgfpathlineto{\pgfqpoint{5.010114in}{5.434687in}}%
\pgfpathlineto{\pgfqpoint{5.121617in}{5.434687in}}%
\pgfpathlineto{\pgfqpoint{5.233119in}{5.434687in}}%
\pgfpathlineto{\pgfqpoint{5.344621in}{5.434687in}}%
\pgfpathlineto{\pgfqpoint{5.456123in}{5.434687in}}%
\pgfpathlineto{\pgfqpoint{5.567626in}{5.521569in}}%
\pgfpathlineto{\pgfqpoint{5.679128in}{5.521569in}}%
\pgfpathlineto{\pgfqpoint{5.790630in}{5.521569in}}%
\pgfpathlineto{\pgfqpoint{5.902132in}{5.608450in}}%
\pgfpathlineto{\pgfqpoint{6.013634in}{5.608450in}}%
\pgfpathlineto{\pgfqpoint{6.125137in}{5.608450in}}%
\pgfpathlineto{\pgfqpoint{6.236639in}{5.695332in}}%
\pgfpathlineto{\pgfqpoint{6.348141in}{5.695332in}}%
\pgfpathlineto{\pgfqpoint{6.459643in}{5.695332in}}%
\pgfpathlineto{\pgfqpoint{6.571146in}{5.782214in}}%
\pgfpathlineto{\pgfqpoint{6.682648in}{5.782214in}}%
\pgfpathlineto{\pgfqpoint{6.794150in}{5.782214in}}%
\pgfpathlineto{\pgfqpoint{6.905652in}{5.782214in}}%
\pgfpathlineto{\pgfqpoint{7.017154in}{5.782214in}}%
\pgfpathlineto{\pgfqpoint{7.128657in}{5.782214in}}%
\pgfpathlineto{\pgfqpoint{7.240159in}{5.782214in}}%
\pgfpathlineto{\pgfqpoint{7.351661in}{5.782214in}}%
\pgfpathlineto{\pgfqpoint{7.463163in}{5.869096in}}%
\pgfpathlineto{\pgfqpoint{7.574666in}{5.869096in}}%
\pgfpathlineto{\pgfqpoint{7.686168in}{5.869096in}}%
\pgfpathlineto{\pgfqpoint{7.797670in}{5.869096in}}%
\pgfpathlineto{\pgfqpoint{7.909172in}{5.955977in}}%
\pgfpathlineto{\pgfqpoint{8.020674in}{5.955977in}}%
\pgfpathlineto{\pgfqpoint{8.132177in}{5.955977in}}%
\pgfpathlineto{\pgfqpoint{8.243679in}{5.955977in}}%
\pgfpathlineto{\pgfqpoint{8.355181in}{5.955977in}}%
\pgfpathlineto{\pgfqpoint{8.466683in}{5.955977in}}%
\pgfpathlineto{\pgfqpoint{8.578186in}{5.955977in}}%
\pgfpathlineto{\pgfqpoint{8.689688in}{5.955977in}}%
\pgfpathlineto{\pgfqpoint{8.801190in}{5.955977in}}%
\pgfpathlineto{\pgfqpoint{8.912692in}{5.955977in}}%
\pgfpathlineto{\pgfqpoint{9.024194in}{5.955977in}}%
\pgfpathlineto{\pgfqpoint{9.135697in}{5.955977in}}%
\pgfpathlineto{\pgfqpoint{9.247199in}{5.955977in}}%
\pgfpathlineto{\pgfqpoint{9.358701in}{5.955977in}}%
\pgfpathlineto{\pgfqpoint{9.470203in}{5.955977in}}%
\pgfpathlineto{\pgfqpoint{9.581706in}{5.955977in}}%
\pgfpathlineto{\pgfqpoint{9.693208in}{5.955977in}}%
\pgfpathlineto{\pgfqpoint{9.804710in}{5.955977in}}%
\pgfpathlineto{\pgfqpoint{9.916212in}{5.955977in}}%
\pgfpathlineto{\pgfqpoint{10.027714in}{5.955977in}}%
\pgfpathlineto{\pgfqpoint{10.139217in}{5.955977in}}%
\pgfpathlineto{\pgfqpoint{10.250719in}{5.955977in}}%
\pgfpathlineto{\pgfqpoint{10.362221in}{6.042859in}}%
\pgfpathlineto{\pgfqpoint{10.473723in}{6.042859in}}%
\pgfpathlineto{\pgfqpoint{10.585226in}{6.042859in}}%
\pgfpathlineto{\pgfqpoint{10.696728in}{6.042859in}}%
\pgfpathlineto{\pgfqpoint{10.808230in}{6.042859in}}%
\pgfpathlineto{\pgfqpoint{10.919732in}{6.042859in}}%
\pgfpathlineto{\pgfqpoint{11.031234in}{6.042859in}}%
\pgfpathlineto{\pgfqpoint{11.142737in}{6.042859in}}%
\pgfpathlineto{\pgfqpoint{11.254239in}{6.042859in}}%
\pgfpathlineto{\pgfqpoint{11.365741in}{6.042859in}}%
\pgfpathlineto{\pgfqpoint{11.477243in}{6.042859in}}%
\pgfpathlineto{\pgfqpoint{11.588746in}{6.042859in}}%
\pgfpathlineto{\pgfqpoint{11.700248in}{6.042859in}}%
\pgfpathlineto{\pgfqpoint{11.811750in}{6.042859in}}%
\pgfusepath{stroke}%
\end{pgfscope}%
\begin{pgfscope}%
\pgfpathrectangle{\pgfqpoint{0.661528in}{0.586684in}}{\pgfqpoint{11.150222in}{7.263316in}}%
\pgfusepath{clip}%
\pgfsetrectcap%
\pgfsetroundjoin%
\pgfsetlinewidth{2.007500pt}%
\definecolor{currentstroke}{rgb}{0.839216,0.152941,0.156863}%
\pgfsetstrokecolor{currentstroke}%
\pgfsetdash{}{0pt}%
\pgfpathmoveto{\pgfqpoint{0.661528in}{2.306943in}}%
\pgfpathlineto{\pgfqpoint{0.773030in}{2.567588in}}%
\pgfpathlineto{\pgfqpoint{0.884532in}{2.828234in}}%
\pgfpathlineto{\pgfqpoint{0.996034in}{3.001997in}}%
\pgfpathlineto{\pgfqpoint{1.107537in}{3.088879in}}%
\pgfpathlineto{\pgfqpoint{1.219039in}{3.262643in}}%
\pgfpathlineto{\pgfqpoint{1.330541in}{3.262643in}}%
\pgfpathlineto{\pgfqpoint{1.442043in}{3.349524in}}%
\pgfpathlineto{\pgfqpoint{1.553546in}{3.436406in}}%
\pgfpathlineto{\pgfqpoint{1.665048in}{3.436406in}}%
\pgfpathlineto{\pgfqpoint{1.776550in}{3.436406in}}%
\pgfpathlineto{\pgfqpoint{1.888052in}{3.610170in}}%
\pgfpathlineto{\pgfqpoint{1.999554in}{3.610170in}}%
\pgfpathlineto{\pgfqpoint{2.111057in}{3.610170in}}%
\pgfpathlineto{\pgfqpoint{2.222559in}{3.783933in}}%
\pgfpathlineto{\pgfqpoint{2.334061in}{3.783933in}}%
\pgfpathlineto{\pgfqpoint{2.445563in}{3.870815in}}%
\pgfpathlineto{\pgfqpoint{2.557066in}{3.957697in}}%
\pgfpathlineto{\pgfqpoint{2.668568in}{3.957697in}}%
\pgfpathlineto{\pgfqpoint{2.780070in}{3.957697in}}%
\pgfpathlineto{\pgfqpoint{2.891572in}{4.044578in}}%
\pgfpathlineto{\pgfqpoint{3.003074in}{4.044578in}}%
\pgfpathlineto{\pgfqpoint{3.114577in}{4.044578in}}%
\pgfpathlineto{\pgfqpoint{3.226079in}{4.131460in}}%
\pgfpathlineto{\pgfqpoint{3.337581in}{4.131460in}}%
\pgfpathlineto{\pgfqpoint{3.449083in}{4.131460in}}%
\pgfpathlineto{\pgfqpoint{3.560586in}{4.131460in}}%
\pgfpathlineto{\pgfqpoint{3.672088in}{4.131460in}}%
\pgfpathlineto{\pgfqpoint{3.783590in}{4.218342in}}%
\pgfpathlineto{\pgfqpoint{3.895092in}{4.218342in}}%
\pgfpathlineto{\pgfqpoint{4.006594in}{4.218342in}}%
\pgfpathlineto{\pgfqpoint{4.118097in}{4.305224in}}%
\pgfpathlineto{\pgfqpoint{4.229599in}{4.305224in}}%
\pgfpathlineto{\pgfqpoint{4.341101in}{4.305224in}}%
\pgfpathlineto{\pgfqpoint{4.452603in}{4.305224in}}%
\pgfpathlineto{\pgfqpoint{4.564106in}{4.305224in}}%
\pgfpathlineto{\pgfqpoint{4.675608in}{4.305224in}}%
\pgfpathlineto{\pgfqpoint{4.787110in}{4.305224in}}%
\pgfpathlineto{\pgfqpoint{4.898612in}{4.305224in}}%
\pgfpathlineto{\pgfqpoint{5.010114in}{4.305224in}}%
\pgfpathlineto{\pgfqpoint{5.121617in}{4.305224in}}%
\pgfpathlineto{\pgfqpoint{5.233119in}{4.305224in}}%
\pgfpathlineto{\pgfqpoint{5.344621in}{4.305224in}}%
\pgfpathlineto{\pgfqpoint{5.456123in}{4.392106in}}%
\pgfpathlineto{\pgfqpoint{5.567626in}{4.392106in}}%
\pgfpathlineto{\pgfqpoint{5.679128in}{4.392106in}}%
\pgfpathlineto{\pgfqpoint{5.790630in}{4.392106in}}%
\pgfpathlineto{\pgfqpoint{5.902132in}{4.392106in}}%
\pgfpathlineto{\pgfqpoint{6.013634in}{4.392106in}}%
\pgfpathlineto{\pgfqpoint{6.125137in}{4.392106in}}%
\pgfpathlineto{\pgfqpoint{6.236639in}{4.392106in}}%
\pgfpathlineto{\pgfqpoint{6.348141in}{4.392106in}}%
\pgfpathlineto{\pgfqpoint{6.459643in}{4.478987in}}%
\pgfpathlineto{\pgfqpoint{6.571146in}{4.478987in}}%
\pgfpathlineto{\pgfqpoint{6.682648in}{4.565869in}}%
\pgfpathlineto{\pgfqpoint{6.794150in}{4.565869in}}%
\pgfpathlineto{\pgfqpoint{6.905652in}{4.565869in}}%
\pgfpathlineto{\pgfqpoint{7.017154in}{4.565869in}}%
\pgfpathlineto{\pgfqpoint{7.128657in}{4.565869in}}%
\pgfpathlineto{\pgfqpoint{7.240159in}{4.565869in}}%
\pgfpathlineto{\pgfqpoint{7.351661in}{4.565869in}}%
\pgfpathlineto{\pgfqpoint{7.463163in}{4.565869in}}%
\pgfpathlineto{\pgfqpoint{7.574666in}{4.565869in}}%
\pgfpathlineto{\pgfqpoint{7.686168in}{4.565869in}}%
\pgfpathlineto{\pgfqpoint{7.797670in}{4.565869in}}%
\pgfpathlineto{\pgfqpoint{7.909172in}{4.565869in}}%
\pgfpathlineto{\pgfqpoint{8.020674in}{4.565869in}}%
\pgfpathlineto{\pgfqpoint{8.132177in}{4.565869in}}%
\pgfpathlineto{\pgfqpoint{8.243679in}{4.565869in}}%
\pgfpathlineto{\pgfqpoint{8.355181in}{4.565869in}}%
\pgfpathlineto{\pgfqpoint{8.466683in}{4.565869in}}%
\pgfpathlineto{\pgfqpoint{8.578186in}{4.565869in}}%
\pgfpathlineto{\pgfqpoint{8.689688in}{4.565869in}}%
\pgfpathlineto{\pgfqpoint{8.801190in}{4.565869in}}%
\pgfpathlineto{\pgfqpoint{8.912692in}{4.565869in}}%
\pgfpathlineto{\pgfqpoint{9.024194in}{4.565869in}}%
\pgfpathlineto{\pgfqpoint{9.135697in}{4.565869in}}%
\pgfpathlineto{\pgfqpoint{9.247199in}{4.565869in}}%
\pgfpathlineto{\pgfqpoint{9.358701in}{4.565869in}}%
\pgfpathlineto{\pgfqpoint{9.470203in}{4.565869in}}%
\pgfpathlineto{\pgfqpoint{9.581706in}{4.565869in}}%
\pgfpathlineto{\pgfqpoint{9.693208in}{4.652751in}}%
\pgfpathlineto{\pgfqpoint{9.804710in}{4.652751in}}%
\pgfpathlineto{\pgfqpoint{9.916212in}{4.652751in}}%
\pgfpathlineto{\pgfqpoint{10.027714in}{4.652751in}}%
\pgfpathlineto{\pgfqpoint{10.139217in}{4.652751in}}%
\pgfpathlineto{\pgfqpoint{10.250719in}{4.652751in}}%
\pgfpathlineto{\pgfqpoint{10.362221in}{4.652751in}}%
\pgfpathlineto{\pgfqpoint{10.473723in}{4.652751in}}%
\pgfpathlineto{\pgfqpoint{10.585226in}{4.652751in}}%
\pgfpathlineto{\pgfqpoint{10.696728in}{4.652751in}}%
\pgfpathlineto{\pgfqpoint{10.808230in}{4.652751in}}%
\pgfpathlineto{\pgfqpoint{10.919732in}{4.652751in}}%
\pgfpathlineto{\pgfqpoint{11.031234in}{4.652751in}}%
\pgfpathlineto{\pgfqpoint{11.142737in}{4.652751in}}%
\pgfpathlineto{\pgfqpoint{11.254239in}{4.652751in}}%
\pgfpathlineto{\pgfqpoint{11.365741in}{4.652751in}}%
\pgfpathlineto{\pgfqpoint{11.477243in}{4.652751in}}%
\pgfpathlineto{\pgfqpoint{11.588746in}{4.652751in}}%
\pgfpathlineto{\pgfqpoint{11.700248in}{4.652751in}}%
\pgfpathlineto{\pgfqpoint{11.811750in}{4.652751in}}%
\pgfusepath{stroke}%
\end{pgfscope}%
\begin{pgfscope}%
\pgfpathrectangle{\pgfqpoint{0.661528in}{0.586684in}}{\pgfqpoint{11.150222in}{7.263316in}}%
\pgfusepath{clip}%
\pgfsetrectcap%
\pgfsetroundjoin%
\pgfsetlinewidth{2.007500pt}%
\definecolor{currentstroke}{rgb}{0.580392,0.403922,0.741176}%
\pgfsetstrokecolor{currentstroke}%
\pgfsetdash{}{0pt}%
\pgfpathmoveto{\pgfqpoint{0.661528in}{1.525007in}}%
\pgfpathlineto{\pgfqpoint{0.773030in}{1.959416in}}%
\pgfpathlineto{\pgfqpoint{0.884532in}{1.959416in}}%
\pgfpathlineto{\pgfqpoint{0.996034in}{2.046298in}}%
\pgfpathlineto{\pgfqpoint{1.107537in}{2.133180in}}%
\pgfpathlineto{\pgfqpoint{1.219039in}{2.133180in}}%
\pgfpathlineto{\pgfqpoint{1.330541in}{2.133180in}}%
\pgfpathlineto{\pgfqpoint{1.442043in}{2.133180in}}%
\pgfpathlineto{\pgfqpoint{1.553546in}{2.133180in}}%
\pgfpathlineto{\pgfqpoint{1.665048in}{2.133180in}}%
\pgfpathlineto{\pgfqpoint{1.776550in}{2.220061in}}%
\pgfpathlineto{\pgfqpoint{1.888052in}{2.220061in}}%
\pgfpathlineto{\pgfqpoint{1.999554in}{2.220061in}}%
\pgfpathlineto{\pgfqpoint{2.111057in}{2.220061in}}%
\pgfpathlineto{\pgfqpoint{2.222559in}{2.220061in}}%
\pgfpathlineto{\pgfqpoint{2.334061in}{2.220061in}}%
\pgfpathlineto{\pgfqpoint{2.445563in}{2.306943in}}%
\pgfpathlineto{\pgfqpoint{2.557066in}{2.306943in}}%
\pgfpathlineto{\pgfqpoint{2.668568in}{2.306943in}}%
\pgfpathlineto{\pgfqpoint{2.780070in}{2.393825in}}%
\pgfpathlineto{\pgfqpoint{2.891572in}{2.393825in}}%
\pgfpathlineto{\pgfqpoint{3.003074in}{2.393825in}}%
\pgfpathlineto{\pgfqpoint{3.114577in}{2.393825in}}%
\pgfpathlineto{\pgfqpoint{3.226079in}{2.393825in}}%
\pgfpathlineto{\pgfqpoint{3.337581in}{2.393825in}}%
\pgfpathlineto{\pgfqpoint{3.449083in}{2.567588in}}%
\pgfpathlineto{\pgfqpoint{3.560586in}{2.567588in}}%
\pgfpathlineto{\pgfqpoint{3.672088in}{2.567588in}}%
\pgfpathlineto{\pgfqpoint{3.783590in}{2.654470in}}%
\pgfpathlineto{\pgfqpoint{3.895092in}{2.654470in}}%
\pgfpathlineto{\pgfqpoint{4.006594in}{2.654470in}}%
\pgfpathlineto{\pgfqpoint{4.118097in}{2.654470in}}%
\pgfpathlineto{\pgfqpoint{4.229599in}{2.654470in}}%
\pgfpathlineto{\pgfqpoint{4.341101in}{2.654470in}}%
\pgfpathlineto{\pgfqpoint{4.452603in}{2.654470in}}%
\pgfpathlineto{\pgfqpoint{4.564106in}{2.654470in}}%
\pgfpathlineto{\pgfqpoint{4.675608in}{2.654470in}}%
\pgfpathlineto{\pgfqpoint{4.787110in}{2.654470in}}%
\pgfpathlineto{\pgfqpoint{4.898612in}{2.654470in}}%
\pgfpathlineto{\pgfqpoint{5.010114in}{2.741352in}}%
\pgfpathlineto{\pgfqpoint{5.121617in}{2.741352in}}%
\pgfpathlineto{\pgfqpoint{5.233119in}{2.828234in}}%
\pgfpathlineto{\pgfqpoint{5.344621in}{2.828234in}}%
\pgfpathlineto{\pgfqpoint{5.456123in}{2.828234in}}%
\pgfpathlineto{\pgfqpoint{5.567626in}{2.828234in}}%
\pgfpathlineto{\pgfqpoint{5.679128in}{2.828234in}}%
\pgfpathlineto{\pgfqpoint{5.790630in}{2.828234in}}%
\pgfpathlineto{\pgfqpoint{5.902132in}{2.828234in}}%
\pgfpathlineto{\pgfqpoint{6.013634in}{2.828234in}}%
\pgfpathlineto{\pgfqpoint{6.125137in}{3.001997in}}%
\pgfpathlineto{\pgfqpoint{6.236639in}{3.001997in}}%
\pgfpathlineto{\pgfqpoint{6.348141in}{3.001997in}}%
\pgfpathlineto{\pgfqpoint{6.459643in}{3.001997in}}%
\pgfpathlineto{\pgfqpoint{6.571146in}{3.001997in}}%
\pgfpathlineto{\pgfqpoint{6.682648in}{3.001997in}}%
\pgfpathlineto{\pgfqpoint{6.794150in}{3.001997in}}%
\pgfpathlineto{\pgfqpoint{6.905652in}{3.001997in}}%
\pgfpathlineto{\pgfqpoint{7.017154in}{3.001997in}}%
\pgfpathlineto{\pgfqpoint{7.128657in}{3.001997in}}%
\pgfpathlineto{\pgfqpoint{7.240159in}{3.001997in}}%
\pgfpathlineto{\pgfqpoint{7.351661in}{3.001997in}}%
\pgfpathlineto{\pgfqpoint{7.463163in}{3.001997in}}%
\pgfpathlineto{\pgfqpoint{7.574666in}{3.001997in}}%
\pgfpathlineto{\pgfqpoint{7.686168in}{3.001997in}}%
\pgfpathlineto{\pgfqpoint{7.797670in}{3.088879in}}%
\pgfpathlineto{\pgfqpoint{7.909172in}{3.088879in}}%
\pgfpathlineto{\pgfqpoint{8.020674in}{3.088879in}}%
\pgfpathlineto{\pgfqpoint{8.132177in}{3.088879in}}%
\pgfpathlineto{\pgfqpoint{8.243679in}{3.088879in}}%
\pgfpathlineto{\pgfqpoint{8.355181in}{3.088879in}}%
\pgfpathlineto{\pgfqpoint{8.466683in}{3.088879in}}%
\pgfpathlineto{\pgfqpoint{8.578186in}{3.088879in}}%
\pgfpathlineto{\pgfqpoint{8.689688in}{3.088879in}}%
\pgfpathlineto{\pgfqpoint{8.801190in}{3.088879in}}%
\pgfpathlineto{\pgfqpoint{8.912692in}{3.088879in}}%
\pgfpathlineto{\pgfqpoint{9.024194in}{3.088879in}}%
\pgfpathlineto{\pgfqpoint{9.135697in}{3.088879in}}%
\pgfpathlineto{\pgfqpoint{9.247199in}{3.088879in}}%
\pgfpathlineto{\pgfqpoint{9.358701in}{3.088879in}}%
\pgfpathlineto{\pgfqpoint{9.470203in}{3.088879in}}%
\pgfpathlineto{\pgfqpoint{9.581706in}{3.088879in}}%
\pgfpathlineto{\pgfqpoint{9.693208in}{3.088879in}}%
\pgfpathlineto{\pgfqpoint{9.804710in}{3.088879in}}%
\pgfpathlineto{\pgfqpoint{9.916212in}{3.088879in}}%
\pgfpathlineto{\pgfqpoint{10.027714in}{3.088879in}}%
\pgfpathlineto{\pgfqpoint{10.139217in}{3.088879in}}%
\pgfpathlineto{\pgfqpoint{10.250719in}{3.088879in}}%
\pgfpathlineto{\pgfqpoint{10.362221in}{3.088879in}}%
\pgfpathlineto{\pgfqpoint{10.473723in}{3.088879in}}%
\pgfpathlineto{\pgfqpoint{10.585226in}{3.088879in}}%
\pgfpathlineto{\pgfqpoint{10.696728in}{3.088879in}}%
\pgfpathlineto{\pgfqpoint{10.808230in}{3.088879in}}%
\pgfpathlineto{\pgfqpoint{10.919732in}{3.088879in}}%
\pgfpathlineto{\pgfqpoint{11.031234in}{3.088879in}}%
\pgfpathlineto{\pgfqpoint{11.142737in}{3.088879in}}%
\pgfpathlineto{\pgfqpoint{11.254239in}{3.088879in}}%
\pgfpathlineto{\pgfqpoint{11.365741in}{3.088879in}}%
\pgfpathlineto{\pgfqpoint{11.477243in}{3.088879in}}%
\pgfpathlineto{\pgfqpoint{11.588746in}{3.088879in}}%
\pgfpathlineto{\pgfqpoint{11.700248in}{3.088879in}}%
\pgfpathlineto{\pgfqpoint{11.811750in}{3.088879in}}%
\pgfusepath{stroke}%
\end{pgfscope}%
\begin{pgfscope}%
\pgfpathrectangle{\pgfqpoint{0.661528in}{0.586684in}}{\pgfqpoint{11.150222in}{7.263316in}}%
\pgfusepath{clip}%
\pgfsetrectcap%
\pgfsetroundjoin%
\pgfsetlinewidth{2.007500pt}%
\definecolor{currentstroke}{rgb}{0.549020,0.337255,0.294118}%
\pgfsetstrokecolor{currentstroke}%
\pgfsetdash{}{0pt}%
\pgfpathmoveto{\pgfqpoint{0.661528in}{0.916835in}}%
\pgfpathlineto{\pgfqpoint{0.773030in}{1.264362in}}%
\pgfpathlineto{\pgfqpoint{0.884532in}{1.264362in}}%
\pgfpathlineto{\pgfqpoint{0.996034in}{1.438125in}}%
\pgfpathlineto{\pgfqpoint{1.107537in}{1.438125in}}%
\pgfpathlineto{\pgfqpoint{1.219039in}{1.438125in}}%
\pgfpathlineto{\pgfqpoint{1.330541in}{1.438125in}}%
\pgfpathlineto{\pgfqpoint{1.442043in}{1.438125in}}%
\pgfpathlineto{\pgfqpoint{1.553546in}{1.438125in}}%
\pgfpathlineto{\pgfqpoint{1.665048in}{1.438125in}}%
\pgfpathlineto{\pgfqpoint{1.776550in}{1.438125in}}%
\pgfpathlineto{\pgfqpoint{1.888052in}{1.438125in}}%
\pgfpathlineto{\pgfqpoint{1.999554in}{1.438125in}}%
\pgfpathlineto{\pgfqpoint{2.111057in}{1.438125in}}%
\pgfpathlineto{\pgfqpoint{2.222559in}{1.438125in}}%
\pgfpathlineto{\pgfqpoint{2.334061in}{1.438125in}}%
\pgfpathlineto{\pgfqpoint{2.445563in}{1.438125in}}%
\pgfpathlineto{\pgfqpoint{2.557066in}{1.438125in}}%
\pgfpathlineto{\pgfqpoint{2.668568in}{1.438125in}}%
\pgfpathlineto{\pgfqpoint{2.780070in}{1.525007in}}%
\pgfpathlineto{\pgfqpoint{2.891572in}{1.525007in}}%
\pgfpathlineto{\pgfqpoint{3.003074in}{1.525007in}}%
\pgfpathlineto{\pgfqpoint{3.114577in}{1.611889in}}%
\pgfpathlineto{\pgfqpoint{3.226079in}{1.611889in}}%
\pgfpathlineto{\pgfqpoint{3.337581in}{1.611889in}}%
\pgfpathlineto{\pgfqpoint{3.449083in}{1.611889in}}%
\pgfpathlineto{\pgfqpoint{3.560586in}{1.611889in}}%
\pgfpathlineto{\pgfqpoint{3.672088in}{1.611889in}}%
\pgfpathlineto{\pgfqpoint{3.783590in}{1.611889in}}%
\pgfpathlineto{\pgfqpoint{3.895092in}{1.611889in}}%
\pgfpathlineto{\pgfqpoint{4.006594in}{1.611889in}}%
\pgfpathlineto{\pgfqpoint{4.118097in}{1.611889in}}%
\pgfpathlineto{\pgfqpoint{4.229599in}{1.611889in}}%
\pgfpathlineto{\pgfqpoint{4.341101in}{1.611889in}}%
\pgfpathlineto{\pgfqpoint{4.452603in}{1.611889in}}%
\pgfpathlineto{\pgfqpoint{4.564106in}{1.611889in}}%
\pgfpathlineto{\pgfqpoint{4.675608in}{1.611889in}}%
\pgfpathlineto{\pgfqpoint{4.787110in}{1.611889in}}%
\pgfpathlineto{\pgfqpoint{4.898612in}{1.611889in}}%
\pgfpathlineto{\pgfqpoint{5.010114in}{1.611889in}}%
\pgfpathlineto{\pgfqpoint{5.121617in}{1.611889in}}%
\pgfpathlineto{\pgfqpoint{5.233119in}{1.611889in}}%
\pgfpathlineto{\pgfqpoint{5.344621in}{1.698771in}}%
\pgfpathlineto{\pgfqpoint{5.456123in}{1.698771in}}%
\pgfpathlineto{\pgfqpoint{5.567626in}{1.698771in}}%
\pgfpathlineto{\pgfqpoint{5.679128in}{1.698771in}}%
\pgfpathlineto{\pgfqpoint{5.790630in}{1.698771in}}%
\pgfpathlineto{\pgfqpoint{5.902132in}{1.698771in}}%
\pgfpathlineto{\pgfqpoint{6.013634in}{1.698771in}}%
\pgfpathlineto{\pgfqpoint{6.125137in}{1.698771in}}%
\pgfpathlineto{\pgfqpoint{6.236639in}{1.698771in}}%
\pgfpathlineto{\pgfqpoint{6.348141in}{1.698771in}}%
\pgfpathlineto{\pgfqpoint{6.459643in}{1.698771in}}%
\pgfpathlineto{\pgfqpoint{6.571146in}{1.698771in}}%
\pgfpathlineto{\pgfqpoint{6.682648in}{1.698771in}}%
\pgfpathlineto{\pgfqpoint{6.794150in}{1.698771in}}%
\pgfpathlineto{\pgfqpoint{6.905652in}{1.698771in}}%
\pgfpathlineto{\pgfqpoint{7.017154in}{1.698771in}}%
\pgfpathlineto{\pgfqpoint{7.128657in}{1.698771in}}%
\pgfpathlineto{\pgfqpoint{7.240159in}{1.698771in}}%
\pgfpathlineto{\pgfqpoint{7.351661in}{1.698771in}}%
\pgfpathlineto{\pgfqpoint{7.463163in}{1.785652in}}%
\pgfpathlineto{\pgfqpoint{7.574666in}{1.785652in}}%
\pgfpathlineto{\pgfqpoint{7.686168in}{1.785652in}}%
\pgfpathlineto{\pgfqpoint{7.797670in}{1.785652in}}%
\pgfpathlineto{\pgfqpoint{7.909172in}{1.785652in}}%
\pgfpathlineto{\pgfqpoint{8.020674in}{1.785652in}}%
\pgfpathlineto{\pgfqpoint{8.132177in}{1.785652in}}%
\pgfpathlineto{\pgfqpoint{8.243679in}{1.785652in}}%
\pgfpathlineto{\pgfqpoint{8.355181in}{1.785652in}}%
\pgfpathlineto{\pgfqpoint{8.466683in}{1.872534in}}%
\pgfpathlineto{\pgfqpoint{8.578186in}{1.872534in}}%
\pgfpathlineto{\pgfqpoint{8.689688in}{1.872534in}}%
\pgfpathlineto{\pgfqpoint{8.801190in}{1.872534in}}%
\pgfpathlineto{\pgfqpoint{8.912692in}{1.872534in}}%
\pgfpathlineto{\pgfqpoint{9.024194in}{1.872534in}}%
\pgfpathlineto{\pgfqpoint{9.135697in}{1.872534in}}%
\pgfpathlineto{\pgfqpoint{9.247199in}{1.872534in}}%
\pgfpathlineto{\pgfqpoint{9.358701in}{1.872534in}}%
\pgfpathlineto{\pgfqpoint{9.470203in}{1.872534in}}%
\pgfpathlineto{\pgfqpoint{9.581706in}{1.872534in}}%
\pgfpathlineto{\pgfqpoint{9.693208in}{1.872534in}}%
\pgfpathlineto{\pgfqpoint{9.804710in}{1.872534in}}%
\pgfpathlineto{\pgfqpoint{9.916212in}{1.872534in}}%
\pgfpathlineto{\pgfqpoint{10.027714in}{1.872534in}}%
\pgfpathlineto{\pgfqpoint{10.139217in}{1.872534in}}%
\pgfpathlineto{\pgfqpoint{10.250719in}{1.872534in}}%
\pgfpathlineto{\pgfqpoint{10.362221in}{1.872534in}}%
\pgfpathlineto{\pgfqpoint{10.473723in}{1.872534in}}%
\pgfpathlineto{\pgfqpoint{10.585226in}{1.872534in}}%
\pgfpathlineto{\pgfqpoint{10.696728in}{1.872534in}}%
\pgfpathlineto{\pgfqpoint{10.808230in}{1.872534in}}%
\pgfpathlineto{\pgfqpoint{10.919732in}{1.872534in}}%
\pgfpathlineto{\pgfqpoint{11.031234in}{1.872534in}}%
\pgfpathlineto{\pgfqpoint{11.142737in}{1.872534in}}%
\pgfpathlineto{\pgfqpoint{11.254239in}{1.872534in}}%
\pgfpathlineto{\pgfqpoint{11.365741in}{1.872534in}}%
\pgfpathlineto{\pgfqpoint{11.477243in}{1.872534in}}%
\pgfpathlineto{\pgfqpoint{11.588746in}{1.872534in}}%
\pgfpathlineto{\pgfqpoint{11.700248in}{1.872534in}}%
\pgfpathlineto{\pgfqpoint{11.811750in}{1.872534in}}%
\pgfusepath{stroke}%
\end{pgfscope}%
\begin{pgfscope}%
\pgfsetrectcap%
\pgfsetmiterjoin%
\pgfsetlinewidth{0.803000pt}%
\definecolor{currentstroke}{rgb}{0.000000,0.000000,0.000000}%
\pgfsetstrokecolor{currentstroke}%
\pgfsetdash{}{0pt}%
\pgfpathmoveto{\pgfqpoint{0.661528in}{0.586684in}}%
\pgfpathlineto{\pgfqpoint{0.661528in}{7.850000in}}%
\pgfusepath{stroke}%
\end{pgfscope}%
\begin{pgfscope}%
\pgfsetrectcap%
\pgfsetmiterjoin%
\pgfsetlinewidth{0.803000pt}%
\definecolor{currentstroke}{rgb}{0.000000,0.000000,0.000000}%
\pgfsetstrokecolor{currentstroke}%
\pgfsetdash{}{0pt}%
\pgfpathmoveto{\pgfqpoint{11.811750in}{0.586684in}}%
\pgfpathlineto{\pgfqpoint{11.811750in}{7.850000in}}%
\pgfusepath{stroke}%
\end{pgfscope}%
\begin{pgfscope}%
\pgfsetrectcap%
\pgfsetmiterjoin%
\pgfsetlinewidth{0.803000pt}%
\definecolor{currentstroke}{rgb}{0.000000,0.000000,0.000000}%
\pgfsetstrokecolor{currentstroke}%
\pgfsetdash{}{0pt}%
\pgfpathmoveto{\pgfqpoint{0.661528in}{0.586684in}}%
\pgfpathlineto{\pgfqpoint{11.811750in}{0.586684in}}%
\pgfusepath{stroke}%
\end{pgfscope}%
\begin{pgfscope}%
\pgfsetrectcap%
\pgfsetmiterjoin%
\pgfsetlinewidth{0.803000pt}%
\definecolor{currentstroke}{rgb}{0.000000,0.000000,0.000000}%
\pgfsetstrokecolor{currentstroke}%
\pgfsetdash{}{0pt}%
\pgfpathmoveto{\pgfqpoint{0.661528in}{7.850000in}}%
\pgfpathlineto{\pgfqpoint{11.811750in}{7.850000in}}%
\pgfusepath{stroke}%
\end{pgfscope}%
\begin{pgfscope}%
\pgfsetbuttcap%
\pgfsetmiterjoin%
\definecolor{currentfill}{rgb}{1.000000,1.000000,1.000000}%
\pgfsetfillcolor{currentfill}%
\pgfsetfillopacity{0.800000}%
\pgfsetlinewidth{1.003750pt}%
\definecolor{currentstroke}{rgb}{0.800000,0.800000,0.800000}%
\pgfsetstrokecolor{currentstroke}%
\pgfsetstrokeopacity{0.800000}%
\pgfsetdash{}{0pt}%
\pgfpathmoveto{\pgfqpoint{0.758750in}{6.496418in}}%
\pgfpathlineto{\pgfqpoint{2.230687in}{6.496418in}}%
\pgfpathquadraticcurveto{\pgfqpoint{2.258465in}{6.496418in}}{\pgfqpoint{2.258465in}{6.524196in}}%
\pgfpathlineto{\pgfqpoint{2.258465in}{7.752778in}}%
\pgfpathquadraticcurveto{\pgfqpoint{2.258465in}{7.780556in}}{\pgfqpoint{2.230687in}{7.780556in}}%
\pgfpathlineto{\pgfqpoint{0.758750in}{7.780556in}}%
\pgfpathquadraticcurveto{\pgfqpoint{0.730972in}{7.780556in}}{\pgfqpoint{0.730972in}{7.752778in}}%
\pgfpathlineto{\pgfqpoint{0.730972in}{6.524196in}}%
\pgfpathquadraticcurveto{\pgfqpoint{0.730972in}{6.496418in}}{\pgfqpoint{0.758750in}{6.496418in}}%
\pgfpathlineto{\pgfqpoint{0.758750in}{6.496418in}}%
\pgfpathclose%
\pgfusepath{stroke,fill}%
\end{pgfscope}%
\begin{pgfscope}%
\pgfsetrectcap%
\pgfsetroundjoin%
\pgfsetlinewidth{2.007500pt}%
\definecolor{currentstroke}{rgb}{0.121569,0.466667,0.705882}%
\pgfsetstrokecolor{currentstroke}%
\pgfsetdash{}{0pt}%
\pgfpathmoveto{\pgfqpoint{0.786528in}{7.668088in}}%
\pgfpathlineto{\pgfqpoint{0.925417in}{7.668088in}}%
\pgfpathlineto{\pgfqpoint{1.064306in}{7.668088in}}%
\pgfusepath{stroke}%
\end{pgfscope}%
\begin{pgfscope}%
\definecolor{textcolor}{rgb}{0.000000,0.000000,0.000000}%
\pgfsetstrokecolor{textcolor}%
\pgfsetfillcolor{textcolor}%
\pgftext[x=1.175417in,y=7.619477in,left,base]{\color{textcolor}{\rmfamily\fontsize{10.000000}{12.000000}\selectfont\catcode`\^=\active\def^{\ifmmode\sp\else\^{}\fi}\catcode`\%=\active\def%{\%}CPLEX}}%
\end{pgfscope}%
\begin{pgfscope}%
\pgfsetrectcap%
\pgfsetroundjoin%
\pgfsetlinewidth{2.007500pt}%
\definecolor{currentstroke}{rgb}{1.000000,0.498039,0.054902}%
\pgfsetstrokecolor{currentstroke}%
\pgfsetdash{}{0pt}%
\pgfpathmoveto{\pgfqpoint{0.786528in}{7.464231in}}%
\pgfpathlineto{\pgfqpoint{0.925417in}{7.464231in}}%
\pgfpathlineto{\pgfqpoint{1.064306in}{7.464231in}}%
\pgfusepath{stroke}%
\end{pgfscope}%
\begin{pgfscope}%
\definecolor{textcolor}{rgb}{0.000000,0.000000,0.000000}%
\pgfsetstrokecolor{textcolor}%
\pgfsetfillcolor{textcolor}%
\pgftext[x=1.175417in,y=7.415620in,left,base]{\color{textcolor}{\rmfamily\fontsize{10.000000}{12.000000}\selectfont\catcode`\^=\active\def^{\ifmmode\sp\else\^{}\fi}\catcode`\%=\active\def%{\%}PACS, $\rho =.1$}}%
\end{pgfscope}%
\begin{pgfscope}%
\pgfsetrectcap%
\pgfsetroundjoin%
\pgfsetlinewidth{2.007500pt}%
\definecolor{currentstroke}{rgb}{0.172549,0.627451,0.172549}%
\pgfsetstrokecolor{currentstroke}%
\pgfsetdash{}{0pt}%
\pgfpathmoveto{\pgfqpoint{0.786528in}{7.256508in}}%
\pgfpathlineto{\pgfqpoint{0.925417in}{7.256508in}}%
\pgfpathlineto{\pgfqpoint{1.064306in}{7.256508in}}%
\pgfusepath{stroke}%
\end{pgfscope}%
\begin{pgfscope}%
\definecolor{textcolor}{rgb}{0.000000,0.000000,0.000000}%
\pgfsetstrokecolor{textcolor}%
\pgfsetfillcolor{textcolor}%
\pgftext[x=1.175417in,y=7.207897in,left,base]{\color{textcolor}{\rmfamily\fontsize{10.000000}{12.000000}\selectfont\catcode`\^=\active\def^{\ifmmode\sp\else\^{}\fi}\catcode`\%=\active\def%{\%}PACS, $\rho =.25$}}%
\end{pgfscope}%
\begin{pgfscope}%
\pgfsetrectcap%
\pgfsetroundjoin%
\pgfsetlinewidth{2.007500pt}%
\definecolor{currentstroke}{rgb}{0.839216,0.152941,0.156863}%
\pgfsetstrokecolor{currentstroke}%
\pgfsetdash{}{0pt}%
\pgfpathmoveto{\pgfqpoint{0.786528in}{7.048785in}}%
\pgfpathlineto{\pgfqpoint{0.925417in}{7.048785in}}%
\pgfpathlineto{\pgfqpoint{1.064306in}{7.048785in}}%
\pgfusepath{stroke}%
\end{pgfscope}%
\begin{pgfscope}%
\definecolor{textcolor}{rgb}{0.000000,0.000000,0.000000}%
\pgfsetstrokecolor{textcolor}%
\pgfsetfillcolor{textcolor}%
\pgftext[x=1.175417in,y=7.000174in,left,base]{\color{textcolor}{\rmfamily\fontsize{10.000000}{12.000000}\selectfont\catcode`\^=\active\def^{\ifmmode\sp\else\^{}\fi}\catcode`\%=\active\def%{\%}PACS, $\rho =.5$}}%
\end{pgfscope}%
\begin{pgfscope}%
\pgfsetrectcap%
\pgfsetroundjoin%
\pgfsetlinewidth{2.007500pt}%
\definecolor{currentstroke}{rgb}{0.580392,0.403922,0.741176}%
\pgfsetstrokecolor{currentstroke}%
\pgfsetdash{}{0pt}%
\pgfpathmoveto{\pgfqpoint{0.786528in}{6.841062in}}%
\pgfpathlineto{\pgfqpoint{0.925417in}{6.841062in}}%
\pgfpathlineto{\pgfqpoint{1.064306in}{6.841062in}}%
\pgfusepath{stroke}%
\end{pgfscope}%
\begin{pgfscope}%
\definecolor{textcolor}{rgb}{0.000000,0.000000,0.000000}%
\pgfsetstrokecolor{textcolor}%
\pgfsetfillcolor{textcolor}%
\pgftext[x=1.175417in,y=6.792451in,left,base]{\color{textcolor}{\rmfamily\fontsize{10.000000}{12.000000}\selectfont\catcode`\^=\active\def^{\ifmmode\sp\else\^{}\fi}\catcode`\%=\active\def%{\%}PACS, $\rho =.75$}}%
\end{pgfscope}%
\begin{pgfscope}%
\pgfsetrectcap%
\pgfsetroundjoin%
\pgfsetlinewidth{2.007500pt}%
\definecolor{currentstroke}{rgb}{0.549020,0.337255,0.294118}%
\pgfsetstrokecolor{currentstroke}%
\pgfsetdash{}{0pt}%
\pgfpathmoveto{\pgfqpoint{0.786528in}{6.633340in}}%
\pgfpathlineto{\pgfqpoint{0.925417in}{6.633340in}}%
\pgfpathlineto{\pgfqpoint{1.064306in}{6.633340in}}%
\pgfusepath{stroke}%
\end{pgfscope}%
\begin{pgfscope}%
\definecolor{textcolor}{rgb}{0.000000,0.000000,0.000000}%
\pgfsetstrokecolor{textcolor}%
\pgfsetfillcolor{textcolor}%
\pgftext[x=1.175417in,y=6.584729in,left,base]{\color{textcolor}{\rmfamily\fontsize{10.000000}{12.000000}\selectfont\catcode`\^=\active\def^{\ifmmode\sp\else\^{}\fi}\catcode`\%=\active\def%{\%}PACS, $\rho =.9$}}%
\end{pgfscope}%
\end{pgfpicture}%
\makeatother%
\endgroup%
}
%     \end{minipage}%
%     \hfill
%     \begin{minipage}{0.4\columnwidth} 
%         \centering
%         \resizebox{\linewidth}{!}{%% Creator: Matplotlib, PGF backend
%%
%% To include the figure in your LaTeX document, write
%%   \input{<filename>.pgf}
%%
%% Make sure the required packages are loaded in your preamble
%%   \usepackage{pgf}
%%
%% Also ensure that all the required font packages are loaded; for instance,
%% the lmodern package is sometimes necessary when using math font.
%%   \usepackage{lmodern}
%%
%% Figures using additional raster images can only be included by \input if
%% they are in the same directory as the main LaTeX file. For loading figures
%% from other directories you can use the `import` package
%%   \usepackage{import}
%%
%% and then include the figures with
%%   \import{<path to file>}{<filename>.pgf}
%%
%% Matplotlib used the following preamble
%%   \def\mathdefault#1{#1}
%%   \everymath=\expandafter{\the\everymath\displaystyle}
%%   \IfFileExists{scrextend.sty}{
%%     \usepackage[fontsize=10.000000pt]{scrextend}
%%   }{
%%     \renewcommand{\normalsize}{\fontsize{10.000000}{12.000000}\selectfont}
%%     \normalsize
%%   }
%%   
%%   \ifdefined\pdftexversion\else  % non-pdftex case.
%%     \usepackage{fontspec}
%%     \setmainfont{DejaVuSerif.ttf}[Path=\detokenize{/home/bisca/.global/lib/python3.12/site-packages/matplotlib/mpl-data/fonts/ttf/}]
%%     \setsansfont{DejaVuSans.ttf}[Path=\detokenize{/home/bisca/.global/lib/python3.12/site-packages/matplotlib/mpl-data/fonts/ttf/}]
%%     \setmonofont{DejaVuSansMono.ttf}[Path=\detokenize{/home/bisca/.global/lib/python3.12/site-packages/matplotlib/mpl-data/fonts/ttf/}]
%%   \fi
%%   \makeatletter\@ifpackageloaded{underscore}{}{\usepackage[strings]{underscore}}\makeatother
%%
\begingroup%
\makeatletter%
\begin{pgfpicture}%
\pgfpathrectangle{\pgfpointorigin}{\pgfqpoint{8.000000in}{5.000000in}}%
\pgfusepath{use as bounding box, clip}%
\begin{pgfscope}%
\pgfsetbuttcap%
\pgfsetmiterjoin%
\definecolor{currentfill}{rgb}{1.000000,1.000000,1.000000}%
\pgfsetfillcolor{currentfill}%
\pgfsetlinewidth{0.000000pt}%
\definecolor{currentstroke}{rgb}{1.000000,1.000000,1.000000}%
\pgfsetstrokecolor{currentstroke}%
\pgfsetdash{}{0pt}%
\pgfpathmoveto{\pgfqpoint{0.000000in}{0.000000in}}%
\pgfpathlineto{\pgfqpoint{8.000000in}{0.000000in}}%
\pgfpathlineto{\pgfqpoint{8.000000in}{5.000000in}}%
\pgfpathlineto{\pgfqpoint{0.000000in}{5.000000in}}%
\pgfpathlineto{\pgfqpoint{0.000000in}{0.000000in}}%
\pgfpathclose%
\pgfusepath{fill}%
\end{pgfscope}%
\begin{pgfscope}%
\pgfsetbuttcap%
\pgfsetmiterjoin%
\definecolor{currentfill}{rgb}{1.000000,1.000000,1.000000}%
\pgfsetfillcolor{currentfill}%
\pgfsetlinewidth{0.000000pt}%
\definecolor{currentstroke}{rgb}{0.000000,0.000000,0.000000}%
\pgfsetstrokecolor{currentstroke}%
\pgfsetstrokeopacity{0.000000}%
\pgfsetdash{}{0pt}%
\pgfpathmoveto{\pgfqpoint{0.706528in}{0.395972in}}%
\pgfpathlineto{\pgfqpoint{7.850000in}{0.395972in}}%
\pgfpathlineto{\pgfqpoint{7.850000in}{4.850000in}}%
\pgfpathlineto{\pgfqpoint{0.706528in}{4.850000in}}%
\pgfpathlineto{\pgfqpoint{0.706528in}{0.395972in}}%
\pgfpathclose%
\pgfusepath{fill}%
\end{pgfscope}%
\begin{pgfscope}%
\pgfpathrectangle{\pgfqpoint{0.706528in}{0.395972in}}{\pgfqpoint{7.143472in}{4.454028in}}%
\pgfusepath{clip}%
\pgfsetbuttcap%
\pgfsetmiterjoin%
\definecolor{currentfill}{rgb}{0.121569,0.466667,0.705882}%
\pgfsetfillcolor{currentfill}%
\pgfsetlinewidth{0.000000pt}%
\definecolor{currentstroke}{rgb}{0.000000,0.000000,0.000000}%
\pgfsetstrokecolor{currentstroke}%
\pgfsetstrokeopacity{0.000000}%
\pgfsetdash{}{0pt}%
\pgfpathmoveto{\pgfqpoint{1.031231in}{0.395972in}}%
\pgfpathlineto{\pgfqpoint{1.926964in}{0.395972in}}%
\pgfpathlineto{\pgfqpoint{1.926964in}{4.637903in}}%
\pgfpathlineto{\pgfqpoint{1.031231in}{4.637903in}}%
\pgfpathlineto{\pgfqpoint{1.031231in}{0.395972in}}%
\pgfpathclose%
\pgfusepath{fill}%
\end{pgfscope}%
\begin{pgfscope}%
\pgfpathrectangle{\pgfqpoint{0.706528in}{0.395972in}}{\pgfqpoint{7.143472in}{4.454028in}}%
\pgfusepath{clip}%
\pgfsetbuttcap%
\pgfsetmiterjoin%
\definecolor{currentfill}{rgb}{1.000000,0.498039,0.054902}%
\pgfsetfillcolor{currentfill}%
\pgfsetlinewidth{0.000000pt}%
\definecolor{currentstroke}{rgb}{0.000000,0.000000,0.000000}%
\pgfsetstrokecolor{currentstroke}%
\pgfsetstrokeopacity{0.000000}%
\pgfsetdash{}{0pt}%
\pgfpathmoveto{\pgfqpoint{2.150898in}{0.395972in}}%
\pgfpathlineto{\pgfqpoint{3.046631in}{0.395972in}}%
\pgfpathlineto{\pgfqpoint{3.046631in}{3.645973in}}%
\pgfpathlineto{\pgfqpoint{2.150898in}{3.645973in}}%
\pgfpathlineto{\pgfqpoint{2.150898in}{0.395972in}}%
\pgfpathclose%
\pgfusepath{fill}%
\end{pgfscope}%
\begin{pgfscope}%
\pgfpathrectangle{\pgfqpoint{0.706528in}{0.395972in}}{\pgfqpoint{7.143472in}{4.454028in}}%
\pgfusepath{clip}%
\pgfsetbuttcap%
\pgfsetmiterjoin%
\definecolor{currentfill}{rgb}{0.172549,0.627451,0.172549}%
\pgfsetfillcolor{currentfill}%
\pgfsetlinewidth{0.000000pt}%
\definecolor{currentstroke}{rgb}{0.000000,0.000000,0.000000}%
\pgfsetstrokecolor{currentstroke}%
\pgfsetstrokeopacity{0.000000}%
\pgfsetdash{}{0pt}%
\pgfpathmoveto{\pgfqpoint{3.270564in}{0.395972in}}%
\pgfpathlineto{\pgfqpoint{4.166297in}{0.395972in}}%
\pgfpathlineto{\pgfqpoint{4.166297in}{3.598020in}}%
\pgfpathlineto{\pgfqpoint{3.270564in}{3.598020in}}%
\pgfpathlineto{\pgfqpoint{3.270564in}{0.395972in}}%
\pgfpathclose%
\pgfusepath{fill}%
\end{pgfscope}%
\begin{pgfscope}%
\pgfpathrectangle{\pgfqpoint{0.706528in}{0.395972in}}{\pgfqpoint{7.143472in}{4.454028in}}%
\pgfusepath{clip}%
\pgfsetbuttcap%
\pgfsetmiterjoin%
\definecolor{currentfill}{rgb}{0.839216,0.152941,0.156863}%
\pgfsetfillcolor{currentfill}%
\pgfsetlinewidth{0.000000pt}%
\definecolor{currentstroke}{rgb}{0.000000,0.000000,0.000000}%
\pgfsetstrokecolor{currentstroke}%
\pgfsetstrokeopacity{0.000000}%
\pgfsetdash{}{0pt}%
\pgfpathmoveto{\pgfqpoint{4.390231in}{0.395972in}}%
\pgfpathlineto{\pgfqpoint{5.285964in}{0.395972in}}%
\pgfpathlineto{\pgfqpoint{5.285964in}{2.862285in}}%
\pgfpathlineto{\pgfqpoint{4.390231in}{2.862285in}}%
\pgfpathlineto{\pgfqpoint{4.390231in}{0.395972in}}%
\pgfpathclose%
\pgfusepath{fill}%
\end{pgfscope}%
\begin{pgfscope}%
\pgfpathrectangle{\pgfqpoint{0.706528in}{0.395972in}}{\pgfqpoint{7.143472in}{4.454028in}}%
\pgfusepath{clip}%
\pgfsetbuttcap%
\pgfsetmiterjoin%
\definecolor{currentfill}{rgb}{0.580392,0.403922,0.741176}%
\pgfsetfillcolor{currentfill}%
\pgfsetlinewidth{0.000000pt}%
\definecolor{currentstroke}{rgb}{0.000000,0.000000,0.000000}%
\pgfsetstrokecolor{currentstroke}%
\pgfsetstrokeopacity{0.000000}%
\pgfsetdash{}{0pt}%
\pgfpathmoveto{\pgfqpoint{5.509897in}{0.395972in}}%
\pgfpathlineto{\pgfqpoint{6.405630in}{0.395972in}}%
\pgfpathlineto{\pgfqpoint{6.405630in}{1.949572in}}%
\pgfpathlineto{\pgfqpoint{5.509897in}{1.949572in}}%
\pgfpathlineto{\pgfqpoint{5.509897in}{0.395972in}}%
\pgfpathclose%
\pgfusepath{fill}%
\end{pgfscope}%
\begin{pgfscope}%
\pgfpathrectangle{\pgfqpoint{0.706528in}{0.395972in}}{\pgfqpoint{7.143472in}{4.454028in}}%
\pgfusepath{clip}%
\pgfsetbuttcap%
\pgfsetmiterjoin%
\definecolor{currentfill}{rgb}{0.549020,0.337255,0.294118}%
\pgfsetfillcolor{currentfill}%
\pgfsetlinewidth{0.000000pt}%
\definecolor{currentstroke}{rgb}{0.000000,0.000000,0.000000}%
\pgfsetstrokecolor{currentstroke}%
\pgfsetstrokeopacity{0.000000}%
\pgfsetdash{}{0pt}%
\pgfpathmoveto{\pgfqpoint{6.629564in}{0.395972in}}%
\pgfpathlineto{\pgfqpoint{7.525297in}{0.395972in}}%
\pgfpathlineto{\pgfqpoint{7.525297in}{1.286961in}}%
\pgfpathlineto{\pgfqpoint{6.629564in}{1.286961in}}%
\pgfpathlineto{\pgfqpoint{6.629564in}{0.395972in}}%
\pgfpathclose%
\pgfusepath{fill}%
\end{pgfscope}%
\begin{pgfscope}%
\pgfsetbuttcap%
\pgfsetroundjoin%
\definecolor{currentfill}{rgb}{0.000000,0.000000,0.000000}%
\pgfsetfillcolor{currentfill}%
\pgfsetlinewidth{0.803000pt}%
\definecolor{currentstroke}{rgb}{0.000000,0.000000,0.000000}%
\pgfsetstrokecolor{currentstroke}%
\pgfsetdash{}{0pt}%
\pgfsys@defobject{currentmarker}{\pgfqpoint{0.000000in}{-0.048611in}}{\pgfqpoint{0.000000in}{0.000000in}}{%
\pgfpathmoveto{\pgfqpoint{0.000000in}{0.000000in}}%
\pgfpathlineto{\pgfqpoint{0.000000in}{-0.048611in}}%
\pgfusepath{stroke,fill}%
}%
\begin{pgfscope}%
\pgfsys@transformshift{1.479098in}{0.395972in}%
\pgfsys@useobject{currentmarker}{}%
\end{pgfscope}%
\end{pgfscope}%
\begin{pgfscope}%
\definecolor{textcolor}{rgb}{0.000000,0.000000,0.000000}%
\pgfsetstrokecolor{textcolor}%
\pgfsetfillcolor{textcolor}%
\pgftext[x=1.479098in,y=0.298750in,,top]{\color{textcolor}{\rmfamily\fontsize{10.000000}{12.000000}\selectfont\catcode`\^=\active\def^{\ifmmode\sp\else\^{}\fi}\catcode`\%=\active\def%{\%}CPLEX}}%
\end{pgfscope}%
\begin{pgfscope}%
\pgfsetbuttcap%
\pgfsetroundjoin%
\definecolor{currentfill}{rgb}{0.000000,0.000000,0.000000}%
\pgfsetfillcolor{currentfill}%
\pgfsetlinewidth{0.803000pt}%
\definecolor{currentstroke}{rgb}{0.000000,0.000000,0.000000}%
\pgfsetstrokecolor{currentstroke}%
\pgfsetdash{}{0pt}%
\pgfsys@defobject{currentmarker}{\pgfqpoint{0.000000in}{-0.048611in}}{\pgfqpoint{0.000000in}{0.000000in}}{%
\pgfpathmoveto{\pgfqpoint{0.000000in}{0.000000in}}%
\pgfpathlineto{\pgfqpoint{0.000000in}{-0.048611in}}%
\pgfusepath{stroke,fill}%
}%
\begin{pgfscope}%
\pgfsys@transformshift{2.598764in}{0.395972in}%
\pgfsys@useobject{currentmarker}{}%
\end{pgfscope}%
\end{pgfscope}%
\begin{pgfscope}%
\definecolor{textcolor}{rgb}{0.000000,0.000000,0.000000}%
\pgfsetstrokecolor{textcolor}%
\pgfsetfillcolor{textcolor}%
\pgftext[x=2.598764in,y=0.298750in,,top]{\color{textcolor}{\rmfamily\fontsize{10.000000}{12.000000}\selectfont\catcode`\^=\active\def^{\ifmmode\sp\else\^{}\fi}\catcode`\%=\active\def%{\%}PACS, $\rho=.1$}}%
\end{pgfscope}%
\begin{pgfscope}%
\pgfsetbuttcap%
\pgfsetroundjoin%
\definecolor{currentfill}{rgb}{0.000000,0.000000,0.000000}%
\pgfsetfillcolor{currentfill}%
\pgfsetlinewidth{0.803000pt}%
\definecolor{currentstroke}{rgb}{0.000000,0.000000,0.000000}%
\pgfsetstrokecolor{currentstroke}%
\pgfsetdash{}{0pt}%
\pgfsys@defobject{currentmarker}{\pgfqpoint{0.000000in}{-0.048611in}}{\pgfqpoint{0.000000in}{0.000000in}}{%
\pgfpathmoveto{\pgfqpoint{0.000000in}{0.000000in}}%
\pgfpathlineto{\pgfqpoint{0.000000in}{-0.048611in}}%
\pgfusepath{stroke,fill}%
}%
\begin{pgfscope}%
\pgfsys@transformshift{3.718431in}{0.395972in}%
\pgfsys@useobject{currentmarker}{}%
\end{pgfscope}%
\end{pgfscope}%
\begin{pgfscope}%
\definecolor{textcolor}{rgb}{0.000000,0.000000,0.000000}%
\pgfsetstrokecolor{textcolor}%
\pgfsetfillcolor{textcolor}%
\pgftext[x=3.718431in,y=0.298750in,,top]{\color{textcolor}{\rmfamily\fontsize{10.000000}{12.000000}\selectfont\catcode`\^=\active\def^{\ifmmode\sp\else\^{}\fi}\catcode`\%=\active\def%{\%}PACS, $\rho=.25$}}%
\end{pgfscope}%
\begin{pgfscope}%
\pgfsetbuttcap%
\pgfsetroundjoin%
\definecolor{currentfill}{rgb}{0.000000,0.000000,0.000000}%
\pgfsetfillcolor{currentfill}%
\pgfsetlinewidth{0.803000pt}%
\definecolor{currentstroke}{rgb}{0.000000,0.000000,0.000000}%
\pgfsetstrokecolor{currentstroke}%
\pgfsetdash{}{0pt}%
\pgfsys@defobject{currentmarker}{\pgfqpoint{0.000000in}{-0.048611in}}{\pgfqpoint{0.000000in}{0.000000in}}{%
\pgfpathmoveto{\pgfqpoint{0.000000in}{0.000000in}}%
\pgfpathlineto{\pgfqpoint{0.000000in}{-0.048611in}}%
\pgfusepath{stroke,fill}%
}%
\begin{pgfscope}%
\pgfsys@transformshift{4.838097in}{0.395972in}%
\pgfsys@useobject{currentmarker}{}%
\end{pgfscope}%
\end{pgfscope}%
\begin{pgfscope}%
\definecolor{textcolor}{rgb}{0.000000,0.000000,0.000000}%
\pgfsetstrokecolor{textcolor}%
\pgfsetfillcolor{textcolor}%
\pgftext[x=4.838097in,y=0.298750in,,top]{\color{textcolor}{\rmfamily\fontsize{10.000000}{12.000000}\selectfont\catcode`\^=\active\def^{\ifmmode\sp\else\^{}\fi}\catcode`\%=\active\def%{\%}PACS, $\rho=.5$}}%
\end{pgfscope}%
\begin{pgfscope}%
\pgfsetbuttcap%
\pgfsetroundjoin%
\definecolor{currentfill}{rgb}{0.000000,0.000000,0.000000}%
\pgfsetfillcolor{currentfill}%
\pgfsetlinewidth{0.803000pt}%
\definecolor{currentstroke}{rgb}{0.000000,0.000000,0.000000}%
\pgfsetstrokecolor{currentstroke}%
\pgfsetdash{}{0pt}%
\pgfsys@defobject{currentmarker}{\pgfqpoint{0.000000in}{-0.048611in}}{\pgfqpoint{0.000000in}{0.000000in}}{%
\pgfpathmoveto{\pgfqpoint{0.000000in}{0.000000in}}%
\pgfpathlineto{\pgfqpoint{0.000000in}{-0.048611in}}%
\pgfusepath{stroke,fill}%
}%
\begin{pgfscope}%
\pgfsys@transformshift{5.957764in}{0.395972in}%
\pgfsys@useobject{currentmarker}{}%
\end{pgfscope}%
\end{pgfscope}%
\begin{pgfscope}%
\definecolor{textcolor}{rgb}{0.000000,0.000000,0.000000}%
\pgfsetstrokecolor{textcolor}%
\pgfsetfillcolor{textcolor}%
\pgftext[x=5.957764in,y=0.298750in,,top]{\color{textcolor}{\rmfamily\fontsize{10.000000}{12.000000}\selectfont\catcode`\^=\active\def^{\ifmmode\sp\else\^{}\fi}\catcode`\%=\active\def%{\%}PACS, $\rho=.75$}}%
\end{pgfscope}%
\begin{pgfscope}%
\pgfsetbuttcap%
\pgfsetroundjoin%
\definecolor{currentfill}{rgb}{0.000000,0.000000,0.000000}%
\pgfsetfillcolor{currentfill}%
\pgfsetlinewidth{0.803000pt}%
\definecolor{currentstroke}{rgb}{0.000000,0.000000,0.000000}%
\pgfsetstrokecolor{currentstroke}%
\pgfsetdash{}{0pt}%
\pgfsys@defobject{currentmarker}{\pgfqpoint{0.000000in}{-0.048611in}}{\pgfqpoint{0.000000in}{0.000000in}}{%
\pgfpathmoveto{\pgfqpoint{0.000000in}{0.000000in}}%
\pgfpathlineto{\pgfqpoint{0.000000in}{-0.048611in}}%
\pgfusepath{stroke,fill}%
}%
\begin{pgfscope}%
\pgfsys@transformshift{7.077430in}{0.395972in}%
\pgfsys@useobject{currentmarker}{}%
\end{pgfscope}%
\end{pgfscope}%
\begin{pgfscope}%
\definecolor{textcolor}{rgb}{0.000000,0.000000,0.000000}%
\pgfsetstrokecolor{textcolor}%
\pgfsetfillcolor{textcolor}%
\pgftext[x=7.077430in,y=0.298750in,,top]{\color{textcolor}{\rmfamily\fontsize{10.000000}{12.000000}\selectfont\catcode`\^=\active\def^{\ifmmode\sp\else\^{}\fi}\catcode`\%=\active\def%{\%}PACS, $\rho=.9$}}%
\end{pgfscope}%
\begin{pgfscope}%
\pgfpathrectangle{\pgfqpoint{0.706528in}{0.395972in}}{\pgfqpoint{7.143472in}{4.454028in}}%
\pgfusepath{clip}%
\pgfsetbuttcap%
\pgfsetroundjoin%
\pgfsetlinewidth{0.803000pt}%
\definecolor{currentstroke}{rgb}{0.690196,0.690196,0.690196}%
\pgfsetstrokecolor{currentstroke}%
\pgfsetstrokeopacity{0.500000}%
\pgfsetdash{{2.960000pt}{1.280000pt}}{0.000000pt}%
\pgfpathmoveto{\pgfqpoint{0.706528in}{0.395972in}}%
\pgfpathlineto{\pgfqpoint{7.850000in}{0.395972in}}%
\pgfusepath{stroke}%
\end{pgfscope}%
\begin{pgfscope}%
\pgfsetbuttcap%
\pgfsetroundjoin%
\definecolor{currentfill}{rgb}{0.000000,0.000000,0.000000}%
\pgfsetfillcolor{currentfill}%
\pgfsetlinewidth{0.803000pt}%
\definecolor{currentstroke}{rgb}{0.000000,0.000000,0.000000}%
\pgfsetstrokecolor{currentstroke}%
\pgfsetdash{}{0pt}%
\pgfsys@defobject{currentmarker}{\pgfqpoint{-0.048611in}{0.000000in}}{\pgfqpoint{-0.000000in}{0.000000in}}{%
\pgfpathmoveto{\pgfqpoint{-0.000000in}{0.000000in}}%
\pgfpathlineto{\pgfqpoint{-0.048611in}{0.000000in}}%
\pgfusepath{stroke,fill}%
}%
\begin{pgfscope}%
\pgfsys@transformshift{0.706528in}{0.395972in}%
\pgfsys@useobject{currentmarker}{}%
\end{pgfscope}%
\end{pgfscope}%
\begin{pgfscope}%
\definecolor{textcolor}{rgb}{0.000000,0.000000,0.000000}%
\pgfsetstrokecolor{textcolor}%
\pgfsetfillcolor{textcolor}%
\pgftext[x=0.520940in, y=0.343211in, left, base]{\color{textcolor}{\rmfamily\fontsize{10.000000}{12.000000}\selectfont\catcode`\^=\active\def^{\ifmmode\sp\else\^{}\fi}\catcode`\%=\active\def%{\%}0}}%
\end{pgfscope}%
\begin{pgfscope}%
\pgfpathrectangle{\pgfqpoint{0.706528in}{0.395972in}}{\pgfqpoint{7.143472in}{4.454028in}}%
\pgfusepath{clip}%
\pgfsetbuttcap%
\pgfsetroundjoin%
\pgfsetlinewidth{0.803000pt}%
\definecolor{currentstroke}{rgb}{0.690196,0.690196,0.690196}%
\pgfsetstrokecolor{currentstroke}%
\pgfsetstrokeopacity{0.500000}%
\pgfsetdash{{2.960000pt}{1.280000pt}}{0.000000pt}%
\pgfpathmoveto{\pgfqpoint{0.706528in}{1.367414in}}%
\pgfpathlineto{\pgfqpoint{7.850000in}{1.367414in}}%
\pgfusepath{stroke}%
\end{pgfscope}%
\begin{pgfscope}%
\pgfsetbuttcap%
\pgfsetroundjoin%
\definecolor{currentfill}{rgb}{0.000000,0.000000,0.000000}%
\pgfsetfillcolor{currentfill}%
\pgfsetlinewidth{0.803000pt}%
\definecolor{currentstroke}{rgb}{0.000000,0.000000,0.000000}%
\pgfsetstrokecolor{currentstroke}%
\pgfsetdash{}{0pt}%
\pgfsys@defobject{currentmarker}{\pgfqpoint{-0.048611in}{0.000000in}}{\pgfqpoint{-0.000000in}{0.000000in}}{%
\pgfpathmoveto{\pgfqpoint{-0.000000in}{0.000000in}}%
\pgfpathlineto{\pgfqpoint{-0.048611in}{0.000000in}}%
\pgfusepath{stroke,fill}%
}%
\begin{pgfscope}%
\pgfsys@transformshift{0.706528in}{1.367414in}%
\pgfsys@useobject{currentmarker}{}%
\end{pgfscope}%
\end{pgfscope}%
\begin{pgfscope}%
\definecolor{textcolor}{rgb}{0.000000,0.000000,0.000000}%
\pgfsetstrokecolor{textcolor}%
\pgfsetfillcolor{textcolor}%
\pgftext[x=0.432575in, y=1.314652in, left, base]{\color{textcolor}{\rmfamily\fontsize{10.000000}{12.000000}\selectfont\catcode`\^=\active\def^{\ifmmode\sp\else\^{}\fi}\catcode`\%=\active\def%{\%}50}}%
\end{pgfscope}%
\begin{pgfscope}%
\pgfpathrectangle{\pgfqpoint{0.706528in}{0.395972in}}{\pgfqpoint{7.143472in}{4.454028in}}%
\pgfusepath{clip}%
\pgfsetbuttcap%
\pgfsetroundjoin%
\pgfsetlinewidth{0.803000pt}%
\definecolor{currentstroke}{rgb}{0.690196,0.690196,0.690196}%
\pgfsetstrokecolor{currentstroke}%
\pgfsetstrokeopacity{0.500000}%
\pgfsetdash{{2.960000pt}{1.280000pt}}{0.000000pt}%
\pgfpathmoveto{\pgfqpoint{0.706528in}{2.338855in}}%
\pgfpathlineto{\pgfqpoint{7.850000in}{2.338855in}}%
\pgfusepath{stroke}%
\end{pgfscope}%
\begin{pgfscope}%
\pgfsetbuttcap%
\pgfsetroundjoin%
\definecolor{currentfill}{rgb}{0.000000,0.000000,0.000000}%
\pgfsetfillcolor{currentfill}%
\pgfsetlinewidth{0.803000pt}%
\definecolor{currentstroke}{rgb}{0.000000,0.000000,0.000000}%
\pgfsetstrokecolor{currentstroke}%
\pgfsetdash{}{0pt}%
\pgfsys@defobject{currentmarker}{\pgfqpoint{-0.048611in}{0.000000in}}{\pgfqpoint{-0.000000in}{0.000000in}}{%
\pgfpathmoveto{\pgfqpoint{-0.000000in}{0.000000in}}%
\pgfpathlineto{\pgfqpoint{-0.048611in}{0.000000in}}%
\pgfusepath{stroke,fill}%
}%
\begin{pgfscope}%
\pgfsys@transformshift{0.706528in}{2.338855in}%
\pgfsys@useobject{currentmarker}{}%
\end{pgfscope}%
\end{pgfscope}%
\begin{pgfscope}%
\definecolor{textcolor}{rgb}{0.000000,0.000000,0.000000}%
\pgfsetstrokecolor{textcolor}%
\pgfsetfillcolor{textcolor}%
\pgftext[x=0.344210in, y=2.286093in, left, base]{\color{textcolor}{\rmfamily\fontsize{10.000000}{12.000000}\selectfont\catcode`\^=\active\def^{\ifmmode\sp\else\^{}\fi}\catcode`\%=\active\def%{\%}100}}%
\end{pgfscope}%
\begin{pgfscope}%
\pgfpathrectangle{\pgfqpoint{0.706528in}{0.395972in}}{\pgfqpoint{7.143472in}{4.454028in}}%
\pgfusepath{clip}%
\pgfsetbuttcap%
\pgfsetroundjoin%
\pgfsetlinewidth{0.803000pt}%
\definecolor{currentstroke}{rgb}{0.690196,0.690196,0.690196}%
\pgfsetstrokecolor{currentstroke}%
\pgfsetstrokeopacity{0.500000}%
\pgfsetdash{{2.960000pt}{1.280000pt}}{0.000000pt}%
\pgfpathmoveto{\pgfqpoint{0.706528in}{3.310296in}}%
\pgfpathlineto{\pgfqpoint{7.850000in}{3.310296in}}%
\pgfusepath{stroke}%
\end{pgfscope}%
\begin{pgfscope}%
\pgfsetbuttcap%
\pgfsetroundjoin%
\definecolor{currentfill}{rgb}{0.000000,0.000000,0.000000}%
\pgfsetfillcolor{currentfill}%
\pgfsetlinewidth{0.803000pt}%
\definecolor{currentstroke}{rgb}{0.000000,0.000000,0.000000}%
\pgfsetstrokecolor{currentstroke}%
\pgfsetdash{}{0pt}%
\pgfsys@defobject{currentmarker}{\pgfqpoint{-0.048611in}{0.000000in}}{\pgfqpoint{-0.000000in}{0.000000in}}{%
\pgfpathmoveto{\pgfqpoint{-0.000000in}{0.000000in}}%
\pgfpathlineto{\pgfqpoint{-0.048611in}{0.000000in}}%
\pgfusepath{stroke,fill}%
}%
\begin{pgfscope}%
\pgfsys@transformshift{0.706528in}{3.310296in}%
\pgfsys@useobject{currentmarker}{}%
\end{pgfscope}%
\end{pgfscope}%
\begin{pgfscope}%
\definecolor{textcolor}{rgb}{0.000000,0.000000,0.000000}%
\pgfsetstrokecolor{textcolor}%
\pgfsetfillcolor{textcolor}%
\pgftext[x=0.344210in, y=3.257535in, left, base]{\color{textcolor}{\rmfamily\fontsize{10.000000}{12.000000}\selectfont\catcode`\^=\active\def^{\ifmmode\sp\else\^{}\fi}\catcode`\%=\active\def%{\%}150}}%
\end{pgfscope}%
\begin{pgfscope}%
\pgfpathrectangle{\pgfqpoint{0.706528in}{0.395972in}}{\pgfqpoint{7.143472in}{4.454028in}}%
\pgfusepath{clip}%
\pgfsetbuttcap%
\pgfsetroundjoin%
\pgfsetlinewidth{0.803000pt}%
\definecolor{currentstroke}{rgb}{0.690196,0.690196,0.690196}%
\pgfsetstrokecolor{currentstroke}%
\pgfsetstrokeopacity{0.500000}%
\pgfsetdash{{2.960000pt}{1.280000pt}}{0.000000pt}%
\pgfpathmoveto{\pgfqpoint{0.706528in}{4.281738in}}%
\pgfpathlineto{\pgfqpoint{7.850000in}{4.281738in}}%
\pgfusepath{stroke}%
\end{pgfscope}%
\begin{pgfscope}%
\pgfsetbuttcap%
\pgfsetroundjoin%
\definecolor{currentfill}{rgb}{0.000000,0.000000,0.000000}%
\pgfsetfillcolor{currentfill}%
\pgfsetlinewidth{0.803000pt}%
\definecolor{currentstroke}{rgb}{0.000000,0.000000,0.000000}%
\pgfsetstrokecolor{currentstroke}%
\pgfsetdash{}{0pt}%
\pgfsys@defobject{currentmarker}{\pgfqpoint{-0.048611in}{0.000000in}}{\pgfqpoint{-0.000000in}{0.000000in}}{%
\pgfpathmoveto{\pgfqpoint{-0.000000in}{0.000000in}}%
\pgfpathlineto{\pgfqpoint{-0.048611in}{0.000000in}}%
\pgfusepath{stroke,fill}%
}%
\begin{pgfscope}%
\pgfsys@transformshift{0.706528in}{4.281738in}%
\pgfsys@useobject{currentmarker}{}%
\end{pgfscope}%
\end{pgfscope}%
\begin{pgfscope}%
\definecolor{textcolor}{rgb}{0.000000,0.000000,0.000000}%
\pgfsetstrokecolor{textcolor}%
\pgfsetfillcolor{textcolor}%
\pgftext[x=0.344210in, y=4.228976in, left, base]{\color{textcolor}{\rmfamily\fontsize{10.000000}{12.000000}\selectfont\catcode`\^=\active\def^{\ifmmode\sp\else\^{}\fi}\catcode`\%=\active\def%{\%}200}}%
\end{pgfscope}%
\begin{pgfscope}%
\definecolor{textcolor}{rgb}{0.000000,0.000000,0.000000}%
\pgfsetstrokecolor{textcolor}%
\pgfsetfillcolor{textcolor}%
\pgftext[x=0.288654in,y=2.622986in,,bottom,rotate=90.000000]{\color{textcolor}{\rmfamily\fontsize{10.000000}{12.000000}\selectfont\catcode`\^=\active\def^{\ifmmode\sp\else\^{}\fi}\catcode`\%=\active\def%{\%}Integral Value}}%
\end{pgfscope}%
\begin{pgfscope}%
\pgfsetrectcap%
\pgfsetmiterjoin%
\pgfsetlinewidth{0.803000pt}%
\definecolor{currentstroke}{rgb}{0.000000,0.000000,0.000000}%
\pgfsetstrokecolor{currentstroke}%
\pgfsetdash{}{0pt}%
\pgfpathmoveto{\pgfqpoint{0.706528in}{0.395972in}}%
\pgfpathlineto{\pgfqpoint{0.706528in}{4.850000in}}%
\pgfusepath{stroke}%
\end{pgfscope}%
\begin{pgfscope}%
\pgfsetrectcap%
\pgfsetmiterjoin%
\pgfsetlinewidth{0.803000pt}%
\definecolor{currentstroke}{rgb}{0.000000,0.000000,0.000000}%
\pgfsetstrokecolor{currentstroke}%
\pgfsetdash{}{0pt}%
\pgfpathmoveto{\pgfqpoint{7.850000in}{0.395972in}}%
\pgfpathlineto{\pgfqpoint{7.850000in}{4.850000in}}%
\pgfusepath{stroke}%
\end{pgfscope}%
\begin{pgfscope}%
\pgfsetrectcap%
\pgfsetmiterjoin%
\pgfsetlinewidth{0.803000pt}%
\definecolor{currentstroke}{rgb}{0.000000,0.000000,0.000000}%
\pgfsetstrokecolor{currentstroke}%
\pgfsetdash{}{0pt}%
\pgfpathmoveto{\pgfqpoint{0.706528in}{0.395972in}}%
\pgfpathlineto{\pgfqpoint{7.850000in}{0.395972in}}%
\pgfusepath{stroke}%
\end{pgfscope}%
\begin{pgfscope}%
\pgfsetrectcap%
\pgfsetmiterjoin%
\pgfsetlinewidth{0.803000pt}%
\definecolor{currentstroke}{rgb}{0.000000,0.000000,0.000000}%
\pgfsetstrokecolor{currentstroke}%
\pgfsetdash{}{0pt}%
\pgfpathmoveto{\pgfqpoint{0.706528in}{4.850000in}}%
\pgfpathlineto{\pgfqpoint{7.850000in}{4.850000in}}%
\pgfusepath{stroke}%
\end{pgfscope}%
\end{pgfpicture}%
\makeatother%
\endgroup%
}
%     \end{minipage}
%     \caption{Success Rate vs. Computation Time for **}
%     \label{fig:}
% \end{figure}


% \begin{figure}[thpb]
%     \centering
%     \begin{minipage}{0.6\columnwidth}
%         \centering
%         \resizebox{\linewidth}{!}{\input{chapter/img/}}
%     \end{minipage}%
%     \hfill
%     \begin{minipage}{0.4\columnwidth}
%         \centering
%         \resizebox{\linewidth}{!}{\input{chapter/img/}}
%     \end{minipage}
%     \caption{Success Rate vs. MIP Gap Plot for **}
%     \label{fig:}
% \end{figure}
