% \pdfinfo{
%     /Author (Francesco Biscaccia Carrara)
%     /Title (Enhancing the ACS Heuristic: Parameter Tuning and Computational Experiments for Solving MIP Problems)
%     /Subject (Master Degree Thesis in Computer Engineering UniPD 2024/2025)
%     /Keyword (ACS, Heuristic, MIP, Parallelism, MIPLIB2017)
% }
\pdfminorversion=7
\begin{filecontents*}[overwrite]{\jobname.xmpdata}
    \Title{Enhancing the ACS Heuristic: Parameter Tuning and Computational Experiments for Solving MIP Problems}
    \Author{Francesco Biscaccia Carrara}
    \Language{en}
    \Subject{Master Degree Thesis in Computer Engineering UniPD 2024/2025}
    \Keywords{ACS\sep Heuristic\sep MIP\sep Parallelism\sep MIPLIB2017}
\end{filecontents*}

% Libraries

\documentclass[a4paper,12pt]{report}
%\documentclass[a4paper,12pt,twoside]{report} %solo per stampa
\usepackage{colorprofiles}
\usepackage[a-2b,mathxmp]{pdfx}
\hypersetup{pdfstartview=}

\usepackage[utf8]{inputenc}
\usepackage{bookmark}
\usepackage{hyperref}
\usepackage[english]{babel}
\usepackage{fancyhdr}%%%%%%%%%%
\usepackage{sectsty}
\usepackage[inner=3cm,top=2cm,bottom=2cm,outer=2cm]{geometry}
\usepackage{setspace}
\usepackage[hang,small,sf,font=small, labelfont=bf]{caption}
\usepackage{subcaption}
\usepackage{graphicx}
\graphicspath{{./chapter/img/}}
\usepackage{cancel}
\usepackage{amsmath,amssymb}
\usepackage{xcolor}
\usepackage{colortbl}
\usepackage{tocloft}
\usepackage{algorithm}
\usepackage{algpseudocode}
\usepackage{tabto}
\usepackage{pgf}
\usepackage{pgfplots}

\usepackage{tikz}
\usepackage{circuitikz}

\setcounter{tocdepth}{3} %subsubsection on index

\usepackage{amsmath}
\usepackage{amsthm}
\usepackage[sorting=none]{biblatex} % For biblatex
\addbibresource{refs.bib} % Path to your .bib file
\usetikzlibrary{shapes.geometric, arrows.meta, positioning, fit, backgrounds}
\usetikzlibrary{patterns}
\usetikzlibrary{math}
\pgfplotsset{width=10cm,compat=1.9}

\usepackage{indentfirst}
\setlength{\arrayrulewidth}{1pt}
\usepackage{afterpage}
\newcommand\blankpage{%
    \null%
    \thispagestyle{empty}%
    \addtocounter{page}{-1}%
    \newpage}

% Mandatory settings
\onehalfspacing%
\hypersetup{
    colorlinks,
    citecolor=black,
    filecolor=black,
    linkcolor=black,
    urlcolor=black,
    pdfpagelabels
}

% Subsections
\renewcommand{\cftpartleader}{\cftdotfill{\cftdotsep}} % for parts
\renewcommand{\cftchapleader}{\cftdotfill{\cftdotsep}} % for chapters
\renewcommand{\cftsecleader}{\cftdotfill{\cftdotsep}} % for sections


%\pagestyle{fancy}
%\renewcommand{\chaptermark}[1]{\markboth{\chaptername\ \thechapter.\ #1 }{}}
%\renewcommand{\sectionmark}[1]{\markright{\thesection\ #1}{}}
%\fancyhead{}
%\fancyhead[LE,RO]{\sffamily \thepage}
%\fancyhead[RE]{\sffamily \leftmark}
%\fancyhead[LO]{\sffamily \rightmark}
%\fancyfoot{}

%\fancypagestyle{plain}{ \fancyhead{} \fancyfoot{}
%\fancyfoot[C]{\sffamily \thepage}
%\renewcommand{\headrulewidth}{0pt}}%%%%%%%%%%

% Start document
\begin{document}
\begin{titlepage}
\begin{center}

\includegraphics[height=0.13\textheight]{logo_unipd.png}
\hfill
\includegraphics[height=0.13\textheight]{logo_dei.png}
\newline
\newline

\vspace{0.8cm}
\textsc{\LARGE Universit\`{a} degli Studi di Padova}\\
\vspace{1.6cm}
\textsc{\large School of Engineering Department of Information Engineering}\\
\vspace{0.4cm}

\textsc{\large Master Degree in Computer Engineering}\\
\vfill
{ \LARGE \bfseries Enhancing the ACS Heuristic: Parameter Tuning and Computational Experiments for Solving MIP Problems}\\
\vfill

\textit{\large Supervisor:} \hfill \textit{\large Candidate:}\\
\textsc{\large Prof.\ Domenico Salvagnin} \hfill \textsc{Francesco Biscaccia Carrara}\\
\textit{\large {}} \hfill \textsc{2120934}\\

\vfill
{\large Academic Year 2024/2025}\\
{Date 07/10/2025} 
\end{center}
\end{titlepage}

\thispagestyle{empty}
\cleardoublepage%

\clearpage\null\newpage

\pagenumbering{roman}
\thispagestyle{empty}
\clearpage{\pagestyle{plain}\cleardoublepage}


% Abstract
\newcommand\summaryname{Abstract}
\newenvironment{Abstract}%
    {\begin{center}%
    \bfseries{\summaryname} \end{center}}

\begin{Abstract}
    This thesis focuses on enhancing the Alternating Criteria Search (ACS) heuristic for solving large and complex Mixed Integer Programming (MIP) problems. While the Parallel ACS (PACS) approach has proven effective in generating high-quality solutions through large-scale parallelization, it remains limited by reliance on specialized hardware and by the need for manual parameter calibration. To overcome these issues, this work develops a refined PACS framework that adapts automatically to different problem instances and computing environments. The improvements center on a dynamic mechanism for adjusting the variable fixing strategy, a lightweight initialization method that avoids parameter tuning, and a reformulation of subproblems that leverages slack-based constraints for greater efficiency. Extensive computational experiments on hard instances from MIPLIB 2017 confirm that the refined PACS consistently achieves better solution quality than the baseline approach, while remaining robust, architecture-agnostic, and easy to integrate into modern MIP solvers.
\end{Abstract}
\newpage
\begin{Abstract}
    Questa tesi si concentra sul perfezionamento dell’euristica Alternating Criteria Search (ACS) per la risoluzione di problemi di Programmazione Mista Intera (MIP) di grande dimensione e complessità. Sebbene l’approccio Parallel ACS (PACS) abbia dimostrato di essere efficace nel generare soluzioni di alta qualità attraverso la parallelizzazione su larga scala, esso presenta ancora limiti legati alla dipendenza da hardware specializzato e alla necessità di una calibrazione manuale dei parametri. Per superare tali criticità, questo lavoro propone un framework PACS perfezionato, capace di adattarsi automaticamente a diversi tipi di problemi e a differenti ambienti di calcolo. Le migliorie principali riguardano un meccanismo dinamico per l’adattamento della strategia di fissaggio delle variabili, un metodo di inizializzazione leggero che elimina la necessità di tarature manuali, e una riformulazione dei sottoproblemi basata su vincoli slack per garantire maggiore efficienza. Ampi test computazionali su istanze complesse tratte da MIPLIB 2017 confermano come il PACS perfezionato raggiunga sistematicamente risultati di qualità superiore rispetto all’approccio di riferimento, mantenendosi al contempo robusto, indipendente dall’architettura e facilmente integrabile nei moderni solver MIP.
\end{Abstract}

\afterpage{\blankpage}

% Index
\clearpage{\pagestyle{plain}\cleardoublepage}
\tableofcontents
%\listoffigures
%\listoftables%

\afterpage{\blankpage}
% \afterpage{\blankpage}

\clearpage{\pagestyle{plain}\cleardoublepage}
\pagenumbering{arabic}

% Introduction
\clearpage{\pagestyle{plain}\cleardoublepage}
\chapter{Introduction}
\section{Mixed Integer Programming (MIP)}
Mixed Integer Programming (MIP) is a powerful optimization technique used to model complex real-world problems where discrete decisions are essential—such as resource allocation, scheduling, and logistics.
MIP models aim to find the best solution according to a given objective function, which is either maximized or minimized.
A generic mixed-integer program (MIP) will be defined as
\begin{equation}
\begin{cases}
\text{min} \quad & c^T x \\
\text{s.t.} \quad & Ax = b \\
                        & x_i \in \mathbb{Z},\; \forall i \in \mathcal{I} \\
                        & l \le x \le u
\end{cases}
\end{equation}
where $c \in \mathbb{R}^n$, $A \in \mathbb{R}^{m \times n}$ and $\mathcal{I}\subseteq\{1\dots n\}$ is the subset of integer variables indices. The solution vector $x$ is bounded by $l \in \bar{\mathbb{R}}^n$ and $u \in \bar{\mathbb{R}}^n$ where $\bar{\mathbb{R}}= \mathbb{R} \cup \{-\infty,\infty\}$.
Here, some decision variables are constrained to take integer values, which increases the model’s complexity.
\subsection{Heuristic in MIP solving}
The presence of integer variables significantly increases the computational complexity of MIP problems, especially as the number of variables and constraints grows. Solvers such as IBM ILOG CPLEX$^\text{\cite{cplex}}$ and GUROBI$^\text{\cite{gurobi}}$ incorporate a variety of built-in strategies to improve performance. Among these strategies are heuristics, which are specialized algorithms designed to quickly find feasible (but not necessarily optimal) solutions. Although heuristics do not guarantee optimality, they are computationally efficient and can significantly accelerate the solution 
process by providing good initial solutions or guiding the search within the solution space.

\section{Alternating Criteria Search (ACS)}
Finding high-quality feasible solutions is a key aspect of the discrete optimization process, and constructing them has been one of the main focuses of research in the last few decades.  
Primal heuristics differ in whether they require a starting feasible solution or not. In the first case, the heuristics are called starting heuristics, and state-of-the-art techniques rely on Large Neighborhood Search (LNS)$^\text{\cite{LNS}}$ to find high-quality feasible solutions. The effectiveness of such strategies relies on the power of the MIP solver to solve the subproblems generated during the LNS heuristic.  
Improvement heuristics, in contrast, require a feasible starting solution and aim to improve its quality with respect to the objective. The simplest improvements are the 1-opt and 2-opt$^\text{\cite{2opt}}$ methods, but—as with starting heuristics—there are many improvement heuristics based on LNS ideas. The neighborhoods explored are defined based on branch-and-bound information—such as the best incumbent and the LP relaxation.  
This is essentially a limitation in terms of diversification: the strategies may reach good solutions, but it is difficult to find them in the early stages of the search.  
Therefore, to address this issue, the Alternating Criteria Search (ACS)$^\text{\cite{ACS}}$—or Parallel ACS, its straightforward parallel implementation—can be utilized. The search neighborhoods are defined based on randomization instead of branch-and-bound information, allowing for a wider exploration of the search space. The aim of this strategy is to find high-quality feasible solutions at the beginning of the search, increasing the heuristic's effectiveness.

\subsection{Parallelization of ACS}
Parallelism can be exploited in MIP optimization to accelerate computation and enhance solver performance, in terms of execution time and scalability.
The exploration of the branch-and-bound tree in parallel is now incorporated in most state-of-the-art MIP solvers—such as CPLEX or GUROBI—and it essentially consists of solving multiple subproblems of the original MIP simultaneously.  
However, this strategy may not scale well to a large number of cores$^\text{\cite{largeParallel}}$.
To address this limitation, the proposed heuristic—Parallel ACS (PACS)—is a parallel algorithm that integrates elements of both starting solution and improvement heuristics, offering the capability of generating starting solutions and improving them with respect to the original objective.  
To leverage parallelism, Parallel ACS performs a large number of LNS runs simultaneously over a diversified set of neighborhoods, with the aim of increasing the chances of finding higher-quality solutions. After each LNS exploration, the strategy consolidates the local improvements through a recombination phase.  
PACS combines parallelism and diversified LNS in order to address large instances of MIPs arising from various application domains.

\subsection{Limitations of the PACS Approach}\label{sec:lim_PACS}
Although the experiments about the PACS strategy bring evidence about the effectiveness on hard MIP instances, there are several considerations that must be addressed.
First of all, the PACS algorithm was executed on an 8-node computing cluster, each equipped with two Intel Xeon X5650 6-core processors and 24 GB of memory, totaling 96 cores and 192 GB of RAM. Such computing resources are not typically available in consumer-grade or general-purpose environments. Furthermore, as discussed in the original study, the PACS strategy could assist the MIP solver in accelerating convergence. This suggests that integrating PACS heuristics within a MIP solver could provide a meaningful contribution to solver performance. 
Second, before the PACS execution, an initial calibration phase is required to tune the parameters prior to execution. This undermines the objective of designing a self-contained, efficient heuristic.
With these considerations in mind, the goal of this thesis is to refine the PACS strategy to make it more compatible with MIP solvers. In particular, the aim is to establish a fixed, robust parameter configuration that remains effective regardless of the input instance.

\clearpage\null\newpage

% % Chapter 1
\clearpage{\pagestyle{plain}\cleardoublepage}
\chapter{The ACS Framework}
The Alternating Criteria Search (ACS) heuristic is designed to pursue a twofold objective: to identify a feasible solution and to subsequently enhance its quality with respect to the objective function $c^T x$. To this end, ACS employs a Large Neighborhood Search (LNS) strategy, which iteratively solves two auxiliary mixed-integer programming (MIP) subproblems in order to address both objectives.
The heuristic requires an initial vector, which is not mandated to be a feasible solution for the original MIP. This vector is progressively refined by solving sub-MIPs in which a subset of variables is fixed to the values of the initial input.
\section{Feasibility-MIP (FMIP)}
In linear programming theory, it is well established that the feasibility problem can be addressed using the two-phase Simplex method, in which an auxiliary optimization problem is solved to obtain a feasible starting basis, and hence a feasible solution.
In a similar manner, the following auxiliary MIP problem, denoted as the Feasibility-MIP (FMIP), is formulated to identify a feasible starting solution:
\begin{equation}
\begin{cases}
\text{min} \quad & \sum_{i=0}^m \Delta_i^{+}+\Delta_i^{-} \\ \text{s.t.} \quad & Ax + I_m\Delta^+ - I_m\Delta^- =b\\ & x_i = \hat{x}_i, \; \forall i \in F\\ & l \le x \le u\\ & x_i \in \mathbb{Z}, \; \forall i \in \mathcal{I} \\ & \Delta^+ \ge 0, \Delta^- \ge 0 
\end{cases}
\end{equation}
Here, $I_m$ denotes an $m \times m$ identity matrix, while $\Delta^+$ and $\Delta^-$ are vectors in $R^m$ corresponding to the $m$ constraints. These are introduced as slack variables, and the objective is to minimize their sum. 
Analogous to the two-phase Simplex method, a vector $x$ is feasible for the original MIP if and only if it can be extended to a solution of value $0$ for the associated FMIP. Since solving an FMIP is as computationally demanding as solving the original problem, the neighborhoods are restricted by fixing a given subset $F$ of the integer variables to the values of an input vector $[\hat{x}, \Delta^+, \Delta^-]$. Because of the introduction of slack variables, the vector $\hat{x}$ is not required to be a feasible solution itself, but it must be integral and within the variable bounds to preserve the feasibility of the model. In this way, the FMIP guarantees that feasibility is preserved under any arbitrary variable-fixing scheme. Once a feasible solution is obtained, any LNS-based improving heuristic—such as RINS$^\text{\cite{RINS}}$, DINS$^\text{\cite{DINS}}$, or local branching$^\text{\cite{localBranching}}$—can then be applied to refine the solution.
\section{Optimality-MIP (OMIP)}
Rather than executing the FMIP until convergence to a feasible solution vector, the following auxiliary MIP problem, denoted as Optimality-MIP (OMIP), is designed to improve a partially feasible solution $[\hat{x}, \hat{\Delta}^+, \hat{\Delta}^-]$, which satisfies $\sum_{i=1}^m (\hat{\Delta}_i^+ + \hat{\Delta}_i^-) \neq 0$, with respect to the original objective $c^T x$:
\begin{equation}
\begin{cases}
\text{min} \quad & c^T x \\ \text{s.t.} \quad & Ax + I_m\Delta^+ - I_m\Delta^- = b\\ & \sum_{i=0}^m \Delta_i^{+}+\Delta_i^{-} \le \sum_{i=0}^m \hat\Delta_i^{+}+\hat\Delta_i^{-}\\ & x_i = \hat{x}_i, \; \forall i \in F\\ & l \le x \le u\\ & x_i \in \mathbb{Z} \quad \forall i \in \mathcal{I} \\ & \Delta^+ \ge 0, \; \Delta^- \ge 0 
\end{cases}
\end{equation}
Analogous to the FMIP, the OMIP represents a reformulation of the original MIP model in which auxiliary slack variables are introduced for each constraint. This formulation enables the OMIP to enhance a solution even if it is not feasible for the original MIP. Moreover, to ensure that the optimal solution of the OMIP does not exceed the infeasibility of the input solution $\hat{x}$, an additional budget constraint is imposed, limiting the total slack to $\sum_{i=0}^m \hat\Delta_i^{+}+\hat\Delta_i^{-}$.

By iteratively solving subproblems of both auxiliary MIPs, the ACS heuristic is designed to converge—although convergence is not formally guaranteed—to a high-quality feasible solution. By construction, infeasibility decreases monotonically after each iteration. However, the quality of the solution with respect to the original objective function may fluctuate.

\section{Parallelization of ACS}
The parallelization of ACS exploits parallelism by generating a diversified set of large neighborhood searches, which are solved simultaneously. Exploring multiple search neighborhoods in parallel is expected to increase the likelihood of identifying high-quality solutions.
Following this parallelization step, the improvements obtained in parallel must be combined efficiently. To this end, an additional search subproblem is generated in which variables with identical values across different solutions are fixed. Consequently, the recombination phase constitutes a crucial step in PACS: by merging the improvements achieved during the parallel phase, the subsequent phase can explore a newly diversified set of large neighborhood searches based on the recombined solution, thereby enhancing the probability of further improvements.
The pseudocode[\ref{alg:PACS}] illustrates the overall workflow.
\begin{algorithm}[h]
\caption{Parallel Alternating Criteria Search}\label{alg:PACS}
\begin{algorithmic}
\Function{FMIP\_LNS}{$\mathcal{F}, \hat{x}$}
    \State \Return $\min\{\sum_i \Delta^+_i + \Delta^-_i \mid A x + I_m \Delta^+ - I_m \Delta^- = b, \; x_j = \hat{x}_j \; \forall j \in F, \; x_j \in \mathbb{Z} \; \forall j \in \mathcal{I}\}$
\EndFunction
\end{algorithmic}
\vspace{1em}
\begin{algorithmic}
\Function{OMIP\_LNS}{$\mathcal{F}, \hat{x}, \hat{\Delta}$}
    \State \Return $\min\{c^t x \mid A x + I_m \Delta^+ - I_m \Delta^- = b, \; \sum_i \Delta^+_i + \Delta^-_i \leq \hat{\Delta}, \; x_j = \hat{x}_j \; \forall j \in F, \; x_j \in \mathbb{Z} \; \forall j \in \mathcal{I}\}$
\EndFunction
\end{algorithmic}
\vspace{1em}
\begin{algorithmic}[1]
\Ensure{Feasible solution $\hat{x}$ if found}
\State Initialize $[\hat{x}, \Delta^+, \Delta^-]$ as an integer solution
\State $T := numThreads()$
\While{$timeElapsed()\le $ timeLimit}
    \If{$\sum_i \Delta_i^+ + \Delta_i^- > 0$}
        \ForAll{threads $t_i \in \{0, T-1\}$ \textbf{in parallel}}
            \State $F_{t_i} :=$ randomized variable index subset, $F_{t_i} \subseteq \mathcal{I}$ 
            \State $[x^{t_i}, \Delta^{+t_i}, \Delta^{-t_i}] :=$ FMIP\_LNS$(F_{t_i}, \hat{x})$
        \EndFor
        \State $U := \{ j \in \mathcal{I} | x^{t_i}_j = x^{t_k}_j, \; 0 \leq i < k < T\}$
        \State $[\hat{x}, \Delta^+, \Delta^-] :=$ FMIP\_LNS$(U, x^{t_0})$
    \EndIf
    \State $\Delta^{UB} := \sum_i \Delta^+_i + \Delta^-_i$
    \ForAll{threads $t_i \in \{0, T-1\}$ \textbf{in parallel}}
            \State $F_{t_i} :=$ randomized variable index subset, $F_{t_i} \subseteq \mathcal{I}$
            \State $[x^{t_i}, \Delta^{+t_i}, \Delta^{-t_i}] :=$ OMIP\_LNS$(F_{t_i}, \hat{x}, \Delta^{UB})$
    \EndFor
    \State $U := \{ j \in \mathcal{I} | x^{t_i}_j = x^{t_k}_j, \; 0 \leq i < k < T\}$
    \State $[\hat{x}, \Delta^+, \Delta^-] :=$ OMIP\_LNS$(U, x^{t_0}, \Delta^{UB})$
    
\EndWhile
\State \Return $[\hat{x}, \Delta^+, \Delta^-]$
\end{algorithmic}
\end{algorithm}
Each processor generates a set of randomized variable fixings and solves the associated sub-MIP—either FMIP or OMIP—until the allotted time limit is reached.
Subsequently, the solutions are exchanged, and the set $U$, containing the indices of variables common across solutions, is constructed. The recombination MIP then consists of a subproblem, again associated with either FMIP or OMIP, in which the variables in $U$ are fixed.
The best solution obtained will be either the most feasible or the most optimal, depending on whether a recombination FMIP or OMIP is employed. Moreover, every solution used as input can be incorporated as a MIP start, as it remains feasible under any variable-fixing strategy.
Processor synchronization and memory communication are handled via the Message Passing Interface (MPI)$^\text{\cite{MPI}}$, owing to its efficient all-to-all collective communication primitives in distributed large-scale architectures.

\section{Initialization of ACS}
As introduced earlier in this chapter, ACS only requires a starting vector that is integer feasible and within the variable bounds. However, a stronger starting point is a solution that is as feasible as possible with respect to the objective function of the FMIP.
The proposed algorithm provides a lightweight heuristic that seeks to minimize the infeasibility of the initial solution. This algorithm can be described by the pseudocode[\ref{alg:starting_vector}].
\begin{algorithm}
\caption{Starting vector heuristic}\label{alg:starting_vector}
\begin{algorithmic}[1]
\Require{Percentage of variables to fix $\theta$, $0 < \theta \leq 100$, Fixed bound constant $c_b$}
\Ensure{Starting integer-feasible vector $\hat{x}$}
\State $V :=$ list of integer variables sorted by increasing bound range $u-l$
\State $F := \emptyset$
\While{$\hat{x}$ is not integer feasible \textbf{AND} $F \neq \mathcal{I}$}
    \State $\mathcal{K} :=$ top $\theta \%$ of unfixed variables from $V$
    \For{$k \in \mathcal{K}$}
        \State $\hat{x}_k :=$ random integer value between $[\max(l_k, -c_b), \min(u_k, c_b)]$
    \EndFor
    \State $F := F \cup \mathcal{K}$
    \State $[x, \Delta^+, \Delta^-] := \min\{\sum_i \Delta_i^+ + \Delta_i^- \mid A x + I_m \Delta^+ - I_m \Delta^- = b, \; x_j = \hat{x}_j \;\; \forall j \in F\}$
    \State $Q :=$ index set of integer variables of $x$ with integer value
    \State $\hat{x}_q = x_q, \;\; \forall q \in Q$
    \State $F := F \cup Q$
\EndWhile
\State \Return $\hat{x}$
\end{algorithmic}
\end{algorithm}
The algorithm first sorts the list of integer variables in order of increasing bound range. It then fixes the top $\theta$\% of variables to random integer values within their respective bounds. The input parameter $\theta$ controls the trade-off between the difficulty of the LP relaxation and the quality of the resulting starting solution. In cases where the bounds are infinite, a constant value of $10^6$ is used to clamp the bounds.
The rationale behind the sorting step is to prioritize binary variables first, followed by the remaining integer variables.
Until all integer variables are fixed, the LP relaxation of the FMIP is solved to optimize the unfixed variables. Any variables that attain integer values in this process are then fixed.
Since at least $\theta$\% of the variables are fixed at each iteration, the algorithm is guaranteed to terminate after at most $\lceil 100 / \theta \rceil$ iterations.
\section{Variable Fixing Strategy}\label{sec:PACS_var_fix}
Selecting an appropriate variable fixing scheme is a challenging task: an overly restrictive strategy may fail to yield improvements, whereas an excessively loose strategy can lead to a search space that is too large to explore efficiently within a reasonable timespan.
The proposed algorithm constitutes a simple yet intuitive variable fixing method. It incorporates randomness to promote diversification and allows for controlling the number of variables to be fixed through an adjustable parameter.
The algorithm is described in the  pseudocode[\ref{alg:variable_fixing}].
\begin{algorithm}
\caption{Variable Fixing Selection Algorithm}\label{alg:variable_fixing}
\begin{algorithmic}[1]
\Require{Fraction of variables to fix $\rho$, $0 < \rho < 1$}
\Ensure{Set of integer indices $F$}
\Function{RandomFixings}{$\rho$}
    \State $i :=$ random element in $\mathcal{I}$
    \State $F :=$ first $\rho \cdot |\mathcal{I}|$ consecutive integer variable indices starting from $i$ in a circular fashion
    \State \Return $F$
\EndFunction
\end{algorithmic}
\end{algorithm}
The fixings are determined by selecting a random integer variable $x'$ and fixing a consecutive sequence of integer variables starting from $x'$ up to a cap determined by $\rho$, an input parameter that specifies the number of variables to be fixed. The fixing is performed in a circular fashion: if the end of the set $\mathcal{I}$ is reached before the required number of variables are fixed, the algorithm continues from the beginning of $\mathcal{I}$.
The effectiveness of this strategy relies on the fact that, for many problems—such as network flow and routing—the variables are arranged consecutively, often defining a cohesive substructure within the problem.
\clearpage\null\newpage

% % Chapter 2
\clearpage{\pagestyle{plain}\cleardoublepage}
\chapter{The PACS Tuning}
% As discussed in the section[\ref{sec:lim_PACS}], the baseline PACS has a some limitations that make it not suitable for the general-purpose environments. Hence, the PACS parameter and PACS itself are tuned up with this aim: make PACS hardware and input independent, in order to be embedded into a state-of-the-art MIP solver-such as IBM ILOG CPLEX or GUROBI-.
% Parameter to tune up: theta, rho, timing 

\section{PACS with Generalized Fixing}
The baseline PACS algorithm enforces both the starting vector construction and the fixing scheme to operate exclusively on the set $\mathcal{I}$ of integer variables. While this restriction may appear efficient and straightforward, it can, in fact, be limiting. In particular, if the MIP instance contains only a small fraction of integer variables relative to the total, focusing solely on them may reduce diversification and cause the algorithm to converge prematurely to a local minimum, which is undesirable in an optimization process.
To overcome this issue, Algorithm \ref{alg:starting_vector} and Algorithm \ref{alg:variable_fixing} are generalized into Algorithm \ref{alg:gen_starting_vector} and Algorithm \ref{alg:gen_variable_fixing}, respectively, thereby allowing both integer and continuous variables to be considered.
\begin{algorithm}[H]
\caption{Generalized Starting vector heuristic}\label{alg:gen_starting_vector}
\begin{algorithmic}[1]
\Require{Percentage of variables to fix $\theta$, $0 < \theta \leq 100$, Fixed bound constant $c_b$}
\Ensure{Starting integer-feasible vector $\hat{x}$}
\State $V :=$ list of \cancel{integer} variables sorted by increasing bound range $u-l$
\State $F := \emptyset$
\While{$\hat{x}$ is not integer feasible \textbf{AND} $F \neq V$}
    \State $\mathcal{K} :=$ top $\theta \%$ of unfixed variables from $V$
    \For{$k \in \mathcal{K}$}
        \State $\hat{x}_k :=$ random integer value between $[\max(l_k, -c_b), \min(u_k, c_b)]$
    \EndFor
    \State $F := F \cup \mathcal{K}$
    \State $[x, \Delta^+, \Delta^-] := \min\{\sum_i \Delta_i^+ + \Delta_i^- \mid A x + I_m \Delta^+ - I_m \Delta^- = b, \; x_j = \hat{x}_j \;\; \forall j \in F\}$
    \State $Q :=$ index set of \cancel{integer} variables of $x$ with integer value
    \State $\hat{x}_q = x_q, \;\; \forall q \in Q$
    \State $F := F \cup Q$
\EndWhile
\State \Return $\hat{x}$
\end{algorithmic}
\end{algorithm}
\begin{algorithm}[H]
\caption{Generalized Variable Fixing Selection Algorithm}\label{alg:gen_variable_fixing}
\begin{algorithmic}[1]
\Require{Fraction of variables to fix $\rho$, $0 < \rho < 1$}
\Ensure{Set of integer indices $F$}
\Function{RandomFixings}{$\rho$}
    \State $i :=$ random element in $\{1\dots n\}$ \Comment $n$: number of variables in the original MIP 
    \State $F :=$ first $\rho \cdot n$ consecutive \cancel{integer} variable indices starting from $i$ in a circular fashion
    \State \Return $F$
\EndFunction
\end{algorithmic}
\end{algorithm}
Since these algorithms are executed on the auxiliary MIP problems -FMIP or OMIP-, the variables subject to fixing correspond exactly to those defined in the original MIP formulation.  
By generalizing the fixing strategy to include both integer and continuous variables, diversification is enhanced, thereby reducing the likelihood of stagnation in local minima and potentially improving the exploration of the solution space.

\section{Architecture-Agnostic Parallelization}
In the original study, the Message Passing Interface (MPI) was employed to synchronize processors at each recombination phase. While this approach is well suited to high-performance computing environments, it may be unnecessarily complex in general-purpose scenarios, where a simpler multi-threading implementation is often preferable.  
In this thesis, communication is instead managed through a set of logical threads, which may differ from the number of available hardware threads. This abstraction ensures that, even on machines with fewer physical cores, the algorithm can reproduce the same behavior across different architectures, provided sufficient computational time is allowed.  
More specifically, during the coordination phase, either the most feasible or the most optimal solution—depending on whether a recombination FMIP or OMIP is performed—is shared among the logical processors. Each processor then continues working independently on its own copy, with updates to the incumbent solution handled exclusively through a thread-safe update function, formally described in Algorithm~\ref{alg:inc_update}.  
\begin{algorithm}[H]
\caption{Parallel ACS Incumbent Update Procedure}\label{alg:inc_update}
\begin{algorithmic}[1]
\Require Candidate solution $x$ with slack sum $S(x)$ and objective value $C(x)$; Incumbent $\tilde{x}$; Zero-tolerance $\epsilon$
\Ensure Updated incumbent $\tilde{x}$ in a thread-safe manner
\Function{UpdateIncumbent}{$x$}
    \State acquire lock
    \If{$(|S(x)| < |S(\tilde{x})|) \;\;\lor\;\; (|S(x)| < \epsilon \;\land\; C(x) < C(\tilde{x}))$}
        \State $\tilde{x} \gets x$
    \EndIf
    \State release lock
\EndFunction
\end{algorithmic}
\end{algorithm}

This mechanism is crucial to ensure that the algorithm consistently improves and converges within the given time limit. The incumbent solution is updated whenever a better solution is identified: either one with a smaller slack sum, indicating improved feasibility, or one with a lower objective value $c^T x$ provided that the slack sum is less than the tolerance parameter $\epsilon$, the zero-feasibility threshold.

\section{Eliminating Calibration in PACS Parameter Selection}
After adapting the PACS algorithm to a more general-purpose environment, another important challenge arises: parameter selection. Since PACS must be applicable to a wide range of hard MIP instances, it is necessary to identify parameter settings that generally perform well, both in terms of computational efficiency and heuristic solution quality.  
The goal is to remove the need for an explicit calibration phase, while still ensuring reliable performance. The tuning process specifically concerns the following parameters:
\begin{enumerate}
    \item The time span assigned to each sub-MIP, either FMIP\_LNS or OMIP\_LNS
    \item The parameter $\rho$ governing the fixing strategy 
    \item The parameter $\theta$ governing the starting vector
\end{enumerate}

\subsection{Adaptive Determination of Sub-MIP Time Span}
To guarantee determinism in the implementation, each sub-MIP is assigned both a time limit equal to the remaining computation time and a deterministic time limit. The latter is defined as the maximum number of instructions that the solver may execute before termination.  
The deterministic time limit is computed according to the following formula:
$$
TL_{DET} = \max\Big(x, \min\Big(\frac{nz}{y}, X\Big)\Big),
$$
which provides a dynamic mechanism for adapting the computational effort of each sub-MIP.  
In this formulation, $x$ and $X$ denote the minimum and maximum allowable deterministic time limits, respectively, $nz$ represents the number of nonzeros in the constraint matrix $A$ of the MIP problem, and $y$ acts as a scaling factor. The chosen parameter values are:
\begin{enumerate}
\item Minimum deterministic time limit: $x = 10^3$  
\item Scaling factor: $y = 10^2$  
\item Maximum deterministic time limit: $X = 10^7$  
\end{enumerate}
This dynamic adjustment removes the need to explicitly set a fixed time limit for each subproblem, thereby eliminating the necessity of parameter tuning in this respect.

\subsection{Adaptive Variable Fixing through $\rho$ Adjustment}
\subsubsection{Fixed $\rho$ Initialization}
As discussed in Section \ref{sec:PACS_var_fix}, selecting an appropriate variable fixing scheme is a non-trivial task. In particular, the random fixings scheme presents additional challenges: if the fraction $\rho$ of variables to be fixed is set too high, the procedure may fail to yield meaningful improvements; conversely, if $\rho$ is set too low, the resulting search space may become excessively large, making it computationally intractable within a reasonable time frame. \\
For this reason, in the first instance, the parameter $\rho$ is selected from a set of predetermined candidate values. Anticipating the results presented in Section \ref{}, it can be observed that certain values of $\rho$ are more effective in terms of solution quality, as they grant the solver greater flexibility during the optimization process.

\subsubsection{Dynamic Adjustment of Parameter $\rho$}
Since the LNS heuristics in the PACS algorthm restrict the search space by fixing a number of variables according to the value of $\rho$, it is natural to design a mechanism for dynamically adapting this parameter in order to increase the likelihood of discovering high-quality solutions.  
\begin{algorithm}[H]
\caption{Parallel ACS Rho Update (Parallel Phases)}\label{alg:rho_update_MT}
\begin{algorithmic}[1]
\Require Status code $MIP_{code}$ returned by the solver; Adjustment step $\Delta_\rho$ for $\rho$; The variable fixing parameter $\rho$; Number of parallel sub-MIPs $num_{MIP}$
\Ensure Updated value of $\rho$, synchronized across parallel sub-MIPs
\Function{DynRhoUpdate}{$MIP_{code}$, $\Delta_\rho$, $\rho$, $num_{MIP}$}
    \State $\hat\Delta_\rho \gets {\Delta_\rho \over{num_{MIP}}}$
    \State acquire lock
    \If{$MIP_{code} = OPT \;\lor\; MIP_{code} = OPT_{TOL}$}
        \State $\rho \gets \rho - \hat\Delta_\rho$
    \EndIf
    \If{$MIP_{code} = FEAS_{TL} \;\lor\; MIP_{code} = FEAS_{DET\_TL}$}
        \State $\rho \gets \rho + \hat\Delta_\rho$
    \EndIf
    \State Clap $\rho$ within $[0.01,0.99]$
    \If{*Tie Case detected*}
        \State $\rho \gets \rho - \Delta_\rho$
    \EndIf
    \State release lock
\EndFunction
\end{algorithmic}
\end{algorithm}
The procedure, described in Algorithm \ref{alg:rho_update_MT}, is applied after each sub-MIP optimization phase in the parallel step. Based on the status returned by the solver, the value of $\rho$ is updated as follows:
\begin{itemize}
    \item  If the solver hits the time limit—either the deterministic limit or the global remaining time—while still producing a feasible solution, $\rho$ is increased by $\frac{\Delta_\rho}{num_{MIP}}$. This adjustment suggests fixing more variables in subsequent phases, thereby simplifying the subproblem to be solved. 
    \item Conversely, if the solver converges to an optimal solution within the tolerance, this indicates that the corresponding region of the search space has already been sufficiently explored. In this case, $\rho$ is decreased by $\frac{\Delta_\rho}{num_{MIP}}$, enlarging the search space and granting the solver greater freedom in the following optimization steps.  
\end{itemize}
Since each sub-MIP independently attempts to modify the value of $\rho$, synchronization through locking is required to prevent inconsistencies. The final update is determined as the average of the adjustments proposed by the parallel optimization phases. In case of a tie, a deterministic rule is applied: $\rho$ is decreased by $\Delta_\rho$.  
\begin{algorithm}[H]
\caption{Parallel ACS Rho Update (Recombination Phases)}\label{alg:rho_update}
\begin{algorithmic}[1]
\Require Status code $MIP_{code}$ returned by the solver; Adjustment step $\Delta_\rho$ for $\rho$; The variable fixing parameter $\rho$; Number of parallel sub-MIPs $num_{MIP}$
\Ensure Updated value of $\rho$ after recombination adjustment
\Function{DynRhoUpdate}{$MIP_{code}$, $\Delta_\rho$, $\rho$, $num_{MIP}$}
    \State $\hat\Delta_\rho \gets {2\Delta_\rho \over{num_{MIP}}}$
    \If{$MIP_{code} = OPT \;\lor\; MIP_{code} = OPT_{TOL}$}
        \State $\rho \gets \rho - \hat\Delta_\rho$
    \EndIf
    \If{$MIP_{code} = FEAS_{TL} \;\lor\; MIP_{code} = FEAS_{DET\_TL}$}
        \State $\rho \gets \rho + \hat\Delta_\rho$
    \EndIf
    \State Clap $\rho$ within $[0.01,0.99]$
\EndFunction
\end{algorithmic}
\end{algorithm}
For the recombination phase, the procedure is slightly modified into Algorithm \ref{alg:rho_update}, where the adjustment step is doubled, i.e. $2\Delta_\rho / num_{MIP}$, in order to resolve potential tie cases.\\
Finally, although an initial value of $\rho$ must be provided to start the fixing process, experimental results in Section \ref{} show that performance is only marginally affected by this initialization. Consequently, the effectiveness of the method does not critically depend on the initial choice of $\rho$.

\subsection{Determination of Parameter $\theta$}
For simplicity, the parameter $\theta$ is initially fixed to $0.25$. A more effective initialization strategy will be discussed in the Section \ref{}, where a new heuristic is introduced.  



\clearpage\null\newpage

% % Chapter 3
\clearpage{\pagestyle{plain}\cleardoublepage}
\chapter{Experimental Results}
%intro%
\section{Experimental Setup}
The experiments are designed to compare the PACS framework with the standalone IBM ILOG CPLEX Optimization Studio (version 22.1).  
The PACS framework is implemented in C++ for performance reasons, interfacing with the solver through the C API. The source code is available under a non-commercial MIP license at \textbf{[repository link]}.  
All experiments were conducted on a cluster within the UniPD DEI Blade infrastructure, running Rocky Linux 8.10. Each node is equipped with an Intel(R) Xeon(R) E5-2623 v3 processor (4 cores, 3.00 GHz) and 16 GB of RAM. Although the infrastructure is cluster-based, the computational power of a single node is comparable to that of a general-purpose laptop, or even lower.  
For fairness in comparison, PACS executions use 4 logical threads, implemented as C++ standard threads, each running a CPLEX instance restricted to a single core. The baseline counterpart is a standalone CPLEX execution restricted to 4 cores.

\subsection{Dataset}
As anticipated in the previous sections, the benchmark for this comparison must consist of hard MIP instances. To this end, the set of hard instances from MIPLIB2017$^\text{\cite{MIPLIB}}$ has been used as the test bed.\footnote{The instance \textit{tpl-tub-ss16} was excluded, as its execution was terminated due to insufficient computational resources.}  
While CPLEX directly processes each MIP instance, the PACS framework additionally requires a random seed in order to reproduce the same randomized choices across different runs. For this purpose, PACS was evaluated on each instance using the seeds $\{38472910, 56473829, 27384910, 91827364, 8374659\}$. This setup increases statistical reliability and can be considered representative of the algorithm’s standard behavior.  
Consequently, for each instance, five independent PACS runs and one CPLEX run were executed.

\subsection{Metrics}
Both PACS and plain CPLEX terminate as soon as an incumbent solution (i.e., a feasible solution) is identified, subject to a global time limit of five minutes.
Since the PACS framework was executed five times for each instance, the reported solution quality and computation time are given as the mean across all runs.
Moreover, if the majority of PACS executions for a given instance---at least three out of five seeds---fail to produce a solution, the instance is recorded as unsolved.
Instead of directly comparing the objective values of the solutions returned by the two approaches, the MIP gap metric$^\text{\cite{MIPGAP}}$ has been adopted. The MIP gap is defined as follows: given a solution $\hat{x}$ for a MIP with an optimal solution $x$, the primal gap $\gamma(\hat{x}) \in [0,1]$ is
\begin{equation}
\gamma(\hat{x}) =
\begin{cases}
0 & \text{if } |c^T x| = |c^T \hat{x}| = 0, \\
1 & \text{if } c^T x \cdot c^T \hat{x}< 0, \\
\frac{|c^T x-c^T \hat{x}|}{\max\{|c^T \hat{x}|, |c^T x|\}} & \text{otherwise}.
\end{cases}
\end{equation}
This metric captures the relative gap between an incumbent and the best-known solution, normalized to the interval $[0,1]$.  
Since the algorithms operate under a strict time limit, the metric must be extended accordingly. Given a time limit $t_{\max}$, the primal function $p:[0,t_{\max}] \to [0,1]$ is defined as:
\begin{equation}
p(\hat{x}) =
\begin{cases}
1 & \text{if no solution is found up to time $t$}, \\
\gamma(\hat{x}(t)) & \text{if $\hat{x}(t)$ is the incumbent at time $t$}.
\end{cases}
\end{equation}
For each test, two types of plots are produced to summarize and compare performance:
\begin{enumerate}
    \item \textbf{Success Rate vs. Computation Time:}  
    The $x$-axis represents elapsed time, and the $y$-axis reports the cumulative number of instances for which an incumbent was found. The closer a curve is to the top-left corner, the more efficient the algorithm.  
    \item \textbf{Success Rate vs. MIP Gap:}  
    The $x$-axis represents the MIP gap, and the $y$-axis reports the cumulative number of instances solved with a gap less than or equal to the given value. Consistently, curves closer to the top-left corner indicate better performance 
\end{enumerate}

\subsection{Tolerance Parameters}
Because both algorithms involve floating-point computations and comparisons, PACS employs a set of tolerance parameters. In these experiments, the parameters were selected to be reasonable relative to the defaults used by CPLEX:
\begin{itemize}
    \item \textbf{Zero-tolerance parameter} $\epsilon$: values smaller than $\epsilon$ are treated as zero. In all tests, $\epsilon$ was set to $10^{-5}$.
    \item \textbf{Absolute maximum constraint violation}: the maximum permissible violation of any constraint under a candidate solution $\hat{x}$. In these tests, it was set equal to $\epsilon$, i.e., $10^{-5}$.
    \item \textbf{Absolute maximum integrality violation}: the maximum deviation allowed for variables constrained to take integer values. This tolerance was likewise set to $\epsilon = 10^{-5}$.
    \item \textbf{Relative objective error}: the maximum admissible relative error between the recomputed objective value $c^T \hat{x}$ and the value stored internally by PACS. This threshold was again set to $\epsilon = 10^{-5}$.  
\end{itemize}  
In the latter three cases, if the corresponding tolerance is exceeded, the algorithm terminates with an error and the run is recorded as unsolved.  

\section{Results}
%
%\subsection for the single experiments
%
%
\clearpage\null\newpage

% % Chapter 5
\clearpage{\pagestyle{plain}\cleardoublepage}
\chapter{Conclusion and Future Work}
Questa tesi presenta diversi approcci per la riduzione della polarizzazione nella navigazione nei grafi, ovvero gli algoritmi RePBubLik e ShuffLik.
RePBubLik permette di calcolare un'approssimazione accurata per il problema della riduzione del \emph{structural bias}.
In seguito alle considerazioni teoriche, son stati svolti dei confronti tra RePBubLik+, la versione più veloce e pratica dell'
algoritmo RePBubLik, e le attuali tecniche per la riduzione della polarizzaione dei grafi: ROV e 
Node2Vec. Dalle analisi, è emerso che RePBubLik+ permette di ridurre drasticamente il \emph{structural bias} 
del grafo con un numero di archi significativamente inferiore, rispetto agli altri approcci utilizzati. 
\\
In aggiunta, è stata brevemente descritta un tecnica alternativa a RePBubLik, che permette di aumentare
la navigabilità di una rete effettuando \emph{edge swapping}. 
Questo algoritmo prende il nome di ShuffLik ed esiste una sua versione più pratica, ShuffLik+.
Importante notare che ShuffLik risulta efficace in contesti dove non è possibile eseguire \emph{edge inserction} per motivi di costo o di vincoli funzionali, 
operando quindi interventi meno invasivi.
\\
Possiamo osservare diverse opportunità di sviluppo futuro:
\begin{enumerate}
    \item Nella pratica, potrebbe essere difficile stimare le probabilità di transizione all'interno del grafo. I pesi da assegnare agli archi potrebbero 
    essere appresi mediante un algoritmo di machine learning sui dataset.
    \item La determinazione di valori realistici del parametro $t$ non è semplice. È un problema rilevante individuare il valore ideale di random walk da fornire all'algoritmo, in funzione del trade-off tra guadagno e tempo di esecuzione.
    \item Nell'estensione a più ``colori'' dell'algoritmo, diventa importante determinare quale coppia di ``colori'' permette, con un numero ridotto ti archi, di ridurre notevolmente il bias strutturale.
\end{enumerate} 
\clearpage\null\newpage

% % Biblio
\printbibliography[nottype=online]

\end{document}